\section{活用法}
\numberParagraph
梵語の活用法は語根に若干の變化を加へてこれに語尾を
附するにある。卽ち yaj (祭る)+ a = yaja + ti (二人稱單數,現
實法語尾),yajati (彼は祭る)。

梵語にては能動調受動調ある外に能動調を更に爲他,爲自の二
類に分ち,語尾を異にす。されどその意義に至つては爲他も爲自
も同樣である。

三數は格例法の如くである。

\numberParagraph
時法に關しては異れる四類がある。1. 現在組織は現實
法,第一過去,可能法,命令法である(これら四類は共通の語基
を有す)。2. 未來時。3. 第二過去時,4. 第三過去時。この中
2, 3, 4 類は 1 類が特定の語基を有するに反し,語根へ直接に語尾
が附けられる。受動も亦さうである。只第十類動詞(\ref{np:59}, \ref{np:73} 條)の
みは大抵の形が現在語基から作られる。

\numberParagraph \label{np:59}
現在語基から作られる形,現實法,第一過去,可能法,命
令法。語基の作られる樣式によりて語基は二種十類に分たれる。

\fbox{(A) 第一種變化} 現在語基がすべて a にて終る。第一類,
第六類,第四類,第十類がこれに屬す。

第一類。語根に a を附加して語基を作る。その母音は重韻と
なる。bhū (ある)~ bhavati (彼はあり)。

第六類 語根に語勢ある a を附加して語基を作る。語根の母
音は變化せず。tub (打つ)~ tudati (彼は打つ)。

第四類。語根に ya を附加して語基を作る。div (博戯す)~
dīvyati (彼は博戯す)。

第十類。語根に aya を附加して語基を作る。語根の母音は通
常重韻化せらる。cur (盗む)~ corayati (彼は盗む)。

\fbox{(B) 第二種變化} 語根に强弱を分つ。これに屬するものは
第二類,第三類,第五類,第七類,第八類,第九類である。

第二類は語根に直に語尾を附加する。ad (食ふ)~ atti (彼は
食ふ)。

第三類。語根が重複せらる。hu (供ふ)~ juhoti (彼は供ふ),
juhumas (我々は供ふ)。

第七類。强語基に na 弱語基に n を挿んで現在語基を作る。
bhid (破る)~ bhinatti (彼は破る),bhindmas (我々は破る)。

第五類。强語基に no, 弱語基に nu を加へる。su (搾る)~
sunoti (彼は搾る),sunumas (我々は搾る)。

第八類。强語基に o, 弱語基に u を附加する。tan (擴ぐ)~
tanoti (彼は擴ぐ),tanumas (我々は擴ぐ)。

第九類。語根に nā を附加して强語基,子音語尾の前には nī,
母音語尾の前には n を附加して弱語基を作る。aś (食ふ)~
aśnāti (彼は食ふ),aśnīmas (我々は食ふ),aśnate (彼等は食
ふ)。

その他,受動並に派生動詞,卽ち催起動詞,重複動詞,求欲動詞
名稱動詞がある。

\begin{center}現在組織\end{center}
\subsection{第一種變化}
\subsubsection[第一類 a 級 現在]{第一類 a 級}
\numberParagraph \label{np:60}
a 級の構成。語根に a を附加す。語根の母音は重韻化す
る。聲中に在りて本來長きと位置によりて長きは重韻化せず。nī
(導く)現在語基 ne + a = naya (\ref{np:31}條)。nind (嘲る)~ ninda.

\numberParagraph \textbf{現實法の語尾。}

爲他 mi, si, ti; vas, thas, tas; mas, tha, anti.

爲自 e, se, te; vahe, āthe, āte; mahe, dhve, ante.

語基の a は m, v にて始まる語尾の前に延長せられ,anti,
ante の前に省略せられ,āthe, āte の ā と合して e となる。

\numberParagraph
a 級の語根 bhū (ある),現在語基,bho + a = bhava.
\begin{center}
\begin{tabular}{c*{3}{p{0.15\hsize}}}
  \multicolumn{4}{c}{\textbf{現實法}} \\
  \multicolumn{4}{c}{爲他} \\
     & 單       & 兩        & 複 \\
  1. & bhavāmi  & bhavāvas  & bhavāmas \\
  2. & bhavasi  & bhavathas & bhavatha \\
  3. & bhavati  & bhavatas  & bhavanti \\
  \multicolumn{4}{c}{爲自} \\
  1. & bhave   & bhavāvahe & bhavāmahe \\
  2. & bhavase & bhavethe  & bhavadhve \\
  3. & bhavate & bhavete   & bhabante
\end{tabular}
\end{center}

\numberParagraph
若干の不規則なる語根を擧ぐれば
\begin{enumerate}[label=(\alph*)]
\item guh (覆ふ)は語根の母音を重韻化せず。只延長す。
gūhati.
\item kram (歩む)は ya 級によりても變化せられ(\ref{np:69}條),
爲他にありて母音を延長し爲自に於ては延長せず。
\item 或る語根は鼻音を失ふ。

\indent daṃś (咬む)~ daśati.

\indent rañj (染める)~ rajati.

\indent sañj (着く)~ sajati.

\indent svañj (抱く)~ svajati.
\item gam (行く),yam (與ふ),iṣ (欲す),ṛ (達す)は夫々
語基 gaccha, yaccha, iccha, ṛccha に作る。
\item sad (坐す)~ sīdati, sthā (立つ)~ tiṣṭhati, pā (飲む)
~ pibati, ghrā (嗅ぐ)~ jighrati.
\end{enumerate}

\ex{第二}

\begin{longtable}{c*{2}{p{0.45\hsize}}}
 1. & kva gacchasi. & 汝は何處に行くか。\\
 2. & gacchāmi grāmam. & 我は村へ行く。\\
 3. & mitraṃ hvayāmi. & 我は友を喚ぶ。\\
 4. & kuto dhāvathaḥ. & 何故に汝等兩人は走るか。\\
 5. & gajasya bhayād ghāvāvaḥ. & 象の恐怖の故に我等兩人は
走る。\\
 6. & ghaṭas tale patati. & 甕は地上へ落ちたり。\\
 7. & pāpā janāḥ svargaṃ na gacchanti. & 罪ある人々は天國へ行かぬ。\\
 8. & agnis tiṣṭhati gūḍho dāruṣu. & 火は薪の中に隱れてあり。\\
 9. & mitrasya phalaṃ prayacchāmi. & 我は友に果物を與ふ。\\
10. & Devadatto 'nnaṃ pacati. & デーヷダッタは果物を煮る。\\
11. & vane vṛkam īkṣāmahe. & 我々は林の中に狼を見る。\\
12. & ācāryaḥ śiṣyaṃ nindati. & 師は弟子を責む。\\
13. & prāsādasya samīpe hrado bhavati. & 宮殿の側に池がある。\\
14. & nṛpaḥ sainyena Pāṭaliputraṃ praṭiṣṭhati. & 王は軍隊と共にパータリプトラへ出發せり。\\
15. & Godāvaryā jale gajau viha\-rataḥ. & ゴーダーヷリー河の水に於て二つの象が遊ぶ。\\
16. & padmasya pattreṣu jalaṃ na sajati. & 蓮華の葉に於て水は着かぬ。\\
17. & anilasya vaśena vṛkṣāḥ kam\-pante. & 風の力によりて樹々は動
く。\\
18. & puṣpāṇi vasante prasphoṭanti. & 花は春に於いて開く。\\
19. & na niścayād viramanti dhīrāḥ. & 勇者は計劃を中止せぬ。\\
20. & adya brāhmaṇau nagaraṃ tyajataḥ. & 今日二人の婆羅門が市城
を去つた。\\
21. & candrasyāloke kumudaṃ vika\-sati. & 月の出現に於いて蓮華は開
く。\\
22. & lobhāt krodhaḥ prabhavati & 貪慾より憤怒は生ず。\\
    & lobhāt kāmaḥ pravartate, & 貪慾より愛慾は起る。\\
    & lobhān mohaś ca nāśaś ca & 貪慾より愚痴と破滅と
(あり)。\\
    & lobhaḥ pāpasya kāraṇam. & 貪慾は罪惡の原因なり。
\end{longtable}


\subsubsection{第六類 á 級}
\begin{center}第一過去變化\end{center}

\numberParagraph
第六類では語根の母音は變化しない。附加せらるる a は
語勢を有するのが特徴である。

\numberParagraph 第一過去の語尾。

爲他 am, s, t; va, tam, tām; ma, ta, an.

爲自 i, thās, ta; vahi, āthām, ātām; mahi, dhvam, anta.

\numberParagraph
第一過去はこれらの語尾を附加する外に過去符 a を語根
の前に加ふ。又 m, v にて始まる語尾の前の語基の a は延長せら
れ,語尾 an 並に anta の前に a は省略せられ,āthām, ātām
の ā と合して e に變ず。

過去符 a は前置詞の後語根の前に加ふ。卽ち tyaj (捨)+ pari
(斷念す)三單.第一過去 paryatyajat (\ref{np:13} 條參照)。

\numberParagraph
語根 tud (打つ),現在語基 tuda.
\begin{center}
\begin{tabular}{c*{3}{p{0.15\hsize}}}
  \multicolumn{4}{c}{\textbf{第一過去}} \\
  \multicolumn{4}{c}{爲他} \\
     & 單      & 兩       & 複 \\
  1. & atudam  & atudāva  & atudāma \\
  2. & atudas  & atudatam & atudata \\
  3. & atudat. & atudatām & atudan \\
  \multicolumn{4}{c}{爲自} \\
  1. & atude(a+i) & atudāvahi & atudāmahi \\
  2. & atudathās  & atudethām & atudata \\
  3. & atudata    & atudetām  & atudan
\end{tabular}
\end{center}

\numberParagraph
á 級の不規則なる語根。
\begin{enumerate}[label=(\alph*)]
\item 或る語根は結尾の子音の前に微韻を挿入する。

\indent muc (解く)muñcati.

\indent lip (塗る)limpati.

\indent sic (\ruby{灌}{そそ}ぐ)siñcati.

\indent kṛt (斷つ)kṛntati.

\indent vid (見出す)vindati.

\indent lup (碎く)lumpati.
\item prach (問ふ)はpṛcchati, iṣ (欲す)は icchati とな
る。(第一類の gam 等と比較)。
\end{enumerate}

\ex{第三}

\begin{longtable}{c*{2}{p{0.45\hsize}}}
 1. & vṛkṣasya cchāyāyāṃ munir asīdat. & 樹の蔭に於て聖者は坐したり。\\
 2. & Kālidāsaṃ kaviṃ sevāmahe. & カーリダーサなる詩人を我々は尊敬する。\\
 3. & kanyā Gaṅgāyās tīre 'krī\-ḍan. & 少女等は恒河の岸に於て遊戯した。\\
 4. & gajasya siṃhena saha yud\-dham abhavat. & 獅子と共に象の爭ひがあつた。\\
 5. & tṛṣṇā pathikam abādhata. & 渇が旅人を苦しめた。\\
 6. & putrasya śokād Daśaratho nṛpo jīvitaṃ paryatyajat. & 子を悲しみて十車王は命を捨てた。\\
 7. & śiṣyau gṛhasthasya bhāryāṃ bhikṣāṃ ayācetām. & 二人の弟子は長者の妻に施物を請へり。\\
 8. & Prayāge Gaṅgā Yamunayā saha saṃgacchate. & プラヤーガに於て恒河はヤムナーと出會ふ。\\
 9. & dāsyo 'nnam ānayan. & 婢女等は食物を持ち來つた。\\
10. & saṃkaṭe dhīro dhṛtiṃ na muñcati. & 危難に際し勇者は堅持を捨てない。\\
11. & lajjayā kanyā na pratyabhā\-ṣata. & 羞恥を以て少女は答へなかつた。\\
12. & kīrtiṃ labhante kavayaḥ. & 詩人等は名譽を得る。\\
13. & bubhukṣayā pīḍitaḥ śṛgālo vānān nagaram adhāvat. & 飢に苦しめられ\ruby{野干}{や|かん}は林から町へ走つた。\\
14. & ācāryasya gṛhe Śūdrakeṇa ka\-vinā kṛtāṃ Mṛcchakaṭikām apaṭhāva. & 師の家に於てシュードラカなる詩人によつて作られ
しムリッチュㇵカティカーを我々二人は讀めり。\\
15. & hastena śilām akṣipan naraḥ. & 人は手を以て石を投げた。\\
16. & siṃhaḥ Pāṇineḥ priyān prā\-ṇān aharat. & 獅子はパーニニの愛する生命を奪へり。\\
17. & lubdho naro na visṛjaty & 貪慾の人は貧窮(となる)\\
    & arthaṃ daridratāyāḥ śaṅkayā. & 心配のために富を施さない。\\
18. & bālo vāri pāṇinā kūpād ud\-dharati. & 小兒は手を以て井より水を汲む。\\
19. & lobhena buddhiś calati. & 貪慾によりて思慮は動搖する。\\
20. & makṣikā vraṇam icchanti, & 蠅は瘡傷を欲し,\\
    & dhanam icchanti pārthivāḥ, & 王者は富を欲し,\\
    & nīcāḥ kalaham icchanti, & 下賤なものは爭を欲し,\\
    & śāntim icchanti sādhavaḥ. & 善人は寂靜を欲す。
\end{longtable}

\subsubsection[第四類 ya 級 可能法]{第四類 ya 級}
\begin{center}可能法\end{center}

\numberParagraph \label{np:69}
第四類の動詞は語根に ya を附加して語基を作る。語勢は
語根にあり。語根 sidh (成就す)~ sidhya.

\numberParagraph
可能法の語尾は第一種變化現在語基の結尾の a と合して
次の如くなる。

爲他 eyam, es, et; eva, etam, etām; ema, eta, eyus.

爲自 eya, ethās, eta; evahi, eyāthām, eyātām; emahi,
edhvam, eran.

\numberParagraph
ya 級の變化。語根 div (博戯する)~ dīvya.

\begin{center}
\begin{tabular}{c*{3}{p{0.15\hsize}}}
  \multicolumn{4}{c}{\textbf{可能法}} \\
  \multicolumn{4}{c}{爲他} \\
     & 單       & 兩       & 複 \\
  1. & dīvyeyam & dīvyeva  & dīvyema \\
  2. & dīvyes   & dīvyetam & dīvyeta \\
  3. & dīvyet   & dīvyetām & dīvyeyus \\
  \multicolumn{4}{c}{爲自} \\
  1. & dīvyeya   & dīvyevahi   & dīvyemahi \\
  2. & dīvyethās & dīvyeyāthām & dīvyedhvam \\
  3. & dīvyeta   & dīvyeyātām  & dīvyeran
\end{tabular}
\end{center}

\numberParagraph
若干の不規則なる語根。
\begin{enumerate}[label=(\alph*)]
\item am に終る語根及び mad (醉ふ)はその a を延長す。dam
(馴らす) dāmyati, śam (靜める) śāmyati, kram (歩む)
krāmyati, bhram (彷徨す)には bhrāmyati と bhramyati
の兩形があり。又第一類の bhramati にも變化せらる。
\item jan (生る)は jāyate に作る。
\end{enumerate}

\ex{第四}
\begin{longtable}{c*{2}{p{0.45\hsize}}}
 1. & sevakaḥ prabhuṃ praṇamati. & 僕は主人に\ruby{敬禮}{けい|れい}する。\\
 2. & satyena jayed anṛtam. & 人は正義を以て虚僞に勝つべきである。\\
 3. & sādhavaḥ pratijñāyā na ca\-lanti kadācana. & 善人は何時も約束から動かない。\\
 4. & śatror api guṇān vaded doṣāṃś ca guror api. & 人は敵に對しても功績を,
 而して師に對しても過失を語るべきだ。\\
 5. & parasya duḥkhena sādhur nityaṃ duḥkhito bhavati. & 善人は他の苦によりて常に
 苦しめられてある。\\
 6. & pratyahaṃ pratyavekṣeta na\-raś caritam. & 日日人は行を檢察すべきだ。\\
 7. & asādhuḥ sādhur vā bhavati khalu jātyaiva puruṣaḥ. & 人は生れながらに惡人若く
 は善人である。\\
 8. & astamite sūrye vihagā vilī\-yante, paṅkajā nimīlanti, māla\-tyaś ca vikasanti. & 日沒する時鳥は隱れ,蓮花
は閉ぢ,素馨は咲く。\\
 9. & janakaḥ prītyā putram āśliṣyati. & 父は喜を以て子を抱擁せり。\\
10. & mitrair jñātibhiś ca parihīṇo daridro drutam anaśyat. & 友と親族に捨てられし貧人は速かに亡びた。\\
11. & vasantasya kāle 'layo bhrām\-yanti mukhena ca madhu pibanti. & 春の時に於て蜜蜂は飛び回
り口を以て蜜を吸ふ。\\
12. & sūtasya daṇḍena praṇuditā aśvā na śrāmyanti. & 御者の笞に勵まされて馬は
疲勞しない。\\
13. & grāmasyārthe kulaṃ tyajet. & 人は村邑のために家族を捨てるべきだ。\\
14. & udyamena hi sidhyanti kār\-yāṇi na manorathaiḥ. & 勤勉によりて事は成就す。
希望によりてに非ず。\\
& na hi suptasya siṃhasya pra\-viśanti mukhe mṛgāḥ. & 蓋し眠れる獅子の口に鹿
は入るものでないから。
\end{longtable}

\subsubsection[第十類 aya 級 命令法]{第十類 aya 級}
\begin{center}命令\end{center}

\numberParagraph \label{np:73}
第十類の構成は語根に aya を附加す。語勢は前の a に
在り。聲中にある i, u, ṛ は單子音の前には重韻化し,聲の終に
在る。i, u, ṛ 又は單子音の中間にある a は複重韻化し,多子音
の前にあると長母音の場合は變化しない。cur (盗む)~ coraya,
cint (考ふ)~ cintaya. この第十類は實際 催起動詞,名稱動詞と
同類である。只現在語基が大體に於て變化しないで用ひられるの
を異とする。

\numberParagraph
命令法の語尾。

爲他 āni, dhi, tu; āva, tam, tām; āma, ta, antu.

爲自 ai, sva, tām: āvahai, āthām, ātām; āmahai,
dhvam, antām.

\numberParagraph
二人稱の dhi は第一種變化に於ては通じて附加しない。
現在語基の結尾 a は antu, antām の前に消滅し。āthām, ātām
の ā と合して e に變ず。

\numberParagraph
aya 級。命令法の語尾。

\begin{center}
\begin{tabular}{c*{3}{p{0.15\hsize}}}
  \multicolumn{4}{c}{爲他} \\
     & 單       & 兩        & 複 \\
  1. & corayāṇi & corayāva  & corayāma \\
  2. & coraya   & corayatam & corayata \\
  3. & corayatu & corayatām & corayantu \\
  \multicolumn{4}{c}{爲自} \\
  1. & corayai   & corayāvahai & corayāmahai \\
  2. & corayasva & corayethām  & corayadhvam \\
  3. & corayatām & corayetām   & corayantām
\end{tabular}
\end{center}

\ex{第五}
\begin{longtable}{c*{2}{p{0.45\hsize}}}
 1. & adhunā muñca śayyām. & 今臥床を離れよ。\\
 2. & vaidyo rogārtasyauṣadhaṃ yacchatu. & 醫師をして病に苦しむもの
 に藥を與へしめよ。\\
 3. & adya pitur gṛham āgaccha\-tam. & 今汝等兩人は父の家に歸れかし。\\
 4. & bhadre, maivaṃ vada. & 善き婦人よ是の如く語る勿れ。\\
 5. & Pāṇiner vyākaraṇasya karur buddhiṃ śaṃsāmaḥ. & 我々は文法作家パーニニの
 聰明を讚む。\\
 6. & bhāryā patyuḥ snihyatu. & 妻をして夫を愛せしめよ。\\
 7. & tyaja nīcānāṃ saṃsargam. & 下劣なるものとの交際を避けよ。\\
 8. & kṣamāṃ bhaja. & 堪忍せよ。\\
 9. & dharme ratir bhavatu. & 法に於て快樂があれかし。\\
10. & āpṛcchasva priyaṃ sakhāyam. & 愛する共に吿別せよ。\\
11. & durjaneṣv api mā pāpaṃ cin\-tayasva kadācana. & 何時にても惡人に對してす
らも惡を考ふること勿れ。\\
12. & bhūpatayaḥ sarvadā prajā dharmeṇa rakṣantu. & 常に王をして人民を法を以て守らしめよ。\\
13. & mitrasya dhanaṃ prayaccha. & 共に財物を與へよ。\\
14. & kumbhakāro daṇḍena ghaṭam akhaṇḍayat. & 甕作りは杖にて甕を粉碎せり。\\
15. & bho sakhe kṣaṇam atra tiṣṭha. & おゝ友よ暫しそこに立て。\\
16. & śokasya hetuṃ vada. & 悲哀の原因を語れ。\\
17. & śiṣyā guroḥ pādau pūjayanti. & 弟子等は師の兩足を尊敬せり。\\
18. & stenā rātrau gṛhaṃ pravi\-śanti janānāṃ ca dhanaṃ cora\-yanti. & 盗人は夜に於て家に入り,
而して人々の富を盗む。\\
19. & kauliko nṛpasya duhitaraṃ paryaṇayat. & 織師は王女に婚せり。\\
20. & muktim icchasi ced viṣam iva viṣayāṃs tyaja. & 若しも汝が解脱を欲するな
らば毒の如くに欲境を捨てよ。
\end{longtable}

\numberParagraph \addcontentsline{toc}{subsubsection}{現在組織總覽}
以上第一種變化の動詞に現在組織の語尾を附加せるもの
を擧げた。勿論この四種の法は各類に應用せらるべきもの。故に
今試みに次に bhū (ある)に就きて全部の語尾表を擧げる。


\begin{center}
\begin{tabular}{c*{3}{p{0.15\hsize}}}
  \multicolumn{4}{c}{\textbf{現實法}} \\
  \multicolumn{4}{c}{爲他} \\
     & 單       & 兩       & 複 \\
  1. & bhavāmi  & bhavāvas  & bhavāmas \\
  2. & bhavasi  & bhavathas & bhavatha \\
  3. & bhavatai & bhavantas & bhavanti \\
  \multicolumn{4}{c}{爲自} \\
  1. & bhave   & bhavāvahe & bhavāmahe \\
  2. & bhavase & bhavethe  & bhavadhve \\
  3. & bhavate & bhavete   & bhavante
\end{tabular}
\end{center}
\begin{center}
\begin{tabular}{c*{3}{p{0.15\hsize}}}
  \multicolumn{4}{c}{\textbf{第一過去}} \\
  \multicolumn{4}{c}{爲他} \\
     & 單       & 兩       & 複 \\
  1. & abhavam & abhavāva  & abavāma \\
  2. & abhavas & abhavatam & abhavata \\
  3. & abhavat & abhavatām & abhavan \\
  \multicolumn{4}{c}{爲自} \\
  1. & abhave     & abhavāvahi & abhavāmahi \\
  2. & abhavathās & abhavethām & abhadhvam \\
  3. & abhavata   & abhavetām  & abhavanta
\end{tabular}
\end{center}
\begin{center}
\begin{tabular}{c*{3}{p{0.15\hsize}}}
  \multicolumn{4}{c}{\textbf{可能法}} \\
  \multicolumn{4}{c}{爲他} \\
     & 單       & 兩       & 複 \\
  1. & bhaveyam & bhaveva  & bhavema \\
  2. & bhaves   & bhavetam & bhaveta \\
  3. & bhavet   & bhavetām & bhaveyus \\
  \multicolumn{4}{c}{爲自} \\
  1. & bhaveya   & bhavevahi   & bhavemahi \\
  2. & bhavethās & bhaveyāthām & bhavedhvam \\
  3. & bhaveta   & bhaveyātām  & bhaveran
\end{tabular}
\end{center}
\begin{center}
\begin{tabular}{c*{3}{p{0.15\hsize}}}
  \multicolumn{4}{c}{\textbf{命令法}} \\
  \multicolumn{4}{c}{爲他} \\
     & 單      & 兩       & 複 \\
  1. & bhavāni & bhavāva  & bhavāma \\
  2. & bhava   & bhavatam & bhavata \\
  3. & bhavatu & bhavatām & bhavantu \\
  \multicolumn{4}{c}{爲自} \\
  1. & bhavai   & bhavāvahai & bhavāmahai \\
  2. & bhavasva & bhavethām  & bhavadhvam \\
  3. & bhavatām & bhavetām   & bhavantām
\end{tabular}
\end{center}

(活用法は次 \ref{np:136} 條に連續す)。

%%% Local Variables:
%%% mode: latex
%%% TeX-master: "IntroductionToSanskrit"
%%% End:
