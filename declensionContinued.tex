\section{格例法(續き)}
\begin{center}\textbf{子音語基}\end{center}

\numberParagraph \label{np:78}
子音に終る語基は次の如きものである。

\begin{itemize}[label=\hspace{2\zw}]
\item 語根語基
\item as, is, us に終るもの
\item in に終るもの
\item ac に終るもの
\item an, man, van に終るもの
\item at に終るもの
\item mat, vat に終るもの
\item vas に終る過去分詞
\item yas に終る比較級
\end{itemize}

\numberParagraph
語尾は各語基を通じて規則正しく次のものが用ひられる。
この點母音語基の如く不規則ではない。男性,女性は次の如くで
ある。

\begin{center}
\begin{tabular}{c*{3}{p{0.24\hsize}}}
     & 單                  & 兩                     & 複 \\
  主 & s                   & \rdelim\}{2}{*}[au]    & \multirow{2}{*}{as} \\
  業 & am                  &                        & \\
  具 & ā                   & \rdelim\}{3}{*}[bhyām] & bhis \\
  爲 & e                   &                        & \rdelim\}{2}{*}[bhyas] \\
  從 & \rdelim\}{2}{*}[as] &                        & \\
  屬 &                     & \rdelim\}{2}{*}[os]    & ām \\
  於 & i                   &                        & su
\end{tabular}
\end{center}

呼格は兩複に於て必ず主格に同じ。單に於ては大抵語基の形で
あるが,或は主格の形であることもある。中性は主業格が單に於
て語尾を有せず。兩に於て ī, 複に於て i となる外,男女性と同じ
である。

\begin{center}\textbf{(1) 語根語基}\end{center}\addcontentsline{toc}{subsection}{1) 語根語基}

\numberParagraph
語根語基及それに準ずるもの。男性,語基 marut (風),
女性 vāc (言語)。

\begin{center}
\begin{tabular}{c*{4}{p{0.15\hsize}}}
     & \multicolumn{2}{c}{單}                           & \multicolumn{2}{c}{複} \\
  主 & \rdelim\}{2}{*}[marut]   & \multirow{2}{*}{vāk (\ref{np:17}條)} & \rdelim\}{3}{*}[marutas]    & \multirow{3}{*}{vācas} \\
  呼 &                          &                                        &                             & \\
  業 & marutam                  & vācam                                  &                             & \\
  具 & marutā                   & vācā                                   & marudbhis (\ref{np:18}條) & vāgbhis \\
  爲 & marute                   & vāce                                   & marudbhyas                  & vāgbhyas \\
  從 & \rdelim\}{2}{*}[marutas] & \multirow{2}{*}{vācas}                 & \rdelim\}{2}{*}[marutām]    & \multirow{2}{*}{vācām}\\
  屬 &                          &                                        &                             & \\
  於 & maruti                   & vāci                                   & marutsu                     & vākṣu \\
     &                          &                                        &                             & (\ref{np:17} 條,\ref{np:39}條)
\end{tabular}
\end{center}

\begin{center}
\begin{tabular}{c*{2}{p{0.24\hsize}}}
     & \multicolumn{2}{c}{兩} \\
  主 & \rdelim\}{3}{*}[marutau]    & \multirow{3}{*}{vācau} \\
  呼 &                             & \\
  業 &                             & \\
  具 & \rdelim\}{3}{*}[marudbhyām] & \multirow{3}{*}{vāgbhyām} \\
  爲 &                             & \\
  從 &                             & \\
  屬 & \rdelim\}{2}{*}[marutos]    & \multirow{2}{*}{vācos} \\
  於 &                             &
\end{tabular}
\end{center}
中性語の主業複は最後の音の次前にその音に相當する鼻音を挿
む。hṛd (心)は hṛndi, asṛj (血)は asṛñji, jagat (世界), jaganti.
これは語基の强弱あるものと見做しても不可でない。

\numberParagraph
\textbf{この種類に屬する若干の語基。}

suhṛd 男(友)。

主 suhṛt, 業 suhṛdam, 具複 suhṛdbhis

dharmabudh 男,女,中(法に明かなる)。

主 dharmabhut, 業 dharmabudham, 具複 dharmabhudbhi (\ref{np:35}條)

vaṇij 男(商人)。

主 vaṇik, 業 vaṇijam, 具複 vaṇigbhis.

diś 女(方)。

主 dik, 業 diśam, 具複 digbhis.

tviṣ 女(光)。

主 tviṭ (\ref{np:17}條),業 tviṣam, 具複 tviḍbhis, 於 tviṭsu 又は
tviṭtsu.

ir, ur に終る語基は單の主呼と並に子音が次に來る場合 i, u を
延長す。

語基 gir 女(歌),主 gīr, 業 giram, 具複 gīrbhis, 於 gīrṣu.

pur 女(城)。主 pūr, 業 puram, 具複 pūrbhis, 於 pūrṣu.

ap 女(水)は常に複數にして,主 āpas, 業 apas, 具 abdhis,
爲從 adbhyas, 屬 apām, 於 apsu である。

puṃs 男(人)は全く不規則で,單 pumān puman, pumāṃsam,
puṃsā puṃse, puṃsas, puṃsi, 兩 pumāṃsau, pumbhyās,
puṃsos, 複 pumāṃsas, puṃsas, pumbhis, pumbhyas, puṃ\-%
sām, puṃsu.

\begin{center}\textbf{(2) as, is, us に終る語基}\end{center}

\numberParagraph
男女性にあつては主單に as の a を延長する。中性では
主呼業の複に a, i, u を延長し,且つ隨韻が挿入せられる。

語基 sumanas 男,女,中(善意ある),cakṣus 中(眼)。

\begin{center}
\begin{tabular}{c*{3}{p{0.2\hsize}}}
  \multicolumn{4}{c}{單} \\
     & 男女                       & \multicolumn{2}{c}{中} \\
  主 & sumanās                    & \rdelim\}{3}{*}[sumanas]   & \multirow{3}{*}{cakṣus} \\
  呼 & sumanas                    &                            & \\
  業 & sumanasam                  &                            & \\
  具 & sumanasā                   & sumanasā                   & cakṣuṣā \\
  爲 & sumanase                   & sumanase                   & cakṣuṣe \\
  從 & \rdelim\}{2}{*}[sumanasas] & \multirow{2}{*}{sumanasas} & \multirow{2}{*}{cakṣuṣas} \\
  屬 &                            &                            & \\
  於 & sumanasi                   & sumanasi                   & cakṣuṣi
\end{tabular}
\end{center}
\begin{center}
\begin{tabular}{c*{3}{p{0.2\hsize}}}
  \multicolumn{4}{c}{兩} \\
     & 男女                                                                                            & \multicolumn{2}{c}{中} \\
  主 & \rdelim\}{3}{*}[sumanasau]                                                                      & \multirow{3}{*}{sumanasī}    & \multirow{3}{*}{cakṣuṣī} \\
  呼 &                                                                                                 &                              & \\
  業 &                                                                                                 &                              & \\
  具 & \rdelim\}{3}{*}[\parbox{8cm-\tabcolsep-\widthof{$\Bigg\}$}}{sumananobhyām \\(\ref{np:24}條)}] & \multirow{3}{*}{sumanobhyām} & \multirow{3}{*}{\parbox{8cm-\tabcolsep-\widthof{$\Bigg\}$}}{cakṣurbhyām \\(\ref{np:23}條)}} \\
  爲 &                                                                                                 &                              & \\
  從 &                                                                                                 &                              & \\
  屬 & \rdelim\}{2}{*}[sumananasos]                                                                    & \multirow{2}{*}{sumanasos}   & \multirow{2}{*}{cakṣuṣos} \\
  於 &                                                                                                 &                              &
\end{tabular}
\end{center}
\begin{center}
\begin{tabular}{c*{3}{p{0.2\hsize}}}
  \multicolumn{4}{c}{複} \\
     & 男女                         & \multicolumn{2}{c}{中} \\
  主 & \rdelim\}{3}{*}[sumanasas]   & \multirow{3}{*}{sumanāṃsi}   & \multirow{3}{*}{cakṣūṃṣi} \\
  呼 &                              &                              & \\
  業 &                              &                              & \\
  具 & sumanobhis                   & sumanobhis                   & cakṣurbhis \\
  爲 & \rdelim\}{2}{*}[sumanobhyas] & \multirow{2}{*}{sumanobhyas} & \multirow{2}{*}{cakṣurbhyas} \\
  從 &                              &                              & \\
  屬 & sumanasām                    & sumanasām                    & cakṣuṣām \\
  於 & sumanaḥsu (\ref{np:22}條)  & sumanaḥsu                    & cakṣuḥṣu (\ref{np:39}條)
\end{tabular}
\end{center}

\begin{center}\textbf{(3) in に終る語基}\end{center}

\numberParagraph
in に終る語基は所有を表はすもので,多くは形容詞であり,
男性中性に變化する。子音で始まる語尾の前に n は消失する。
又同樣に單主,及び中單主業も然り。中單呼にあつては消失せざ
ることもあり,この i は男單主,並に中複主業にあつては延長せ
らる。

語基 balin (力ある)。

\begin{center}
\begin{tabular}{c*{6}{p{0.12\hsize}}}
     & \multicolumn{2}{c}{單}     & \multicolumn{2}{c}{兩}     & \multicolumn{2}{c}{複} \\
     & 男      & 中                                 & 男                       & 中                      & 男                       & 中 \\
  主 & balī    & bali                               & \rdelim\}{3}{*}[balinau] & \multirow{3}{*}{balinī} & \multirow{3}{*}{balinas} & \multirow{3}{*}{balīni} \\
  呼 & balin   & balin, bali                        &                                                    & \\
  業 & balinam & bali                               &                                                    & \\
     & \multicolumn{2}{c}{\upbracefill}             &                          &                         & \multicolumn{2}{c}{\upbracefill} \\
  具 & \multicolumn{2}{c}{balinā}                   & \multicolumn{2}{l}{\rdelim\}{3}{*}[balibhyām]}     & \multicolumn{2}{c}{balibhis} \\
  爲 & \multicolumn{2}{c}{baline}                   &                          &                         & \multicolumn{2}{l}{\rdelim\}{2}{*}[balibhyas]} \\
  從 & \multicolumn{2}{l}{\rdelim\}{2}{*}[balinas]} &                          &                         & \\
  屬 &         &                                    & \multicolumn{2}{l}{\rdelim\}{2}{*}[balinos]}       & \multicolumn{2}{c}{balinām} \\
  於 & \multicolumn{2}{c}{balini}                   &                          &                         & \multicolumn{2}{c}{baliṣu}
\end{tabular}
\end{center}

この語基の女性形容詞は語基に ī を附加して作り nadī (\ref{np:52}條)
に準じて變化せらる。

\ex{第六}
\begin{longtable}{c*{2}{p{0.45\hsize}}}
 1. & sarvaḥ padasthasya suhṛd bandhur āpadi durlabhaḥ. & 顯要の位置にあるものは一
切が友であり不幸に於ては親族が得難い。\\
 2. & yathaā cittaṃ tathā vāco yathā vācas tathā kriyāḥ. & 心の如く此の如く語あり,
語の如く此の如く行爲あり。\\
 3. & durgrāhyḥ pāṇinā vāyur duḥsparśaḥ pāṇinā śikhī. & 風は手を以て捉へ難く,火は手を以て觸れ難い。\\
 4. & kṣamā rūpaṃ tapasvinaḥ. & 忍耐は苦行者の美貌である。\\
 5. & niyato dehināṃ mṛtyur ani\-tyaṃ khalu jīvitam. & 人にとつて死は定まつてゐるが命は不定である。\\
 6. & namanti phalino vṛkṣāḥ na\-manti guṇino janāḥ. & 果實ある樹は曲り德ある人も曲る。\\
 7. & śrutiḥ smṛtiś ca dvijānāṃ cakṣuṣī. & 吠陀と法典(天啓と傳説)とは婆羅門の兩眼である。\\
 8. & vaidyo na prabhur āyuṣaḥ. & 醫は壽命の主でない。\\
 9. & snānāya sarasas tīraṃ sa gacchati. & 沐浴のために湖の岸へ彼は行く。\\
10. & āyur eva paraṃ nidhānam. &  壽命こそ最高の寶なれ。\\
11. & saṃpadas tasya yasya saṃ\-tuṣṭaṃ mānasam. & 滿足せる心ある人には幸福がある。\\
12. & āpadas tasya yasya vittaṃ na vidyate. & 財物なき人には不幸がある。\\
13. & ākiṃcanyaṃ nidhānaṃ vi\-duṣām. & 無一物の狀態は學者の寶である。\\
14. & bhāryāyāḥ sundaraḥ snigdho veśyāyāḥ sundaro dhanī, Śrī\-devyāḥ sundaraḥ śūro Bhāra\-tyāḥ sundaraḥ sundhī. & 妻の好むは情ある人,娼婦の好むは富人,吉祥天の
好むは勇士,辯才天の好むは賢人である。\endnote{底本では「好むは賢人」ではなく「好むはは賢人」。}
\end{longtable}

\numberParagraph
語基には强弱の二語基の場合或は强弱中の三語基の場合
がある。强弱と云ふのは語勢等の關係で音量の多いのが强と名け
られ,少いのが弱と名けられる。中はその中間のものである。而
してこの强弱中の語基は使用せらるゝ場所も一定してゐて決して
混亂はない。卽ち男性,女性にして二語基の場合は主呼業の單兩
及び主呼の複に强語基その他には弱語基を用ふ。三語基の場合は
前の弱語基を用ふる場所の中,大體子音にて始まる語尾の前には
中語基母音にて始まる語尾の前には弱語基が用ひられる。中性で
は複の主業が强語基となる。三語基の場合には單の主業呼が中語
基となる。兩の主業は常に弱語基である。其の餘は男女性に同
じ。

\begin{enumerate}[label=(\alph*)]
\item 二語基の例:强 tudant, 弱 tudat (打ちつゝ)。
\item 三語基の例:强 vidvāṃs, 中 vidvat, 弱 vidus (賢者)。
\end{enumerate}

\begin{center}\textbf{(4) ac に終る語基}\end{center}

\numberParagraph
ac に終る語基は形容詞であつて一部分二語基,一部分は
三語基を有す。卽ち二語基 prāc (東方の)强基 prāñc, 弱基
prāc であり,三語基は pratyac (西方の)强基 pratyañc, 中基
pratyac, 弱基 pratīc.

\begin{center}
\begin{tabular}{c*{4}{p{0.15\hsize}}}
     & \multicolumn{4}{c}{單} \\
     & \cellAlign{c}{男}                       & \cellAlign{c}{中}     & \cellAlign{c}{男}           & \cellAlign{c}{中} \\
  主 & \rdelim\}{2}{*}[prāṅ (\ref{np:17}條)] & \multirow{2}{*}{prāk} & \multirow{2}{*}{pratyaṅ}    & \multirow{2}{*}{pratyak} \\
  呼 &                                         &                       &                             & \\
  業 & prāñcam                                 & prāk                  & pratyañcam                  & pratyak \\
     & \multicolumn{2}{c}{\upbracefill}                                & \multicolumn{2}{c}{\upbracefill} \\
  具 & \multicolumn{2}{c}{prācā}                                       & \multicolumn{2}{c}{pratīcā} \\
  爲 & \multicolumn{2}{c}{prāce}                                       & \multicolumn{2}{c}{pratīce} \\
  從 & \multicolumn{2}{l}{\rdelim\}{2}{*}[prācas]}                     & \multicolumn{2}{c}{\multirow{2}{*}{pratīcas}} \\
  屬 &                                                                 & \\
  於 & \multicolumn{2}{c}{prāci}                                       & \multicolumn{2}{c}{pratīci}
\end{tabular}
\end{center}

\begin{center}
\begin{tabular}{c*{4}{p{0.15\hsize}}}
     & \multicolumn{4}{c}{兩} \\
     & \cellAlign{c}{男}        & \cellAlign{c}{中}      & \cellAlign{c}{男}           & \cellAlign{c}{中} \\
  主 & \rdelim\}{3}{*}[prāñcau] & \multirow{3}{*}{prācī} & \multirow{3}{*}{pratyañcau} & \multirow{3}{*}{pratīcī} \\
  呼 &                          &                        &                             & \\
  業 &                          &                        &                             & \\
     & \multicolumn{2}{c}{\upbracefill}                  & \multicolumn{2}{c}{\upbracefill} \\
  具 & \multicolumn{2}{l}{\rdelim\}{3}{*}[prāgbhyām]}    & \multicolumn{2}{c}{\multirow{3}{*}{pratyagbhyām}} \\
  爲 &                                                   & \\
  從 &                                                   & \\
  屬 & \multicolumn{2}{l}{\rdelim\}{2}{*}{prācos}}       & \multicolumn{2}{c}{\multirow{2}{*}{pratīcos}} \\
  於 &                                                   &
\end{tabular}
\end{center}

\begin{center}
\begin{tabular}{c*{4}{p{0.15\hsize}}}
     & \multicolumn{4}{c}{複} \\
     & \cellAlign{c}{男}        & \cellAlign{c}{中}       & \cellAlign{c}{男}           & \cellAlign{c}{中} \\
  主 & \rdelim\}{2}{*}[prāñcas] & \multirow{2}{*}{prāñci} & \multirow{2}{*}{pratyañcas} & \multirow{2}{*}{pratyañci} \\
  呼 &                          &                         &                             & \\
  業 & prācas                   & prāñci                  & pratīcas                    & pratyañci \\
     & \multicolumn{2}{c}{\upbracefill}                   & \multicolumn{2}{c}{\upbracefill} \\
  具 & \multicolumn{2}{c}{prāgbhis}                       & \multicolumn{2}{c}{pratyagbhis} \\
  爲 & \multicolumn{2}{l}{\rdelim\}{2}{*}[prāgbhyas]}     & \multicolumn{2}{c}{pratyagbhyas} \\
  從 &                                                    & \\
  屬 & \multicolumn{2}{c}{prācām}                         & \multicolumn{2}{c}{pratīcām} \\
  於 & \multicolumn{2}{c}{prākṣu}                         & \multicolumn{2}{c}{pratyakṣu}
\end{tabular}
\end{center}

女性語基は弱語基に ī を附加して作られる。卽ち prācī,
pratīcī であつて nadī (\ref{np:52}條)に準じて變化せられる。

\begin{center}\textbf{(5) an, man, van に終る語基}\end{center}

\numberParagraph
これは皆三語基を有す。强語基では後接字の a は延長せ
られ,中語基では n が省かれ,弱語基では a が除去せらる。
man, van の語基では弱語基にても m 又は v の前に子音が來る
場合 a は保存せらる。單主,男は ā, 中は a にて終る。

語基 rājan 男(王),强語基 rājān, 中語基 rāja, 弱語基 rājñ.
adhvan 男(路),强語基 adhvān, 中語基 adhva, 弱語基 adhvan.
語基 karman 中(業)同樣。

\begin{center}
\begin{tabular}{c*{3}{p{0.15\hsize}}}
     & \multicolumn{3}{c}{單} \\
  主 & rājā                    & adhvā                     & karma \\
  呼 & rājan                   & adhvan                    & karman, karma \\
  業 & rājānam                 & adhvānam                  & karma \\
  具 & rājñā                   & adhvanā                   & karmaṇā \\
  爲 & rājñe                   & adhvane                   & karmaṇe \\
  從 & \rdelim\}{2}{*}[rājñas] & \multirow{2}{*}{adhvanas} & \multirow{2}{*}{karmaṇas} \\
  屬 &                         &                           & \\
  於 & rājñi                   & adhvani                   & karmaṇi
\end{tabular}
\end{center}

\begin{center}
\begin{tabular}{c*{3}{p{0.15\hsize}}}
     & \multicolumn{3}{c}{兩} \\
  主 & \rdelim\}{3}{*}[rājānau]   & \multirow{3}{*}{adhvānau}   & \multirow{3}{*}{karmaṇī} \\
  呼 &                            &                             & \\
  業 &                            &                             & \\
  具 & \rdelim\}{3}{*}[rājabhyām] & \multirow{3}{*}{adhvabhyām} & karmabhyām \\
  爲 &                            &                             & \\
  從 &                            &                             & \\
  屬 & \rdelim\}{2}{*}[rājños]    & \multirow{2}{*}{adhvanos}   & \multirow{2}{*}{karmaṇos} \\
  於 &                            &                             &
\end{tabular}
\end{center}

\begin{center}
\begin{tabular}{c*{3}{p{0.15\hsize}}}
     & \multicolumn{3}{c}{複} \\
  主 & \rdelim\}{2}{*}[rājānas]   & \multirow{2}{*}{adhvānas}   & \rdelim\}{3}{*}[karmāṇi] \\
  呼 &                            &                             & \\
  業 & rājñas                     & adhvanas                    & \\
  具 & rājabhis                   & adhvabhis                   & karmabhis \\
  爲 & \rdelim\}{2}{*}[rājabhyas] & \multirow{2}{*}{adhvabhyas} & \multirow{2}{*}{karmabhyas} \\
  從 &                            &                             & \\
  屬 & rājñām                     & adhvanām                    & karmaṇām \\
  於 & rājasu                     & adhvasu                     & karmasu
\end{tabular}
\end{center}

\nt{三性を通じ單於,及び中性兩の主業には a を保存するを得。卽
ち rājñi と並んで rājani, nāmnī 中(名)と並んで nāmanī.}

\numberParagraph \textbf{若干の不規則なる語基。}
\begin{enumerate}[label=(\alph*)]
\item śvan 男(犬)及び yuvan (若き)は śun yūn なる弱
  語基を有す。
  \begin{description}[font=\normalfont]
  \item[單] śvā, śvānam, śunā, śune 等。
  \item[兩] śvānau, śvabhyām, śunos.
  \item[複] śvānas, śunas, śvabhis, śvabhyas, śunām, śvasu.
  \end{description}
\item panthan 男(路)は强語基 panthān, 中語基 pathi,
  弱語基 path.
  \begin{description}[font=\normalfont]
  \item[單] panthās, panthānam, pathā, pathe, pathas, pathi.
  \item[兩] panthānau, pathibhyām, pathos.
  \item[複] panthānas, pathas, pathibhis, pathibhyas, pathām, pathiṣu.
  \end{description}
\item athan 中(日)は中語基 ahar 若くは ahas.
  \begin{description}[font=\normalfont]
  \item[單] 主呼業 ahar, 具 ahnā 等。
  \item[兩] ahni, ahobhyām, ahnos.
  \item[複] ahāni, 具 ahobhis 等。
  \end{description}
\item 中性 akṣan (眼),asthan (骨)は弱語基の格のみを有
  す。akṣṇā, akṣṇe, akṣṇas, akṣṇi 等,其の餘の格は akṣi, asthi
  の如く i 語基に準じて變化す。\ref{np:49}條\ref{item:49c}參照。
\end{enumerate}

\begin{center}\textbf{(6) at に終る語基}\end{center}

\numberParagraph
これらの語基は殆んどすべて現在又は未來分詞である。
而して强語基は ant 弱語基は at である。

語基 tudat (打ちつゝ),强 tudant, 弱 tudat.
\begin{center}
\begin{tabular}{c*{4}{p{0.15\hsize}}}
     & \multicolumn{2}{c}{單}                            & \multicolumn{2}{c}{複} \\
     & \cellAlign{c}{男}        & \cellAlign{c}{中}      & \cellAlign{c}{男}             & \cellAlign{c}{中} \\
  主 & \rdelim\}{2}{*}[tudan]   & \rdelim\}{3}{*}[tudat] & \multirow{2}{*}{tudantas}     & \multirow{3}{*}{tudanti} \\
  呼 &                          &                        &                               & \\
  業 & tudantam                 &                        & tudatas                       & \\
     & \multicolumn{2}{c}{\upbracefill}                  & \multicolumn{2}{c}{\upbracefill} \\
  具 & \multicolumn{2}{c}{tudatā}                        & \multicolumn{2}{c}{tudadbhis} \\
  爲 & \multicolumn{2}{c}{tudate}                        & \multicolumn{2}{c}{tudadbhyas} \\
  從 & \multicolumn{2}{c}{\rdelim\}{2}{*}[tudatas]}      & \multicolumn{2}{c}{\multirow{2}{*}{tudatām}} \\
  屬 &                                                   & \\
  於 & \multicolumn{2}{c}{tudati}                        & \multicolumn{2}{c}{tudatsu}
\end{tabular}
\end{center}

\begin{center}
\begin{tabular}{c*{2}{p{0.24\hsize}}}
     & \multicolumn{2}{c}{兩} \\
     & \cellAlign{c}{男}           & \cellAlign{c}{中} \\
  主 & \rdelim\}{3}{*}[tudantau]   & \multirow{3}{*}{tudantī, tudatī} \\
  呼 &                             & \\
  業 &                             & \\
     & \multicolumn{2}{c}{\upbracefill} \\
  具 & \multicolumn{2}{c}{\rdelim\}{3}{*}[tudadbhyām]} \\
  爲 &                             & \\
  從 &                             & \\
  屬 & \multicolumn{2}{c}{\rdelim\}{2}{*}[tudatos]} \\
  於 &                             &
\end{tabular}
\end{center}

\numberParagraph \label{np:89}
これらの語基の女性は\ref{np:90}條に準じ弱基又は强基に ī を附
加して作る。例:tudatī 又は tudantī.

\numberParagraph \label{np:90}
中性の兩主業に於て並に女性の構成に於て强基弱基の孰
れを取るかに關しては次の規定がある。
\begin{enumerate}[label=(\alph*)]
\item a 級 ya 級 aya 級並びに派生動詞に關しては强形(ant)
  が作られねばならぬ。卽ち bhavat, 女性 bhavantī.
\item á 級,語根級の ā に終る語根例へば yā, 並に爲他未
  來分詞に關しては强弱兩形が作られる。卽ち tudat, 女性
  tudatī 又は tudantī.
\item 餘の級の語根からは必ず弱形(at)が作られねばなら
  ぬ。卽ち kurvat, 女性 kurvatī.
\end{enumerate}

\numberParagraph
重複語根(\ref{np:145}條)は弱語基で總ての格を構成する。卽ち
datat (與へつゝ)男主 dadat, 業 dadatām 等。只中性複の主
業には兩形並び行はれる。

\numberParagraph
mahat (大なる)は强基 ānt である。
\begin{description}[font=\normalfont]
\item[男,單] mahān, mahāntam, mahatā 等。
\item[兩] mahāntau.
\item[複] mahāntas, mahatas, mahadbhis 等。
\item[中] mahat, mahatī, mahānti.
\end{description}

\begin{center}\textbf{(7) mat 及び vat に終る語基}\end{center}

\numberParagraph
mat, vat に終る所有を表はす形容詞は at に終る分詞と
同樣に變化する。但し主單男は mān 及び vān となる。卽ち
balavat (强き),主 balavān.

\begin{center}\textbf{(8) vas に終る過去能動分詞}\end{center}

\numberParagraph \label{np:94}
過去能動分詞(\ref{np:177}條)は三語基を有す。强基は vāṃs,
中基 vat, 弱基は us である。主單は vān, 呼單は van.

語基 vidvas (知れる),强形 vidvāṃs, 中形 vidvat, 弱形
vidus.
\begin{center}
\begin{tabular}{c*{4}{p{0.15\hsize}}}
     & \multicolumn{2}{c}{單}                       & \multicolumn{2}{c}{複} \\
     & \cellAlign{c}{男} & \cellAlign{c}{中}        & \cellAlign{c}{男}          & \cellAlign{c}{中} \\
  主 & vidvān            & \rdelim\}{3}{*}[vidvat]  & \rdelim\}{2}{*}[vidvāṃsas] & \rdelim\}{3}{*}[vidvāṃsi] \\
  呼 & vidvan            &                          &                            & \\
  業 & vidvāṃsam         &                          & viduṣas                    & \\
     & \multicolumn{2}{c}{\upbracefill}             & \multicolumn{2}{c}{\upbracefill} \\
  具 & \multicolumn{2}{c}{viduṣā}                   & \multicolumn{2}{c}{vidvadbhis} \\
  爲 & \multicolumn{2}{c}{viduṣe}                   & \multicolumn{2}{c}{vidvadbhyas} \\
  從 & \multicolumn{2}{c}{\rdelim\}{2}{*}[viduṣas]} & \multicolumn{2}{c}{\multirow{2}{*}{viduṣām}} \\
  屬 &                                              & \\
  於 & \multicolumn{2}{c}{viduṣi}                   & \multicolumn{2}{c}{vidvatsu}
\end{tabular}
\end{center}

\begin{center}
\begin{tabular}{c*{2}{p{0.24\hsize}}}
     & \multicolumn{2}{c}{兩} \\
     & \cellAlign{c}{男}          & \cellAlign{c}{中} \\
  主 & \rdelim\}{3}{*}[vidvāṃsau] & \multirow{3}{*}{tudantī, tudatī} \\
  呼 &                            & \\
  業 &                            & \\
     & \multicolumn{2}{c}{\upbracefill} \\
  具 & \multicolumn{2}{c}{\rdelim\}{3}{*}[vidvadbhyām]} \\
  爲 &                            & \\
  從 &                            & \\
  屬 & \multicolumn{2}{c}{\rdelim\}{2}{*}[viduṣos]} \\
  於 &                            &
\end{tabular}
\end{center}

女性語基は弱基に ī を附加して作る viduṣī.

\begin{center}\textbf{(9) yas に終る比較級形容詞}\end{center}

\numberParagraph
yas に終る比較級(\ref{np:98}條)は二語基を有す卽ち强形 yāṃs
及び弱形 yas.

語基 śreyas (よりよき),强形 śreyāṃs. 弱形 śreyas.
\begin{center}
\begin{tabular}{c*{4}{p{0.15\hsize}}}
     & \multicolumn{2}{c}{單}                        & \multicolumn{2}{c}{複} \\
     & \cellAlign{c}{男} & \cellAlign{c}{中}         & \cellAlign{c}{男}          & \cellAlign{c}{中} \\
  主 & śreyān            & \rdelim\}{3}{*}[śreyas]   & \rdelim\}{2}{*}[śreyāṃsas] & \rdelim\}{3}{*}[śreyāṃsi] \\
  呼 & śreyan            &                           &                            & \\
  業 & śreyāṃsam         &                           & śreyasas                   & \\
     & \multicolumn{2}{c}{\upbracefill}              & \multicolumn{2}{c}{\upbracefill} \\
  具 & \multicolumn{2}{c}{śreyasā}                   & \multicolumn{2}{c}{śreyobhis} \\
  爲 & \multicolumn{2}{c}{śreyase}                   & \multicolumn{2}{c}{\rdelim\}{2}{*}[śreyobhyaas]} \\
  從 & \multicolumn{2}{c}{\rdelim\}{2}{*}[śreyasas]} & \\
  屬 &                   &                           & \multicolumn{2}{c}{śreyasām} \\
  於 & \multicolumn{2}{c}{śreyasi}                   & \multicolumn{2}{c}{śreyaḥsu}
\end{tabular}
\end{center}

\begin{center}
\begin{tabular}{c*{2}{p{0.24\hsize}}}
     & \multicolumn{2}{c}{兩} \\
     & \cellAlign{c}{男}          & \cellAlign{c}{中} \\
  主 & \rdelim\}{3}{*}[śreyāṃsau] & \multirow{3}{*}{śreyasī} \\
  呼 &                            & \\
  業 &                            & \\
     & \multicolumn{2}{c}{\upbracefill} \\
  具 & \multicolumn{2}{c}{\rdelim\}{3}{*}[śreyobhyām]} \\
  爲 &                            & \\
  從 &                            & \\
  屬 & \multicolumn{2}{c}{\rdelim\}{2}{*}[śreyasos]} \\
  於 &                            &
\end{tabular}
\end{center}

女性語基は弱基に ī を附加して作る。卽ち śreyas, 女性
śreyasī; garīyas (より重き),女性 garīyasī.

\ex{第七}
\begin{longtable}{c*{2}{p{0.45\hsize}}}
 1. & rājño gṛhe mahān utsavo 'bhavat. & 王の家に於て大なる祭があつた。\\
 2. & vāyur ambhasi nāvaṃ ha\-rati. & 風は水に於て舟を運び行く。\\
 3. & vaṇijaḥ sutā sakhībhiḥ sa\-hārāme krīḍanāya gacchati sma & 商人の娘は友達と共に國へ遊戯のために行けり。\\
 4. & mahati vipadaḥ sāgare pra\-mādena nāpito 'patat. & 理髪師は不注意によつて不幸の大海に落ちた。\\
 5. & vṛddho dhanī vaṇig yuvatiṃ nirdhanasya duhitaraṃ parya\-ṇayat. & 老いたる富める商人は若き貧人の娘と結婚せり。\\
 6. & asamarthānāṃ puṃsāṃ kopa ātmana upadravāya bhavati. & 無能力なる人々の怒は自らの危難を招く。\\
 7. & ravir niśāyās tamo 'paharati. & 太陽は夜の暗を拂ひ去る。\\
 8. & sūryasya tejasā saṃtaptaḥ pānthaś chāyām āśrayate. & 日の熱に熱せられ旅人は蔭に近寄る。\\
 9. & medhāvī śuddhaṃ jīvitam ācaret. & 知者は淨き生を送るべきだ。\\
10. & śuṣkavat saras tyajanti sā\-rasāḥ. & 鶴は涸れたる池を去る。\\
11. & balavatī dantānāṃ vedanā brāhmaṇaṃ bādhate. & 太だしい齒痛は婆羅門を苦しめる。\\
12. & hariṇo lubdhakasya śarāṇāṃ prahārād ubbhāritaḥ kṛcchreṇa saraḥ praviṣṭaḥ. & 鹿は獵師の箭の打擊から逃
れてやつとのことで池に入つた。\\
13. & munis tapasi rato vane tiṣṭhati. & 聖者は苦行を樂みて林に住す。\\
14. & kāvyānāṃ śāstrāṇāṃ ca vino\-dena kālo gacchati dhīmatām. & 詩書の樂に賢者の時は過ぎ行く。\\
15. & śunaḥ puccham iva vyarthaṃ jīvitaṃ vidyayā vinā. & 知識なくしては生は犬の尾の如く價値無し。\\
16. & śriyā striyo haranti puṃsāṃ manāṃsi ca cakṣūṃsi ca. & 婦女は美によりて人々の心と目とを奪ふ。\\
17. & balaṃ vidyā ca viprāṇāṃ rājñāṃ sainyaṃ balaṃ tathā. &知識は婆羅門の力,此の如く王者の力は軍隊。\\
& balaṃ vittaṃ ca vaiśyānāṃ śūdrāṇāṃ ca kaniṣṭhatā. & 商人にとつては財は力,シュードラにとつては下賤な
ることが(力である)。
\end{longtable}

%%% Local Variables:
%%% mode: latex
%%% TeX-master: "IntroductionToSanskrit"
%%% End:
