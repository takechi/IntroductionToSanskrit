\section{第二過去組織}
第二過去の構成法に二種,重複第二過去と複說第二過去とであ
ある。

\numberParagraph \label{np:192}
重複第二過去は語根の重複によつて作られる。子音に終
る語根に就ては \ref{np:146}條が適用せられる。然し若干の注意すべきも
のがある。ṛ は a によりて重複せられる。i ではない。卽ち kṛ
(作る)~重複第二過去語基 cakṛ.

母音に終る語根に關しては次の重複が注意さるべきである。
\begin{enumerate}[label=(\alph*)]
\item 單一の子音の前に a で始まる語根は a を ā とする。
ad (食ふ)~ ād. 位置によつて長き a 並に ṛ を以て始ま
る語根は ān なる重複をなす。例 ṛc (尊敬する)~ ānarca.
ā にて始まるものは變化しない。āp (得る)~ āp.
\item i, u にて始まる語根は次の如く ī, ū となり,强語基
の場合(\ref{np:193}條)には重韻に强められたる語根の母音 e, o
の前に iy, uv となる。iṣ (欲す)弱基 īṣ, 强基 iyeṣ.
\end{enumerate}

\numberParagraph \label{np:193}
第二過去に於て强語基は單數爲他の一,二,三人稱に用ひ
られ,其の他は弱語基を用ふ。强基は次の方法で强められ弱基か
ら區別せられる。
\begin{enumerate}[label=(\alph*)]
\item ā 以外の母音に終れる,並に中間に a を有する語根は一
單爲他に重韻又は複重韻,二人稱に重韻,三人稱に複重韻の
形を取る。nī (導く)一. 單爲他 ninaya 又は nināya, 二.
ninetha, 三. nināya; kṛ (作る),一. cakara 又は
cakāra, 二. cakartha, 三. cakāra.
\item 其他の短母音は一,二,三人稱を通じて重韻となる。
budh (覺ゆ)~一. bubodha, 二. bubodhitha, 三. bubodha.
\end{enumerate}

\numberParagraph \label{np:194}
第二過去の語尾は或は直接に或は i を介して語根に附加
せられる。それは次の如くである。

\begin{tabular}{ll}
  爲他: & a, tha, a; va, athus, atus; ma, a, us. \\
  爲自: & e, se, e; vahe, āthe, āte; mahe, dhve, re.
\end{tabular}

三單複,爲自の re は常に i を伴ふ。其他の子音で始まる語
尾は tha を除きて大抵は i を挿入する。tha は時には i を挿入
し時には挿入しない。其他のものは任意に i を取り得る。kṛ
(作る),śru (聞く),vṛ (選ぶ),stu (讚む)は慣用的に連結母音
を取らない。

\numberParagraph
語根 kṛ (作る)~强基 cakar, 弱基 cakṛ.

\begin{center}
\begin{tabular}{c*{3}{p{0.23\hsize}}}
  \multicolumn{4}{c}{爲他} \\
     & 單              & 兩        & 複 \\
  1. & cakāra (cakara) & cakṛva    & cakṛma \\
  2. & cakartha        & cakrathus & cakra \\
  3. & cakāra          & cakratus  & cakrus \\
  \multicolumn{4}{c}{爲自} \\
  1. & cakre  & cakṛvahe & cakṛmahe \\
  2. & cakṛṣe & cakrāthe & cakṛdhve \\
  3. & cakre  & cakrāte  & cakrire
\end{tabular}
\end{center}

\numberParagraph
子音を以て終始する語根にして a を中間に有し語根重複
に當り代音を要しないものは强中語基に於て重複するも,弱語基
では通例只 a を e に變ずるのみで重複をなさない。二. 單,爲
他が i を挿入する場合も重複しない。

語根 pac (煮る)~强基 papac, 弱基 pec.

\begin{center}
\begin{tabular}{c*{3}{p{0.23\hsize}}}
  \multicolumn{4}{c}{爲他} \\
     & 單                 & 兩       & 複 \\
  1. & papāca (papaca)    & peciva   & pecima \\
  2. & pecitha (papaktha) & pecathus & peca \\
  3. & papāca             & pecatus  & pecus \\
  \multicolumn{4}{c}{爲自} \\
  1. & pece   & pecivahe & pecimahe \\
  2. & peciṣe & pecāthe  & pecidhve \\
  3. & pece   & pecāte   & pecire
\end{tabular}
\end{center}

\numberParagraph
ā, ai, au に終る語根は一及び三の單爲他に au なる語尾
を用ふ。弱語基にては語根の母音は母音で始まる語尾の前に消
失し,子音で始まる語尾の前にはその母音を弱めて i とする。二
單爲他は ā 又は i に作る。

語根 dā (與ふ)~强基 dadā, 弱基 dad.

\begin{center}
\begin{tabular}{c*{3}{p{0.23\hsize}}}
  \multicolumn{4}{c}{爲他} \\
     & 單                & 兩       & 複 \\
  1. & dadau             & dadiva   & dadima \\
  2. & dadātha (daditha) & dadathus & dada \\
  3. & dadau             & dadatus  & dadus \\
  \multicolumn{4}{c}{爲自} \\
  1. & dade   & dadivahe & dadimahe \\
  2. & dadiṣe & dadāthe  & dadidhve \\
  3. & dade   & dadāte   & dadire
\end{tabular}
\end{center}

\numberParagraph
va を以て始むる語根は u を以て重複音とする。而して
弱語基では語根の va と重複韻を收縮して ū とする。

語根 vac (語る)~强基 uvac, 弱基 ūc.

\begin{center}
\begin{tabular}{c*{3}{p{0.23\hsize}}}
  \multicolumn{4}{c}{爲他} \\
     & 單                 & 兩       & 複 \\
  1. & uvāca (uvaca)      & ūciva    & ūcima \\
  2. & uvacitha (uvaktha) & ūcatheus & ūca \\
  3. & uvāca              & ūcatus   & ūcus \\
  \multicolumn{4}{c}{爲自} \\
  1. & ūce   & ūcivahe & ūcimahe \\
  2. & ūciṣe & ūcāthe  & ūcidhve \\
  3. & ūce   & ūcāte   & ūcire
\end{tabular}
\end{center}

同樣に vad (語る)~强基 uvad, 弱基 ūd, vas (住す)~ uvas,
ūs.

yaj (供ふ)に就ても類推すべきだ。卽ち u の代りに i, ū の
代りに ī となる。iyaj, īj.

\numberParagraph
二つの子音の中間に a を有する或る語根は弱語基に於い
て a を消失する。
gam (行く)~强基 jagam, 弱基 jagm, 一兩,爲他 jagmiva.
同樣に khan (掘る)~ cakhniva, jan (生る)~ jajñiva, han
(殺す)~ jaghniva.

\numberParagraph
稍不規則なるもの。
\begin{enumerate}[label=(\alph*)]
\item bhū (有る)~ babhūva, babhūvitha, babhūva;
babhūviva 乃至 babhūvus.
\item ji (勝つ)~ jigāya, ci (集む)~ cicāya 又は cikāya.
\item 語根 ah (云ふ)\endnote{「云」の字は底本では印刷が乱れている。}は只次の形を存するのみである。單二.
āttha, 三. āha, 兩二. āhathus, 三. āhatus, 複三. āhus.
\end{enumerate}

\numberParagraph
子音を以て終始し中間に i, u, ṛ を有するものは强基に
於て重韻となるのみ。故に只二樣語基を有するのみである。

\begin{tabular}{ll}
  tud (打つ) & tutoda, tutudus. \\
  dṛś (見る) & dadarśa, dadṛśus.
\end{tabular}

vid (知る)は重複しない。只强基に i を e とするのみ。veda,
vettha, veda; vidima, vida, vidus.

\numberParagraph
複說第二過去は語根又は語基に ām を加へ,kṛ, bhū 又
は as の第二過去を加ふ。kṛ は動詞が爲自のみに變化する時は爲
自の形を取る。これは第十類動詞,隨てこれと同樣の構成を有す
る催起動詞又は a, ā 以外の母音に始まりそれが本來或は位置上
長き語根又は ās (坐す)に用ひられ,或は vid (知る),hu (供
ふ)等若干の動詞に用ひることを得。cur (盗む)~ corayāmāsa
tuṣ (滿足す)催起~ toṣaya (滿足せしむ)~ toṣayāmāsa, katha\-%
ya (話す)~ kathayāṃ babhūva, īkṣ (見る)爲自~ īkṣāṃ cakre,
hu (供ふ)~ juhavāṃ cakāra.

\ex{第十五}
\begin{longtable}{c*{2}{p{0.45\hsize}}}
 1. & saras-tīra upaviśya sarasi\-jaṃ dadṛśuḥ. & 池の岸に坐して彼等は蓮を見た。\\
 2. & rāja-puruṣo laguḍena cauraṃ tutoda. & 王の家來は杖で盗人を打つた。\\
 3. & guruḥ śiṣyān nininda. & 師は弟子は叱責した。\\
 4. & kapayo vṛkṣād bhūmau petuḥ. & 猿等は樹より地へ落ちた。\\
 5. & catvāro bālakā rathena naga\-rāya jagmire. & 四人の子供が車で街へ行つた。\\
 6. & mūṣikā bhūmiṃ cakhnire. & 鼠等は血を掘つた。\\
 7. & aham Ayodhyāyāṃ jagāma. & 我は阿輸遮へ行けり。\\
 8. & nṛpo mṛgayāṃ gatvā bahūn mṛgāṃś caikaṃ vyāghraṃ ca jaghāna.
 & 王は獵に行き多くの鹿と一疋の虎とを殺せり。\\
 9. & aśvena gacchantaṃ nṛpaṃ tasya parivārā anujagmuḥ. & 馬にて行く王に彼の家來達
は從つた。\\
10. & rājñaḥ puruṣāḥ sva-rājānaṃ na viduḥ. & 王の家來等は自分の王を知
らざりき。\\
11. & mūṣiko bhūmiṃ khanitvā sva\-vivaraṃ cakāra. & 鼠は地を掘りて自らの穴を作りき。\\
12. & vṛkṣasya mūle dvau bālakau cikrīḍatuḥ. & 樹の下に於いて二人の子は遊べり。\\
13. & tāv etaṃ bālakaṃ tutudatur iti śrutvāhaṃ tau nininda. & 彼等二人が此の小兒を打て
りと聞きて我は彼等二人を叱責せり。\\
14. & kiṃ tvaṃ mahā-śabdaṃ ca\-kartha. & 何故に汝は大なる聲をなせるか。\\
15. & mahā-siṃhaṃ dṛṣṭvā tad\-bhayād vayaṃ mahā-śabdaṃ cakṛma. & かの少女は美はしき歌を歌へり。\\
16. & kavir kīrtiṃ iyeṣa. & 詩人は名譽を欲せり。\\
17. & sa vaṇik-putraḥ pratyahaṃ saktuṃ brāhmaṇebhyo dadau. & 彼の商人の子は每日麥粉を
婆羅門に與へたり。\\
18. & sa bālakaḥ kūpāj jalaṃ ānināya. & かの子供は井戸から水を持ち來つた。\\
19. & ghṛtena narā devam ījuḥ. & 人々は酥にて神を祀れり。\\
20. & ete janā nṛpam ūcuḥ. & これ等の人々は王に云へり。\\
21. & rājña ājñāṃ śrutvā rāja\-puruṣaḥ pratyuvāca: yāthā deva ājñāpayatītī.
& 王の命令を聞きて王の家來は答へて云へり。王の命のまゝにと。\\
22. & aśvāt patito rājā mamāra. & 馬より落ちし王は死せり。\\
23. & rājño gṛhe dvau brāhmaṇāv ūṣatuḥ. & 王の家に於いて二人の婆羅門は住めりき。\\
24. & nṛpasya śataṃ putrā ba\-bhūvuḥ. & 王は百人の子等ありき。\\
25. & taṃ sa Bhīmaḥ prajākāmas toṣayāmāsa dharmavit. & 子を欲せる法に達せるブヒ
ーマは彼を滿足せしめた \\
26. & taṃ ca dṛḍhaṃ pariṣvajyā\-śrubhiḥ snapayāmāsa. & 而して彼をかたく抱きて淚
もて沐浴せしめた。\\
27. & svīyaṃ rūpaṃ taṃ darśa\-yāmāsa. & 自分の美貌を彼に示した。
\end{longtable}

%%% Local Variables:
%%% mode: latex
%%% TeX-master: "IntroductionToSanskrit"
%%% End:
