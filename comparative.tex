\section{比較法}
\numberParagraph
形容詞の比較級は男性の語基に tara を,最上級は tama
を附加して作る。變化は a 語基に準ず。女性は終の母音を ā に
作り kanyā の變化に準ず。priya (愛する)~ priyatara (より
愛する),priyatama (最も愛する)。

\numberParagraph
二語基ある場合は弱基に,三語基ある場合は中基に附加せ
らる。例:prāc (東方の)~ prāktara, prāktama.

\numberParagraph \label{np:98}
他の方法は後接字 īyas, iṣṭha を語基に附加して作る。
語基の母音は重韻化又は延長によりて强められる。kṣipra (速か
なる)は語根 kṣip から比較級 kṣepīyas, 最上級 kṣepiṣṭha が
作られ,mṛdu (軟かき)は mradīyas, mradiṣṭha; dūra (遠き)
は davīyas, daviṣṭha.

時には原級の形が比較級最上級の形と異るものがある。例せば
śreyas, śreṣṭha の原級は śrī にあらずして praśasya (勝れた
る)であり,kanīyas, kaniṣṭha の原級は alpa (小なる)である。

%%% Local Variables:
%%% mode: latex
%%% TeX-master: "IntroductionToSanskrit"
%%% End:
