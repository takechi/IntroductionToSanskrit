\chapter{語尾曲法}\label{cha:flection}
\section{格例法}
\numberParagraph
言語の各類に就て變化するものと變化しないものとがあ
る。變化しないものとは接續詞,間投詞,副詞の如きである。今
語尾曲法に於て取扱はうとするものはこれら変化せざるものを暫
らく除外して變化するものに就て論ずる。變化する詞にも二樣の
種類が分れる。一は卽ち名詞的變化であつて性と數と格によつて
語尾を異にするもの。この種に屬するものは名詞,代名詞,形容
詞,數詞,分詞である。他は卽ち動詞的變化であつて人稱と數と
時と法とに隨つて語尾を異にする。前者の取扱ひを格例法と云
ひ,後者の取扱ひを活用法と云ふ。

\numberParagraph
梵語では格例法に三性 三數 八格を認める。三性とは男
女 中であり,三數とは單 兩 複であり,八格とは主 業 具 爲
從 屬 於 呼である。活用法には三の人稱と 三數と 三時と 四
法を認める。曰く 一人稱 二人稱 三人稱,曰く單 兩 複,曰く
現在 過去 未來,曰く現實法,可能法,命令法,條件法である。
これらの意義は後節に説く。

\numberParagraph
梵語では名詞的變化は語基を以て,動詞的變化は語根を以
て單位とする。單位とは分解の終極を意味し,それ以上溯る必要
を見ない形である。この形で辭典に記録されるものと知るべきで
ある。嚴密に云はば如何なる詞もその語根まで溯るべきである
が,名詞,代名詞,形容詞,數詞,分詞は必ずしも語根の詮索の
手を延すを要しない。出來上つた語基の形に語尾が附けられる。
然し動詞は語根によりて論ぜられる。之を要するに語根は梵語の
あらゆる語彙を分析してこれを原始の狀態に還元したものであ
る。印度平原に花咲く言語から得た收穫である。萬を以て數へら
れる言語が數百の語根に還元せらるゝことは實に言語學の寄與貢
献の偉大を示すものである。

\numberParagraph
語根は多くは單綴の音であり,意義はあるも未だ言語では
ない。云はば言語のエキスである。今語根に若干の変化を施して
語基を作り,これに語尾を附加すれば始めて活用する言語とな
る。或は之を逆に考へて活用する言語を比較し分析して語尾を切
離し,語基を整理して最後の形を得たとすればそれが語根であ
る。

\numberParagraph
\textbf{名詞的變化}に就て\textbf{語基が母音}で終るものと\textbf{子音で終るも
  の}との二類を分つ。前者を母音語基,後者を子音語基と呼ぶ。母
音語基は語尾が附加せられて太だ不規則なる變化をなすから,稍
複雜ではあるが,語基に强弱を認むる必要あまりなきと,數に於
いて飛び離れて多いとの理由から,これを最初に取扱ふことゝす
る。先づ最初に a にて終る男性中性の語基を擧げる。女性は a
にて終る場合が無い。

\numberParagraph
\hfil 1) \fbox{a 語基,男性 deva(神)} \hfil\,
\begin{center}
\begin{tabular}{c*{3}{p{0.2\hsize}}}
     & \cellAlign{c}{單} & \cellAlign{c}{兩}          & \cellAlign{c}{複} \\
  主 & devas             & \rdelim\}{3}{*}[devau]     & \rdelim\}{2}{*}[devās] \\
  呼 & deva              &                            & \\
  業 & devam             &                            & devān \vspace{0.5\zw} \\
  具 & devena            & \rdelim\}{3}{*}[devābhyām] & devais \\
  爲 & devāya            &                            & \rdelim\}{2}{*}[devebhyas] \\
  從 & devāt             &                            & \vspace{0.5\zw} \\
  屬 & devasya           & \rdelim\}{2}{*}[devayos]   & devānām \\
  於 & deve              &                            & deveṣu
\end{tabular}
\end{center}

\numberParagraph
\hfil 2) \fbox{a 語基,中性 dāna(施} \hfil\,
\begin{center}
\begin{tabular}{c*{3}{p{0.2\hsize}}}
     & \cellAlign{c}{單}      & \cellAlign{c}{兩}          & \cellAlign{c}{複} \\
  主 & \rdelim\}{2}{*}[dānam] & \rdelim\}{3}{*}[dāne]      & \rdelim\}{3}{*}[dānāni] \\
  業 &                        &                            & \\
  呼 & dāna                   &                            &
\end{tabular}
\end{center}
餘は全く男性の如く變化する。

\numberParagraph
a 語基の男中性形容詞もこれに準じて變化する。例:基
pāpa (罪ある),主男 pāpas,中 pāpam 等。形容詞の女性は ā
又は ī に終る。格變化はその條を参照。形容詞の中性單數主格
の形は副詞の意義を有す。例:śīghra (速かなる),śīghram (速
かに)。

\numberParagraph
\hfil 3) \fbox{i 語基,男女中性} \hfil\,

男性 kavi (詩人),女性 mati (意),中性 vāri (水).

\begin{center}
\begin{tabular}{c*{3}{p{0.2\hsize}}}
  \multicolumn{4}{c}{單} \\
     & \cellAlign{c}{男}      & \cellAlign{c}{女}              & \cellAlign{c}{中} \\
  主 & kavis                  & matis                          & vāri \\
  呼 & kave                   & mate                           & vāri, vāre \\
  業 & kavim                  & matim                          & vāri \\
  具 & kavinā                 & matyā                          & vāriṇā \\
  爲 & kavaye                 & mataye, matyai                 & vāriṇe \\
  從 & \rdelim\}{2}{*}[kaves] & \multirow{2}{*}{mates, matyās} & \multirow{2}{*}{vāriṇas} \\
  屬 &                        &                                & \\
  於 & kavau                  & matau, matyām                  & vāriṇi
\end{tabular}
\end{center}
\begin{center}
\begin{tabular}{c*{3}{p{0.2\hsize}}}
  \multicolumn{4}{c}{兩} \\
  主 & \rdelim\}{3}{*}[kavī]      & \multirow{3}{*}{matī}      & \multirow{3}{*}{vāriṇī} \\
  呼 &                            &                            & \\
  業 &                            &                            & \\
  具 & \rdelim\}{3}{*}[kavibhyām] & \multirow{3}{*}{matibhyām} & \multirow{3}{*}{vāribhyām} \\
  爲 &                            &                            & \\
  從 &                            &                            & \\
  屬 & \rdelim\}{2}{*}[kavyos]    & \multirow{2}{*}{matyos}    & \multirow{2}{*}{vāriṇos} \\
  於 &                            &                            &
\end{tabular}
\end{center}
\begin{center}
\begin{tabular}{c*{3}{p{0.2\hsize}}}
  \multicolumn{4}{c}{複} \\
  主 & \rdelim\}{2}{*}[kavayas]   & \multirow{2}{*}{matayas}   & \rdelim\}{3}{*}[vārīṇi] \\
  呼 &                            &                            & \\
  業 & kavīn                      & matīs                      & \\
  具 & kavibhis                   & matibhis                   & vāribhis \\
  爲 & \rdelim\}{2}{*}[kavibhyas] & \multirow{2}{*}{matibhyas} & \multirow{2}{*}{vāribhyas} \\
  從 &                            &                            & \\
  屬 & kavīnām                    & matīnām                    & vārīṇām \\
  於 & kaviṣu                     & matiṣu                     & vāriṣu
\end{tabular}
\end{center}

\numberParagraph \label{np:49}
\hfil 4) \fbox{若干の不規則なるもの} \hfil\,
\begin{enumerate}[label=(\alph*)]
\item sakhi (男性,友)。
\begin{description}[font=\normalfont]
\item[單] sakhā, sakhe, sakhāyam, sakhyā, sakhye, sakhyus,
sakhyou.
\item[兩] sakhāyau, sakhibhyām, sakhyos.
\item[複] sakhāyaś, sakhīn, sakhibhis. 餘は kavi に準ず。
\end{description}
\item pati は「夫」の意味の場合,單具 patyā,爲 patye,從屬
patyus,於 patyau. 又「主」の意味に用ひらるゝか又は bhū-pati
(地主)の如く合成語の終にある時は kavi に準じて變化す。
\item \label{item:49c} akṣi (眼),asthi (骨),dadhi (酸乳),sakthi (腿)は an
語基(86條參照)に準じて變化する部分を有す。單具 akṣṇā 等
(87條)。
\end{enumerate}

\numberParagraph
\hfil 5) \fbox{u 語基、男性 guru (師),女性 dhenu (牝牛),中性 madhu (蜜)} \hfil\,
\begin{center}
\begin{tabular}{c*{3}{p{0.2\hsize}}}
  \multicolumn{4}{c}{單} \\
     & \cellAlign{c}{男}      & \cellAlign{c}{女}                & \cellAlign{c}{中} \\
  主 & gurus                  & dhenus                           & madhu \\
  呼 & guruo                  & dheno                            & madhu, madho \\
  業 & gurum                  & dhenum                           & madhu \\
  具 & guruṇā                 & dhenvā                           & madhunā \\
  爲 & gurave                 & dhenave, dhenvai                 & madhune \\
  從 & \rdelim\}{2}{*}[guros] & \multirow{2}{*}{dhenos, dhenvās} & \multirow{2}{*}{madhunas} \\
  屬 &                        &                                  & \\
  於 & gurau                  & dhenau, dhenvām                  & madhuni
\end{tabular}
\end{center}
\begin{center}
\begin{tabular}{c*{3}{p{0.2\hsize}}}
  \multicolumn{4}{c}{兩} \\
  主 & \rdelim\}{3}{*}[gurū]      & \multirow{3}{*}{dhenū}      & \multirow{3}{*}{madhum} \\
  呼 &                            &                             & \\
  業 &                            &                             & \\
  具 & \rdelim\}{3}{*}[gurubhyām] & \multirow{3}{*}{dhenubhyām} & \multirow{3}{*}{madhubbyām} \\
  爲 &                            &                             & \\
  從 &                            &                             & \\
  屬 & \rdelim\}{2}{*}[gurvos]    & \multirow{2}{*}{dhenvos}    & \multirow{2}{*}{madhunos} \\
  於 &                            &                             &
\end{tabular}
\end{center}
\begin{center}
\begin{tabular}{c*{3}{p{0.2\hsize}}}
  \multicolumn{4}{c}{複} \\
  主 & \rdelim\}{2}{*}[guravas]   & \multirow{2}{*}{dhenavas}   & \rdelim\}{3}{*}[madhūni] \\
  呼 &                            &                             & \\
  業 & gurūn                      & dhenūs                      & \\
  具 & gurubhyas                  & dhenubhis                   & madhubhis \\
  爲 & \rdelim\}{2}{*}[gurubhyas] & \multirow{2}{*}{dhenubhyas} & \multirow{2}{*}{madhubhyas} \\
  從 &                            &                             & \\
  屬 & gurūṇām                    & dhenūnām                    & madhūnām \\
  於 & guruṣu                     & dhenuṣu                     & madhuṣu
\end{tabular}
\end{center}

\numberParagraph
i, u 語基形容詞男女中性これに準ず。例:語基 śuci (淸
き)主,男,女 śucis 中,śuci. 語基 tanu (痩せたる)主,男,女
tanus, 中 tanu. 中性にてはこの語尾以外爲從屬於の單,屬於
の兩には之に相當せる男性の語尾を用ふるを得。

\hfil 6) \fbox{ā, ī, ū 語基} \hfil\,

\numberParagraph \label{np:52}
ā, ī, ū 語基は女性である。その中 ī, ū 語基にありては
多綴のものと單綴のものと稍變化を異にす。例:kanyā (少女),
nadī (河),vadhū (婦).
\begin{center}
\begin{tabular}{c*{3}{p{0.2\hsize}}}
  \multicolumn{4}{c}{單} \\
  主 & kanyā                     & nadī                    & vadhū \\
  呼 & kanye                     & nadi                    & vadhu \\
  業 & kanyām                    & nadīm                   & vadhūm \\
  具 & kanyayā                   & nadyā                   & vadhvā \\
  爲 & kanyāyai                  & nadyai                  & vadhvai \\
  從 & \rdelim\}{2}{*}[kanyāyās] & \multirow{2}{*}{nadyās} & \multirow{2}{*}{vadhvās} \\
  屬 &                           &                         & \\
  於 & kanyāyām                  & nadyām                  & vadhvām
\end{tabular}
\end{center}
\begin{center}
\begin{tabular}{c*{3}{p{0.2\hsize}}}
  \multicolumn{4}{c}{兩} \\
  主 & \rdelim\}{3}{*}[kanye]      & \multirow{3}{*}{nadyau}    & \multirow{3}{*}{vadhvau} \\
  呼 &                             &                            & \\
  業 &                             &                            & \\
  具 & \rdelim\}{3}{*}[kanyābhyām] & \multirow{3}{*}{nadībhyām} & \multirow{3}{*}{vadhūbhyām} \\
  爲 &                             &                            & \\
  從 &                             &                            & \\
  屬 & \rdelim\}{2}{*}[kanyayos]   & \multirow{2}{*}{nadyos}    & \multirow{2}{*}{vadhvos} \\
  於 &                             &                            &
\end{tabular}
\end{center}
\begin{center}
\begin{tabular}{c*{3}{p{0.2\hsize}}}
  \multicolumn{4}{c}{複} \\
  主 & \rdelim\}{3}{*}[kanyās]     & \rdelim\}{2}{*}[nadyas]    & \multirow{2}{*}{vadhvas} \\
  呼 &                             &                            & \\
  業 &                             & nadīs                      & vadhūs \\
  具 & kanyābhis                   & nadhībhis                  & vadhūbhis \\
  爲 & \rdelim\}{2}{*}[kanyābhyas] & \multirow{2}{*}{nadibhyas} & \multirow{2}{*}{vadhūbhyas} \\
  從 &                             &                            & \\
  屬 & kanyānām                    & nadīnām                    & vadhūnām \\
  於 & kanyāsu                     & nadīṣu                     & vadhuṣu
\end{tabular}
\end{center}

\numberParagraph
單綴の ī, ū 語基。śrī (幸福), bhū (地).
\begin{center}
\begin{tabular}{c*{2}{p{0.3\hsize}}}
  \multicolumn{3}{c}{單} \\
  主 & \rdelim\}{2}{*}[śrīs]           & bhūs \\
  呼 &                                 & \\
  業 & śriyam                          & bhuvam \\
  具 & śriyā                           & bhuvā \\
  爲 & śriye, śriyai                   & bhuve, bhuvai \\
  從 & \rdelim\}{2}{*}[śriyas, śriyās] & \multirow{2}{*}{bhuvas, bhuvās} \\
  屬 &                                 & \\
  於 & śriyi, śriyām                   & bhuvi, bhuvām
\end{tabular}
\end{center}
\begin{center}
\begin{tabular}{c*{2}{p{0.3\hsize}}}
  \multicolumn{3}{c}{兩} \\
  主 & \rdelim\}{3}{*}[śriyau]   & \multirow{3}{*}{bhūs} \\
  呼 &                           & \\
  業 &                           & \\
  具 & \rdelim\}{3}{*}[śribhyām] & \multirow{3}{*}{bhūbhyām} \\
  爲 &                           & \\
  從 &                           & \\
  屬 & \rdelim\}{2}{*}[śriyos]   & \multirow{2}{*}{bhuvos} \\
  於 &                           &
\end{tabular}
\end{center}
\begin{center}
\begin{tabular}{c*{2}{p{0.3\hsize}}}
  \multicolumn{3}{c}{複} \\
  主 & \rdelim\}{3}{*}[śriyas]   & \multirow{3}{*}{bhuvas} \\
  呼 &                           & \\
  業 &                           & \\
  具 & śrībhis                   & bhūbhyas \\
  爲 & \rdelim\}{2}{*}[śribhyas] & \multirow{2}{*}{bhūbhyas} \\
  從 &                           & \\
  屬 & śriyām, śrīṇām            & bhuvām, bhūnām \\
  於 & śrīṣu                     & bhūṣu
\end{tabular}
\end{center}

\nt{
  strī は不規則にして單主 strī, 呼 stri, 業 striyam 又は strīm,
  爲 striyai, 從屬 striyās, 具 striyām, 複業 strīs 又は striyas, 屬
  strīṇām.}

\numberParagraph
複母音語基,nau (舟),go (牛)。\endnote{底本では両数形について主・呼・業ではなく
  呼・業が \} でまとめられている(主格の欄は空欄)。また具・為をまとめる \} が
  複数形ではなく単数形の列に付いている。}
\begin{center}
\begin{tabular}{c*{6}{p{0.12\hsize}}}
     & \multicolumn{2}{c}{單}                         & \multicolumn{2}{c}{兩}                               & \multicolumn{2}{c}{複} \\
  主 & \rdelim\}{2}{*}[naus]  & \multirow{2}{*}{gaus} & \rdelim\}{3}{*}[nāvau]    & \multirow{3}{*}{gāvau}   & \rdelim\}{3}{*}[nāvas]   & \rdelim\}{2}{*}[gāvas] \\
  呼 &                        &                       &                           &                          &                          & \\
  業 & nāvam                  & gām                   &                           &                          &                          & gās \\
  具 & nāvā                   & gavā                  & \rdelim\}{3}{*}[naubhyām] & \multirow{3}{*}{gobhyām} & \rdelim\}{2}{*}[naubhis] & \multirow{2}{*}{gobhis} \\
  爲 & nāve                   & gave                  &                           &                          &                          & \\
  從 & \rdelim\}{2}{*}[nāvas] & \multirow{2}{*}{gos}  &                           &                          & nāvām                    & gavām \\
  屬 &                        &                       & \rdelim\}{2}{*}[nāvos]    & \multirow{2}{*}{gavos}   & \multirow{2}{*}{nauṣu}   & \multirow{2}{*}{goṣu} \\
  於 & nāvi                   & gavi                  &                           &                          &                          &
\end{tabular}
\end{center}

\numberParagraph
ṛ 語基 これは一部分 作者名詞であり,一部分は 親族名
詞である。男 dātṛ (施者),女 svasṛ (姉妹)。
\begin{center}
\begin{tabular}{c*{4}{p{0.12\hsize}}}
     & \multicolumn{2}{c}{單}                           & \multicolumn{2}{c}{複} \\
  主 & dātā                   & svasā                   & \rdelim\}{2}{*}[dātāras]   & \multirow{2}{*}{svasāras} \\
  呼 & dātar                  & svasar                  &                            & \\
  業 & dātāram                & svasāram                & dātrīn                     & svasṝs \\
  具 & dātrā                  & svasrā                  & dātṛbhis                   & svasṛbhis \\
  爲 & dātre                  & svasre                  & \rdelim\}{2}{*}[dātṛbhyas] & \multirow{2}{*}{svasṛbhyas} \\
  從 & \rdelim\}{2}{*}[dātur] & \multirow{2}{*}{svasur} &                            & \\
  屬 &                        &                         & dātṝṇām                    & svasṝṇām \\
  於 & dātari                 & svasari                 & dātrṣu                     & svasṛṣu
\end{tabular}
\end{center}

\begin{center}
\begin{tabular}{c*{2}{p{0.24\hsize}}}
     & \multicolumn{2}{c}{兩} \\
  主 & \rdelim\}{3}{*}[dātārau]   & \multirow{3}{*}{svasārau} \\
  呼 &                            & \\
  業 &                            & \\
  具 & \rdelim\}{3}{*}[dātṛbhyām] & \multirow{3}{*}{svasṛbhyām} \\
  爲 &                            & \\
  從 &                            & \\
  屬 & \rdelim\}{2}{*}[dātros]    & \multirow{2}{*}{svasros} \\
  於 &                            &
\end{tabular}
\end{center}
中性の變化は ī, ū の中性に準ず。卽ち單主業 dātṛṇā,
爲 dātṛṇe, 兩主呼業 dātṛṇī, 複主呼業 dātṝṇī.

\numberParagraph
親族名詞〔naptṛ (孫),svasṛ (姉妹)を除く〕は單,業,
兩主業呼,及複主に於て異る。a が短音である。

\begin{tabular}{cl}
  語基 & pitṛ (父) pitaram, pitarau, pitaras. \\
  〃   & mātṛ (母) mātram, mātarau, mātaras. \\
  〃   & duhitṛ (娘) duhitaram, huhitarau, huhitaras.
\end{tabular}

(注意)格例法は次 \ref{np:78} 條に連續する。

\ex{第一}

\begin{longtable}{c*{2}{p{0.45\hsize}}}
   1. & andhasya dīpo vidyā.             & 知識は盲冥の燈火なり。\\
   2. & ācāraḥ pradhāno dharmaḥ.         & 善行は最上の法なり。\\
   3. & kālasya kuṭilā gatiḥ.            & 時の路は曲れり。\\
   4. & śīlaṃ narasya bhūṣaṇam.          & 戒は人の莊嚴なり。\\
   5. & alpaḥ kālo bahavaś ca vigh\-nāḥ. & 短い時と多くの障害。\\
   6. & yathā vṛkṣās tathā phalāni.      & 樹の如くかくの如く果(あり)。\\
   7. & aputrasya gṛhaṃ śūnyam.          & 子なき者の家は空虚なり。\\
   8. & duḥkhaṃ kadāpi sukhasya          & 苦は往々樂の因なり。\\
   9. & yatra yatra dhūmas tatra tatra vahniḥ. & 烟ある所其處に火あり。\\
  10. & pādapānāṃ bhayaṃ vātaḥ padmānāṃ śiśiro bhayam. & 樹の恐は風,蓮花の恐は冷氣。\\
      & parvatānāṃ bhayaṃ vajraḥ sādhūnāṃ durjano bhayam. & 山の恐は雷,善人の恐は悪人なり。\\
  11. & upadeśo hi mūrkhāṇāṃ pra\-kopāya na śāntaye. & 論議は愚者の怒を招く,鎭靜に役立たない。\\
  12.& nagaraṃ devena jitam. & 都城は王によりて打ち勝たれた。\\
  13. & naraḥ sarpeṇa daṣṭo mṛtaś ca. & 人は蛇によりて嚙まれて\ruby{而}{しこう}して死せり。\\
  14. & siṃho vyādhasya śareṇa hataḥ. & 獅子は獵師の\ruby{箭}{や}によりて殺された。\\
  15. & mūṣikā śyenena gṛhītā bhak\-ṣitā ca. & 鼠は鷹によりて捉えられ而して食はれた。\\
  16. & raṇe nṛpasya senayārayo jitāḥ. & 戰鬪に於て王の軍隊のために敵は征服された。\\
  17. & nirdhanasya kutaḥ sukham. & 貧しきものには何處にか幸福あらむ。\\
  18  & vidyā mitraṃ pravāse ca & 旅行に於て知識は友であり \\
      & bhāryā mitraṃ gṛheṣu ca. & 妻は家に於て友であり,\\
      & vyādhitasyauṣadhaṃ mitram dharmo mitraṃ mṛtasya ca & 病めるものに藥は友であり,而して死者にとりては正義は友である。
\end{longtable}

\clearpage
\section{活用法}
\numberParagraph
梵語の活用法は語根に若干の變化を加へてこれに語尾を
附するにある。卽ち yaj (祭る)+ a = yaja + ti (二人稱單數,現
實法語尾),yajati (彼は祭る)。

梵語にては能動調受動調ある外に能動調を更に爲他,爲自の二
類に分ち,語尾を異にす。されどその意義に至つては爲他も爲自
も同様である。

三數は格例法の如くである。

\numberParagraph
時法に關しては異れる四類がある。1. 現在組織は現實
法,第一過去,可能法,命令法である(これら四類は共通の語基
を有す)。2. 未來時。3. 第二過去時,4. 第三過去時。この中
2, 3, 4 類は 1 類が特定の語基を有するに反し,語根へ直接に語尾
が附けられる。受動も亦さうである。只第十類動詞(59, 73 條)の
みは大抵の形が現在語基から作られる。

\numberParagraph
現在語基から作られる形,現實法,第一過去,可能法,命
令法。語基の作られる樣式によりて語基は二種十類に分たれる。

\fbox{(A) 第一種變化} 現在語基がすべて a にて終る。第一類,
第六類,第四類,第十類がこれに屬す。

第一類。語根に a を附加して語基を作る。その母音は重韻と
なる。bhū (ある)~ bhavati (彼はあり)。

第六類 語根に語勢ある a を附加して語基を作る。語根の母
音は變化せず。tub (打つ)~ tudati (彼は打つ)。

第四類。語根に ya を附加して語基を作る。div (博戯す)~
dīvyati (彼は博戯す)。

第十類。語根に aya を附加して語基を作る。語根の母音は通
常重韻化せらる。cur (盗む)~ corayati (彼は盗む)。

\fbox{(B) 第二種變化} 語根に强弱を分つ。これに屬するものは
第二類,第三類,第五類,第七類,第八類,第九類である。

第二類は語根に直に語尾を附加する。ad (食ふ)~ atti (彼は
食ふ)。

第三類。語根が重複せらる。hu (供ふ)~ juhoti (彼は供ふ),
juhumas (我々は供ふ)。

第七類。强語基に na 弱語基に n を挿んで現在語基を作る。
bhid (破る)~ bhinatti (彼は破る),bhindmas (我々は破る)。

第五類。强語基に no, 弱語基に nu を加へる。su (搾る)~
sunoti (彼は搾る),sunumas (我々は搾る)。

第八類。强語基に o, 弱語基に u を附加する。tan (擴ぐ)~
tanoti (彼は擴ぐ),tanumas (我々は擴ぐ)。

第九類。語根に nā を附加して强語基,子音語尾の前には nī,
母音語尾の前には n を附加して弱語基を作る。aś (食ふ)~
aśnāti (彼は食ふ),aśnīmas (我々は食ふ),aśnate (彼等は食
ふ)。

その他,受動並に派生動詞,卽ち催起動詞,重複動詞,求欲動詞
名稱動詞がある。

\begin{center}現在組織\end{center}
\subsection{第一種變化}
\subsubsection{第一類 a 級}
\numberParagraph
a 級の構成。語根に a を附加す。語根の母音は重韻化す
る。聲中に在りて本來長きと位置によりて長きは重韻かせず。nī
(導く)現在語基 ne + a = naya (\ref{np:31}條)。nind (嘲る)~ ninda.

\numberParagraph \textbf{現実法の語尾。}

爲他 mi, si, ti; vas, thas, tas; mas, tha, anti.

爲自 e, se, te; vahe, āthe, āte; mahe, dhve, ante.

語基の a は m, v にて始まる語尾の前に延長せられ,anti,
ante の前に省略せられ,āthe, āte の ā と合して e となる。

\numberParagraph
a 級の語根 bhū (ある),現在語基,bho + a = bhava.
\begin{center}
\begin{tabular}{c*{3}{p{0.15\hsize}}}
  \multicolumn{4}{c}{\textbf{現實法}} \\
  \multicolumn{4}{c}{爲他} \\
     & 單       & 兩        & 複 \\
  1. & bhavāmi  & bhavāvas  & bhavāmas \\
  2. & bhavasi  & bhavathas & bhavatha \\
  3. & bhavati  & bhavatas  & bhavanti \\
  \multicolumn{4}{c}{爲自} \\
  1. & bhave   & bhavāvahe & bhavāmahe \\
  2. & bhavase & bhavethe  & bhavadhve \\
  3. & bhavate & bhavete   & bhabante
\end{tabular}
\end{center}

\numberParagraph
若干の不規則なる語根を擧ぐれば
\begin{enumerate}[label=(\alph*)]
\item guh (覆ふ)は語根の母音を重韻化せず。只延長す。
gūhati.
\item kram (歩む)は ya 級によりても變化せられ(\ref{np:69}條),
爲他にありて母音を延長し爲自に於ては延長せず。
\item 或る語根は鼻音を失ふ。

\indent daṃś (咬む)~ daśati.

\indent rañj (染める)~ rajati.

\indent sañj (着く)~ sajati.

\indent svañj (抱く)~ svajati.
\item gam (行く),yam (與ふ),iṣ (欲す),ṛ (達す)は夫々
語基 gaccha, yaccha, iccha, ṛccha に作る。
\item sad (坐す)~ sīdati, sthā (立つ)~ tiṣṭhati, pā (飲む)
~ pibati, ghrā (嗅ぐ)~ jighrati.
\end{enumerate}

\ex{第二}

\begin{longtable}{c*{2}{p{0.45\hsize}}}
 1. & kva gacchasi. & 汝は何處に行くか。\\
 2. & gacchāmi grāmam. & 我は村へ行く。\\
 3. & mitraṃ hvayāmi. & 我は友を喚ぶ。\\
 4. & kuto dhāvathaḥ. & 何故に汝等兩人は走るか。\\
 5. & gajasya bhayād ghāvāvaḥ. & 象の恐怖の故に我等兩人は
走る。\\
 6. & ghaṭas tale patati. & 甕は地上へ落ちたり。\\
 7. & pāpā janāḥ svargaṃ na gacchanti. & 罪ある人々は天國へ行かぬ。\\
 8. & agnis tiṣṭhati gūḍho dāruṣu. & 火は薪の中に隱れてあり。\\
 9. & mitrasya phalaṃ prayacchāmi. & 我は友に果物を與ふ。\\
10. & Devadatto 'nnaṃ pacati. & デーヷダッタは果物を煮る。\\
11. & vane vṛkam īkṣāmahe. & 我々は林の中に狼を見る。\\
12. & ācāryaḥ śiṣyaṃ nindati. & 師は弟子を責む。\\
13. & prāsādasya samīpe hrado bhavati. & 宮殿の側に池がある。\\
14. & nṛpaḥ sainyena Pāṭaliputraṃ praṭiṣṭhati. & 王は軍隊と共にパータリプトラへ出發せり。\\
15. & Godāvaryā jale gajau viha\-rataḥ. & ゴーダーヷリー河の水に於て二つの象が遊ぶ。\\
16. & padmasya pattreṣu jalaṃ na sajati. & 蓮華の葉に於て水は着かぬ。\\
17. & anilasya vaśena vṛkṣāḥ kam\-pante. & 風の力によりて樹々は動
く。\\
18. & puṣpāṇi vasante prasphoṭanti. & 花は春に於いて開く。\\
19. & na niścayād viramanti dhīrāḥ. & 勇者は計劃を中止せぬ。\\
20. & adya brāhmaṇau nagaraṃ tyajataḥ. & 今日二人の婆羅門が市城
を去つた。\\
21. & candrasyāloke kumudaṃ vika\-sati. & 月の出現に於いて蓮華は開
く。\\
22. & lobhāt krodhaḥ prabhavati & 貪慾より憤怒は生ず。\\
    & lobhāt kāmaḥ pravartate, & 貪慾より愛慾は起る。\\
    & lobhān mohaś ca nāśaś ca & 貪慾より愚痴と破滅と
(あり)。\\
    & lobhaḥ pāpasya kāraṇam. & 貪慾は罪惡の原因なり。
\end{longtable}


\subsubsection{第六類 á 級}
\begin{center}第一過去變化\end{center}

\numberParagraph
第六類では語根の母音は變化しない。附加せらるる a は
語勢を有するのが特徴である。

\numberParagraph 第一過去の語尾。

爲他 am, s, t; va, tam, tām; ma, ta, an.

爲自 i, thās, ta; vahi, āthām, ātām; mahi, dhvam, anta.

\numberParagraph
第一過去はこれらの語尾を附加する外に過去符 a を語根
の前に加ふ。又 m, v にて始まる語尾の前の語基の a は延長せら
れ,語尾 an 並に anta の前に a は省略せられ,āthām, ātām
の ā と合して e に變ず。

過去符 a は前置詞の後語根の前に加ふ。卽ち tyaj (捨)+ pari
(斷念す)三單.第一過去 paryatyajat (\ref{np:13} 條參照)。

\numberParagraph
語根 tud (打つ),現在語基 tuda.
\begin{center}
\begin{tabular}{c*{3}{p{0.15\hsize}}}
  \multicolumn{4}{c}{\textbf{第一過去}} \\
  \multicolumn{4}{c}{爲他} \\
     & 單      & 兩       & 複 \\
  1. & atudam  & atudāva  & atudāma \\
  2. & atudas  & atudatam & atudata \\
  3. & atudat. & atudatām & atudan \\
  \multicolumn{4}{c}{爲自} \\
  1. & atude(a+i) & atudāvahi & atudāmahi \\
  2. & atudathās  & atudethām & atudata \\
  3. & atudata    & atudetām  & atudan
\end{tabular}
\end{center}

\numberParagraph
á 級の不規則なる語根。
\begin{enumerate}[label=(\alph*)]
\item 或る語根は結尾の子音の前に微韻を挿入する。

\indent muc (解く)muñcati.

\indent lip (塗る)limpati.

\indent sic (\ruby{灌}{そそ}ぐ)siñcati.

\indent kṛt (斷つ)kṛntati.

\indent vid (見出す)vindati.

\indent lup (碎く)lumpati.
\item prach (問ふ)はpṛcchati, iṣ (欲す)は icchati とな
る。(第一類の gam 等と比較)。
\end{enumerate}

\ex{第三}

\begin{longtable}{c*{2}{p{0.45\hsize}}}
 1. & vṛkṣasya cchāyāyāṃ munir asīdat. & 樹の蔭に於て聖者は坐したり。\\
 2. & Kālidāsaṃ kaviṃ sevāmahe. & カーリダーサなる詩人を我々は尊敬する。\\
 3. & kanyā Gaṅgāyās tīre 'krī\-ḍan. & 少女等は恒河の岸に於て遊戯した。\\
 4. & gajasya siṃhena saha yud\-dham abhavat. & 獅子と共に象の爭ひがあつた。\\
 5. & tṛṣṇā pathikam abādhata. & 渇が旅人を苦しめた。\\
 6. & putrasya śokād Daśaratho nṛpo jīvitaṃ paryatyajat. & 子を悲しみて十車王は命を捨てた。\\
 7. & śiṣyau gṛhasthasya bhāryāṃ bhikṣāṃ ayācetām. & 二人の弟子は長者の妻に施物を請へり。\\
 8. & Prayāge Gaṅgā Yamunayā saha saṃgacchate. & プラヤーガに於て恒河はヤムナーと出會ふ。\\
 9. & dāsyo 'nnam ānayan. & 婢女等は食物を持ち來つた。\\
10. & saṃkaṭe dhīro dhṛtiṃ na muñcati. & 危難に際し勇者は堅持を捨てない。\\
11. & lajjayā kanyā na pratyabhā\-ṣata. & 羞恥を以て少女は答へなかつた。\\
12. & kīrtiṃ labhante kavayaḥ. & 詩人等は名譽を得る。\\
13. & bubhukṣayā pīḍitaḥ śṛgālo vānān nagaram adhāvat. & 飢に苦しめられ\ruby{野干}{や|かん}は林から町へ走つた。\\
14. & ācāryasya gṛhe Śūdrakeṇa ka\-vinā kṛtāṃ Mṛcchakaṭikām apaṭhāva. & 師の家に於てシュードラカなる詩人によつて作られ
しムリッチュㇵカティカーを我々二人は讀めり。\\
15. & hastena śilām akṣipan naraḥ. & 人は手を以て石を投げた。\\
16. & siṃhaḥ Pāṇineḥ priyān prā\-ṇān aharat. & 獅子はパーニニの愛する生命を奪へり。\\
17. & lubdho naro na visṛjaty & 貪慾の人は貧窮(となる)\\
    & arthaṃ daridratāyāḥ śaṅkayā. & 心配のために富を施さない。\\
18. & bālo vāri pāṇinā kūpād ud\-dharati. & 小兒は手を以て井より水を汲む。\\
19. & lobhena buddhiś calati. & 貪慾によりて思慮は動搖する。\\
20. & makṣikā vraṇam icchanti, & 蠅は瘡傷を欲し,\\
    & dhanam icchanti pārthivāḥ, & 王者は富を欲し,\\
    & nīcāḥ kalaham icchanti, & 下賤なものは爭を欲し,\\
    & śāntim icchanti sādhavaḥ. & 善人は寂靜を欲す。
\end{longtable}

\subsubsection{第四類 ya 級}
\begin{center}可能法\end{center}

\numberParagraph \label{np:69}
第四類の動詞は語根に ya を附加して語基を作る。語勢は
語根にあり。語根 sidh (成就す)~ sidhya.

\numberParagraph
可能法の語尾は第一種變化現在語基の結尾の a と合して
次の如くなる。

爲他 eyam, es, et; eva, etam, etām; ema, eta, eyus.

爲自 eya, ethās, eta; evahi, eyāthām, eyātām; emahi,
edhvam, eran.

\numberParagraph
ya 級の變化。語根 div (博戯する)~ dīvya.

\begin{center}
\begin{tabular}{c*{3}{p{0.15\hsize}}}
  \multicolumn{4}{c}{\textbf{可能法}} \\
  \multicolumn{4}{c}{爲他} \\
     & 單       & 兩       & 複 \\
  1. & dīvyeyam & dīvyeva  & dīvyema \\
  2. & dīvyes   & dīvyetam & dīvyeta \\
  3. & dīvyet   & dīvyetām & dīvyeyus \\
  \multicolumn{4}{c}{爲自} \\
  1. & dīvyeya   & dīvyevahi   & dīvyemahi \\
  2. & dīvyethās & dīvyeyāthām & dīvyedhvam \\
  3. & dīvyeta   & dīvyeyātām  & dīvyeran
\end{tabular}
\end{center}

\numberParagraph
若干の不規則なる語根。
\begin{enumerate}[label=(\alph*)]
\item am に終る語根及び mad (醉ふ)はその a を延長す。dam
(馴らす) dāmyati, śam (靜める) śāmyati, kram (歩む)
krāmyati, bhram (彷徨す)には bhrāmyati と bhramyati
の兩形があり。又第一類の bhramati にも變化せらる。
\item jan (生る)は jāyate に作る。
\end{enumerate}

\ex{第四}
\begin{longtable}{c*{2}{p{0.45\hsize}}}
 1. & sevakaḥ prabhuṃ praṇamati. & 僕は主人に\ruby{敬禮}{けい|れい}する。\\
 2. & satyena jayed anṛtam. & 人は正義を以て虚僞に勝つべきである。\\
 3. & sādhavaḥ pratijñāyā na ca\-lanti kadācana. & 善人は何時も約束から動かない。\\
 4. & śatror api guṇān vaded doṣāṃś ca guror api. & 人は敵に對しても功績を,
 而して師に對しても過失を語るべきだ。\\
 5. & parasya duḥkhena sādhur nityaṃ duḥkhito bhavati. & 善人は他の苦によりて常に
 苦しめられてある。\\
 6. & pratyahaṃ pratyavekṣeta na\-raś caritam. & 日日人は行を檢察すべきだ。\\
 7. & asādhuḥ sādhur vā bhavati khalu jātyaiva puruṣaḥ. & 人は生れながらに惡人若く
 は善人である。\\
 8. & astamite sūrye vihagā vilī\-yante, paṅkajā nimīlanti, māla\-tyaś ca vikasanti. & 日沒する時鳥は隱れ,蓮花
は閉ぢ,素馨は咲く。\\
 9. & janakaḥ prītyā putram āśliṣyati. & 父は喜を以て子を抱擁せり。\\
10. & mitrair jñātibhiś ca parihīṇo daridro drutam anaśyat. & 友と親族に捨てられし貧人は速かに亡びた。\\
11. & vasantasya kāle 'layo bhrām\-yanti mukhena ca madhu pibanti. & 春の時に於て蜜蜂は飛び回
り口を以て蜜を吸ふ。\\
12. & sūtasya daṇḍena praṇuditā aśvā na śrāmyanti. & 御者の笞に勵まされて馬は
疲勞しない。\\
13. & grāmasyārthe kulaṃ tyajet. & 人は村邑のために家族を捨てるべきだ。\\
14. & udyamena hi sidhyanti kār\-yāṇi na manorathaiḥ. & 勤勉によりて事は成就す。
希望によりてに非ず。\\
& na hi suptasya siṃhasya pra\-viśanti mukhe mṛgāḥ. & 蓋し眠れる獅子の口に鹿
は入るものでないから。
\end{longtable}

\subsubsection{第十類 aya 級}
\begin{center}命令\end{center}

\numberParagraph
第十類の構成は語根に aya を附加す。語勢は前の a に
在り。聲中にある i, u, ṛ は單子音の前には重韻化し,聲の終に
在る。i, u, ṛ 又は單子音の中間にある a は複重韻化し,多子音
の前にあると長母音の場合は變化しない。cur (盗む)~ coraya,
cint (考ふ)~ cintaya. この第十類は實際 催起動詞,名稱動詞と
同類である。只現在語基が大體に於て變化しないで用ひられるの
を異とする。

\numberParagraph
命令法の語尾。

爲他 āni, dhi, tu; āva, tam, tām; āma, ta, antu.

爲自 ai, sva, tām: āvahai, āthām, ātām; āmahai,
dhvam, antām.

\numberParagraph
二人稱の dhi は第一種變化に於ては通じて附加しない。
現在語基の結尾 a は antu, antām の前に消滅し。āthām, ātām
の ā と合して e に變ず。

\numberParagraph
aya 級。命令法の語尾。

\begin{center}
\begin{tabular}{c*{3}{p{0.15\hsize}}}
  \multicolumn{4}{c}{爲他} \\
     & 單       & 兩        & 複 \\
  1. & corayāṇi & corayāva  & corayāma \\
  2. & coraya   & corayatam & corayata \\
  3. & corayatu & corayatām & corayantu \\
  \multicolumn{4}{c}{爲自} \\
  1. & corayai   & corayāvahai & corayāmahai \\
  2. & corayasva & corayethām  & corayadhvam \\
  3. & corayatām & corayetām   & corayantām
\end{tabular}
\end{center}

\ex{第五}
\begin{longtable}{c*{2}{p{0.45\hsize}}}
 1. & adhunā muñca śayyām. & 今臥床を離れよ。\\
 2. & vaidyo rogārtasyauṣadhaṃ yacchatu. & 醫師をして病に苦しむもの
 に藥を與へしめよ。\\
 3. & adya pitur gṛham āgaccha\-tam. & 今汝等兩人は父の家に歸れかし。\\
 4. & bhadre, maivaṃ vada. & 善き婦人よ是の如く語る勿れ。\\
 5. & Pāṇiner vyākaraṇasya karur buddhiṃ śaṃsāmaḥ. & 我々は文法作家パーニニの
 聰明を讚む。\\
 6. & bhāryā patyuḥ snihyatu. & 妻をして夫を愛せしめよ。\\
 7. & tyaja nīcānāṃ saṃsargam. & 下劣なるものとの交際を避けよ。\\
 8. & kṣamāṃ bhaja. & 堪忍せよ。\\
 9. & dharme ratir bhavatu. & 法に於て快樂があれかし。\\
10. & āpṛcchasva priyaṃ sakhāyam. & 愛する共に吿別せよ。\\
11. & durjaneṣv api mā pāpaṃ cin\-tayasva kadācana. & 何時にても惡人に對してす
らも惡を考ふること勿れ。\\
12. & bhūpatayaḥ sarvadā prajā dharmeṇa rakṣantu. & 常に王をして人民を法を以て守らしめよ。\\
13. & mitrasya dhanaṃ prayaccha. & 共に財物を與へよ。\\
14. & kumbhakāro daṇḍena ghaṭam akhaṇḍayat. & 甕作りは杖にて甕を粉碎せり。\\
15. & bho sakhe kṣaṇam atra tiṣṭha. & おゝ友よ暫しそこに立て。\\
16. & śokasya hetuṃ vada. & 悲哀の原因を語れ。\\
17. & śiṣyā guroḥ pādau pūjayanti. & 弟子等は師の兩足を尊敬せり。\\
18. & stenā rātrau gṛhaṃ pravi\-śanti janānāṃ ca dhanaṃ cora\-yanti. & 盗人は夜に於て家に入り,
而して人々の富を盗む。\\
19. & kauliko nṛpasya duhitaraṃ paryaṇayat. & 織師は王女に婚せり。\\
20. & muktim icchasi ced viṣam iva viṣayāṃs tyaja. & 若しも汝が解脱を欲するな
らば毒の如くに欲境を捨てよ。
\end{longtable}

\numberParagraph
以上第一種變化の動詞に現在組織の語尾を附加せるもの
を擧げた。勿論この四種の法は各類に應用せらるべきもの。故に
今試みに次に bhū (ある)に就きて全部の語尾表を擧げる。


\begin{center}
\begin{tabular}{c*{3}{p{0.15\hsize}}}
  \multicolumn{4}{c}{\textbf{現實法}} \\
  \multicolumn{4}{c}{爲他} \\
     & 單       & 兩       & 複 \\
  1. & bhavāmi  & bhavāvas  & bhavāmas \\
  2. & bhavasi  & bhavathas & bhavatha \\
  3. & bhavatai & bhavantas & bhavanti \\
  \multicolumn{4}{c}{爲自} \\
  1. & bhave   & bhavāvahe & bhavāmahe \\
  2. & bhavase & bhavethe  & bhavadhve \\
  3. & bhavate & bhavete   & bhavante
\end{tabular}
\end{center}
\begin{center}
\begin{tabular}{c*{3}{p{0.15\hsize}}}
  \multicolumn{4}{c}{\textbf{第一過去}} \\
  \multicolumn{4}{c}{爲他} \\
     & 單       & 兩       & 複 \\
  1. & abhavam & abhavāva  & abavāma \\
  2. & abhavas & abhavatam & abhavata \\
  3. & abhavat & abhavatām & abhavan \\
  \multicolumn{4}{c}{爲自} \\
  1. & abhave     & abhavāvahi & abhavāmahi \\
  2. & abhavathās & abhavethām & abhadhvam \\
  3. & abhavata   & abhavetām  & abhavanta
\end{tabular}
\end{center}
\begin{center}
\begin{tabular}{c*{3}{p{0.15\hsize}}}
  \multicolumn{4}{c}{\textbf{可能法}} \\
  \multicolumn{4}{c}{爲他} \\
     & 單       & 兩       & 複 \\
  1. & bhaveyam & bhaveva  & bhavema \\
  2. & bhaves   & bhavetam & bhaveta \\
  3. & bhavet   & bhavetām & bhaveyus \\
  \multicolumn{4}{c}{爲自} \\
  1. & bhaveya   & bhavevahi   & bhavemahi \\
  2. & bhavethās & bhaveyāthām & bhavedhvam \\
  3. & bhaveta   & bhaveyātām  & bhaveran
\end{tabular}
\end{center}
\begin{center}
\begin{tabular}{c*{3}{p{0.15\hsize}}}
  \multicolumn{4}{c}{\textbf{命令法}} \\
  \multicolumn{4}{c}{爲他} \\
     & 單      & 兩       & 複 \\
  1. & bhavāni & bhavāva  & bhavāma \\
  2. & bhava   & bhavatam & bhavata \\
  3. & bhavatu & bhavatām & bhavantu \\
  \multicolumn{4}{c}{爲自} \\
  1. & bhavai   & bhavāvahai & bhavāmahai \\
  2. & bhavasva & bhavethām  & bhavadhvam \\
  3. & bhavatām & bhavetām   & bhavantām
\end{tabular}
\end{center}

(活用法は次 \ref{np:136} 條に連續す)。

\section{格例法(續き)}
\begin{center}\textbf{子音語基}\end{center}

\numberParagraph \label{np:78}
子音に終る語基は次の如きものである。

\begin{itemize}[label=\hspace{2\zw}]
\item 語根語基
\item as, is, us に終るもの
\item in に終るもの
\item ac に終るもの
\item an, man, van に終るもの
\item at に終るもの
\item mat, vat に終るもの
\item vas に終る過去分詞
\item yas に終る比較級
\end{itemize}

\numberParagraph
語尾は各語基を通じて規則正しく次のものが用ひられる。
この點母音語基の如く不規則ではない。男性,女性は次の如くで
ある。

\begin{center}
\begin{tabular}{c*{3}{p{0.24\hsize}}}
     & 單                  & 兩                     & 複 \\
  主 & s                   & \rdelim\}{2}{*}[au]    & \multirow{2}{*}{as} \\
  業 & am                  &                        & \\
  具 & ā                   & \rdelim\}{3}{*}[bhyām] & bhis \\
  爲 & e                   &                        & \rdelim\}{2}{*}[bhyas] \\
  從 & \rdelim\}{2}{*}[as] &                        & \\
  屬 &                     & \rdelim\}{2}{*}[os]    & ām \\
  於 & i                   &                        & su
\end{tabular}
\end{center}

呼格は兩複に於て必ず主格に同じ。單に於ては大抵語基の形で
あるが,或は主格の形であることもある。中性は主業格が單に於
て語尾を有せず。兩に於て ī, 複に於て i となる外,男女性と同じ
である。

\begin{center}\textbf{(1) 語根語基}\end{center}

\numberParagraph
語根語基及それに準ずるもの。男性,語基 marut (風),
女性 vāc (言語)。

\begin{center}
\begin{tabular}{c*{4}{p{0.15\hsize}}}
     & \multicolumn{2}{c}{單}                           & \multicolumn{2}{c}{複} \\
  主 & \rdelim\}{2}{*}[marut]   & \multirow{2}{*}{vāk (\ref{np:17}條)} & \rdelim\}{3}{*}[marutas]    & \multirow{3}{*}{vācas} \\
  呼 &                          &                                        &                             & \\
  業 & marutam                  & vācam                                  &                             & \\
  具 & marutā                   & vācā                                   & marudbhis (\ref{np:18}條) & vāgbhis \\
  爲 & marute                   & vāce                                   & marudbhyas                  & vāgbhyas \\
  從 & \rdelim\}{2}{*}[marutas] & \multirow{2}{*}{vācas}                 & \rdelim\}{2}{*}[marutām]    & \multirow{2}{*}{vācām}\\
  屬 &                          &                                        &                             & \\
  於 & maruti                   & vāci                                   & marutsu                     & vākṣu \\
     &                          &                                        &                             & (\ref{np:17} 條,\ref{np:39}條)
\end{tabular}
\end{center}

\begin{center}
\begin{tabular}{c*{2}{p{0.24\hsize}}}
     & \multicolumn{2}{c}{兩} \\
  主 & \rdelim\}{3}{*}[marutau]    & \multirow{3}{*}{vācau} \\
  呼 &                             & \\
  業 &                             & \\
  具 & \rdelim\}{3}{*}[marudbhyām] & \multirow{3}{*}{vāgbhyām} \\
  爲 &                             & \\
  從 &                             & \\
  屬 & \rdelim\}{2}{*}[marutos]    & \multirow{2}{*}{vācos} \\
  於 &                             &
\end{tabular}
\end{center}
中性語の主業複は最後の音の次前にその音に相當する鼻音を挿
む。hṛd (心)は hṛndi, asṛj (血)は asṛñji, jagat (世界), jaganti.
これは語基の强弱あるものと見做しても不可でない。

\numberParagraph
\textbf{この種類に屬する若干の語基。}

suhṛd 男(友)。

主 suhṛt, 業 suhṛdam, 具複 suhṛdbhis

dharmabudh 男,女,中(法に明かなる)。

主 dharmabhut, 業 dharmabudham, 具複 dharmabhudbhi (\ref{np:35}條)

vaṇij 男(商人)。

主 vaṇik, 業 vaṇijam, 具複 vaṇigbhis.

diś 女(方)。

主 dik, 業 diśam, 具複 digbhis.

tviṣ 女(光)。

主 tviṭ (\ref{np:17}條),業 tviṣam, 具複 tviḍbhis, 於 tviṭsu 又は
tviṭtsu.

ir, ur に終る語基は單の主呼と並に子音が次に來る場合 i, u を
延長す。

語基 gir 女(歌),主 gīr, 業 giram, 具複 gīrbhis, 於 gīrṣu.

pur 女(城)。主 pūr, 業 puram, 具複 pūrbhis, 於 pūrṣu.

ap 女(水)は常に複數にして,主 āpas, 業 apas, 具 abdhis,
爲從 adbhyas, 屬 apām, 於 apsu である。

puṃs 男(人)は全く不規則で,單 pumān puman, pumāṃsam,
puṃsā puṃse, puṃsas, puṃsi, 兩 pumāṃsau, pumbhyās,
puṃsos, 複 pumāṃsas, puṃsas, pumbhis, pumbhyas, puṃ\-%
sām, puṃsu.

\begin{center}\textbf{(2) as, is, us に終る語基}\end{center}

\numberParagraph
男女性にあつては主單に as の a を延長する。中性では
主呼業の複に a, i, u を延長し,且つ隨韻が挿入せられる。

語基 sumanas 男,女,中(善意ある),cakṣus 中(眼)。

\begin{center}
\begin{tabular}{c*{3}{p{0.2\hsize}}}
  \multicolumn{4}{c}{單} \\
     & 男女                       & \multicolumn{2}{c}{中} \\
  主 & sumanās                    & \rdelim\}{3}{*}[sumanas]   & \multirow{3}{*}{cakṣus} \\
  呼 & sumanas                    &                            & \\
  業 & sumanasam                  &                            & \\
  具 & sumanasā                   & sumanasā                   & cakṣuṣā \\
  爲 & sumanase                   & sumanase                   & cakṣuṣe \\
  從 & \rdelim\}{2}{*}[sumanasas] & \multirow{2}{*}{sumanasas} & \multirow{2}{*}{cakṣuṣas} \\
  屬 &                            &                            & \\
  於 & sumanasi                   & sumanasi                   & cakṣuṣi
\end{tabular}
\end{center}
\begin{center}
\begin{tabular}{c*{3}{p{0.2\hsize}}}
  \multicolumn{4}{c}{兩} \\
     & 男女                                                                                            & \multicolumn{2}{c}{中} \\
  主 & \rdelim\}{3}{*}[sumanasau]                                                                      & \multirow{3}{*}{sumanasī}    & \multirow{3}{*}{cakṣuṣī} \\
  呼 &                                                                                                 &                              & \\
  業 &                                                                                                 &                              & \\
  具 & \rdelim\}{3}{*}[\parbox{8cm-\tabcolsep-\widthof{$\Bigg\}$}}{sumananobhyām \\(\ref{np:24}條)}] & \multirow{3}{*}{sumanobhyām} & \multirow{3}{*}{\parbox{8cm-\tabcolsep-\widthof{$\Bigg\}$}}{cakṣurbhyām \\(\ref{np:23}條)}} \\
  爲 &                                                                                                 &                              & \\
  從 &                                                                                                 &                              & \\
  屬 & \rdelim\}{2}{*}[sumananasos]                                                                    & \multirow{2}{*}{sumanasos}   & \multirow{2}{*}{cakṣuṣos} \\
  於 &                                                                                                 &                              &
\end{tabular}
\end{center}
\begin{center}
\begin{tabular}{c*{3}{p{0.2\hsize}}}
  \multicolumn{4}{c}{複} \\
     & 男女                         & \multicolumn{2}{c}{中} \\
  主 & \rdelim\}{3}{*}[sumanasas]   & \multirow{3}{*}{sumanāṃsi}   & \multirow{3}{*}{cakṣūṃṣi} \\
  呼 &                              &                              & \\
  業 &                              &                              & \\
  具 & sumanobhis                   & sumanobhis                   & cakṣurbhis \\
  爲 & \rdelim\}{2}{*}[sumanobhyas] & \multirow{2}{*}{sumanobhyas} & \multirow{2}{*}{cakṣurbhyas} \\
  從 &                              &                              & \\
  屬 & sumanasām                    & sumanasām                    & cakṣuṣām \\
  於 & sumanaḥsu (\ref{np:22}條)  & sumanaḥsu                    & cakṣuḥṣu (\ref{np:39}條)
\end{tabular}
\end{center}

\begin{center}\textbf{(3) in に終る語基}\end{center}

\numberParagraph
in に終る語基は所有を表はすもので,多くは形容詞であり,
男性中性に變化する。子音で始まる語尾の前に n は消失する。
又同樣に單主,及び中單主業も然り。中單呼にあつては消失せざ
ることもあり,この i は男單主,並に中複主業にあつては延長せ
らる。

語基 balin (力ある)。

\begin{center}
\begin{tabular}{c*{6}{p{0.12\hsize}}}
     & \multicolumn{2}{c}{單}     & \multicolumn{2}{c}{兩}     & \multicolumn{2}{c}{複} \\
     & 男      & 中                                 & 男                       & 中                      & 男                       & 中 \\
  主 & balī    & bali                               & \rdelim\}{3}{*}[balinau] & \multirow{3}{*}{balinī} & \multirow{3}{*}{balinas} & \multirow{3}{*}{balīni} \\
  呼 & balin   & balin, bali                        &                                                    & \\
  業 & balinam & bali                               &                                                    & \\
     & \multicolumn{2}{c}{\upbracefill}             &                          &                         & \multicolumn{2}{c}{\upbracefill} \\
  具 & \multicolumn{2}{c}{balinā}                   & \multicolumn{2}{l}{\rdelim\}{3}{*}[balibhyām]}     & \multicolumn{2}{c}{balibhis} \\
  爲 & \multicolumn{2}{c}{baline}                   &                          &                         & \multicolumn{2}{l}{\rdelim\}{2}{*}[balibhyas]} \\
  從 & \multicolumn{2}{l}{\rdelim\}{2}{*}[balinas]} &                          &                         & \\
  屬 &         &                                    & \multicolumn{2}{l}{\rdelim\}{2}{*}[balinos]}       & \multicolumn{2}{c}{balinām} \\
  於 & \multicolumn{2}{c}{balini}                   &                          &                         & \multicolumn{2}{c}{baliṣu}
\end{tabular}
\end{center}

この語基の女性形容詞は語基に ī を附加して作り nadī (\ref{np:52}條)
に準じて變化せらる。

\ex{第六}
\begin{longtable}{c*{2}{p{0.45\hsize}}}
 1. & sarvaḥ padasthasya suhṛd bandhur āpadi durlabhaḥ. & 顯要の位置にあるものは一
切が友であり不幸に於ては親族が得難い。\\
 2. & yathaā cittaṃ tathā vāco yathā vācas tathā kriyāḥ. & 心の如く此の如く語あり,
語の如く此の如く行爲あり。\\
 3. & durgrāhyḥ pāṇinā vāyur duḥsparśaḥ pāṇinā śikhī. & 風は手を以て捉へ難く,火は手を以て觸れ難い。\\
 4. & kṣamā rūpaṃ tapasvinaḥ. & 忍耐は苦行者の美貌である。\\
 5. & niyato dehināṃ mṛtyur ani\-tyaṃ khalu jīvitam. & 人にとつて死は定まつてゐるが命は不定である。\\
 6. & namanti phalino vṛkṣāḥ na\-manti guṇino janāḥ. & 果實ある樹は曲り德ある人も曲る。\\
 7. & śrutiḥ smṛtiś ca dvijānāṃ cakṣuṣī. & 吠陀と法典(天啓と傳説)とは婆羅門の兩眼である。\\
 8. & vaidyo na prabhur āyuṣaḥ. & 醫は壽命の主でない。\\
 9. & snānāya sarasas tīraṃ sa gacchati. & 沐浴のために湖の岸へ彼は行く。\\
10. & āyur eva paraṃ nidhānam. &  壽命こそ最高の寶なれ。\\
11. & saṃpadas tasya yasya saṃ\-tuṣṭaṃ mānasam. & 滿足せる心ある人には幸福がある。\\
12. & āpadas tasya yasya vittaṃ na vidyate. & 財物なき人には不幸がある。\\
13. & ākiṃcanyaṃ nidhānaṃ vi\-duṣām. & 無一物の狀態は學者の寶である。\\
14. & bhāryāyāḥ sundaraḥ snigdho veśyāyāḥ sundaro dhanī, Śrī\-devyāḥ sundaraḥ śūro Bhāra\-tyāḥ sundaraḥ sundhī. & 妻の好むは情ある人,娼婦の好むは富人,吉祥天の
好むは勇士,辯才天の好むは賢人である。\endnote{底本では「好むは賢人」ではなく「好むはは賢人」。}
\end{longtable}

\numberParagraph
語基には强弱の二語基の場合或は强弱中の三語基の場合
がある。强弱と云ふのは語勢等の關係で音量の多いのが强と名け
られ,少いのが弱と名けられる。中はその中間のものである。而
してこの强弱中の語基は使用せらるゝ場所も一定してゐて決して
混亂はない。卽ち男性,女性にして二語基の場合は主呼業の單兩
及び主呼の複に强語基その他には弱語基を用ふ。三語基の場合は
前の弱語基を用ふる場所の中,大體子音にて始まる語尾の前には
中語基母音にて始まる語尾の前には弱語基が用ひられる。中性で
は複の主業が强語基となる。三語基の場合には單の主業呼が中語
基となる。兩の主業は常に弱語基である。其の餘は男女性に同
じ。

\begin{enumerate}[label=(\alph*)]
\item 二語基の例:强 tudant, 弱 tudat (打ちつゝ)。
\item 三語基の例:强 vidvāṃs, 中 vidvat, 弱 vidus (賢者)。
\end{enumerate}

\begin{center}\textbf{(4) ac に終る語基}\end{center}

\numberParagraph
ac に終る語基は形容詞であつて一部分二語基,一部分は
三語基を有す。卽ち二語基 prāc (東方の)强基 prāñc, 弱基
prāc であり,三語基は pratyac (西方の)强基 pratyañc, 中基
pratyac, 弱基 pratīc.

\begin{center}
\begin{tabular}{c*{4}{p{0.15\hsize}}}
     & \multicolumn{4}{c}{單} \\
     & \cellAlign{c}{男}                       & \cellAlign{c}{中}     & \cellAlign{c}{男}           & \cellAlign{c}{中} \\
  主 & \rdelim\}{2}{*}[prāṅ (\ref{np:17}條)] & \multirow{2}{*}{prāk} & \multirow{2}{*}{pratyaṅ}    & \multirow{2}{*}{pratyak} \\
  呼 &                                         &                       &                             & \\
  業 & prāñcam                                 & prāk                  & pratyañcam                  & pratyak \\
     & \multicolumn{2}{c}{\upbracefill}                                & \multicolumn{2}{c}{\upbracefill} \\
  具 & \multicolumn{2}{c}{prācā}                                       & \multicolumn{2}{c}{pratīcā} \\
  爲 & \multicolumn{2}{c}{prāce}                                       & \multicolumn{2}{c}{pratīce} \\
  從 & \multicolumn{2}{l}{\rdelim\}{2}{*}[prācas]}                     & \multicolumn{2}{c}{\multirow{2}{*}{pratīcas}} \\
  屬 &                                                                 & \\
  於 & \multicolumn{2}{c}{prāci}                                       & \multicolumn{2}{c}{pratīci}
\end{tabular}
\end{center}

\begin{center}
\begin{tabular}{c*{4}{p{0.15\hsize}}}
     & \multicolumn{4}{c}{兩} \\
     & \cellAlign{c}{男}        & \cellAlign{c}{中}      & \cellAlign{c}{男}           & \cellAlign{c}{中} \\
  主 & \rdelim\}{3}{*}[prāñcau] & \multirow{3}{*}{prācī} & \multirow{3}{*}{pratyañcau} & \multirow{3}{*}{pratīcī} \\
  呼 &                          &                        &                             & \\
  業 &                          &                        &                             & \\
     & \multicolumn{2}{c}{\upbracefill}                  & \multicolumn{2}{c}{\upbracefill} \\
  具 & \multicolumn{2}{l}{\rdelim\}{3}{*}[prāgbhyām]}    & \multicolumn{2}{c}{\multirow{3}{*}{pratyagbhyām}} \\
  爲 &                                                   & \\
  從 &                                                   & \\
  屬 & \multicolumn{2}{l}{\rdelim\}{2}{*}{prācos}}       & \multicolumn{2}{c}{\multirow{2}{*}{pratīcos}} \\
  於 &                                                   &
\end{tabular}
\end{center}

\begin{center}
\begin{tabular}{c*{4}{p{0.15\hsize}}}
     & \multicolumn{4}{c}{複} \\
     & \cellAlign{c}{男}        & \cellAlign{c}{中}       & \cellAlign{c}{男}           & \cellAlign{c}{中} \\
  主 & \rdelim\}{2}{*}[prāñcas] & \multirow{2}{*}{prāñci} & \multirow{2}{*}{pratyañcas} & \multirow{2}{*}{pratyañci} \\
  呼 &                          &                         &                             & \\
  業 & prācas                   & prāñci                  & pratīcas                    & pratyañci \\
     & \multicolumn{2}{c}{\upbracefill}                   & \multicolumn{2}{c}{\upbracefill} \\
  具 & \multicolumn{2}{c}{prāgbhis}                       & \multicolumn{2}{c}{pratyagbhis} \\
  爲 & \multicolumn{2}{l}{\rdelim\}{2}{*}[prāgbhyas]}     & \multicolumn{2}{c}{pratyagbhyas} \\
  從 &                                                    & \\
  屬 & \multicolumn{2}{c}{prācām}                         & \multicolumn{2}{c}{pratīcām} \\
  於 & \multicolumn{2}{c}{prākṣu}                         & \multicolumn{2}{c}{pratyakṣu}
\end{tabular}
\end{center}

女性語基は弱語基に ī を附加して作られる。卽ち prācī,
pratīcī であつて nadī (\ref{np:52}條)に準じて變化せられる。

\begin{center}\textbf{(5) an, man, van に終る語基}\end{center}

\numberParagraph
これは皆三語基を有す。强語基では後接字の a は延長せ
られ,中語基では n が省かれ,弱語基では a が除去せらる。
man, van の語基では弱語基にても m 又は v の前に子音が來る
場合 a は保存せらる。單主,男は ā, 中は a にて終る。

語基 rājan 男(王),强語基 rājān, 中語基 rāja, 弱語基 rājñ.
adhvan 男(路),强語基 adhvān, 中語基 adhva, 弱語基 adhvan.
語基 karman 中(業)同樣。

\begin{center}
\begin{tabular}{c*{3}{p{0.15\hsize}}}
     & \multicolumn{3}{c}{單} \\
  主 & rājā                    & adhvā                     & karma \\
  呼 & rājan                   & adhvan                    & karman, karma \\
  業 & rājānam                 & adhvānam                  & karma \\
  具 & rājñā                   & adhvanā                   & karmaṇā \\
  爲 & rājñe                   & adhvane                   & karmaṇe \\
  從 & \rdelim\}{2}{*}[rājñas] & \multirow{2}{*}{adhvanas} & \multirow{2}{*}{karmaṇas} \\
  屬 &                         &                           & \\
  於 & rājñi                   & adhvani                   & karmaṇi
\end{tabular}
\end{center}

\begin{center}
\begin{tabular}{c*{3}{p{0.15\hsize}}}
     & \multicolumn{3}{c}{兩} \\
  主 & \rdelim\}{3}{*}[rājānau]   & \multirow{3}{*}{adhvānau}   & \multirow{3}{*}{karmaṇī} \\
  呼 &                            &                             & \\
  業 &                            &                             & \\
  具 & \rdelim\}{3}{*}[rājabhyām] & \multirow{3}{*}{adhvabhyām} & karmabhyām \\
  爲 &                            &                             & \\
  從 &                            &                             & \\
  屬 & \rdelim\}{2}{*}[rājños]    & \multirow{2}{*}{adhvanos}   & \multirow{2}{*}{karmaṇos} \\
  於 &                            &                             &
\end{tabular}
\end{center}

\begin{center}
\begin{tabular}{c*{3}{p{0.15\hsize}}}
     & \multicolumn{3}{c}{複} \\
  主 & \rdelim\}{2}{*}[rājānas]   & \multirow{2}{*}{adhvānas}   & \rdelim\}{3}{*}[karmāṇi] \\
  呼 &                            &                             & \\
  業 & rājñas                     & adhvanas                    & \\
  具 & rājabhis                   & adhvabhis                   & karmabhis \\
  爲 & \rdelim\}{2}{*}[rājabhyas] & \multirow{2}{*}{adhvabhyas} & \multirow{2}{*}{karmabhyas} \\
  從 &                            &                             & \\
  屬 & rājñām                     & adhvanām                    & karmaṇām \\
  於 & rājasu                     & adhvasu                     & karmasu
\end{tabular}
\end{center}

\nt{三性を通じ單於,及び中性兩の主業には a を保存するを得。卽
ち rājñi と並んで rājani, nāmnī 中(名)と並んで nāmanī.}

\numberParagraph \textbf{若干の不規則なる語基。}
\begin{enumerate}[label=(\alph*)]
\item śvan 男(犬)及び yuvan (若き)は śun yūn なる弱
  語基を有す。
  \begin{description}[font=\normalfont]
  \item[單] śvā, śvānam, śunā, śune 等。
  \item[兩] śvānau, śvabhyām, śunos.
  \item[複] śvānas, śunas, śvabhis, śvabhyas, śunām, śvasu.
  \end{description}
\item panthan 男(路)は强語基 panthān, 中語基 pathi,
  弱語基 path.
  \begin{description}[font=\normalfont]
  \item[單] panthās, panthānam, pathā, pathe, pathas, pathi.
  \item[兩] panthānau, pathibhyām, pathos.
  \item[複] panthānas, pathas, pathibhis, pathibhyas, pathām, pathiṣu.
  \end{description}
\item athan 中(日)は中語基 ahar 若くは ahas.
  \begin{description}[font=\normalfont]
  \item[單] 主呼業 ahar, 具 ahnā 等。
  \item[兩] ahni, ahobhyām, ahnos.
  \item[複] ahāni, 具 ahobhis 等。
  \end{description}
\item 中性 akṣan (眼),asthan (骨)は弱語基の格のみを有
  す。akṣṇā, akṣṇe, akṣṇas, akṣṇi 等,其の餘の格は akṣi, asthi
  の如く i 語基に準じて變化す。\ref{np:49}條\ref{item:49c}参照。
\end{enumerate}

\begin{center}\textbf{(6) at に終る語基}\end{center}

\numberParagraph
これらの語基は殆んどすべて現在又は未來分詞である。
而して强語基は ant 弱語基は at である。

語基 tudat (打ちつゝ),强 tudant, 弱 tudat.
\begin{center}
\begin{tabular}{c*{4}{p{0.15\hsize}}}
     & \multicolumn{2}{c}{單}                            & \multicolumn{2}{c}{複} \\
     & \cellAlign{c}{男}        & \cellAlign{c}{中}      & \cellAlign{c}{男}             & \cellAlign{c}{中} \\
  主 & \rdelim\}{2}{*}[tudan]   & \rdelim\}{3}{*}[tudat] & \multirow{2}{*}{tudantas}     & \multirow{3}{*}{tudanti} \\
  呼 &                          &                        &                               & \\
  業 & tudantam                 &                        & tudatas                       & \\
     & \multicolumn{2}{c}{\upbracefill}                  & \multicolumn{2}{c}{\upbracefill} \\
  具 & \multicolumn{2}{c}{tudatā}                        & \multicolumn{2}{c}{tudadbhis} \\
  爲 & \multicolumn{2}{c}{tudate}                        & \multicolumn{2}{c}{tudadbhyas} \\
  從 & \multicolumn{2}{c}{\rdelim\}{2}{*}[tudatas]}      & \multicolumn{2}{c}{\multirow{2}{*}{tudatām}} \\
  屬 &                                                   & \\
  於 & \multicolumn{2}{c}{tudati}                        & \multicolumn{2}{c}{tudatsu}
\end{tabular}
\end{center}

\begin{center}
\begin{tabular}{c*{2}{p{0.24\hsize}}}
     & \multicolumn{2}{c}{兩} \\
     & \cellAlign{c}{男}           & \cellAlign{c}{中} \\
  主 & \rdelim\}{3}{*}[tudantau]   & \multirow{3}{*}{tudantī, tudatī} \\
  呼 &                             & \\
  業 &                             & \\
     & \multicolumn{2}{c}{\upbracefill} \\
  具 & \multicolumn{2}{c}{\rdelim\}{3}{*}[tudadbhyām]} \\
  爲 &                             & \\
  從 &                             & \\
  屬 & \multicolumn{2}{c}{\rdelim\}{2}{*}[tudatos]} \\
  於 &                             &
\end{tabular}
\end{center}

\numberParagraph
これらの語基の女性は\ref{np:90}條に準じ弱基又は强基に ī を附
加して作る。例:tudatī 又は tudantī.

\numberParagraph \label{np:90}
中性の兩主業に於て並に女性の構成に於て强基弱基の孰
れを取るかに關しては次の規定がある。
\begin{enumerate}[label=(\alph*)]
\item a 級 ya 級 aya 級並びに派生動詞に關しては强形(ant)
  が作られねばならぬ。卽ち bhavat, 女性 bhavantī.
\item á 級,語根級の ā に終る語根例へば yā, 並に爲他未
  來分詞に關しては强弱兩形が作られる。卽ち tudat, 女性
  tudatī 又は tudantī.
\item 餘の級の語根からは必ず弱形(at)が作られねばなら
  ぬ。卽ち kurvat, 女性 kurvatī.
\end{enumerate}

\numberParagraph
重複語根(\ref{np:145}條)は弱語基で總ての格を構成する。卽ち
datat (與へつゝ)男主 dadat, 業 dadatām 等。只中性複の主
業には兩形並び行はれる。

\numberParagraph
mahat (大なる)は强基 ānt である。
\begin{description}[font=\normalfont]
\item[男,單] mahān, mahāntam, mahatā 等。
\item[兩] mahāntau.
\item[複] mahāntas, mahatas, mahadbhis 等。
\item[中] mahat, mahatī, mahānti.
\end{description}

\begin{center}\textbf{(7) mat 及び vat に終る語基}\end{center}

\numberParagraph
mat, vat に終る所有を表はす形容詞は at に終る分詞と
同樣に變化する。但し主單男は mān 及び vān となる。卽ち
balavat (强き),主 balavān.

\begin{center}\textbf{(8) vas に終る過去能動分詞}\end{center}

\numberParagraph
過去能動分詞(\ref{np:177}條)は三語基を有す。强基は vāṃs,
中基 vat, 弱基は us である。主單は vān, 呼單は van.

語基 vidvas (知れる),强形 vidvāṃs, 中形 vidvat, 弱形
vidus.
\begin{center}
\begin{tabular}{c*{4}{p{0.15\hsize}}}
     & \multicolumn{2}{c}{單}                       & \multicolumn{2}{c}{複} \\
     & \cellAlign{c}{男} & \cellAlign{c}{中}        & \cellAlign{c}{男}          & \cellAlign{c}{中} \\
  主 & vidvān            & \rdelim\}{3}{*}[vidvat]  & \rdelim\}{2}{*}[vidvāṃsas] & \rdelim\}{3}{*}[vidvāṃsi] \\
  呼 & vidvan            &                          &                            & \\
  業 & vidvāṃsam         &                          & viduṣas                    & \\
     & \multicolumn{2}{c}{\upbracefill}             & \multicolumn{2}{c}{\upbracefill} \\
  具 & \multicolumn{2}{c}{viduṣā}                   & \multicolumn{2}{c}{vidvadbhis} \\
  爲 & \multicolumn{2}{c}{viduṣe}                   & \multicolumn{2}{c}{vidvadbhyas} \\
  從 & \multicolumn{2}{c}{\rdelim\}{2}{*}[viduṣas]} & \multicolumn{2}{c}{\multirow{2}{*}{viduṣām}} \\
  屬 &                                              & \\
  於 & \multicolumn{2}{c}{viduṣi}                   & \multicolumn{2}{c}{vidvatsu}
\end{tabular}
\end{center}

\begin{center}
\begin{tabular}{c*{2}{p{0.24\hsize}}}
     & \multicolumn{2}{c}{兩} \\
     & \cellAlign{c}{男}          & \cellAlign{c}{中} \\
  主 & \rdelim\}{3}{*}[vidvāṃsau] & \multirow{3}{*}{tudantī, tudatī} \\
  呼 &                            & \\
  業 &                            & \\
     & \multicolumn{2}{c}{\upbracefill} \\
  具 & \multicolumn{2}{c}{\rdelim\}{3}{*}[vidvadbhyām]} \\
  爲 &                            & \\
  從 &                            & \\
  屬 & \multicolumn{2}{c}{\rdelim\}{2}{*}[viduṣos]} \\
  於 &                            &
\end{tabular}
\end{center}

女性語基は弱基に ī を附加して作る viduṣī.

\begin{center}\textbf{(9) yas に終る比較級形容詞}\end{center}

\numberParagraph
yas に終る比較級(\ref{np:98}條)は二語基を有す卽ち强形 yāṃs
及び弱形 yas.

語基 śreyas (よりよき),强形 śreyāṃs. 弱形 śreyas.
\begin{center}
\begin{tabular}{c*{4}{p{0.15\hsize}}}
     & \multicolumn{2}{c}{單}                        & \multicolumn{2}{c}{複} \\
     & \cellAlign{c}{男} & \cellAlign{c}{中}         & \cellAlign{c}{男}          & \cellAlign{c}{中} \\
  主 & śreyān            & \rdelim\}{3}{*}[śreyas]   & \rdelim\}{2}{*}[śreyāṃsas] & \rdelim\}{3}{*}[śreyāṃsi] \\
  呼 & śreyan            &                           &                            & \\
  業 & śreyāṃsam         &                           & śreyasas                   & \\
     & \multicolumn{2}{c}{\upbracefill}              & \multicolumn{2}{c}{\upbracefill} \\
  具 & \multicolumn{2}{c}{śreyasā}                   & \multicolumn{2}{c}{śreyobhis} \\
  爲 & \multicolumn{2}{c}{śreyase}                   & \multicolumn{2}{c}{\rdelim\}{2}{*}[śreyobhyaas]} \\
  從 & \multicolumn{2}{c}{\rdelim\}{2}{*}[śreyasas]} & \\
  屬 &                   &                           & \multicolumn{2}{c}{śreyasām} \\
  於 & \multicolumn{2}{c}{śreyasi}                   & \multicolumn{2}{c}{śreyaḥsu}
\end{tabular}
\end{center}

\begin{center}
\begin{tabular}{c*{2}{p{0.24\hsize}}}
     & \multicolumn{2}{c}{兩} \\
     & \cellAlign{c}{男}          & \cellAlign{c}{中} \\
  主 & \rdelim\}{3}{*}[śreyāṃsau] & \multirow{3}{*}{śreyasī} \\
  呼 &                            & \\
  業 &                            & \\
     & \multicolumn{2}{c}{\upbracefill} \\
  具 & \multicolumn{2}{c}{\rdelim\}{3}{*}[śreyobhyām]} \\
  爲 &                            & \\
  從 &                            & \\
  屬 & \multicolumn{2}{c}{\rdelim\}{2}{*}[śreyasos]} \\
  於 &                            &
\end{tabular}
\end{center}

女性語基は弱基に ī を附加して作る。卽ち śreyas, 女性
śreyasī; garīyas (より重き),女性 garīyasī.

\ex{第七}
\begin{longtable}{c*{2}{p{0.45\hsize}}}
 1. & rājño gṛhe mahān utsavo 'bhavat. & 王の家に於て大なる祭があつた。\\
 2. & vāyur ambhasi nāvaṃ ha\-rati. & 風は水に於て舟を運び行く。\\
 3. & vaṇijaḥ sutā sakhībhiḥ sa\-hārāme krīḍanāya gacchati sma & 商人の娘は友達と共に國へ遊戯のために行けり。\\
 4. & mahati vipadaḥ sāgare pra\-mādena nāpito 'patat. & 理髪師は不注意によつて不幸の大海に落ちた。\\
 5. & vṛddho dhanī vaṇig yuvatiṃ nirdhanasya duhitaraṃ parya\-ṇayat. & 老いたる富める商人は若き貧人の娘と結婚せり。\\
 6. & asamarthānāṃ puṃsāṃ kopa ātmana upadravāya bhavati. & 無能力なる人々の怒は自らの危難を招く。\\
 7. & ravir niśāyās tamo 'paharati. & 太陽は夜の暗を拂ひ去る。\\
 8. & sūryasya tejasā saṃtaptaḥ pānthaś chāyām āśrayate. & 日の熱に熱せられ旅人は蔭に近寄る。\\
 9. & medhāvī śuddhaṃ jīvitam ācaret. & 知者は淨き生を送るべきだ。\\
10. & śuṣkavat saras tyajanti sā\-rasāḥ. & 鶴は涸れたる池を去る。\\
11. & balavatī dantānāṃ vedanā brāhmaṇaṃ bādhate. & 太だしい齒痛は婆羅門を苦しめる。\\
12. & hariṇo lubdhakasya śarāṇāṃ prahārād ubbhāritaḥ kṛcchreṇa saraḥ praviṣṭaḥ. & 鹿は獵師の箭の打擊から逃
れてやつとのことで池に入つた。\\
13. & munis tapasi rato vane tiṣṭhati. & 聖者は苦行を樂みて林に住す。\\
14. & kāvyānāṃ śāstrāṇāṃ ca vino\-dena kālo gacchati dhīmatām. & 詩書の樂に賢者の時は過ぎ行く。\\
15. & śunaḥ puccham iva vyarthaṃ jīvitaṃ vidyayā vinā. & 知識なくしては生は犬の尾の如く價値無し。\\
16. & śriyā striyo haranti puṃsāṃ manāṃsi ca cakṣūṃsi ca. & 婦女は美によりて人々の心と目とを奪ふ。\\
17. & balaṃ vidyā ca viprāṇāṃ rājñāṃ sainyaṃ balaṃ tathā. &知識は婆羅門の力,此の如く王者の力は軍隊。\\
& balaṃ vittaṃ ca vaiśyānāṃ śūdrāṇāṃ ca kaniṣṭhatā. & 商人にとつては財は力,シュードラにとつては下賤な
ることが(力である)。
\end{longtable}

\section{比較法}
\numberParagraph
形容詞の比較級は男性の語基に tara を,最上級は tama
を附加して作る。變化は a 語基に準ず。女性は終の母音を ā に
作り kanyā の變化に準ず。priya (愛する)~ priyatara (より
愛する),priyatama (最も愛する)。

\numberParagraph
二語基ある場合は弱基に,三語基ある場合は中基に附加せ
らる。例:prāc (東方の)~ prāktara, prāktama.

\numberParagraph \label{np:98}
他の方法は後接字 īyas, iṣṭha を語基に附加して作る。
語基の母音は重韻化又は延長によりて强められる。kṣipra (速か
なる)は語根 kṣip から比較級 kṣepīyas, 最上級 kṣepiṣṭha が
作られ,mṛdu (軟かき)は mradīyas, mradiṣṭha; dūra (遠き)
は davīyas, daviṣṭha.

時には原級の形が比較級最上級の形と異るものがある。例せば
śreyas, śreṣṭha の原級は śrī にあらずして praśasya (勝れた
る)であり,kanīyas, kaniṣṭha の原級は alpa (小なる)である。

\section{代名詞}
\subsection{人稱代名詞}
\numberParagraph
一人稱代名詞(吾)語基は 單 mad, 複 asmad.

\begin{center}
\begin{tabular}{c*{3}{p{0.2\hsize}}}
     & 單                    & 兩                        & 複 \\
  主 & \rdelim\}{2}{*}[aham] & \rdelim\}{3}{*}[āvām]     & \rdelim\}{2}{*}[vayam] \\
  呼 &                       &                           & \\
  業 & mām, mā               &                           & asmān, nas \\
  具 & mayā                  & \rdelim\}{3}{*}[āvābhyām] & asmābhis \\
  爲 & mahyam, me            &                           & asmabhyam, nas \\
  從 & mat                   &                           & asmat \\
  屬 & mama, me              & \rdelim\}{2}{*}[āvayos]   & asmākam, nas \\
  於 & mayi                  &                           & asmāsu
\end{tabular}
\end{center}

兩數の業爲屬に nau を用ふることがある。

\numberParagraph
二人稱代名詞(汝)語基は 單 tvad, 複 yuṣmad.

\begin{center}
\begin{tabular}{c*{3}{p{0.2\hsize}}}
     & 單                    & 兩                         & 複 \\
  主 & \rdelim\}{2}{*}[tvam] & \rdelim\}{3}{*}[yuvām]     & \rdelim\}{2}{*}[yūyam] \\
  呼 &                       &                            & \\
  業 & tvām, tvā             &                            & yuṣmān, vas \\
  具 & tvayā                 & \rdelim\}{3}{*}[yuvābhyām] & yuṣmābhis \\
  爲 & tubhyam, te           &                            & yuṣmabhyam, vas \\
  從 & tvat                  &                            & yuṣmat \\
  屬 & tava, te              & \rdelim\}{2}{*}[yuvayos]   & yuṣmākam, vas \\
  於 & tvayi                 &                            & yuṣmāsu
\end{tabular}
\end{center}

兩數の業爲屬に vām を用ふることがある。

\subsection{指示代名詞}
\numberParagraph
語基 tad. これは同時に三人稱代名詞(彼,夫)でもあり得
る,\endnote{底本では女性複数具格(tābhis)が空欄。}

\begin{center}
\begin{tabular}{c*{9}{p{0.085\hsize}}}
     & \multicolumn{3}{c}{單}                                                                                & \multicolumn{3}{c}{兩}                                           & \multicolumn{3}{c}{複} \\
     & \multicolumn{3}{c}{\downbracefill}                                                                    & \multicolumn{3}{c}{\downbracefill}                               & \multicolumn{3}{c}{\downbracefill} \\
     & 男                                                   & 中                   & 女                      & 男                   & 中                  & 女                  & 男                  & 中                     & 女 \\
  主 & \rdelim\}{2}{*}[sas\footnote{\ref{np:24}條註参照。}] & \rdelim\}{3}{*}[tat] & \rdelim\}{2}{*}[sā]     & \rdelim\}{3}{*}[tau] & \multirow{3}{*}{te} & \multirow{3}{*}{te} & \rdelim\}{2}{*}[te] & \rdelim\}{3}{*}[tāni]  & \multirow{3}{*}{tās} \\
  呼 &                                                      &                      &                         &                      &                     &                     &                     &                        & \\
  業 & tam                                                  &                      & tām                     &                      &                     &                     & tān                 &                        & \\
     & \multicolumn{2}{c}{\upbracefill}                                            &                         & \multicolumn{3}{c}{\upbracefill}                                 & \multicolumn{2}{c}{\upbracefill}             & \\
  具 & \multicolumn{2}{c}{tena}                                                    & tayā                    & \multicolumn{3}{c}{\rdelim\}{3}{*}[tābhyām]}                     & \multicolumn{2}{c}{tais}                     & tābhis \\
  爲 & \multicolumn{2}{c}{tasmai}                                                  & tasyai                  &                      &                     &                     & \multicolumn{2}{c}{\rdelim\}{2}{*}[tebhyas]} & \multirow{2}{*}{tābhyas} \\
  從 & \multicolumn{2}{c}{tasmāt}                                                  & \rdelim\}{2}{*}[tasyās] &                      &                     &                     &                   &                          & \\
  屬 & \multicolumn{2}{c}{tasya}                                                   &                         & \multicolumn{3}{c}{\rdelim\}{2}{*}[tayos]}                       & \multicolumn{2}{c}{teṣām}                    & tāsām \\
  於 & \multicolumn{2}{c}{tasmin}                                                  & tasyām                  &                      &                     &                     & \multicolumn{2}{c}{teṣu}                     & tāsu
\end{tabular}
\end{center}

\numberParagraph
語基 etad (此れ)の變化は tad に準ず。

主,單 eṣas, eṣā, etad.

\numberParagraph
語基 enad (彼)は只業の三數と具の單,屬於の兩がある
のみである。卽ち:

\begin{center}
\begin{tabular}{c*{3}{p{0.15\hsize}}}
     & \multicolumn{3}{c}{單} \\
     & 男   & 中                        & 女 \\
  業 & enam & enat                      & enām \\
     & \multicolumn{2}{c}{\upbracefill} & \\
  具 & \multicolumn{2}{c}{enena}        & enayā
\end{tabular}
\end{center}
\begin{center}
\begin{tabular}{c*{3}{p{0.15\hsize}}}
     & \multicolumn{3}{c}{兩} \\
  業 & enau & ene                      & ene \\
     & \multicolumn{3}{c}{\upbracefill} \\
  屬 & \multicolumn{3}{c}{\rdelim\}{2}{*}[enayos]} \\
  於 & \\
\end{tabular}
\end{center}
\begin{center}
\begin{tabular}{c*{3}{p{0.15\hsize}}}
     & \multicolumn{3}{c}{複} \\
  業 & enān & enāni & enās
\end{tabular}
\end{center}

\newpage
\theendnotes


%%% Local Variables:
%%% mode: latex
%%% TeX-master: "IntroductionToSanskrit"
%%% End:
