\chapter{語尾曲法}\label{cha:flection}
\section{格例法}
\numberParagraph
言語の各類に就て變化するものと變化しないものとがあ
る。變化しないものとは接續詞,間投詞,副詞の如きである。今
語尾曲法に於て取扱はうとするものはこれら変化せざるものを暫
らく除外して變化するものに就て論ずる。變化する詞にも二樣の
種類が分れる。一は卽ち名詞的變化であつて性と數と格によつて
語尾を異にするもの。この種に屬するものは名詞,代名詞,形容
詞,數詞,分詞である。他は卽ち動詞的變化であつて人稱と數と
時と法とに隨つて語尾を異にする。前者の取扱ひを格例法と云
ひ,後者の取扱ひを活用法と云ふ。

\numberParagraph
梵語では格例法に三性 三數 八格を認める。三性とは男
女 中であり,三數とは單 兩 複であり,八格とは主 業 具 爲
從 屬 於 呼である。活用法には三の人稱と 三數と 三時と 四
法を認める。曰く 一人稱 二人稱 三人稱,曰く單 兩 複,曰く
現在 過去 未來,曰く現實法,可能法,命令法,條件法である。
これらの意義は後節に説く。

\numberParagraph
梵語では名詞的變化は語基を以て,動詞的變化は語根を以
て單位とする。單位とは分解の終極を意味し,それ以上溯る必要
を見ない形である。この形で辭典に記録されるものと知るべきで
ある。嚴密に云はば如何なる詞もその語根まで溯るべきである
が,名詞,代名詞,形容詞,數詞,分詞は必ずしも語根の詮索の
手を延すを要しない。出來上つた語基の形に語尾が附けられる。
然し動詞は語根によりて論ぜられる。之を要するに語根は梵語の
あらゆる語彙を分析してこれを原始の狀態に還元したものであ
る。印度平原に花咲く言語から得た收穫である。萬を以て數へら
れる言語が數百の語根に還元せらるゝことは實に言語學の寄與貢
献の偉大を示すものである。

\numberParagraph
語根は多くは單綴の音であり,意義はあるも未だ言語では
ない。云はば言語のエキスである。今語根に若干の変化を施して
語基を作り,これに語尾を附加すれば始めて活用する言語とな
る。或は之を逆に考へて活用する言語を比較し分析して語尾を切
離し,語基を整理して最後の形を得たとすればそれが語根であ
る。

\numberParagraph
\textbf{名詞的變化}に就て\textbf{語基が母音}で終るものと\textbf{子音で終るも
  の}との二類を分つ。前者を母音語基,後者を子音語基と呼ぶ。母
音語基は語尾が附加せられて太だ不規則なる變化をなすから,稍
複雜ではあるが,語基に强弱を認むる必要あまりなきと,數に於
いて飛び離れて多いとの理由から,これを最初に取扱ふことゝす
る。先づ最初に a にて終る男性中性の語基を擧げる。女性は a
にて終る場合が無い。

\numberParagraph
\hfil 1) \fbox{a 語基,男性 deva(神)} \hfil\,
\begin{center}
\begin{tabular}{c*{3}{p{0.2\hsize}}}
     & \cellAlign{c}{單} & \cellAlign{c}{兩}          & \cellAlign{c}{複} \\
  主 & devas             & \rdelim\}{3}{*}[devau]     & \rdelim\}{2}{*}[devās] \\
  呼 & deva              &                            & \\
  業 & devam             &                            & devān \vspace{0.5\zw} \\
  具 & devena            & \rdelim\}{3}{*}[devābhyām] & devais \\
  爲 & devāya            &                            & \rdelim\}{2}{*}[devebhyas] \\
  從 & devāt             &                            & \vspace{0.5\zw} \\
  屬 & devasya           & \rdelim\}{2}{*}[devayos]   & devānām \\
  於 & deve              &                            & deveṣu
\end{tabular}
\end{center}

\numberParagraph
\hfil 2) \fbox{a 語基,中性 dāna(施} \hfil\,
\begin{center}
\begin{tabular}{c*{3}{p{0.2\hsize}}}
     & \cellAlign{c}{單}      & \cellAlign{c}{兩}          & \cellAlign{c}{複} \\
  主 & \rdelim\}{2}{*}[dānam] & \rdelim\}{3}{*}[dāne]      & \rdelim\}{3}{*}[dānāni] \\
  業 &                        &                            & \\
  呼 & dāna                   &                            &
\end{tabular}
\end{center}
餘は全く男性の如く變化する。

\numberParagraph
a 語基の男中性形容詞もこれに準じて變化する。例:基
pāpa (罪ある),主男 pāpas,中 pāpam 等。形容詞の女性は ā
又は ī に終る。格變化はその條を参照。形容詞の中性單數主格
の形は副詞の意義を有す。例:śīghra (速かなる),śīghram (速
かに)。

\numberParagraph
\hfil 3) \fbox{i 語基,男女中性} \hfil\,

男性 kavi (詩人),女性 mati (意),中性 vāri (水).

\begin{center}
\begin{tabular}{c*{3}{p{0.2\hsize}}}
  \multicolumn{4}{c}{單} \\
     & \cellAlign{c}{男}      & \cellAlign{c}{女}              & \cellAlign{c}{中} \\
  主 & kavis                  & matis                          & vāri \\
  呼 & kave                   & mate                           & vāri, vāre \\
  業 & kavim                  & matim                          & vāri \\
  具 & kavinā                 & matyā                          & vāriṇā \\
  爲 & kavaye                 & mataye, matyai                 & vāriṇe \\
  從 & \rdelim\}{2}{*}[kaves] & \multirow{2}{*}{mates, matyās} & \multirow{2}{*}{vāriṇas} \\
  屬 &                        &                                & \\
  於 & kavau                  & matau, matyām                  & vāriṇi
\end{tabular}
\end{center}
\begin{center}
\begin{tabular}{c*{3}{p{0.2\hsize}}}
  \multicolumn{4}{c}{兩} \\
  主 & \rdelim\}{3}{*}[kavī]      & \multirow{3}{*}{matī}      & \multirow{3}{*}{vāriṇī} \\
  呼 &                            &                            & \\
  業 &                            &                            & \\
  具 & \rdelim\}{3}{*}[kavibhyām] & \multirow{3}{*}{matibhyām} & \multirow{3}{*}{vāribhyām} \\
  爲 &                            &                            & \\
  從 &                            &                            & \\
  屬 & \rdelim\}{2}{*}[kavyos]    & \multirow{2}{*}{matyos}    & \multirow{2}{*}{vāriṇos} \\
  於 &                            &                            &
\end{tabular}
\end{center}
\begin{center}
\begin{tabular}{c*{3}{p{0.2\hsize}}}
  \multicolumn{4}{c}{複} \\
  主 & \rdelim\}{2}{*}[kavayas]   & \multirow{2}{*}{matayas}   & \rdelim\}{3}{*}[vārīṇi] \\
  呼 &                            &                            & \\
  業 & kavīn                      & matīs                      & \\
  具 & kavibhis                   & matibhis                   & vāribhis \\
  爲 & \rdelim\}{2}{*}[kavibhyas] & \multirow{2}{*}{matibhyas} & \multirow{2}{*}{vāribhyas} \\
  從 &                            &                            & \\
  屬 & kavīnām                    & matīnām                    & vārīṇām \\
  於 & kaviṣu                     & matiṣu                     & vāriṣu
\end{tabular}
\end{center}

\numberParagraph
\hfil 4) \fbox{若干の不規則なるもの} \hfil\,
\begin{enumerate}[label=(\alph*)]
\item sakhi (男性,友)。
\begin{description}[font=\normalfont]
\item[單] sakhā, sakhe, sakhāyam, sakhyā, sakhye, sakhyus,
sakhyou.
\item[兩] sakhāyau, sakhibhyām, sakhyos.
\item[複] sakhāyaś, sakhīn, sakhibhis. 餘は kavi に準ず。
\end{description}
\item pati は「夫」の意味の場合,單具 patyā,爲 patye,從屬
patyus,於 patyau. 又「主」の意味に用ひらるゝか又は bhū-pati
(地主)の如く合成語の終にある時は kavi に準じて變化す。
\item akṣi (眼),asthi (骨),dadhi (酸乳),sakthi (腿)は an
語基(86條參照)に準じて變化する部分を有す。單具 akṣṇā 等
(87條)。
\end{enumerate}

\numberParagraph
\hfil 5) \fbox{u 語基、男性 guru (師),女性 dhenu (牝牛),中性 madhu (蜜)} \hfil\,
\begin{center}
\begin{tabular}{c*{3}{p{0.2\hsize}}}
  \multicolumn{4}{c}{單} \\
     & \cellAlign{c}{男}      & \cellAlign{c}{女}                & \cellAlign{c}{中} \\
  主 & gurus                  & dhenus                           & madhu \\
  呼 & guruo                  & dheno                            & madhu, madho \\
  業 & gurum                  & dhenum                           & madhu \\
  具 & guruṇā                 & dhenvā                           & madhunā \\
  爲 & gurave                 & dhenave, dhenvai                 & madhune \\
  從 & \rdelim\}{2}{*}[guros] & \multirow{2}{*}{dhenos, dhenvās} & \multirow{2}{*}{madhunas} \\
  屬 &                        &                                  & \\
  於 & gurau                  & dhenau, dhenvām                  & madhuni
\end{tabular}
\end{center}
\begin{center}
\begin{tabular}{c*{3}{p{0.2\hsize}}}
  \multicolumn{4}{c}{兩} \\
  主 & \rdelim\}{3}{*}[gurū]      & \multirow{3}{*}{dhenū}      & \multirow{3}{*}{madhum} \\
  呼 &                            &                             & \\
  業 &                            &                             & \\
  具 & \rdelim\}{3}{*}[gurubhyām] & \multirow{3}{*}{dhenubhyām} & \multirow{3}{*}{madhubbyām} \\
  爲 &                            &                             & \\
  從 &                            &                             & \\
  屬 & \rdelim\}{2}{*}[gurvos]    & \multirow{2}{*}{dhenvos}    & \multirow{2}{*}{madhunos} \\
  於 &                            &                             &
\end{tabular}
\end{center}
\begin{center}
\begin{tabular}{c*{3}{p{0.2\hsize}}}
  \multicolumn{4}{c}{複} \\
  主 & \rdelim\}{2}{*}[guravas]   & \multirow{2}{*}{dhenavas}   & \rdelim\}{3}{*}[madhūni] \\
  呼 &                            &                             & \\
  業 & gurūn                      & dhenūs                      & \\
  具 & gurubhyas                  & dhenubhis                   & madhubhis \\
  爲 & \rdelim\}{2}{*}[gurubhyas] & \multirow{2}{*}{dhenubhyas} & \multirow{2}{*}{madhubhyas} \\
  從 &                            &                             & \\
  屬 & gurūṇām                    & dhenūnām                    & madhūnām \\
  於 & guruṣu                     & dhenuṣu                     & madhuṣu
\end{tabular}
\end{center}

\numberParagraph
i, u 語基形容詞男女中性これに準ず。例:語基 śuci (淸
き)主,男,女 śucis 中,śuci. 語基 tanu (痩せたる)主,男,女
tanus, 中 tanu. 中性にてはこの語尾以外爲從屬於の單,屬於
の兩には之に相當せる男性の語尾を用ふるを得。

\hfil 6) \fbox{ā, ī, ū 語基} \hfil\,

\numberParagraph
ā, ī, ū 語基は女性である。その中 ī, ū 語基にありては
多綴のものと單綴のものと稍變化を異にす。例:kanyā (少女),
nadī (河),vadhū (婦).
\begin{center}
\begin{tabular}{c*{3}{p{0.2\hsize}}}
  \multicolumn{4}{c}{單} \\
  主 & kanyā                     & nadī                    & vadhū \\
  呼 & kanye                     & nadi                    & vadhu \\
  業 & kanyām                    & nadīm                   & vadhūm \\
  具 & kanyayā                   & nadyā                   & vadhvā \\
  爲 & kanyāyai                  & nadyai                  & vadhvai \\
  從 & \rdelim\}{2}{*}[kanyāyās] & \multirow{2}{*}{nadyās} & \multirow{2}{*}{vadhvās} \\
  屬 &                           &                         & \\
  於 & kanyāyām                  & nadyām                  & vadhvām
\end{tabular}
\end{center}
\begin{center}
\begin{tabular}{c*{3}{p{0.2\hsize}}}
  \multicolumn{4}{c}{兩} \\
  主 & \rdelim\}{3}{*}[kanye]      & \multirow{3}{*}{nadyau}    & \multirow{3}{*}{vadhvau} \\
  呼 &                             &                            & \\
  業 &                             &                            & \\
  具 & \rdelim\}{3}{*}[kanyābhyām] & \multirow{3}{*}{nadībhyām} & \multirow{3}{*}{vadhūbhyām} \\
  爲 &                             &                            & \\
  從 &                             &                            & \\
  屬 & \rdelim\}{2}{*}[kanyayos]   & \multirow{2}{*}{nadyos}    & \multirow{2}{*}{vadhvos} \\
  於 &                             &                            &
\end{tabular}
\end{center}
\begin{center}
\begin{tabular}{c*{3}{p{0.2\hsize}}}
  \multicolumn{4}{c}{複} \\
  主 & \rdelim\}{3}{*}[kanyās]     & \rdelim\}{2}{*}[nadyas]    & \multirow{2}{*}{vadhvas} \\
  呼 &                             &                            & \\
  業 &                             & nadīs                      & vadhūs \\
  具 & kanyābhis                   & nadhībhis                  & vadhūbhis \\
  爲 & \rdelim\}{2}{*}[kanyābhyas] & \multirow{2}{*}{nadibhyas} & \multirow{2}{*}{vadhūbhyas} \\
  從 &                             &                            & \\
  屬 & kanyānām                    & nadīnām                    & vadhūnām \\
  於 & kanyāsu                     & nadīṣu                     & vadhuṣu
\end{tabular}
\end{center}

\numberParagraph
單綴の ī, ū 語基。śrī (幸福), bhū (地).
\begin{center}
\begin{tabular}{c*{2}{p{0.3\hsize}}}
  \multicolumn{3}{c}{單} \\
  主 & \rdelim\}{2}{*}[śrīs]           & bhūs \\
  呼 &                                 & \\
  業 & śriyam                          & bhuvam \\
  具 & śriyā                           & bhuvā \\
  爲 & śriye, śriyai                   & bhuve, bhuvai \\
  從 & \rdelim\}{2}{*}[śriyas, śriyās] & \multirow{2}{*}{bhuvas, bhuvās} \\
  屬 &                                 & \\
  於 & śriyi, śriyām                   & bhuvi, bhuvām
\end{tabular}
\end{center}
\begin{center}
\begin{tabular}{c*{2}{p{0.3\hsize}}}
  \multicolumn{3}{c}{兩} \\
  主 & \rdelim\}{3}{*}[śriyau]   & \multirow{3}{*}{bhūs} \\
  呼 &                           & \\
  業 &                           & \\
  具 & \rdelim\}{3}{*}[śribhyām] & \multirow{3}{*}{bhūbhyām} \\
  爲 &                           & \\
  從 &                           & \\
  屬 & \rdelim\}{2}{*}[śriyos]   & \multirow{2}{*}{bhuvos} \\
  於 &                           &
\end{tabular}
\end{center}
\begin{center}
\begin{tabular}{c*{2}{p{0.3\hsize}}}
  \multicolumn{3}{c}{複} \\
  主 & \rdelim\}{3}{*}[śriyas]   & \multirow{3}{*}{bhuvas} \\
  呼 &                           & \\
  業 &                           & \\
  具 & śrībhis                   & bhūbhyas \\
  爲 & \rdelim\}{2}{*}[śribhyas] & \multirow{2}{*}{bhūbhyas} \\
  從 &                           & \\
  屬 & śriyām, śrīṇām            & bhuvām, bhūnām \\
  於 & śrīṣu                     & bhūṣu
\end{tabular}
\end{center}

\nt{
  strī は不規則にして單主 strī, 呼 stri, 業 striyam 又は strīm,
  爲 striyai, 從屬 striyās, 具 striyām, 複業 strīs 又は striyas, 屬
  strīṇām.}

\numberParagraph
複母音語基,nau (舟),go (牛)。\endnote{底本では両数形について主・呼・業ではなく
  呼・業が \} でまとめられている(主格の欄は空欄)。また具・為をまとめる \} が
  単数形の複数形ではなく単数形の列に付いている。}
\begin{center}
\begin{tabular}{c*{6}{p{0.12\hsize}}}
     & \multicolumn{2}{c}{單}                         & \multicolumn{2}{c}{兩}                               & \multicolumn{2}{c}{複} \\
  主 & \rdelim\}{2}{*}[naus]  & \multirow{2}{*}{gaus} & \rdelim\}{3}{*}[nāvau]    & \multirow{3}{*}{gāvau}   & \rdelim\}{3}{*}[nāvas]   & \rdelim\}{2}{*}[gāvas] \\
  呼 &                        &                       &                           &                          &                          & \\
  業 & nāvam                  & gām                   &                           &                          &                          & gās \\
  具 & nāvā                   & gavā                  & \rdelim\}{3}{*}[naubhyām] & \multirow{3}{*}{gobhyām} & \rdelim\}{2}{*}[naubhis] & \multirow{2}{*}{gobhis} \\
  爲 & nāve                   & gave                  &                           &                          &                          & \\
  從 & \rdelim\}{2}{*}[nāvas] & \multirow{2}{*}{gos}  &                           &                          & nāvām                    & gavām \\
  屬 &                        &                       & \rdelim\}{2}{*}[nāvos]    & \multirow{2}{*}{gavos}   & \multirow{2}{*}{nauṣu}   & \multirow{2}{*}{goṣu} \\
  於 & nāvi                   & gavi                  &                           &                          &                          &
\end{tabular}
\end{center}

\numberParagraph
ṛ 語基 これは一部分 作者名詞であり,一部分は 親族名
詞である。男 dātṛ (施者),女 svasṛ (姉妹)。
\begin{center}
\begin{tabular}{c*{4}{p{0.12\hsize}}}
     & \multicolumn{2}{c}{單}                           & \multicolumn{2}{c}{複} \\
  主 & dātā                   & svasā                   & \rdelim\}{2}{*}[dātāras]   & \multirow{2}{*}{svasāras} \\
  呼 & dātar                  & svasar                  &                            & \\
  業 & dātāram                & svasāram                & dātrīn                     & svasṝs \\
  具 & dātrā                  & svasrā                  & dātṛbhis                   & svasṛbhis \\
  爲 & dātre                  & svasre                  & \rdelim\}{2}{*}[dātṛbhyas] & \multirow{2}{*}{svasṛbhyas} \\
  從 & \rdelim\}{2}{*}[dātur] & \multirow{2}{*}{svasur} &                            & \\
  屬 &                        &                         & dātṝṇām                    & svasṝṇām \\
  於 & dātari                 & svasari                 & dātrṣu                     & svasṛṣu
\end{tabular}
\end{center}

\begin{center}
\begin{tabular}{c*{2}{p{0.24\hsize}}}
     & \multicolumn{2}{c}{兩} \\
  主 & \rdelim\}{3}{*}[dātārau]   & \multirow{3}{*}{svasārau} \\
  呼 &                            & \\
  業 &                            & \\
  具 & \rdelim\}{3}{*}[dātṛbhyām] & \multirow{3}{*}{svasṛbhyām} \\
  爲 &                            & \\
  從 &                            & \\
  屬 & \rdelim\}{2}{*}[dātros]    & \multirow{2}{*}{svasros} \\
  於 &                            &
\end{tabular}
\end{center}
中性の變化は ī, ū の中性に準ず。卽ち單主業 dātṛṇā,
爲 dātṛṇe, 兩主呼業 dātṛṇī, 複主呼業 dātṝṇī.

\numberParagraph
親族名詞〔naptṛ (孫),svasṛ (姉妹)を除く〕は單,業,
兩主業呼,及複主に於て異る。a が短音である。

\begin{tabular}{cl}
  語基 & pitṛ (父) pitaram, pitarau, pitaras. \\
  〃   & mātṛ (母) mātram, mātarau, mātaras. \\
  〃   & duhitṛ (娘) duhitaram, huhitarau, huhitaras.
\end{tabular}

(注意)格例法は次 78 條に連續する。

\ex{第一}

\begin{longtable}{c*{2}{p{0.45\hsize}}}
   1. & andhasya dīpo vidyā.             & 知識は盲冥の燈火なり。\\
   2. & ācāraḥ pradhāno dharmaḥ.         & 善行は最上の法なり。\\
   3. & kālasya kuṭilā gatiḥ.            & 時の路は曲れり。\\
   4. & śīlaṃ narasya bhūṣaṇam.          & 戒は人の莊嚴なり。\\
   5. & alpaḥ kālo bahavaś ca vigh\-nāḥ. & 短い時と多くの障害。\\
   6. & yathā vṛkṣās tathā phalāni.      & 樹の如くかくの如く果(あり)。\\
   7. & aputrasya gṛhaṃ śūnyam.          & 子なき者の家は空虚なり。\\
   8. & duḥkhaṃ kadāpi sukhasya          & 苦は往々樂の因なり。\\
   9. & yatra yatra dhūmas tatra tatra vahniḥ. & 烟ある所其處に火あり。\\
  10. & pādapānāṃ bhayaṃ vātaḥ padmānāṃ śiśiro bhayam. & 樹の恐は風,蓮花の恐は冷氣。\\
      & parvatānāṃ bhayaṃ vajraḥ sādhūnāṃ durjano bhayam. & 山の恐は雷,善人の恐は悪人なり。\\
  11. & upadeśo hi mūrkhāṇāṃ pra\-kopāya na śāntaye. & 論議は愚者の怒を招く,鎭靜に役立たない。\\
  12.& nagaraṃ devena jitam. & 都城は王によりて打ち勝たれた。\\
  13. & naraḥ sarpeṇa daṣṭo mṛtaś ca. & 人は蛇によりて嚙まれて\ruby{而}{しこう}して死せり。\\
  14. & siṃho vyādhasya śareṇa hataḥ. & 獅子は獵師の\ruby{箭}{や}によりて殺された。\\
  15. & mūṣikā śyenena gṛhītā bhak\-ṣitā ca. & 鼠は鷹によりて捉えられ而して食はれた。\\
  16. & raṇe nṛpasya senayārayo jitāḥ. & 戰鬪に於て王の軍隊のために敵は征服された。\\
  17. & nirdhanasya kutaḥ sukham. & 貧しきものには何處にか幸福あらむ。\\
  18  & vidyā mitraṃ pravāse ca & 旅行に於て知識は友であり \\
      & bhāryā mitraṃ gṛheṣu ca. & 妻は家に於て友であり,\\
      & vyādhitasyauṣadhaṃ mitram dharmo mitraṃ mṛtasya ca & 病めるものに藥は友であり,而して死者にとりては正義は友である。
\end{longtable}

\clearpage
\section{活用法}
\numberParagraph
梵語の活用法は語根に若干の變化を加へてこれに語尾を
附するにある。卽ち yaj (祭る)+ a = yaja + ti (二人稱單數,現
實法語尾),yajati (彼は祭る)。

梵語にては能動調受動調ある外に能動調を更に爲他,爲自の二
類に分ち,語尾を異にす。されどその意義に至つては爲他も爲自
も同様である。

三數は格例法の如くである。


\newpage
\theendnotes


%%% Local Variables:
%%% mode: latex
%%% TeX-master: "IntroductionToSanskrit"
%%% End:
