\chapter{語尾曲法}\label{cha:flection}
\section{格例法}
\numberParagraph
言語の各類に就て變化するものと變化しないものとがあ
る。變化しないものとは接續詞,間投詞,副詞の如きである。今
語尾曲法に於て取扱はうとするものはこれら変化せざるものを暫
らく除外して變化するものに就て論ずる。變化する詞にも二樣の
種類が分れる。一は卽ち名詞的變化であつて性と數と格によつて
語尾を異にするもの。この種に屬するものは名詞,代名詞,形容
詞,數詞,分詞である。他は卽ち動詞的變化であつて人稱と數と
時と法とに隨つて語尾を異にする。前者の取扱ひを格例法と云
ひ,後者の取扱ひを活用法と云ふ。

\numberParagraph
梵語では格例法に三性 三數 八格を認める。三性とは男
女 中であり,三數とは單 兩 複であり,八格とは主 業 具 爲
從 屬 於 呼である。活用法には三の人稱と 三數と 三時と 四
法を認める。曰く 一人稱 二人稱 三人稱,曰く單 兩 複,曰く
現在 過去 未來,曰く現實法,可能法,命令法,條件法である。
これらの意義は後節に説く。

\numberParagraph
梵語では名詞的變化は語基を以て,動詞的變化は語根を以
て單位とする。單位とは分解の終極を意味し,それ以上溯る必要
を見ない形である。この形で辭典に記録されるものと知るべきで
ある。嚴密に云はば如何なる詞もその語根まで溯るべきである
が,名詞,代名詞,形容詞,數詞,分詞は必ずしも語根の詮索の
手を延すを要しない。出來上つた語基の形に語尾が附けられる。
然し動詞は語根によりて論ぜられる。之を要するに語根は梵語の
あらゆる語彙を分析してこれを原始の狀態に還元したものであ
る。印度平原に花咲く言語から得た收穫である。萬を以て數へら
れる言語が數百の語根に還元せらるゝことは實に言語學の寄與貢
献の偉大を示すものである。

\numberParagraph
語根は多くは單綴の音であり,意義はあるも未だ言語では
ない。云はば言語のエキスである。今語根に若干の変化を施して
語基を作り,これに語尾を附加すれば始めて活用する言語とな
る。或は之を逆に考へて活用する言語を比較し分析して語尾を切
離し,語基を整理して最後の形を得たとすればそれが語根であ
る。

\numberParagraph
\textbf{名詞的變化}に就て\textbf{語基が母音}で終るものと\textbf{子音で終るも
  の}との二類を分つ。前者を母音語基,後者を子音語基と呼ぶ。母
音語基は語尾が附加せられて太だ不規則なる變化をなすから,稍
複雜ではあるが,語基に强弱を認むる必要あまりなきと,數に於
いて飛び離れて多いとの理由から,これを最初に取扱ふことゝす
る。先づ最初に a にて終る男性中性の語基を擧げる。女性は a
にて終る場合が無い。

\numberParagraph
\hfil 1) \fbox{a 語基,男性 deva(神)} \hfil\,
\begin{center}
\begin{tabular}{c*{3}{p{0.2\hsize}}}
     & \cellAlign{c}{單} & \cellAlign{c}{兩}          & \cellAlign{c}{複} \\
  主 & devas             & \rdelim\}{3}{*}[devau]     & \rdelim\}{2}{*}[devās] \\
  呼 & deva              &                            & \\
  業 & devam             &                            & devān \vspace{0.5\zw} \\
  具 & devena            & \rdelim\}{3}{*}[devābhyām] & devais \\
  爲 & devāya            &                            & \rdelim\}{2}{*}[devebhyas] \\
  從 & devāt             &                            & \vspace{0.5\zw} \\
  屬 & devasya           & \rdelim\}{2}{*}[devayos]   & devānām \\
  於 & deve              &                            & deveṣu
\end{tabular}
\end{center}

\numberParagraph
\hfil 2) \fbox{a 語基,中性 dāna(施} \hfil\,
\begin{center}
\begin{tabular}{c*{3}{p{0.2\hsize}}}
     & \cellAlign{c}{單}      & \cellAlign{c}{兩}          & \cellAlign{c}{複} \\
  主 & \rdelim\}{2}{*}[dānam] & \rdelim\}{3}{*}[dāne]      & \rdelim\}{3}{*}[dānāni] \\
  業 &                        &                            & \\
  呼 & dāna                   &                            &
\end{tabular}
\end{center}
餘は全く男性の如く變化する。

\numberParagraph
a 語基の男中性形容詞もこれに準じて變化する。例:基
pāpa (罪ある),主男 pāpas,中 pāpam 等。形容詞の女性は ā
又は ī に終る。格變化はその條を参照。形容詞の中性單數主格
の形は副詞の意義を有す。例:śīghra (速かなる),śīghram (速
かに)。

\numberParagraph
\hfil 3) \fbox{i 語基,男女中性} \hfil\,

男性 kavi (詩人),女性 mati (意),中性 vāri (水).

\begin{center}
\begin{tabular}{c*{3}{p{0.2\hsize}}}
  \multicolumn{4}{c}{單} \\
     & \cellAlign{c}{男}      & \cellAlign{c}{女}              & \cellAlign{c}{中} \\
  主 & kavis                  & matis                          & vāri \\
  呼 & kave                   & mate                           & vāri, vāre \\
  業 & kavim                  & matim                          & vāri \\
  具 & kavinā                 & matyā                          & vāriṇā \\
  爲 & kavaye                 & mataye, matyai                 & vāriṇe \\
  從 & \rdelim\}{2}{*}[kaves] & \multirow{2}{*}{mates, matyās} & \multirow{2}{*}{vāriṇas} \\
  屬 &                        &                                & \\
  於 & kavau                  & matau, matyām                  & vāriṇi
\end{tabular}
\end{center}
\begin{center}
\begin{tabular}{c*{3}{p{0.2\hsize}}}
  \multicolumn{4}{c}{兩} \\
  主 & \rdelim\}{3}{*}[kavī]      & \multirow{3}{*}{matī}      & \multirow{3}{*}{vāriṇī} \\
  呼 &                            &                            & \\
  業 &                            &                            & \\
  具 & \rdelim\}{3}{*}[kavibhyām] & \multirow{3}{*}{matibhyām} & \multirow{3}{*}{vāribhyām} \\
  爲 &                            &                            & \\
  從 &                            &                            & \\
  屬 & \rdelim\}{2}{*}[kavyos]    & \multirow{2}{*}{matyos}    & \multirow{2}{*}{vāriṇos} \\
  於 &                            &                            &
\end{tabular}
\end{center}
\begin{center}
\begin{tabular}{c*{3}{p{0.2\hsize}}}
  \multicolumn{4}{c}{複} \\
  主 & \rdelim\}{2}{*}[kavayas]   & \multirow{2}{*}{matayas}   & \rdelim\}{3}{*}[vārīṇi] \\
  呼 &                            &                            & \\
  業 & kavīn                      & matīs                      & \\
  具 & kavibhis                   & matibhis                   & vāribhis \\
  爲 & \rdelim\}{2}{*}[kavibhyas] & \multirow{2}{*}{matibhyas} & \multirow{2}{*}{vāribhyas} \\
  從 &                            &                            & \\
  屬 & kavīnām                    & matīnām                    & vārīṇām \\
  於 & kaviṣu                     & matiṣu                     & vāriṣu
\end{tabular}
\end{center}

\numberParagraph
\hfil 4) \fbox{若干の不規則なるもの} \hfil\,
\begin{enumerate}[label=(\alph*)]
\item sakhi (男性,友)。
\begin{description}[font=\normalfont]
\item[單] sakhā, sakhe, sakhāyam, sakhyā, sakhye, sakhyus,
sakhyou.
\item[兩] sakhāyau, sakhibhyām, sakhyos.
\item[複] sakhāyaś, sakhīn, sakhibhis. 餘は kavi に準ず。
\end{description}
\item pati は「夫」の意味の場合,單具 patyā,爲 patye,從屬
patyus,於 patyau. 又「主」の意味に用ひらるゝか又は bhū-pati
(地主)の如く合成語の終にある時は kavi に準じて變化す。
\item akṣi (眼),asthi (骨),dadhi (酸乳),sakthi (腿)は an
語基(86條參照)に準じて變化する部分を有す。單具 akṣṇā 等
(87條)。
\end{enumerate}

\numberParagraph
\hfil 5) \fbox{u 語基、男性 guru (師),女性 dhenu (牝牛),中性 madhu (蜜)} \hfil\,
\begin{center}
\begin{tabular}{c*{3}{p{0.2\hsize}}}
  \multicolumn{4}{c}{單} \\
     & \cellAlign{c}{男}      & \cellAlign{c}{女}                & \cellAlign{c}{中} \\
  主 & gurus                  & dhenus                           & madhu \\
  呼 & guruo                  & dheno                            & madhu, madho \\
  業 & gurum                  & dhenum                           & madhu \\
  具 & guruṇā                 & dhenvā                           & madhunā \\
  爲 & gurave                 & dhenave, dhenvai                 & madhune \\
  從 & \rdelim\}{2}{*}[guros] & \multirow{2}{*}{dhenos, dhenvās} & \multirow{2}{*}{madhunas} \\
  屬 &                        &                                  & \\
  於 & gurau                  & dhenau, dhenvām                  & madhuni
\end{tabular}
\end{center}
\begin{center}
\begin{tabular}{c*{3}{p{0.2\hsize}}}
  \multicolumn{4}{c}{兩單} \\
  主 & \rdelim\}{3}{*}[gurū]      & \multirow{3}{*}{dhenū}      & \multirow{3}{*}{madhum} \\
  呼 &                            &                             & \\
  業 &                            &                             & \\
  具 & \rdelim\}{3}{*}[gurubhyām] & \multirow{3}{*}{dhenubhyām} & \multirow{3}{*}{madhubbyām} \\
  爲 &                            &                             & \\
  從 &                            &                             & \\
  屬 & \rdelim\}{2}{*}[gurvos]    & \multirow{2}{*}{dhenvos}    & \multirow{2}{*}{madhunos} \\
  於 &                            &                             &
\end{tabular}
\end{center}
\begin{center}
\begin{tabular}{c*{3}{p{0.2\hsize}}}
  \multicolumn{4}{c}{複} \\
  主 & \rdelim\}{2}{*}[guravas]   & \multirow{2}{*}{dhenavas}   & \rdelim\}{3}{*}[madhūni] \\
  呼 &                            &                             & \\
  業 & gurūn                      & dhenūs                      & \\
  具 & gurubhyas                  & dhenubhis                   & madhubhis \\
  爲 & \rdelim\}{2}{*}[gurubhyas] & \multirow{2}{*}{dhenubhyas} & \multirow{2}{*}{madhubhyas} \\
  從 &                            &                             & \\
  屬 & gurūṇām                    & dhenūnām                    & madhūnām \\
  於 & guruṣu                     & dhenuṣu                     & madhuṣu
\end{tabular}
\end{center}

\numberParagraph
i, u 語基形容詞男女中性これに準ず。例:語基 śuci (淸
き)主,男,女 śucis 中,śuci. 語基 tanu (痩せたる)主,男,女
tanus, 中 tanu. 中性にてはこの語尾以外爲從屬於の單,屬於
の兩には之に相當せる男性の語尾を用ふるを得。


%%% Local Variables:
%%% mode: latex
%%% TeX-master: "IntroductionToSanskrit"
%%% End:
