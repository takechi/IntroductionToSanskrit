\texttitle{ナラ王物語}
\addcontentsline{toc}{chapter}{\protect\numberline{}ナラ王物語}%
昔ヴィーラセーナの王子なる力强きナラなる王ありき。

勝れたる德を具し美貌にして馬術に巧みに,

人たる君主の頭首なりき。神の主領なる如く,

太陽の如く,威光を以て一切の上に秀で,

梵行を修し,吠陀に委しく,勇者にしてニシャドハの王なり。

勝負事を好み,眞實を語り,大人物にして軍團の將なり。

勝れし婦女たしに愛せられ,寛厚にして節操正し。

守護者にして弓取るものの最勝者,マヌ自信を親たりに見るが
如くなり。

又ヴィダルブハにブヒーマとて力すぐれたる王あり,

一切の德を具し,勇者なるが,子なくして久しく子を欲せり。

彼の子を欲していみじき努力を爲し,心統一をなせる,

彼へ梵仙ダナマ\endnote{「ダナマ」はママ。}なるものが來れり。ブハーラタよ。

子を欲りせる,法に通ぜる,彼のブヒーマは

夫人と共に,王よ,尊敬もて威光ある彼を喜ばしめぬ。

喜べるダマナは,夫人と共なる彼に賚賜を與へたり,

寶の如き王女と大なる名譽あるすぐれたる三人の王子,

卽ちダマヤンティーと威光あるダマ,ダーンタ,ダマナとなり。

一切の德を具し勇ありて勇氣すぐれたり。

ダマヤンティーは姿めでたく容色すぐれたり。

又幸運を以て世に名を得たり。

\newpage{}
\texttitle{Nalopākhyānam}

āsid rājā Nalo nāma Virasena-suto bali \da

upapanno guṇair iṣṭai rūpavān aśva-kovidaḥ \dd

atiṣṭhan manujendrāṇāṃ mūrdhni devapatir iva \da

upary upari sarveṣām āditya iva tejasā \dd

brahmaṇyo veda-vic chūro Niṣadheṣu mahīpatiḥ \da

akṣa-priyaḥ satya-vādī mahān akṣauhiṇī-patiḥ \dd

īpsito vara-nārīṇām udāraḥ saṃyatendriyaḥ \da

rakṣitā dhanvināṃ śreṣṭhaḥ sākṣād iva Manuḥ svayam \dd

tathaivāsīd Vidarbheṣu Bhīmo bhīma-parākramaḥ \da

śūraḥ sarva-guṇair yuktaḥ prajā-kāmaḥ sa cāprajaḥ \dd

sa prajārthe paraṃ yatnam akarot susamāhitaḥ \da

tam abhyagacchad brahma-rṣir Damano nāma Bhārata \dd

taṃ sa Bhīmaḥ prajā-kāmas toṣayāmāsa dharma-vit \da

mahiṣyā saha rājendra satkāreṇa suvarcasam \dd

tasmai prasanno Damanaḥ sabhāryāya varaṃ dadau \da

kanyā-ratnaṃ kumārāṃś ca trīn udārān mahāyaśāḥ \dd

Damayantīṃ Damaṃ Dāntaṃ Damanaṃ Damanaṃ ca suvarcasam \da

unsapannān guṇaiḥ sarvair bhīmān bhīmaparākramān \dd

Damayantī tu rūpeṇa tejasā yaśasā śriyā \da

saubhāgyena ca lokeṣu yaśaḥ prāpa sumadhyamā \dd

\newpage

さても妙齡に達せし時,装飾つけたる婢女と

侍女との幾百が奉仕せり。シャチーの如くに。

そこにブヒーマの王女は一切の装飾に飾られて輝けり。

侍女の中に避難さるべき點なき彼女は電光の如く,

極めて容色美はしく脩長なる目あり。吉祥天の如く,

天界にもヤクシャ界にも何處にもかくの如き美容もてるは無し。

人界にもその他にも嘗て見られず,又聞かれず。

神々にとりても美はしき心を迷はす若き少女なり。

ナラは又世に於いて人中の虎のやうに强く地上に比びなし。

美貌を以て,親たり形を現ぜしカンダルパのごとくなりき。

彼女の側にては彼等(女達)は好奇心もてナラ王を讚めたり。

ナラ王の側には再三再四ダマヤンティーを讚めたり。

常に樣子を聞ける彼等二人の間にはまだ見ぬ戀心がありき。

クンティーの子よ,互の間にかの戀は增し行けり。

ナラは心もて戀を保ち得ずして

後宮の側なる林に祕かに行きて坐せり。

その時彼は黄金色に輝ける白鳥の群れを見たり。

林に飛び翔る彼等の一羽の鳥を捕へたり。

その時,鳥はナラに人語を語れり,

我をな殺しそ王よ。汝に我はよき事をなすべし。

ニシャドハの王よ,我はダマヤンティーの傍に汝の上を語るべし,

かくて汝以外の男子を彼女は決して考へざるべし。

かくて語りし時王は白鳥を放ちやりぬ,

彼等白鳥は飛び上りてヴィダルブハの方へ行けり。

\newpage

atha tāṃ vayasi prāpte dāsīnāṃ samalaṃkṛtam \da

śatam śatam sakhīnāṃ ca paryupāsac Chacīm iva \dd

tatra sma rājate bhaimī sarvābharaṇa-bhūṣitā \da

sakhī-madhye 'navadyāṅgī vidyut saudāminī yathā \dd

atīva rūpa-saṃpannā Śrīr ivāyata-locanā \da

na deveṣu na yakṣeṣu tādṛg-pūrvāthavā śrutā \da

citta-pramāthinī bālā devānām api sundarī \dd

Nalaś ca nara-śārdūlo lokeṣv apratimo bhuvi \da

Kandarpa iva rūpeṇa mūrtimān abhavat svayam \dd

tasyāḥ samīpe tu Nalaṃ praśaśaṃsuḥ kutūhalāt \da

Naīṣadhasya samīpe tu Damayantīṃ punaḥ punaḥ \dd

tayor adṛṣta-kāmo 'bhūt sṛṇvato satataṃ guṇān \da

anyonyaṃ prati Kauteya sa vyavardhata hṛcchayaḥ \dd

aśaknuvan Nalaḥ kāmaṃ tadā dhārayituṃ hṛdā \da

antaḥpura-samīpa-sthe vana āste raho gataḥ \dd

sa dadarśa tato haṃsān jātarūpa-pariṣkṛtān \da

vane vicaratāṃ teṣām ekaṃ jagrāha pakṣiṇam \dd

tato 'ntarīkṣago vācaṃ vyajahāra Nalaṃ tadā \da

hantavyo 'smi na te rājan kariṣyāmi tava priyam \dd

Damayantī-sakāśe tvāṃ kathayiṣyāmi Naiṣadha \da

yathā tvad-anyaṃ puruṣaṃ na sā maṃsyati karhicit \dd

evam uktas tato haṃsam utsasarja mahīpatiḥ \da

te tu haṃsāḥ samutpatya Vidarbhān agamaṃs tataḥ \dd

\newpage

ヴィダルブハの都城に行きてその時ダマヤンティーの側に

彼等鳥等は飛び下りぬ。かくて彼女は彼等の群を見たり。

彼女はかれら世にもいみじき色を見て侍女たちと共に

心喜び鳥等を捉へんものと急ぎつゝ進み行けり。

時に白鳥等は若き婦女の群の中を遍ねく飛び回れり,

時に少女達は各々彼等白鳥を追ひ回せり,

ダマヤンティーも亦側なるとある白鳥を追ひまわせしが

その白鳥は人語をなしその時ダマヤンティーに云へらく

ダマヤンティーよニシャドハの王なるナラなる君あり,

美貌はアシュヴィンの如く人間に彼に比ぶべきは無し。

若し汝彼の妻たらば,美はしきものよ,

この生れと美貌とはよき果を結ばむ,美しき少女よ。

我らは神々ガンダルヷ人間龍羅刹の世界を

見たるが,未だ曾て我等はかくの如き美しきものを見ず。

汝は又婦女の寶なり,而してナラは人間の最もすぐれしもの。

殊勝なるものが殊勝なるものに合會せんはいみじき事なりかし。

かくの如く白鳥に語りかけられしダマヤンティーは,王よ,

そこに白鳥に言へり,汝はかくナラに語れ。

爾かせむと白鳥はヴィダルブハの少女に云ひて,王よ,

再びニシャドハに來りて王にすべてを吿げたり。
\wosfnt{%
  ナラ王物語は印度二大叙事詩の一なるマハーバーラタの中に見える美し
  い詩篇である。全篇二十六章凡そ九百八十頌のシュローカ調で出來てゐる。
  ナラ王とダマヤンティー姫との間の戀愛から結婚,不幸厄難,而もそれを越
  えて進む勇氣と天祐神助,全卷を貫く誠實貞潔,或る見方ではマハーバー
  ラタ全篇の縮圖とも見られ,印度の理想的人格が手際よく描かれてゐる。
  此に擧げしは最初の部分若干であるが,その文體も平明で華麗,實に典文
  梵語の模範たるに値する。}

\newpage

Vidarbha-nagarīṃ\endnote{底本では ``Vidarbha-'' は ``\rotatebox{90}{V}idarbha-''。} gatvā Damayantyās tadāntike \da

nipetus te garutmantaḥ sā dadarśa ca tān gaṇān \dd

sā tān adbhuta-rūpān vai dṛṣṭvā sakhī-gaṇāvṛtā \da

hṛṣṭā grahītuṃ khagamāṃs tvaramāṇopacakrame \dd

atha haṃsā visasṛpuḥ sarvataḥ pramadāvane \da

ekaikaśas tadā kanyās tān haṃsān samupādravan \dd

Damayantī tu yaṃ haṃsaṃ samupādhāvad antike \da

sa mānuṣīṃ giraṃ kṛtvā Damayantīṃ athābravīt \dd

Damayanti Nalo nāma Niṣadheṣu mahīpatiḥ \da

Aśvinoḥ sadṛśo rūpe na samās tasya mānuṣāḥ \dd

tasya vai yadi bhāryā tvaṃ bhavethā varavarṇini \da

saphalaṃ te bhavej janma rūpaṃ cedaṃ sumadhyame \dd

vayaṃ hi deva-gandharva-mānuṣoraga-rākṣasān \da

dṛṣṭavanto na cāsmābhir dṛṣṭa-pūrvas tathā-vidhaḥ \dd

viśiṣṭāyā tu haṃsena Damayantī viśāṃpate \da

abravīt tatra taṃ haṃsaṃ tvam apy evaṃ Nale vada \dd

tathety uktvāṇḍajaḥ kanyāṃ Vidarbhasya viśāṃpate \da

punar āgatya Niṣadhān Nale sarvaṃ nyavedayat \dd

\rightline{(Mahābhārata III. 6)}


\newpage
\theendnotes

%%% Local Variables:
%%% mode: latex
%%% TeX-master: "IntroductionToSanskrit"
%%% End:
