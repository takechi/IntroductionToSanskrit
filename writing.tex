\chapter{文章法}
\section{總說}
\numberParagraph
梵文學はその大部分が詩で出來てゐる關係上,文章は原
始的で未開拓であるとも云へる。その特徴は同格や動詞狀名詞が
幅を利かせて接續句などに取つて替つてゐることである。それに
間接語法は全然缺如してゐる。その他の特徴としては定動詞形が
極めて稀である。吠陀ではそうでもない。定動詞の代りに過去分
詞とか動詞狀名詞が用ひられる。又受動の文章が好んで用ひられ
る。特に注意すべきは獨立於格の使はれることである。

\subsection{言語の順序}
\numberParagraph \label{np:217}
言語の排列は梵語では第一に主格とその屬性(屬格は主
格に先立つ),次に目的並にその從屬物(それは目的に先つ),而
して最後に動詞である。

副詞卽ち賓辭の擴大されたものは通常最初の方に置かれ,かく
て强調せられない接續的の小辭が最初の語に續く。例

\indent{Janakas tu satvaraṃ svīyaṃ nagaraṃ jagāma.}

\indent{「然しジャナカは急いで彼自身の都城に行けり」。}

若し呼格があれば一般に最初に來る。主語の代りに如何なる他
の語でも强調せらるゝものは最初に置かれる。例

\indent{rātrau tvayāmaṭha-madhye na praveṣṭavyam.}

\indent{「夜に於いて汝は寺院に入つてはならぬ」。}

\begin{enumerate}[label=(\alph*), ref=\alph*]
\item \label{item:217a} 特に强調の必要なくば主語は代名詞で書かぬことになつ
てゐる。定動詞の中に含まれてゐる一般の主語,卽ち人はとか世
人はとか云ふものも動詞だけで示される。例 brūyāt 「人は云ふ
べし」。 āhuḥ 「世人は云ふ」。
\item asti なる助動詞は必要なければ一般に略される。その場
合には賓辭が名詞に先立つのが通常だ。śītalā rātriḥ 「夜は冷か
である」。若し賓辭が强調される必要あらば bhavati が用ひられ
る。例 yo vidyayā tapasā janmanā vā vṛddhaḥ sa pūjyo
dhavati dvijānām 「智慧と行力と門地を以て勝れたる彼は再生
族中に尊敬せられてある」。
\item 屬性が名詞に先立ち,合成語に於いて限定の語が最初に來
るやうに,關係的接續句は主文章に先立つ。而してそれは規則正
しく相關詞で始められる。 yasya dhanaṃ tasya balam 「富める
所の彼には力あり」。同樣に yadā --- tadā, yāvat --- tāvat 等。
\end{enumerate}

\subsection{數}
\numberParagraph \label{np:218}
\begin{enumerate}[label=(\arabic*)]
\item 單數の集合を意味する語が合成語の終に用ひられる。
strī-jana (男)「婦女」。\ruby{恁}{こ}うした集合名詞は時としてそのまゝ複
數に使はれる。lokaḥ 又は lokāḥ 「世界は」,「人々は」。
\item 兩數は嚴格に規則正しく使用せられ,二個のものを複數に
することは決して無い。故に兩數は必ず一對をなすものである。
例へば身體の部分 hastau pādau ca 「手と足と」。男性の兩數が時
として同群の雌雄を意味する。jagataḥ pitarau 「世界の父母」。
\item \label{item:2183}
\begin{enumerate}[label=(\alph*), ref=\alph*]
\item \label{item:2183a} 時として複數が尊敬を表はすこともある。tvam 及び
bhavān と云ふよりも yūyam 及び bhavantaḥ の方が一層恭し
い云ひ方である。例 śrutaṃ bhavadbhiḥ 「陛下は御聽き遊ばせ
しか」(これは單數の意である)。この意味で複數 pādāḥ が兩數の
代りに使はれる。例 eṣa devapādān adhikṣipati 「彼は陛下の御
足を汚し奉つた」。固有名詞が時として同樣に用ひられる。例 iti
śrī-Śaṅkarācāryāḥ 「かく聖者シャンカラ師は云へり」。
\item 一人稱複數が時として說者自身のことを意味することが
ある。vayam api kiṃcit pṛcchāmaḥ 「我れも亦或ることを尋ぬ
べし」。kiṃ kurmaḥ sāṃpratam 「何を我々(=汝と我)は今爲
すべきか」。
\item 國名は複數である。事實は所屬國民の名なのである。例
Vidarbheṣu 「ヴィダルブハに於いて」。
\item 或る名詞は常に複數である。āpaḥ (女)「水」;prāṇāḥ 「生
命」,varṣāḥ 「雨」(雨季のこと),dārāḥ (男)「妻」。
\end{enumerate}
\end{enumerate}

\subsection{用例の一致}
\numberParagraph \label{np:219}
格例,人稱,性,數の一致は語尾曲法を有する言語に於
て一般に同じであるが次の諸點が注意さるべきものである。
\begin{enumerate}[label=(\arabic*)]
\item \label{item:2191} 主格が iti と伴つて呼稱,考慮,認知等の動詞に支配さ
れた賓辭的な業格の位置を取る。例 brāhmaṇa iti māṃ viddhi
「我を婆羅門であると知れ」(brāhmaṇaṃ māṃ viddhi とする
代りに)。
\item 兩數又は複數の動詞が二個又はそれ以上の主體に關係す
る時は,二人稱,三人稱よりも寧ろ一人稱,三人稱よりも寧ろ二人
稱を用ふ。例 tvam ahaṃ ca gacchāvaḥ 「汝と我れとは行く」
\item
\begin{enumerate}[label=(\alph*)]
\item 男性女性名詞に一致する兩數若くは複數の形容詞は
男性に,然し若し中性が男性女性に伴ふ時は中性にする(時とし
て單數とする)。例 mṛgayākṣās tathā pānaṃ garhitāni mahī\-%
bhujām 「狩獵と賭事と飲酒とは王にとつて避難さるべきもので
ある」。pakṣa\-vikalaś ca pakṣī śuṣkaś ca taruḥ saraś ca
jala-hīnaṃ sarpaś coddhṛta-daṃṣṭras tulyaṃ loke daridraś ca
「羽を切られた鳥,枯れた木,干上がつた池,齒を拔かれた蛇,そ
れから貧乏人は世の中に同等のものである」。(tulyam が中性單
數であることに注意)。
\item 時としては屬性や賓辭が文法的の性でなく自然の性をも
つ。tvāṃ cintayanto nirāhārāḥ kṛtāḥ prajāḥ 「汝のことを考
へつゝ(男性)臣下達は(女性)食事をする氣もなくなつた」。
\item 指示代名詞は賓辭と性に於いて一致する。例 asau para\-%
mo mantraḥ 「これは(男)最上の意見(男)だ」。

主格に一致せねばならぬ筈の定動詞の代りに用ひられた分詞が
性に於いて近くにある名詞的賓辭に引き着けられる。例 tvaṃ
me mitraṃ jātam 「汝(男)は我が友(中)となれり(中)」。
\end{enumerate}
\item 單數集合名詞は必然的に單數動詞に從はれる。二個の單
數の主語は兩數の賓辭を要し,三個以上のものは複數の賓辭を要
する。然し時には賓辭が最も近い主語と數に於て一致し,その他
のものに對しては腹で積つて補ふことになる。Kāntimatī rā\-%
jyam idaṃ mama cajīvitam api tvad-adhīnam 「カーンティ
マティーとこの王國と我が生命に至るまでも汝の意(單)に任す」。
\item 同樣に一つの複數主語に一致すべき筈の動詞がその
最も近い名詞賓辭に數の上で引き着けられることもあり得る。
sapta-prakṛtayo hy etāḥ samastaṃ rājyam ucyate 「これら
七の構成部分は全王國なりと云はれる(單)」。
\end{enumerate}

\subsection{代名詞}
\numberParagraph
\begin{enumerate}[label=(\arabic*)]
\item 人稱代名詞。
\begin{enumerate}[label=(\alph*)]
\item 動詞の語尾變化から推測し得る
性質上梵語に人稱代名詞を用ふることは若干節約せられる傾向に
ある(\ref{np:217} \ref{item:217a})。
\item 一人稱,二人稱代名詞の附帶詞(語勢なきもの)は文章の
始に立たない。又呼格の後とか ca vā eva ha の如き小辭の前に
も立つを許されない。例 mama mitram 我が友(me mitram と
は云はぬ),devāsmān pāhi 「神よ,我々を守れ」(asmān で
naḥ とは云はぬ)。tasya mama vā gṛham 「彼若くは我が家\endnote{底本では「我が家」ではなく「我がが家」。}」。
\item bhavān 「貴殿」(女性 bhavatī)は tvam 「汝」の敬稱形
であつて(同じ文章中でさへ交替に用ふ)三人稱の動詞を伴ふ。
例 kim āha bhavān 「貴下は何を仰せられますか」。複數 bha\-%
vantaḥ (女性 bhavatyḥ)同樣に解釋される。それは屢々單數の
意味を持つ(\ref{np:218} \ref{item:2183} \ref{item:2183a})。bhavān の二種の合成語が戯曲に見える。
atra-bhavān は呼びかけられた人でも第三者としての人でも或る
そこに居る人に係る。「此に在す貴殿」と云ふことである。tatra\-%
bhavān 「あの貴殿」は舞臺にゐない或る人に係り第三者のこと
に限る。共に三人稱單數の動詞を伴ふ。
\end{enumerate}
\item \textbf{指示代名詞。}
\begin{enumerate}[label=(\alph*)]
\item eṣa 及び ayam 「此れ」は近き又そこに
ゐるものに係る。前者の意味一層强し。共に一,三,單に於て主
語に一致して使はれ「此に」の意味がある。例 eṣa tapasvī
tiṣṭhati 「此に行者がゐる」,ayam asmi 「此に我あり」,ayam
āgatas tava putraḥ 「此に汝の子は來れり」,ayaṃ janaḥ 「此
の人」なる形は屢々「我」と同意味に用ひられる。
\item sa 「彼」及び asau 「それ」は居ない遠方のものに係る。sa
はこの中でも一層限定された指示代名詞で,例へば前に叙べた
ものに關係がある。次の如き特殊な用例がある。それは譯すれば
「あの有名な」とか「例の話の」とのとか云ふ意味が加はる。例
sā ramyā nagarī 「あの有名な美しい都城」。それは「前に云つ
た」と云ふことにもなる。例 so 'ham 「前に云つたやうなさうし
た私は」。名詞に伴はれない場合は三人稱代名詞として役立つ。
ayam も asau も同樣に使はれる。最後に tad が繰り返される時
には「種々の」「多樣の」「總ての種類の」と云ふ意味となる。例
tāni tāni śāstrāṇy adhyaita 「彼は種々の論書を讀めり」。
\end{enumerate}
\item 所有代名詞。īya なる後接字を附加して madīya (私の),
tvadīya (汝の)の如きものが作られる。然しこの用例は人稱代
名詞の屬格が用ひられるから比較的少い。bhavat から bhava\-%
dīya, bhavatka と云ふ二人稱の敬稱を表はす所有代名詞が作ら
れてゐる。
\end{enumerate}

\section{格の用法}
\subsection{主格}
\numberParagraph
アーリアの言語の中では梵語は文章中の主語として主格
をあまり使はない傾向にある。その代りに具格が受動の動詞と共
に用ひられる。例 kenāpi sasya-rakṣakeṇaikānte sthitam 「あ
る穀物守護者によりて一方に於いて立たれた」。この意味はある
穀物守護者が一方に立つてゐたと云ふことに歸する。

\numberParagraph
主格は「である」「と成る」「と見える」と云ふ意味の動
詞と共に又は呼稱,思惟,送致,造作の意味の動詞の受動と共
に說明的に用ひられる。例 tena muninā kukkuro vyāghraḥ
kṛtaḥ 「狗は聖者によりて虎となされたり」。iti によりて從はれ
たる主格は或る場合には業格の代用をなす。\ref{np:219} 條\ref{item:2191}。

\subsubsection{業格}
\numberParagraph
通常他動詞の目的となるのであるが,其の他に業格には
次のやうな用法がある。
\begin{enumerate}[label=(\arabic*)]
\item 動作の到着點。例 sa Vidarbhān agamat. 「彼はヴィダ
ブㇵへ行けり」。
\
\begin{enumerate}[label=(\alph*)]
\item gam, yā の如き「行く」と云ふ意味の動詞は通常抽象名
詞と結び付き「……となる」若くは自動詞で表はされるやうな意
味となる。例 sa kīrtiṃ yāti. 「彼は名譽に行く」(有名となれ
り),pañcatvaṃ gacchati. 「彼は死へ行く」(彼は死せり)。
\end{enumerate}
\item 時の繼續,場所の擴がりを意味する。例 māsam adhīte
「彼は一箇月間學べり」。yojanaṃ gacchati 「彼は一由旬行け
り」。
\end{enumerate}

\numberParagraph
\begin{enumerate}[label=(\arabic*)]
\item 二重業格は呼稱,思惟,認知,造作,選定の意義を
有する動詞に支配せられる。例 jānāmi tvāṃ prakṛti-puru\-%
ṣam 「我は汝を主なる人物なりと知れり」。
\item 其他語る (brū, vac, ah), 問ふ (prach), 乞ふ(yāc,
prārthaya), 敎ふ (anuśās), 罰する (daṇḍaya), 勝つ (ji),
搾る (duh) の意味を有する動詞。例 antarikṣago vācaṃ vyā\-%
jahāra Nalam 「鳥はナラへ語を發せり」,sākṣyaṃ pṛcched
ṛtaṃ dvijān 「彼は眞の證據を婆羅門達に尋ねたり」,baliṃ yācate
vasudhām 「彼大地に對して供物を求む」。yad anuśāsti mām 「彼
が私に命ずる所のそのもの」,tān sahasraṃ daṇḍayet 「彼は一
千(パナ)を彼等に課すべし」,jitvā rājyaṃ Nalam 「ナラより
王國を勝ち取りて」,ratnāni duduhur dharitrīm 「彼らは寶玉を
大地から搾れり」。
\begin{enumerate}[label=(\alph*)]
\item kathaya (吿ぐ),vedaya (知らしむ)及び ā-diś (命ず)
は話しかけられた人に對して業格の形を取らず。爲格(又は屬格)
となる。
\end{enumerate}
\item 持來る,運ぶ,導く,送るの意味を有する動詞。例 grā\-%
mam ajāṃ nayati. 「彼は山羊を村に持ち行く」,Śakuntalāṃ
patikulaṃ visṛjya 「シャクンタラーを夫の家に送つて」。
\item 催起動詞。例 Rāmaṃ vedam adhyāpayati 「彼はラー
マをして吠陀を學ばしむ」。若しも行爲者に强調の必要ある時は
それを具格となすことを得。tāṃ śvabhiḥ khādayet 「彼女を狗
に食はしむべし」。
\end{enumerate}

\subsubsection{具格}
\numberParagraph
具格は「……によりて」又は「……と共に」を以て表さ
れるがその根本的槪念は動作が行はれる行爲者,作具(方法)又
は並立者を示す。例 tenoktam 「彼によりて云はれたり」(彼は云
へり),sa khaḍgena vyāpāditaḥ 「彼は劍を以て殺されたり」。
yasya mitreṇa saṃlāpas tato nāstīha puṇyavān 「友と語る
より幸福なるもの世に無し」。

\numberParagraph \label{np:226}
次に示す所は具格の意味の變形である。
\begin{enumerate}[label=(\arabic*)]
\item \label{item:2261} 理由。例 bhavato 'nugraheṇa 「汝の恩惠の故に」,
tenāparādhena tvāṃ daṇḍayāmi 「我れその過失の故に汝を罰
す」,vyāghrabuddhyā 「虎の思想によりて」(彼がそれは虎なり
と考へたが故に),sukha-bhrāntyā 「幸福の妄想の故に」。
\item 一致。例 prakṛtyā 「性質上」,jātyā. 「生れながらに」,
sa mama matena vartate. 「彼は我が意見によりて(一致して)
動作す」。
\item 値段。例 rūpaka-śatena vikrīyamāṇaṃ pustakam 「一
百ルーピーで賣られる書物」,ātmānaṃ satataṃ rakṣed dārair
api dhanair api 「人は常に妻や財產を投じても自己を守らねば
ならぬ」。
\item 時間の繼續。例 dvādaśair varṣair vyākaraṇaṃ śrū\-%
yate 「文典は十二年間に學ばれる」。
\item 動作がなされる方法,媒介物,又は身體の一部分。例
katamena mārgeṇa pranaṣṭāḥ kākāḥ 「鴉は何の方法に於て消
え失せしや」,vājinā carati 「彼は馬によりて行けり」,sa
śvānaṃ skandhenovāha 「彼は狗を肩で運べり」。
\item 優性,劣性又は缺如を意味する語と共に「……に關して」。
例 etābhyāṃ śauryeṇa hīnaḥ 「彼等二人より(從格)勇氣劣れ
り」,pūrvān mahābhāga tayātiśeṣe 「お,幸福なるものよ,汝
はそれを以て祖先を凌駕す」,akṣṇā kāṇaḥ 「一目眇たり」。
\item 必要,用途 arthaḥ prayojanam (疑問又は否定に用ひら
れる)或は kim (何)(kṛ と共に若くは単獨に)等の語と伴ひて
「……について」「……を以て」と譯すべきもの。例 deva-pādā\-%
nāṃ sevakair na prayojanam 「陛下の足に對し奴僕を以て要
なし」,kiṃ tayā kriyate dhenvā 「その牝牛を以て何が爲され
るか」,kiṃ na etena 「我々はこれを以て何かせむ」,kṛtam,
alam 等の用例もこれと同樣である。kṛtam abhyutthānena
「起立する勿れ」。
\item \label{item:2268} 並列又は倶存の意味は副詞 saha, sākam, sārdham, sa\-%
mam を補つて表はされる。時には分離,對立が示される。例
putreṇa saha pitā gataḥ 「父は子供と共に行けり」。mitreṇa
saha citta-viśleṣaḥ 「友との中たがひ」,sa tena vidadhe samaṃ
yuddham. 「彼は彼と戰に從事せり」。この意義が適用されて更に
\begin{enumerate}[label=(\alph*)]
\item 同伴の狀態を表はすもの。例 tau dampatī mahatā
snehena vasataḥ 「夫妻は深き愛を以て住せり」,mahatā
sukhena. 「大なる幸福を以て」。
\item \label{item:2268b} 伴隨,結合,賦與,具有,對立の意味の動詞の受動と伴
ふもの。例 tvayā sahitaḥ 「彼に伴はれて」,dhanena saṃ\-%
panno vihīno vā 「富を具有せる又は缺如せる」,prāṇair viyu\-%
ktaḥ 「生命なき」。
\item 等同類似の意味の形容詞 sama, samāna, sadṛśa, tulya
と伴ふもの。例 Śakreṇa samaḥ 「インドラに等しき」,anena
sadṛśa 「彼の如き」,ayaṃ na me pāda-rajasāpi tulyaḥ 「彼
は我が足の塵にも及ばず」。屬格もこれらの形容詞と共に用ひら
れる。
\end{enumerate}
\end{enumerate}

\subsubsection{爲格}
\numberParagraph
爲格は間接目的(通常は人)か若くは動作の目的を表は
す。
\begin{enumerate}[label=(\Alph*)]
\item 関節目的の爲格は
\begin{enumerate}[label=(\arabic*)]
\item 直接目的を有して若くはなしに次の他動詞と共に用ひら
れる。
\begin{enumerate}[label=(\alph*)]
\item 與へる (dā, arpaya), 吿げる (cakṣ, śaṃs, kathaya,
khyāpaya, nivedaya) 約束する (prati 又は ā-śru, prati-jñā),
示す (darśaya)。例 viprāya gāṃ dadāti 「彼は牝牛を婆羅門
に與ふ」,kathayāmi te bhūtārtham 「我は汝に眞實を吿ぐ」。
\item 送る,投ぐ。例 Bhojena dūto Raghave visṛṣṭaḥ 「使は
ボージャによりてラグフに送られた」。 śūlāṃś cikṣipū Rāmāya
「彼らは鎗をラーマに投げたり」。
\end{enumerate}
\item 自動詞にして喜ぶ (ruc), 願ふ (lubh, spṛh), 怒る (asūya,
kup, krudh), 害する (druh)。例 rocate mahyam 「我にまで\ruby[S]{樂}{ねが}
はし」(そは我を喜ばす),na rājyāya spṛhaye 「我は王國を願は
ず」,kiṃkarāya kupyati 「彼は奴僕を怒る」(krudh 及び druh
が前加語を有すれば業格を支配す)。
\item 挨拶の語と共に。例 Gaṇeśāya namaḥ 「ガネーシャに
歸敬す」,kuśalaṃ te 「汝にまで健康あれ」,Rāmāya svasti 「ラ
ーマに幸福あれ」,svāgataṃ devyai 「王后にまで歡迎」。
\end{enumerate}
目的の爲格は動作のなされた目的を表はし屢々不定法と
同意義を有す。例 muktaye Hariṃ bhajati 「解脱のためにハ
リを崇拜す」。phalebhyo yāti 「彼は果實のために行く」(果實を
得んために)。asmatputrāṇāṃ nīti-śāstropadeśāya bhavantaḥ
pramāṇam 「貴下は倫理書の敎訓に關して吾が子等の證權であ
る」,yuddhāya prasthitaḥ 「彼は戰爭に出立つた」,punar
darśanāya 「再會を期して,さよなら」。
\begin{enumerate}[label=(\arabic*)]
\item 「適當である」,「……に歸す,(或は)に資す」(kḷp, saṃ-pad,
pra-bhū)。例 bhaktir jñānāya kalpate 「信仰は知識に資す」。
\begin{enumerate}[label=(\alph*)]
\item as と bhū が同樣の方法で用ひられる。然し屢々省略せ
られる。例 laghūnām api saṃśrayo rakṣāyai bhavati 「弱き
ものゝ附着は保護に資す」,ārta-trāṇāya vaḥ śastram 「汝の
武器は苦しめられたるものゝ救ひに役立つ」。
\end{enumerate}
\item 能ふ,始む,努む,決心す,命ず,任ず。例 iyaṃ kathā
kṣatriyasyākarṣaṇāyāśakat 「この物語は勇士の心を引くに堪
ふ」,prāvartata śapathāya 「彼は誓を始めた」,tad-anveṣaṇāya
yatiṣye 「我はそれを求めんと努力すべし」,tena jīvotsargāya
vyavasitam 「彼は彼の命を捨てんと決心せり」,duhitaram
atithi-satkārāyādiśya 「娘に賓客の接待を委かせて」,Rāvaṇo\-%
cchittaye devair niyojitaḥ 「ラーヷナの討滅にと神々によりて任
命せられた」。
\begin{enumerate}[label=(\alph*)]
\item 副詞 alam (十分に)は「……と匹敵す」,「……と相競ふ」
の意に用ひられる。例 daityebhyo Harir alam 「ハリは惡魔
に敵對す」。
\end{enumerate}
\end{enumerate}
\end{enumerate}

\subsubsection{從格}
\numberParagraph
從格は元來何かゞ進出する所の出發點若は源泉地を表は
すものである。それは「何處から」と云ふ問に答へる。隨て一
般に「から」と譯される。例 aham asmad vanād gantum
icchāmi 「私はこの林から出立せんと欲する」,pāpān nāśa ud\-%
bhavati. 「罪過より破滅は生ず」,niścayān na cacāla saḥ 「彼は決
心から動かざりき」,sva-janebhyaḥ suta-vināśaṃ śuśrāva 「彼は
親族から子の死を聞けり」,tāṃ bandhanād vimucya 「彼女を束
縛から解放して」,virama karmaṇo 'smāt 「この行爲を止めよ」,
pāhi māṃ narakāt 「我を地獄から護れ」(地獄に墮ちぬやうに)。
\begin{enumerate}[label=(\alph*)]
\item \ruby{危懼}{き|く}の原因は恐怖の動詞 (bhī, ud-vij) と共に從格にす
る。例 lubdhakād bibheṣi 「汝は獵師を恐る」,saṃmānād
brāhmaṇo nityam udvijeta 「婆羅門は常に尊敬を受くることを
恐るべきだ」。
\item 別離を表はす動詞は自然に從格を取る。例 bhavabdhyo
viyojitaḥ 「汝から別れた」,sā pati-lokāc ca hīyate 「而して彼女
は夫の處から離されてゐる」(この種の語は具格にもなる)(\ref{np:226}條
\ref{item:2268} \ref{item:2268b})。これと連絡して vañcaya (欺く)の用例(結局は正しい狀
態から離れることである)。vañcayituṃ brāhmaṇaṃ chāgalāt
「婆羅門を山羊で欺かんと」。
\item 從格は起點 (terminus a quo) を表はすものだから遠
方を意味する總ての語及び方位の名稱に伴ふものである。例
dūraṃ grāmāt 「遙かに村から」,grāmāt pūrvo giriḥ 「山は村
の東方にあり」。
\item 同樣に從格は又何かゞ起つた後の時間を表はす。例
bahor dṛṣṭaṃ kālāt 「長い時の後それは見られたり」,saptāhāt
「七日の後」。
\item ārabhya (……より始めて),prabhṛti (以來)等の語に
伴ふ。例 tataḥ prabhṛti 「それより以來」。又 ā なる前置詞に
伴ふ。例 ā nagarāt 「市城に至るまで」,ā mūlāt 「根本より」。
\end{enumerate}

この根本的な意義から次のやうな意義を表はす。
\begin{enumerate}[label=(\arabic*)]
\item 原因,理由,動機。例 laulyād māṃsaṃ bhakṣayati
「彼は貪慾の故に肉を食ふ」。この用法では特に tva を附加した
抽象名詞で註釋などがなされるのを普通とする。例 parvato
'gnimān dhūmatvāt 「山は燃えてあり,烟あるが故に」(具格もこ
の意味に用ひられる(\ref{np:226} \ref{item:2261})。
\item 比較。
\begin{enumerate}[label=(\alph*)]
\item 比較を表はす例 Govindād Rāmo vidvat\-%
tarah 「ラーマはゴーヴィンダよりも學識あり」。karmaṇo jñānam
atiricyate 「知識は行爲よりも優れたり」。この意味で原級の言ひ
表はし方もある。例 bhāryā sarva-lokād api vallabhā bhavati
「妻は全世界と比しても親愛なり」。
\item 「他の」,「異れる」と云ふ意味の語 (itara, apara, bhinna,
pṛthak) と共に。例 Kṛṣṇād anyo Govindaḥ 「ゴーヴィンダは
クリシュナに異る」。
\item 比較を表はす從格に連關して「二倍」とか「三倍」とか
の語と共に用ひられることがある。例 mūlyāt pañca-guṇo
daṇḍaḥ 「價よりご倍の罰金」。
\end{enumerate}
\end{enumerate}

\subsubsection{屬格}
\numberParagraph \label{np:229}
屬格の本來の意味は疑似的形容詞であり,他の名詞を限
定して「……に屬する」とか「……に關する」とかの意義を有す
る。伴ふ名詞に對してその用法から所有的,主觀的,客觀的,表
分的の區別がある。例 rājñaḥ puruṣaḥ 「王の臣下」(所有)。
rākṣasa-kalatra-pracchādanaṃ bhavataḥ 「貴下の羅刹女隱匿」
(汝が隱した)(主觀)。śaṅkatā tasyāḥ 「彼女の疑によりて」(彼
女ならむとの疑)(客観)(彼女は目的)。dhuryo dhanavatām
「富める人々の中での首班」(部分を表はす)。
\begin{enumerate}[label=(\arabic*)]
\item 屬格は多くの動詞と共に用ひられる。
\begin{enumerate}[label=(\alph*)]
\item 所有的意義では pra-bhū (……を支配する),as, bhū (有
る),vidyate (存在す)。例 ātmanaḥ prabhaviṣyāmi 「我は自
己を支配すべし」。mama pustakaṃ vidyate 「我が書はあり」。
\item 客觀的意義に於いて(業格と同樣に) day (惠む),smṛ
(念ず),anu-kṛ (擬す)と共に用ひられる。例 ete tava dayantām
「これらの人々をして汝の上に惠あらしめよ」,smarati te pra\-%
sādānām 「彼は汝の惠を念ず」,Bhīmasyānukariṣyāmi 「我はブ
ヒーマに倣ふ」。
\item 客觀的意義に於いて(於格と同樣に)「……に益する又は害
をなす」(upa-kṛ, pra-sad, apa-kṛ, apa-rādh), 「……に信賴す
る」(vi-śvas) 「……を忍ぶ」(kṣam) と共に用ひられる。例
mitrāṇām upakurvāṇaḥ 「友を利しつゝ」,kiṃ mayā tasyā
apakṛtam 「如何に我によりて彼女に害をなせしか」,kṣamasva
me 「我を許せ」。
\item \label{item:229d} 「……に就て語る」,「……を期待す」なる意義の動詞に伴
ふ。例 mamādoṣasyāpy evaṃ vadati 「彼は過失なき我に就て
すらかく語る」,sarvam asya mūrkhasya saṃbhāvyate 「一切
はその愚者に對して期待せらる」。
\item 屢々(爲格の間接目的の代りに)贈與,報知,約束,呈示,
送致,禮敬,歡悅,憤怒を意味する動詞と共に用ひられる。mayā
tasyābhayaṃ pradattam 「我れ彼に無畏を與へたり」,kiṃ tava
rocata eṣaḥ 「彼は如何に汝を喜ばすか」。mamānatikruddho
muniḥ 「聖者は甚しく我に不機嫌でない」。
\item 時として(業格の代りに)「滿足す」「飽く」を意味する動
詞と共に用ひられる。例 nāgnis tṛpyati kāṣṭhānām 「火は材
木にて滿足せしめられない」。
\end{enumerate}
\item 屬格は屢々形容詞と共に用ひられる。
\begin{enumerate}[label=(\alph*)]
\item 他動詞の如き役目をなす。例 jarā vināśinī rūpasya
「老齡は美貌を害す」。
\item 又「……に依存する」「……屬する,着く」「……に親愛な
る」の意義あるもの。例 tavāyattaḥ sa pratīkāraḥ 「療法は
汝に依存す」,yat tvayāsya saktaṃ kiṃcid gṛhītam asti tat
samarpaya 「彼のものと汝が執着せる如何なるものもそれを捨
てよ」,ko nāma rājñāṃ priyḥ 「王に親愛なるは抑も誰ぞ」。
\item 「……を熟知せる」「……に通達せる」「……に慣れたる」の
意義あるもの(於格と同樣に)。abhijñaḥ khalv asi loka-vyava\-%
hārāṇām 「汝は實に世の事柄に通じてゐる」,saṃgrāmāṇām
akovidaḥ 「戰に慣れざる」,ucito janaḥ kleśānām 「苦痛に慣れ
たる人々」。
\item 「……の如き」「……等しき」の意義あるもの(具格と同樣に)。
例 Rāmaḥ Kṛṣṇasya tulyaḥ 「ラーマはクリシュナに等し」。
\end{enumerate}
\item 屬格は受動分詞と共に行爲者を表はす。
\begin{enumerate}[label=(\alph*)]
\item 思惟,認知,禮拜を意味する語根から作られた現在の意
義を有する過去分詞。例 rājñāṃ mataḥ 「王に\ruby{嘉}{よみ}せられる」,
vidito bhavān āśrama-sadām iha-sthaḥ 「貴下は此に住せりと
仙者達に知られてゐる」。
\item 未來分詞(具格も同樣に)。例 mama (mayā) sevyo
Hariḥ 「ハリは私に奉仕せらるべし」。
\end{enumerate}
\item 屬格は tas と云ふ方向の副詞と共に用ひられる。例
grāmasya dakṣiṇataḥ 「村の南方に當りて」。時として亦 ena
を語尾とする福祉と伴ふ(業格と同樣に)。例 uttareṇāsya 「この
北方に」。
\item 時の屬格は次のやうな方法で用ひられる。
\begin{enumerate}[label=(\alph*)]
\item 倍加數又は同樣に他の數にして,ある特定の期間に幾度何
かゞ繰り返されしかを表はす。例 śrāddhaṃ trir (\ref{np:118}條)
abdasya nirvapet 「葬祭を年に三度行ふべきだ」,saṃvatsara\-%
syaikam api caret kṛcchraṃ dvijottamaḥ 「婆羅門は年に一度
でも苦行をなすべきだ」。
\item 時間を表はす語は「後」と云ふ意味で,屬格になされる(從
格の場合のやうに)。例 kati-payāhasya 「若干の日の後」,
cirasya kālasya 「長い時の後」。(cirasya だけでもこの意味にな
る)。
\item 名詞と時の表現をもつ屬格の分詞とで「……より以來」と
云ふ意義を有つ。例 adya daśmo māsas tātasyoparatasya
「今日は父の死せしより第十箇月目である」,daśa kalpā anutta\-%
rāṃ samyaksaṃbodhim abhisaṃbuddhasya 「無上なる正等覺を
\ruby[S]{證}{さと}りしより以來十劫を經た」。この構造は獨立屬格に類するもの
である。獨立屬格のことは後段に詳かにする。
\end{enumerate}
\item 二つの屬格は二者の選擇又は差異を表はす。例 vyasana\-%
sya ca mṛtyoś ca vyasanaṃ kaṣṭam ucyate 「惡德と死とは惡德
の方劇甚なりと云はれる」。etāvān evāyuṣmataḥ Śatakratoś ca
viśeṣaḥ 「それだけが御身とインドラとの差異である」。
\end{enumerate}

\subsubsection{於格}
\numberParagraph
於格は動作の起る場處又は運動の動詞に伴ひ動作の向け
られる場處を示す。前者は大體「……の中に」と譯され,後者は
「……に於いて」と譯される。
\begin{enumerate}[label=(\arabic*)]
\item 次に於格の「何處に」の意味の通常の用例を擧げる。pak\-%
ṣiṇas tasmin vṛkṣe nivasanti 「鳥はその木に棲む」,Vidarbheṣu
「ヴィダルブハに於いて」,ātmānaṃ tava dvāri vyāpādayiṣyāmi
「汝の戸口に於いて我は自殺すべし」,Kāśyām 「カーシーに於い
て」,phalaṃ dṛṣṭaṃ drumeṣu 「果實は木に於いて見られたり」,
āsedur Gaṅgāyām 「彼らはカンガー河に野營せり」,na deveṣu na
yakṣeṣu tādṛg rūpavatī kvacid mānuṣeṣv api cānyeṣu dṛṣṭa\-%
pūrvā 「神々の中にもヤクシャの中にも他の人々の中にもかくの
如き美貌は曾て見られざりき」,mama pārśve 「我の側に」。
\begin{enumerate}[label=(\alph*)]
\item 若し於格が「……の中に」と云ふ意味ある時は表分的屬
格に等しくなる。例 sarveṣu putreṣu Rāmo mama priyata\-%
maḥ 「すべての子等の中にラーマは我が最愛のものである」。
\item 共に住し滯在する所の人は於格の形を取る。例 gurau
vasati 「彼は彼の敎師と共に住む」。
\item 於格は tiṣṭhati (住す)及び vartate (進む)なる動詞と
共に「……に從ふ」「……に應ず」の意を表はす。例 na m
śāsane tiṣṭhasi 「彼は我が命令に從はない」,mātur mate var\-%
tasva 「母の願に應ぜよ」。
\item 於格は原因の結果を表はす。例 daivam eva nṛṇāṃ
vṛddhau kṣaye kāraṇam 「運命こそは人々の繁榮と滅亡の原因
である」。
\item 於格は「摑む」(graph),「結びつく」(bandh)「固着する,
(lag, śliṣ, sañj) 「\ruby{凭}{もた}れる」「依賴又は信用する」(譬喩的に)なる
動詞と共に接觸を表はす。例 keśeṣu gṛhītvā 「髪をつかみて」
pāṇau saṃgṛhya 「手を取りて」,vṛkṣe pāśaṃ babandha 「彼
は縄を木に結べり」,vyasaneṣv asaktaḥ śuraḥ. 「勇士は惡德
に耽らない」,vṛkṣa-mūleṣu saṃśritāḥ 「樹の根に椅れり」,
viśvasiti śatruṣu 「彼は敵に信賴せり」,āśaṃsante surā asyā\-%
dhijye dhanuṣi vijayam 「神々は勝利をその張られた弓に持ち
設けてゐる」。
\item 於格は「……を熟知す」「……に通曉す,練達す」と云ふ
形容詞に伴ふ(屬格と同樣)。例 Rāmo 'kṣadyūte nipuṇaḥ 「ラ
ーマは博戯に巧みである」,nāṭye dakṣā vayam 「我々は劇に練
達す」。
\item 於格は譬喩的に何かある性質狀態が存する人若くは物を
表はす。例 sarvaṃ saṃbhāvayāmy asmin 「我は一切を彼に
期待す」(\ref{np:229} \ref{item:229d}),dṛṣṭa-doṣā mṛgayā svāmini 「獵は王子にとつ
て過惡が見られる」,ārtānām upadeśe na doṣaḥ 「困しめる
ものに對し忠吿するは過失でない」。同樣に言語の意味が說明
されるに當り於格は「……の意義で」と云ふことを表はす。例
kalāpo barhe 「kalāpa と云ふは孔雀の尾と云ふ意味で使はれ
てゐる」。
\item ある動作が起る事情が於格で表はされる。例 āpadi 「不
幸の場合に」,bhāgyeṣu 「幸福に於て」,chidreṣv anarthā
bahulī-bhavanti 「孔隙に於て不幸は增大す」。この一例では於格
が理由を表はす。「弱點があるところで」の意味であり「……に面
して」の意味である,若しも賓辭の分詞が伴ふと獨立於格となる
ものである。
\item 時間の於格,動作の起る時を表はす。これは前の意義の唯
一特殊な適用である。例 varṣāsu 「雨季に於て」,niṣāyām 「夜
に於て」,dine dine 「毎日」。
\item 於格はある事の起つた距離を表はす。例 ito vasati......
adhyardha-yojane maharṣiḥ 「大仙は此處から一由旬半のとこ
ろに住む」。
\end{enumerate}
\item 「何れの點に」といふ疑問に答へる於格は常に落つ,投ぐ,
置くと云ふ動詞並びに業格と同樣に,行く,入る,登る,打つ,齎
らす,送ると云ふ動詞と共に用ひらる。例 bhūmau papāta 「地
上に落ちた」,arau bāṇāt kṣipati 「彼は敵に對して箭を發射
す」,tatraiva bhikaṣā-pātre nidhāya 「その同じ乞鉢に置きて」,
hastam urasi kṛtvā 「胸に手を置きて」,matsyo nadyāṃ
praviveśa 「魚は河に入れり」,samīpa-vartini nagare prasthi\-%
taḥ 「彼は隣りの街へ出立せり」。

この於格の第二義的の用法は次の如くである。
\begin{enumerate}[label=(\alph*)]
\item 動作の向けられ又は關係する所の人若くは物を表はす。
「……の方へ」「……について」「……に關して」の意を有す。例
prāṇiṣu dayāṃ kurvanti sādhvaḥ 「善人は生類に慈愍を行
ふ」,bhava dakṣiṇā parijane 「汝の從者に禮儀正しかれ」,kṣetre
vivadante 「彼等は田地を爭ふ」。
\item 爲格(並に屬格)と同樣に,贈與,報知,約束,購買,賣渡
の意義ある動詞と共に間接目的を表はす。例 sahasrākṣe prati\-%
jñāya 「千眼(のインドラ)に約束して」,śarīraṃ vikrīya dhana\-%
vati 「富人に身體を賣りて」,vitarati guruḥ prājñe vidyām
「師は知識を智慧ある(弟子)に分つ」。
\item 爲格と同樣に,努力,決心,願望,任命,選擇,參加,許
可,可能,適應を意味する語と共に動作の目的を表はす。例
sarvasva-haraṇe yuktaḥ śatruḥ 「敵は一切の奪取を準備せり」,
karmaṇi nyayuṅkta 「彼は事業を任命せり」,patitve varayāmāsa
tam 「彼女は彼を夫と選べり」,asamartho 'yam udara-pūraṇe
'smākam 「彼は我々の口腹を滿すに適せず」。trailokyasyāpi
prabhtvaṃ tasmin yujyate 「三界の主も彼に適當なり」。單
に賓辭的に於格を用ひたるのみで適當を表はす。例 naya-tyāga\-%
śaurya-saṃpanne puruṣe rājyam 處世の知識,捨施,勇氣を
具有せる人に王位は(適當なり)」。
\item 願望,專心,尊敬,友情,信賴,同情,侮蔑,等閑視を意
味する名詞は屢々(屬格と同じく)これらの感情の向けられた
事物の於格と共に用ひられる。例 na kalu Śakuntalāyāṃ
mamābhilāṣaḥ 「吾が愛は決してシャクンタラーに對してあるの
でない」,na me tvayi viśvāsaḥ. 「私は彼に信賴せぬ」。na
laghuṣv api kartavyeṣv anādaraḥ kāryaḥ 「義務の怠慢はたと
ひ小なりともなさるべからず」。
\item 於格は,同樣に好む,專心なる,熱心なる,及びその反對
を意味する形容詞又は過去分詞と共に用ひられる。例 nāryaḥ
kevalaṃ svasukhe ratāḥ 「婦人は只自身の幸福を樂しむ」。
\end{enumerate}
\end{enumerate}

\subsubsection{獨立於格及び屬格}
\numberParagraph \label{np:231}
二個の動作の主體が一文中に在る場合に獨立於格又は屬
格が用ひられる。前者が寧ろ普通に見る所である。例 gacchatsu
dineṣu nṛpo pṛcchati 「日を經て王は尋ねたり」,goṣu dugdhāsu
sa gataḥ 「牡牛の搾られた時彼は出立せり」,karṇaṃ dadāti
mayi bhāṣamāṇe 「我が語る時彼女は耳傾く」。
\begin{enumerate}[label=(\alph*)]
\item 獨立於格の賓辭は實際には常に分詞である。例外は sat
(ある)なる分詞が屢々省略されることである。例 kathaṃ
karma-kriyā-vighnaḥ satāṃ rakṣatari tvayi. 「君が守護者な
る時,如何に善人の義務遂行に障礙がありませう」。
\item 分詞 sat (ある)(又はその同意語 vartamāna 及び sthita)
が屢々無意味に他の獨立分詞に附加せられることがある。例
sūryodaye 'ndhatāṃ prāpteṣūlūkeṣu 「日昇りて梟は盲目
となる時」。
\item 主語は過去分詞が非人稱的に用ひられる時勿論常に省略
せられる。又分詞が evam, tathā, ittham, iti の如き不變化語
に伴ふ時も省略せられる。例 tenābhyupagate 「彼によりて承
認せられし時」,evaṃ gate. 「よくあることだが」(直譯。かく
行きし時」,tathā kṛte sati (tathānuṣṭhite) 「これがなされて
ある所で」。
\end{enumerate}
\begin{enumerate}[label=(\arabic*), start=2]
\item 獨立屬格は獨立於格より遙かに用例少く而もその適用に
一層制限がある。それは同時の動作に限られ,主語は人であらね
ばならず,賓辭は形に於いてか意義に於いてか現在分詞である。
その意味は「……の間に」「……であるのに」「……の時に」と譯すべ
きである。例 paśyato me paribhraman 「我が見つゝあるにも拘
らず彼は彷徨しつゝ」,evaṃ vadatas tasya sa lubdhako nibhṛ\-%
taḥ sthitaḥ 「彼がかくの如く語る間に獵師は匿れゐたりき」,iti
cintayatas tasya tatra toyārtham āyayuḥ striyaḥ 「彼がかく
考えつゝありし間に婦人たちはそこへ水を汲まんと來れり」。
\end{enumerate}

\subsubsection{分詞}
\numberParagraph \label{np:232}
分詞は梵語に於て形容詞の性質を有し,その限定する名
詞と性數格に於て一致する。分詞は屢々動詞の機能を有す。例
śṛgālaḥ kopāviṣṭas tam uvāca 「豺は怒に滿たされ彼に云へり」。
niṣiddhas tvaṃ mayānekaśo na sṛṇoṣi 「汝は幾度となく我に
よりて諫止せられしも汝は聞かざりき」,ajalpato jānatas te
śiro yāsyati khaṇḍaśaḥ 「知りつゝ若しも吿げざるならば汝の頭
は分々に裂くべし」,tāḍayiṣyan Bhīmaṃ punar abhyadravat
「打たんとしつゝ復び彼はブヒーマに走りかゝれし」。
\begin{enumerate}[label=(\arabic*)]
\item 現在分詞。この分詞(並に現在の意味を有する過去)は
asti 又は bhavati (彼はあり),āste (坐す),tiṣṭhati (立つ),
vartate (進行す)と共に續く動作を表はすに用ひられる。例
etad eva vanaṃ yasminn abhūma ciram eva purā vasantaḥ
「これは實に我々が已前長らく住しつゝありし同じ林である」,
bhakṣayann āste 「彼は食しつゝ坐せり」,sā yatnena rakṣya\-%
māṇā tiṣṭhati 「彼女は注意深く護られつゝあり」,paripūrṇo
'yaṃ ghaṭaḥ saktubhir vartate 「この甕は麥粉で滿ちてある」。
\item 過去分詞。ta に終る受動分詞,及び vat に終るその能
動形(時として vas に終る第二過去の能動分詞さへも)屢々定
動詞として用ひられる(繫辭は省略)。例 tenedam uktam 「こ
れは彼によりて云はれたり」,sa idam uktavān 「彼はこれを云
へり」。
\begin{enumerate}[label=(\alph*)]
\item 自動詞の受動は非人稱的に用ひられる。さもなくば能動
の意味を有す。例 mayātra ciraṃ sthitam 「長らくそこに私
は立てり」,sa Gaṅgāṃ gataḥ 「彼はガンガーへ行けり」,sa
pathi mṛtaḥ 「彼は路上に死せり」。
\item 或る ta に終る過去分詞は受動と他能動の兩義を有す。
例 prāpta 「得られたる」「到達したる」,praviṣṭa 「……に入ら
れたる」「入りたる」,pīta 「飲まれたる」「飲みたる」,vismṛta
「忘れられたる」「忘れたる」,vibhakta 「分けられたる」「分けた
る」,prasūta 「生まれたる」「生みたる」,ārūḍha 「乘られたる」
「駕せる」。
\item na に終る受動分詞は決して他能動の意味になることが無
い。
\end{enumerate}
\item 義務分詞。これは必須,義務,適當,蓋然を表はし。未
來受動の意を有す。例 mayāvaśyaṃ deśāntaraṃ gantavyam
「我は必然に外國に行かねばならぬ」,hantavyo 'smi na te rājan
「王よ,汝は我を殺してはならぬ」,tatas tenāpi śabdaḥ kar\-%
tavyaḥ 「かくて彼も亦聲を立てるならむ」。
\begin{enumerate}[label=(\alph*)]
\item 時としては未來受動分詞は純粹未來の意味を有す。例
yuvayoḥ pakṣa-balena mayāpi sukhena gantavyam 「汝の翼
の力にても我も亦易々と行くべし」。
\item bhavitavyam 及び bhāvyam は必須又は高度の蓋然を表
はす。賓辭たる形容詞又は名詞は具格に於て主語に一致す。例
tayā saṃnihitayā bhavitavyam 「彼女は必ずや近くにゐなけれ
ばならぬ」,tasya prāṇino balena sumahatā bhavitavyam 「その
獸の力は甚大なるべし」。
\end{enumerate}
\item 連續體。一の動作が始まる前に他の動作が終ることを表
はす(稀には同時のこともある)。主要動作の文典的主體は主格
と一致し,受動の構成では具格と一致する。然し時としては他の
格に一致することもある。例 taṃ praṇamya sa gataḥ 「彼に
禮して彼は行けり」,atha tenātmānaṃ tasyopari prakṣipya
prāṇāḥ parityaktāḥ 「かくて彼は自身を彼の上に投じて命終れ
り」(「投じて」は tena と一致す)。tasya dṛṣṭvaiva vavṛdhe
kāmas tāṃ cāruhāsinīm 「彼の愛は美しき微笑もつ彼女を見て
增し行けり」(「見て」は tasya に一致す)。
\begin{enumerate}[label=(\alph*)]
\item 連續體は前置詞の如く用ひられる。その用法大體次の如
くである。

具格と共に

\indent{uddiśya 「……の方へ」「……に就て」「……に對して」}

\indent{ādāya, gṛhītvā 「とりて」,nītvā 「……を以て」}

\indent{adhiṣṭhāya, avalambya, āśritya, āsthāya 「……によりて」
  「……のために」}

\indent{muktvā, parityajya, varjayitvā, sthāpayitv;a 「……を除
  きて」}

\indent{adhikṛtya 「……に關して」「……に就て」}

從格と共に

\indent{ārabhya 「……より始めて」「……以來」。}
\item 連續體の原形は名詞の具格である。これが kim (何)又
は alam (十分)なる語と共に用ひられて次のやうに元來の意義を
表はす。

\indent{kiṃ tava gopāyitvā 「隱匿して汝に何の(益)ありや」。}

\indent{alaṃ te vanaṃ gatvā 「林に行くを止めよ」。}

又は受動形で表はされた汎意の主語と共に。

\indent{paśūn hatvā yadi svarge gamyate 「若し人獸を殺して天
  界を往くならば」。}
\end{enumerate}
\end{enumerate}

\subsubsection{不定法}
\numberParagraph
動作の目的を表はすもので一般に目的の爲格に伴ふ。そ
の目的たる事物を業格となし屬格としないので通常の名詞の爲格
と異る。例 taṃ jetuṃ yatate. 「彼は彼に勝たんと努む」と同意であ
る。tasya 屬格)。動詞の直接目的であるから元來の業格の形を
保存する。例 snātuṃ labhate 「彼は沐浴する」。

而して文章の主語とはならぬ。動詞的名詞は通常主語としてそ
の場處を補ふ。例 varaṃ dānaṃ na tu pratigrahaḥ 「與ふる
は取るに勝る」。

不定法は或は名詞と共に用ひられて「時」「機會」を意味し,或
は形容詞と共に用ひられて「適當」「可能」の意となり,或は動詞
と共に「可能」「願望」「開始」の意となる。例 nāyaṃ kālo
vilambitum 「今は遷延すべき時ならず」,avasaro 'yam ātmā\-%
naṃ prakāśayitum 「今は自身を現はす機會なり」,likhitam
api lalāṭe projjhituṃ kaḥ samarthaḥ 「誰か額上に書かれたる
ものを免れ得べき」,ahaṃ tvāṃ praṣṭum āgataḥ 「我れ汝に
尋ねんがため來れり」,kathayituṃ śaknoti 「彼は吿げ得」,
iyeṣa sā kartum 「彼女は爲さんと欲せり」。

\begin{enumerate}[label=(\alph*)]
\item 語根 arh 「値す」の二人稱,三人稱,單數,現實法は鄭
重なる命令の意を含む。例 bhavān māṃ śrotum arhati 「貴
下は何卒私に聽きたまへかし」。
\item 末尾の m を除きたる不定法の形がその動詞の表はす所
をなさんと欲するの意に用ひられる。その合成語は有財釋の形を
取る。例 draṣṭu-kāmaḥ 「見んと欲す」,kiṃ vaktu-manā bha\-%
vān 「何を貴下は語らんとしたまうや」。
\item 梵語の不定法には受動形がない。受動の意味を表はすに
は不定法を支配する動詞が受動に作られる。例 kartuṃ na
yujyate 「そはなされるに適當でない」。例 mayā nītiṃ grāha\-%
yituṃ śakyante 「我によりて儀禮が敎へられ能ふ」,tena
maṇḍapaḥ kārayitum ārabdhaḥ 「彼によりて小屋は作らせられ
始めたり」。
\item 未來,受動分詞 śakya は主語と一致し,又は中性單數
の形となされ得。例 na śakyās te (doṣāḥ) samādhātum 「こ
れらの(過失)は償ひ難し」,sā na śakyam upekṣituṃ kupitā
「怒れる彼女は等閑視すべきでない」,yukta, nyāyya 「適當な
る」も同樣に使用せらる。例 seyaṃ nyāyyā mayā mocayituṃ
bhavattaḥ 「彼女は私によつて貴下から赦免されても宜しい」。
\end{enumerate}

\section{時と法}
\subsection{現在時}
\numberParagraph
現在時の使用法にやや特有なものは
\begin{enumerate}[label=(\arabic*)]
\item 說話の上に歴史的現在が屢々用ひられる,それは持續する
意味を有す。例 Damanakaḥ pṛcchati katham etat 「ダマナ
カは尋ねた。これは何ぞや」,Hiraṇyako bhojanaṃ kṛtvā bile
svapiti 「ヒラヌヤカは食物を作りて穴に於いて眠れり」。
\begin{enumerate}[label=(\alph*)]
\item 時として現在形に purā 「曾て」「以前に」なる副詞が
伴ふ。例 kaśmiṃś cid vṛkṣe purā 'haṃ vāsāmi 「我れ曾て
とある樹下に住したり」。

又屢々 sma なる語が伴ふ。この語は古代梵語では屢々 purā
に伴ひしため遂に單獨に用ひられてもその意味を有するに至つた
ものである。例 kaśmiṃś cid adhiṣṭhāne Somiliko nāma kauliko
asati sma 「とある場處にソーミリカなる織師が住みたりき」。
\item 現在は直接に連絡する過去を意味して使はれる。例
ayam āgacchāmi 「此に我來れり」(今しも恰度來たところである。
現在完了の意)。
\end{enumerate}
\item 現在は又近接せる未來の意を有す。時として purā 「すぐ
に」,yāvat 「丁度」なる語が附加せられる。例 tarhi muktvā
dhanur gacchāmi 「かくて弓を棄てて我は去らむ」,tad yāvac
Chatrughnaṃ preṣayāmi 「故に我は今シャトルグフナを遣はさ
む」。
\begin{enumerate}[label=(\alph*)]
\item 疑問詞と伴ひて將來の動作に關する疑を含む。例 kiṃ
karomi 「我れ何をなすべきや」。
\item 又時には直に或行動をなせと云ふやうな勸吿の意を有す。
例 tarhi gṛham eva praviśāmaḥ 「今や我々は家に入るべ
きだ」。
\end{enumerate}
\end{enumerate}

\subsection{過去時}
\numberParagraph
過去と云はゞ第一,第二,第三は固より,その他 -ta や
-tavat を附した過去分詞,歴史的現在も區別なく遠き時の經過
を表し,等しく一度起つた,又は反覆された,又は連續的な事實
を示す。
\begin{enumerate}[label=(\alph*)]
\item 第二過去は遠き時の經過せし事實の記述に限られ,これ
は記述者の經驗內のことでない。隨てこの過去の一人稱,二人稱
は極めて稀である。
\item 第一過去は歴史的過去を記述する上に記述者自身が目賭
せし過去の事実を述べる。
\item 第三過去は(-ta や -tavat を附加する分詞も含めて)現
在完了の特殊な意味がある。故に對話の上に適用せられる。例
abhūt saṃpādita-svādu-phalo me manorathaḥ 「我が願望は美
はしき結果を齎した」,tubhyaṃ mayā rājyam adāyi 「私は汝
に王位を與ふ」,taṃ dṛṣṭavān asmi 「我れ彼を見たり」。
\item 過去符なき第三過去(稀に第一過去)は否定辭 mā と共
に命令の意を有す。
\item 完全過去の形は梵語に存しない。この意味は前後の關係
より推知せられ,他の過去形連續體を以て表現せられる。
\end{enumerate}

\subsection{未來時}
\numberParagraph
單未來は一般に未來の動作に關する記述に用ひられ,複
說未來は遠き未來に限られ幾分用ひられることも少い。故に雙方
共に同一の動作を記述するに用ひられ屢々互用せられる。
\begin{enumerate}[label=(\alph*)]
\item 未來は時として命令法を伴つて命令の意味を有す。例
bhadre yāsyasi mama tāvad arthitvaṃ śrūyatām 「善良なる
ものよ,汝は行くべし。されどまづ我が願が聽かるべし」。
\end{enumerate}

\subsection{命令法}
\numberParagraph
通常の\ruby{禁遏}{きん|あつ}又は勸吿の意味の外に特殊の用法は次の如き
ものである。
\begin{enumerate}[label=(\alph*)]
\item 第一人稱命令は古代の接續法の痕跡を存するもので「…
…であらう」又は「……をして……せしめよ」と譯すべきだ。例
dīvyāvety (dīvyāva iti) abravīd bhrātā 「兄弟は云へり。い
ざ我々をして博戯せしめよ」,ahaṃ karavāṇi 「我はなすであら
う」。
\item 三人稱,單數,受動,命令法は通常二人稱,能動のそれ
よりも丁寧なる用法である。例 deva śrūyatām 「君よ,願は
くは聽きたまへ」。
\item 願望,祝福を表はすに可能法や願望法を用ふる代りに命令
法が用ひられる。例 ciraṃ jīva 「長壽なれ」,śivās te
panthānaḥ santu 「汝の道程は幸福なれ」。
\item 疑問詞と共に可能性又は疑惑を表はす。例 viṣaṃ bha\-%
vatu mā vāstu phaṭāṭopo bhayaṃkaraḥ 「毒あるも又は無き
も蛇頭の膨大は恐ろし」,kim adhunā karavāma 「我々は今何を爲す
べきか」。
\item 否定辭 mā を伴ふ命令法は比較的稀である。通常は過去
符なき第一過去を以て作られる。又は具格に伴ふ alam 或は
kṛtam によりて表現せられる。
\end{enumerate}

\subsection{可能法}
\numberParagraph
その一般的性質は能力の存在を現はすにあるがその他に
現在梵語に消滅せる接續法に係る意味が種々の點から影の如くに
伴隨する。
\begin{enumerate}[label=(\arabic*)]
\item 主文章に於てはそれは次の如き意味を表はす。
\begin{enumerate}[label=(\alph*)]
\item 願望(屢々 api なる語を加へて)。例 api paśyeyam iha
Rāghavam 「此に我れラーマ王を見るを得たらましかば」。
\item 可能性又は疑惑。例 kadācid go-śabdena budhyate 「い
つか恐らく彼は牛の聲にて目覺めたらむ」,paśyeyuḥ kṣitipata\-%
yaś cāra-dṛṣṭyā 「王達は諜者の目もて見るを得るのだ」。ekaṃ
hanyān na vā hanyād iṣur mukto dhanuṣmatā 「射手の放ち
し箭は一人を傷けしか又は傷けざりしならむ」。
\item 蓋然性。この場合屢々未來に等しい。例 iyaṃ kanyā
nātra tiṣṭhet 「この少女は其處に止まらざるべし」。
\item 勸吿は又は訓誡。例 tvam evaṃ kuryāḥ 「汝はかく爲
すとは\ruby[S]{底}{なに}事ぞや」,āpad arthaṃ dhanaṃ rakṣet 「人は災厄の
ために富を守護すべきである」。
\end{enumerate}
\item 可能法は次の如き接續的の句に用ひられる。
\begin{enumerate}[label=(\alph*)]
\item 一般の關係句。例 kālātikramaṇaṃ vṛtter yo na
kurvīta bhūpatiḥ 「俸給の支拂の時を忽かにせざる所の王は」。
\item 目的の句(「……のために」の意にて)。例 ādiśa me
deśaṃ yatra vaseyam 「私の住すべき所の場處を私に示せ」。
\item 理由を示す句(「……するやうに」の意にて)。例 sa
bhāro bhartavyo yo naraṃ nāvasādayet 「人を壓倒せぬ程の
荷物が持たれるべきである」。
\item 假定句の前件に於て。例 yadi na syān nara-patir
viplaveta naur iva prajā 「若し王なくば人民は船の如く沈沒す
べし」。
\end{enumerate}
\end{enumerate}

\subsection{願望法}
\numberParagraph
これは第三過去の可能法であつて極めて稀に用ひらるる
形であり,祝福の發表に限られ,第一人稱にて說者の願望を述ぶ。
例 vīra-prsavā bhūyāḥ 「御身が勇士を生むやうに」,kṛtārtha
bhūyāsam 「我れ目的を達するやうに」。

命令法も亦この意味に用ひられる。極めて稀ではあるが願望法
が命令法又は通常の可能法と何等區別し得ないこともある。例
idaṃ vaco brūyāsta 「汝等はこの語を宣べよ」,mamaiṣa kāmo
bhūtānāṃ yad bhūyāsur vibhūtayaḥ 「人々が幸福を享受せん
ことは我が欲する所である」。

\subsection{條件法}
\numberParagraph
條件法はその形から云はゞ未來の現實的過去とも云ふべ
く過去の條件を表はすに用ひられ,事實そのことは存在せぬこと
が含まれる。而もそれは過去完了に相當する。又この形は假定句
の前件後件の雙方に用ひられる。例 suvṛṣṭiś ced abhaviṣyad
durbhikṣaṃ nābhaviṣyat 「若しも多量の雨ありせば饑饉はあら
ざりしならむ」。

若し前件に可能法が用ひられたならば後件は假定の意味とな
る。例 yadi na praṇayed rājā daṇḍaṃ śūle matsyān ivā\-%
pakṣyan durbalān balavattarāḥ 「若しも王が處罰を加へない
ならば强者は弱者を串に於いての魚のやうに焙るであらう」。

可能法は假定の現在を云ひ表はすのが持前だが,然しそれが亦
條件法でなくて過去の意味にもなるのである。

%%% Local Variables:
%%% mode: latex
%%% TeX-master: "IntroductionToSanskrit"
%%% End:
