\chapter{書法}
\numberParagraph
Devanāgarī 字母は四十八の文字にて書かれる。母音十三
子音三十五(この中に隨韻アヌスヷーラと止聲ヴィサルガを含
む)。これによつて梵語の音は總て殘る所なく寫されることにな
つてゐる。現時ローマ字に若干の符號を附加してこれを寫す。こ
の方法は曾て 1894 A.D. にゼネヷの東洋學會(Oriental Con\-%
gress)で規定したものに準據することになつてゐる。

\begin{center}
〔母音字母〕
\end{center}

\numberParagraph \label{np:6}
デーヷーナーガリー字母は次の如くである。\endnote{底本は {\dnf अ, आ, ओ, औ} が異体字になっている。
また,{\dnf अ} の右側に {\dnf ◌} ではなく「━」のような文字が入っている。}

\begin{tabular}{cc}
  \begin{minipage}{0.46\hsize}
  \begin{tabular}{cccll}
  {\dnf अ} & {\dnf ◌} & a & ア & \rdelim\}{9}{*}[\textbf{單母音}] \\
  {\dnf आ} & {\dnf ा} & ā & アー & \\
  {\dnf इ} & {\dnf ि} & i & イ & \\
  {\dnf ई} & {\dnf ी} & ī & イー& \\
  {\dnf उ} & {\dnf ु} & u & ウ & \\
  {\dnf ऊ} & {\dnf ू} & ū & ウー & \\
  {\dnf ऋ} & {\dnf ृ} & ṛ & リ & \\
  {\dnf ॠ} & {\dnf ॄ} & ṝ & リー & \\
  {\dnf ऌ} & {\dnf ॢ} & ḷ & リ &
  \end{tabular}
  \end{minipage}
  &
  \begin{minipage}{0.46\hsize}
  \begin{tabular}{cccll}
  {\dnf ए} & {\dnf े} & e & エー & \rdelim\}{4}{*}[\textbf{複母音}] \\
  {\dnf ऐ} & {\dnf ै} & ai & アーイ & \\
  {\dnf ओ} & {\dnf ो} & o & オー & \\
  {\dnf औ} & {\dnf ौ} & au & アーウ &
  \end{tabular}
  \vspace{1\zh}

  \renewcommand{\arraystretch}{1.5}
  \begin{tabular}{ccll}
  {\dnf ः} & ḥ & フ(止聲)& \rdelim\}{2}{*}[\parbox{6\zw}{この二個は母音に隨ふ符號である。}] \\
  {\dnf ँ} & ṃ & ン(隨韻)&
  \end{tabular}
  \renewcommand{\arraystretch}{1}
  \end{minipage}
\end{tabular}

\begin{center}
〔子音字母〕
\end{center}

\begin{tabular}{cc}
  \begin{minipage}{0.46\hsize}
  \begin{tabular}{llp{\widthof{チュハ}}lll}
  {\dnf क} & ka & カ & 無氣 & \rdelim\}{2}{*}[\parbox{1\zw}{無響}] & \rdelim\}{5}{*}[喉音] \\
  {\dnf ख} & kha & クハ & 含氣 & & \\
  {\dnf ग} & ga & ガ & 無氣 & \rdelim\}{2}{*}[\parbox{1\zw}{有響}] & \\
  {\dnf घ} & gha & グハ & 含氣 & & \\
  {\dnf ङ} & ṅa & ンガ & 鼻音 &
  \end{tabular}
  \end{minipage}
  &
  \begin{minipage}{0.46\hsize}
  \begin{tabular}{llllll}
  {\dnf प} & pa & パ & 無氣 & \rdelim\}{2}{*}[\parbox{1\zw}{無響}] & \rdelim\}{5}{*}[\parbox{1\zw}{唇音}] \\
  {\dnf फ} & pha & プハ & 含氣 & & \\
  {\dnf ब} & ba & バ & 無氣 & \rdelim\}{2}{*}[\parbox{1\zw}{有響}] & \\
  {\dnf भ} & bha & ブハ & 含氣 & & \\
  {\dnf म} & ma & マ & 鼻音 &
  \end{tabular}
  \end{minipage}
  \\
  \begin{minipage}{0.46\hsize}
  \begin{tabular}{llp{\widthof{チュハ}}lll}
  {\dnf च} & ca & チャ & 無氣 & \rdelim\}{2}{*}[\parbox{1\zw}{無響}] & \rdelim\}{5}{*}[上顎音] \\
  {\dnf छ} & cha & チュハ & 含氣 & & \\
  {\dnf ज} & ja & ジャ & 無氣 & \rdelim\}{2}{*}[\parbox{1\zw}{有響}] & \\
  {\dnf झ} & jha & ジュハ & 含氣 & & \\
  {\dnf ञ} & ña & ニャ & 鼻音 &
  \end{tabular}
  \end{minipage}
  &
  \begin{minipage}{0.46\hsize}
  \begin{tabular}{llll}
  {\dnf य} & ya & ヤ(上顎音) & \rdelim\}{4}{*}[\parbox{3\zw}{有響音半母音}] \\
  {\dnf र} & ra & ラ(舌音) & \\
  {\dnf ल} & la & ラ(齒音) & \\
  {\dnf व} & va & ヷ(唇音) &
  \end{tabular}
  \end{minipage}
  \\
  \begin{minipage}{0.46\hsize}
  \begin{tabular}{llp{\widthof{チュハ}}lll}
  {\dnf ट} & ṭa & タ & 無氣 & \rdelim\}{2}{*}[\parbox{1\zw}{無響}] & \rdelim\}{5}{*}[舌音] \\
  {\dnf ठ} & ṭha & トハ & 含氣 & & \\
  {\dnf ड} & ḍa & ダ & 無氣 & \rdelim\}{2}{*}[\parbox{1\zw}{有響}] & \\
  {\dnf ढ} & ḍha & ドハ & 含氣 & & \\
  {\dnf ण} & ṇa & ナ & 鼻音 &
  \end{tabular}
  \end{minipage}
  &
  \begin{minipage}{0.46\hsize}
  \begin{tabular}{llll}
  {\dnf श} & śa & シャ(上顎音) & \rdelim\}{3}{*}[\parbox{3\zw}{無響音硬吹氣}] \\
  {\dnf ष} & ṣa & シャ(舌音) & \\
  {\dnf स} & la & サ(齒音) & \\
  {\dnf ह} & ha & ハ(喉音) & 軟吹氣
  \end{tabular}
  \end{minipage}
  \\
  \begin{minipage}{0.46\hsize}
  \begin{tabular}{llp{\widthof{チュハ}}lll}
  {\dnf त} & ta & タ & 無氣 & \rdelim\}{2}{*}[\parbox{1\zw}{無響}] & \rdelim\}{5}{*}[齒音] \\
  {\dnf थ} & tha & トハ & 含氣 & & \\
  {\dnf द} & da & ダ & 無氣 & \rdelim\}{2}{*}[\parbox{1\zw}{有響}] & \\
  {\dnf ध} & dha & ドハ & 含氣 & & \\
  {\dnf न} & na & ナ & 鼻音 &
  \end{tabular}
  \end{minipage}
  &
\end{tabular}

この順序は辭典の語彙排列等の標準となり一定不變のものである。

\begin{enumerate}[label=(\alph*)]
\item 子音に用ひられし文字は常に a 音を伴つてゐる。{\dnf क} = ka
{\dnf त} = ta 等。故に純粹の子音を表はすには Virāma なる符號,
卽ち格字の下に斜線を加ふ。例 {\dnf क्} = k,{\dnf त्} = t.
\item デーヷーナーガリー字を書くには各字の特殊なる部分
先づ書き次に垂直線,最後に水平線を引くを原則とす。\endnote{底本では {\dnf त्} の筆順が例示されているが,省略。}
\item 語首の {\dnf अ} a が聲音の規定により消失する時は(\ref{np:14} 條)
Avagrapha と呼ばるる符號 {\dnf ऽ} を以て表はす。例 {\dnf तेऽपि} te 'pi
(彼等も亦)は {\dnf ते अपि} のことである。
\end{enumerate}

\nt{
\begin{enumerate}[label=(\arabic*)]
\item 五種の鼻音が夫々自己と同類の子音に先立つ時に簡便を尚
ぶより隨韻を以つて代用することあるも,これは正しい方法ではない。
{\dnf अङ्कित} aṇkita を {\dnf अंकित} aṃkita とし {\dnf कम्पित} kampita を {\dnf कंपित}
kaṃpita とする類である。勿論隨韻を代用しても發音は夫々五種
の鼻音通りになすべきである。又同樣に文章の終の {\dnf म} m が時として隨
韻で寫される。{\dnf अहम} aham を {\dnf अहं} ahaṃ とするが如きである。然し
これは誤謬を惹起し易いから避ける方が望ましい。甚だまぎらはしき方法である。隨韻
の使用は \ref{np:28} 條の範圍を以て大體の限界とし原則を立てるべきである。前
加語 sam の如きは特に獨立語の如く見做されるので saṃ とするは當然
である。所が印度の寫本などはとかく簡便をよいことにして代用隨韻を濫
用する傾向がある。學者はなるべくこれに隨はないやうにしたい。
\item 隨韻と止聲とはその字母中に於ける順序位置に關して初學を惑は
すことも少くない。辭典の檢索に於いて次のことは心得置くべきである。
\item 半母音吹氣音に先立つ隨韻は總ての子音に先立つ。故に saṃ\-%
vara, saṃśaya は sa-ka より先に置かれる。これを純粹隨韻と稱す。
\item 之に對して變化隨韻又は代用隨韻なるものがある。それは喉等五
類の子音の前にある場合は隨韻の形を取るもそのまゝその類の鼻音に代用
し得べきものである。これは夫々その鼻音の位置にあるものである。例せ
ば saṃkāśa は sa-ga の前に置かれる。saṃ とあるも saṅ であるが如
く見做すべきものである。
\item これと同樣に,硬喉唇音に先立つ純粹止聲は總ての子音に先立つ。
故に antaḥkaraṇa や antaḥpura は anta の後で anta-ka の前に置
かれる。然し硬吹氣音に先立つ止聲は變化又は代用止聲であつて夫々の硬
吹氣音と見做されねばならぬ。卽ち ḥś, ḥṣ, ḥs は,各々 ś, ṣ, s の後に
置かれる。
\end{enumerate}}

\begin{enumerate}[label=(\alph*), start=4]
\item a 以外の母音が子音に續く場合は夫々の母音の略符を附
加して文字を作る。今 {\dnf क} ka, を以て例すれば {\dnf का} kā, {\dnf कि} ki,
{\dnf की} kī, {\dnf कु} ku, {\dnf कू} kū, {\dnf कृ} kṛ, {\dnf कॄ} kṝ, {\dnf कॢ} kḷ, {\dnf कॆ} ke, {\dnf कै} kai, {\dnf कॊ} ko,
{\dnf कौ} kau. となす。其他の字も推知すべきである。

稍特殊の形を例示せば {\dnf रु} ru, {\dnf रू} rū. 等。
\item {\dnf र्} r は子音または {\dnf ऋ} ṛ 母音に先立つ時は c の如く記
され字の上に置かれる。例 {\dnf अर्क} arka {\dnf निरृतिः} nirṛtiḥ 等。
又 {\dnf र्} r が子音の後に來る時は右より左へ下に向ふ斜線にて
記さる。例 {\dnf क्र} kra 等。
\end{enumerate}

\printParagraphCounter
子音の結合の主要なるものは次の如くである。

% {\dnf क्क} k-ka, {\dnf क्ख} k-kha, {\dnf क्च} k-ca, {\dnf कण} k-ṇa, {\dnf क्त} k-ta, {\dnf क्त्य} k-t-ya,
(略)\endnote{底本のグリフ(リガチャ)を正確に再現できるフォントがないため,省略。}

數字は次の如くである。

\begin{center}
\begin{tabular}{cccccccccc}
  {\dnf १} & {\dnf २} & {\dnf ३} & {\dnf ४} & {\dnf ५} & {\dnf ६} & {\dnf ७} & {\dnf ८} & {\dnf ९} & {\dnf ०} \\
  1 & 2 & 3 & 4 & 5 & 6 & 7 & 8 & 9 & 0
\end{tabular}
\end{center}

句讀の符號は小段落に {\dnf ।},大段落に {\dnf ॥} を用ふ。

\newpage
\def\enotesize{\normalsize}
\theendnotes

%%% Local Variables:
%%% mode: latex
%%% TeX-master: "IntroductionToSanskrit"
%%% End:
