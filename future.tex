\section{未來組織}
未來時の構成法に二種あり。(1)單未來。(2)複說未來。

\subsection{單未來}
\numberParagraph \label{np:186}
單未來は語根に sya を附加して作る。その母音は重韻化
する。sya は直接或は母音 i を介して添加せらる。人稱語尾は現
在の如くである。

\numberParagraph

\begin{tabular}{cl}
  語根 & dā (與ふ)~ dāsya \\
  〃   & kṛ (作る)~ kariṣya
\end{tabular}

\begin{center}
\begin{tabular}{c*{3}{p{0.23\hsize}}}
  \multicolumn{4}{c}{爲他} \\
     & 單      & 兩        & 複 \\
  1. & dāsyāmi & dāsyāvas  & dāsyāmas \\
  2. & dāsyasi & dāsyathas & dāsyatha \\
  3. & dāsyati & dāsyatas  & dāsyanti \\
  \multicolumn{4}{c}{爲自} \\
  1. & dāsye   & dāsyāvahe & dāsyāmahe \\
  2. & dāsyase & dāsyethe  & dāsyadhve \\
  3. & dāsyate & dāsyete   & dāsyante
\end{tabular}
\end{center}
\begin{center}
\begin{tabular}{c*{3}{p{0.23\hsize}}}
  \multicolumn{4}{c}{爲他} \\
     & 單        & 兩          & 複 \\
  1. & kariṣyāmi & kariṣyāvas  & kariṣyāmas \\
  2. & kariṣyasi & kariṣyathas & kariṣyatha \\
  3. & kariṣyati & kariṣyatas  & kariṣyanti \\
  \multicolumn{4}{c}{爲自} \\
  1. & kariṣye   & kariṣyāvahe & kariṣyāmahe \\
  2. & kariṣyase & kariṣyethe  & kariṣyadhve \\
  3. & kariṣyate & kariṣyete   & kariṣyante
\end{tabular}
\end{center}

\numberParagraph
未來時の記號なる sya を附加するに當り子音に終る語根
に會して聲音上の變化あることに注意すべきである。

\begin{tabular}{*{2}{p{0.4\hsize}}}
  śak (能ふ) śakṣyati    & pac (煮る) pakṣyati \\
  prach (問ふ) prakṣyati & tyaj (捨つ) tyakṣyati \\
  labh (得る) labpyati   & vas (住す) vatsyati
\end{tabular}

尙次の如きものもある。

\begin{tabular}{*{2}{p{0.4\hsize}}}
  nī (導く) neṣyati    & budh (覺る) bodhiṣyati \\
  gam (行く) gamiṣyati & grah (取る) grahīṣyati \\
  dṛś (見る) drakṣyati & ji (勝つ) jeṣyati \\
  gai (歌ふ) gāsyati   &
\end{tabular}

\subsection{條件法}
\numberParagraph
未來語基から條件法が作られる。卽ち過去符と第一過去
の語尾を附加する。dā (與ふ)未來語基 dāsya, 條件法 adāsyat (彼
は與へたであらう)。同樣に bhū (あり)~ bhaviṣya ~ abhaviṣyat
(彼は有りしならむ),kṛ (爲す)~ kariṣya ~ akariṣyat (彼は爲せ
しならむ)。

\subsection{複說未來}
\numberParagraph
複說未來は tṛ に終る作者名詞(\ref{np:55}條)と as (有り)と云
ふ助動詞とで構成せられる。三人稱は總て三數に隨つて用ふる(dātā, dā\-%
tārau, dātaras の如く)。其の他の人稱では as の現,爲他,爲
自の形が添へられる。dā (與ふ)作者名詞 dātṛ 主單 dātā + asmi
= dātāsmi, bhū (有り)~ bhavitṛ ~ bhavitā + asi = bhavitāsi 等。

\begin{center}
\begin{tabular}{c*{3}{p{0.23\hsize}}}
  \multicolumn{4}{c}{爲他} \\
     & 單         & 兩           & 複 \\
  1. & bodhitāsmi & bodhitāsvas  & bodhitāsmas \\
  2. & bodhitāsi  & bodhitāsthas & bodhitāstha \\
  3. & bodhitā    & bodhitārau   & bodhitāras \\
  \multicolumn{4}{c}{爲自} \\
  1. & bodhitāhe & bodhitāsvahe & bodhitāsmahe \\
  2. & bodhitāse & bodhitāsāthe & bodhitādhve \\
  3. & bodhitā   & bodhitārau   & bodhitāras
\end{tabular}
\end{center}

其他の例:nī (導く)~ netāsmi, dṛś (見る)~ drṣṭāsmi, jīv
(生く)~ jīvitāsmi.

\numberParagraph
單未來は今日中に生ずべき事實又は汎く將來に於て早晩
發生すべき事實を敍ぶるに用ふ。 通例 śvas (明
日)等の語と共に用ひられる。

\ex{第十四}
\begin{longtable}{c*{2}{p{0.45\hsize}}}
 1. & aham adya gamiṣyāmi. & 私は今日行くだらう。\\
 2. & tau gantārau. & 彼等二人は明日行くであらう。\\
 3. & yena tvaṃ śvo yoddhāsi taṃ dhruvaṃ tvaṃ jetāsi. & 汝は明日戰うであらう所の
人を必ず征服するであらう。\\
 4. & adya varṣiṣyati śvo 'pi vraṣṭā. & 今日雨降らむ。明日も亦雨降るべし。\\
 5. & sūdo 'nnaṃ pakṣyati. & 庖人は食を煮む。\\
 6. & gṛhāt prasthitvā vane vats\-yāmi. & 家より出でゝ我は林中に住まむ。\\
 7. & Gangā-tīraṃ gatvā nadīṃ drakṣyāmi. & 恒河の岸に行きて我は河を見るべし。\\
 8. & asau śīghraṃ nīrogo bhavi\-ṣyati. & 彼は速かに健康となるべし。\\
 9. & astaṃgate sūrye śaśī śiśira\-kara udeṣyati. & 日沒したる時冷かなる光あ
る月は昇るべし。\\
10. & yadi tvaṃ mahā\-pañke pati\-ṣyati tad asaṃśataṃ mariṣyasi.
& 汝若し大なる沼に陷らば必ず死すべし。\\
11. & yadi na pāpaṃ sa naro 'tyakṣyat tadā duḥkhaṃ tasyā\-bhaviṣyat.
& 彼もし罪を捨てざりせばその時彼に不幸がありしならむ。\\
12. & yadi vayaṃ bhūmau patitam agniṃ niravāpayiṣyāma tad agnis tasya gṛham adhakṣyat.
& 若し我等地上に落ちたる火を消さざりせばその火は彼の家を燒きしならむ。\\
13. & yat tvaṃ prakṣyasi tad ahaṃ yathā-śakti prativakṣyāmi. & 汝の問はむことを我は力の
限り答ふべし。\\
14. & kiṃ kariṣyanti vaktāraḥ śrotā yatra na vidyate, nagna\-kṣapaṇake deśe rajakāḥ kiṃ
kariṣyanti. & 聞くものなき所には語るものは何をかなさむ。裸形者の國土にては洗濯人は何をかなさむ。
\end{longtable}


%%% Local Variables:
%%% mode: latex
%%% TeX-master: "IntroductionToSanskrit"
%%% End:
