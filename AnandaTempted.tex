\newpage

\texttitle{阿難誘惑せらる}
\addcontentsline{toc}{chapter}{\protect\numberline{}阿難誘惑せらる}%
是の如く我れによりて聞かれたり。ある時世尊はシュラーヷス
ティー逝多林なる給孤獨の園に住したまへり。時に\ruby{具壽}{ぐ|じゅ}阿難陀は
午前に衣を整へ鉢と上衣を取りシュラーヷスティー大城に行乞のた
めに進み行けり。時に具壽阿難陀は行乞をなし,飯食の後,とある
井の方に進み行けり,その時にその井に於いてプラクリティと名
くる賤族の少女は水を汲みてゐたりき。時に具壽阿難陀は賤族の
少女にこれを云へり。「我に水を得させよ,姉妹よ,我れは飲まん
とす」。かく云はれたる賤族の少女プラクリティは具壽なる阿難
陀にこれを云へり。「我れは賤族の少女なり,大德阿難陀よ」。
「姉妹よ,我は種族生家を問へるにあらず。たゞ若し許さるべくば
水を與へよ我は飲まんとす」。時に賤族の少女プラクリティは具壽
阿難陀に水を與へたり。時に具壽なる阿難陀は水を飲みて進み行
けり。時に賤族の少女プラクリティは具壽阿難陀の身に於いて顔
に於いて一切善く審らかに相を執りて如理に作意して愛戀の心を
生ぜり。聖者阿難陀は我が夫たるべしと。而して我母は大咒師な
り,彼は聖者阿難陀を引き出すを得べし。時に賤族の少女プラク
リティは水瓶を一方に投げて自からの母にこれを云へり。「阿母よ

\newpage
\texttitle{Ānanda ākṛṣṭaḥ}
evaṃ mayā śrutam \da{} ekasmin samayo Bhagavāṃś Chrā\-%
vastyāṃ viharati sma Jetavane 'nāthapiṇḍadasyārame \da{}
athāyuṣmān Ānandaḥ pūrvāhne nivāsya pātra-cīvaram ādāya
Śrāvastīṃ mahānagarīṃ piṇḍāya prāvikṣat \da{} atha āyuṣmān
Ānandaḥ Śrāvastīṃ piṇḍāya caritvā kṛta-bhakta-kṛtyo
yenānyatamam udapānaṃ tenopasaṃkrāntaḥ \da{} tena khalu
samayena tasminn udapāne Prakṛtir nāma mātaṅgadāriko\-%
dakam uddharate sma \da{} athāyuṣmān Ānandaḥ Prakṛtiṃ
mātaṅgadārikām etad avocat \da{} dehi me bhagini pānīyaṃ
pāsyāmi \da{} evam ukte Prakṛtir mātaṅgadārikā āyuṣmantam
Ānandam idam avocat \da{} mātaṅgadārikāham asmi bhadan\-%
tānanda \da{} nāhaṃ he bhagini kulaṃ vā jātiṃ vā pṛcchāmi
api tu sacet te parityaktaṃ pānīyaṃ dehi pāsyāmi \da{} atha
Prakṛtir mātaṅgadārikā ayuṣmaṭa Ānandāya pānīyam adāt \da{}
athāyuṣmān Ānandaḥ pānīyaṃ pītvā prakrāntaḥ \da{} atha
Prakṛtir mātaṅgadārikā āyuṣmato Ānandasya śarīre mukhe
sarve sādhu ca suṣṭhu ca nimittam udgṛhītvā yoniśo mana\-%
sikāreṇāviṣṭā saṃrāga-cittam utpādayati sma \da{} āryo me
Ānandaḥ svāmī syād iti \da{} mātā ca me mahā-vidyā-dharī sā
śakṣyaty āryam Ānandaṃ ānayitum \da{} atha Prakṛtir mātaṅ\-%
gadārikā pānīya-ghaṭam ekānte nikṣipya svāṃ jananīm idam

\newpage
\noindent
今や知らるべし。阿難陀と名くる沙門瞿答摩\endnote{「瞿答摩」はおそらく Gautama の音訳。}の聲聞の弟子あり。
我は彼を夫となさんとす。阿母よ,彼を引き出し得べきや」。彼女
は娘に云へり。「娘よ死せるか離慾ならざる限り我は阿難陀を引
出し得。されどコーサラの王波斯匿は沙門瞿答摩に極めて尊敬奉
事せり,若し彼にして知らばこの旃陀羅族の不幸は生ずべし。而
して沙門瞿答摩は離慾なりと聞く。離慾なる彼は一切の咒を超越
せり」。かくの如く云はれし時,賤族の女プラクリティは母にこれ
を云へり。「阿母よ,若し沙門瞿答摩は離慾にて,彼の前より沙門
阿難陀を得ざるならば我は生命を捨つべし。若し得ば我は生くべ
し」。「娘よ,汝は生命を棄つる勿れ。我れ汝に沙門阿難陀を引き
出すべし」。時に賤族の女プラクリティの母は家の中に牛糞の塗料
を作りて壇を塗り。吉祥草を散らし,火を燃し,百八瓣のアルカ
の花を取りて咒を唱へつゝ一々のアルカの花を咒しつつ火中に投
じたり。そこにこの咒あり。
\begin{addmargin}[2em]{0em}
\indent
アマレー,ヴィマレー,クンクメー,スマネー,それによ
りて汝は覺者なり,希望もて輝かさる。神は雨らし輝く。驚
きより響く。大王の增加せしめんために\ 諸神に諸人に諸乾
闥婆は火執の故に明火執の故に。阿難陀の來現と交會と進出
と執受のために\ 我は供養す。莎婆訶。
\end{addmargin}

\newpage
avocat \da{} yat khalv evam amba jānīyā Ānando nāma śramaṇa-
Gautamasya śrāvaka upasthāyakas tam ahaṃ svāminam
icchāmi śakṣyasi tam amba ānayitum \da{} sā tām avocat \da{}
śaktāhaṃ putri Ānandam ānayituṃ sthāpayitvā yo mṛtaḥ
syād yo vā vīta-rāgaḥ \da{} api ca rājā Prasenajit Kauśalaḥ
śramaṇa-Gautamam atīva sevate bhajate paryupāsate \da{} yadi
jānīyāt so 'yaṃ caṇḍāla-kulasyānarthāya pratipadyeta \da{}
śramaṇaś ca Gautamo vīta-rāgaḥ śrūyate vīta-rāgas sa punaḥ
sarva-mantrān abhibhavati \da{} evam uktā Prakṛtir mātaṅga\-%
dārikā mātaram idam avocat \da{} saced amba śramaṇo Gautamo
vīta-rāgas tasyāntikāc chramaṇam Ānandaṃ na pratilapsye
jīvitaṃ parityajeyaṃ sacet pratilapsye jīvāmi \da{} mā tvaṃ
putri jīvitaṃ parityakṣyasi ānayāmi te śramaṇam Ānandam \da{}
atha Prakṛter mātaṇgadārikāyā mātā madhye gṛhāṅgaṇasya
gomayenālepanaṃ kṛtvā vedīm ālipya darbhān saṃstīryāgniṃ
prajvālyāṣṭaśatam arka-puṣpāṇāṃ gṛhītvā mantrān āvar\-%
tayamānā ekaikam arka-puṣpaṃ parijapyāgnau pratikṣipati
sma \da{} tatreyaṃ vidyā bhavati \da{}
\begin{addmargin}[2em]{0em}
\indent
amale vimale kuṅkume sumane yena buddho 'si vidyuta
icchayā devo varṣati vidyotati garjati vismayān mahā\-%
rājasya samabhivardhayituṃ devebhyo manuṣyebhyo gan\-%
dharvebhyaḥ śikhigrahād eva viśikhigrahād eva Ānanda-
syāgamanāya saṃgamanāya kramaṇāya grahaṇāya grahaṇāya ju\-%
homi svāhā \dd
\end{addmargin}

\newpage
時に具壽阿難陀の心は引き寄せられたり。彼は精舎より出でゝ
旃陀羅の屋舎の方に往けり。旃陀羅の女は具壽阿難陀の來れる
を遠く望み見て娘プラクリティにこれを云へり。「娘よ,ここに彼
れ沙門阿難陀は來れり,臥床を設けよ」。時に賤族の少女は歡喜滿
足欣悅の意もて具壽阿難陀の臥床を設けたり。時に具壽阿難陀は
旃陀羅女の屋舎の方に往けり,往きて壇に近づきて立てり。一方
に立ち又彼具壽阿難陀は啼泣せり,淚を流しつゝかくの如く云へり。
「我は厄難に臨めり,而して世尊は我を憶念したまはざるなり」。
時に世尊は具壽阿難陀を憶念したまへり。憶念して正覺者の咒に
よりて旃陀羅の咒を破壞したまへり。咒に云く
\begin{addmargin}[2em]{0em}
「一切有情にまで堅固不死不導吉祥\\
\ 湖水の如く清澄無過失寂靜一切處無畏\\
\ 災禍は鎮靜し怖畏は移動する所に\\
\ 天よ一切成就の修行者は彼を敬禮す。\\
\ この眞實語を以て比丘阿難陀に吉祥あれ」
\end{addmargin}
時に具壽阿難陀は旃陀羅の咒破られて旃陀羅の屋舎より出でて
自らの精舎の方に往き始めたり。賤族の女プラクリティは具壽阿
難陀の歸り去るを見て自らの母にこれを云へり。「こゝに母よ,か
の沙門阿難陀は歸り去る」。彼へ母は云へり。「娘よ,定めて沙門

\newpage
athāyuṣmata Ānandasya cittam ākṣiptam \da{} sa vihārān
niṣkramya yena caṇḍāla-gṛhaṃ tenopasaṃkrāmati \da{} adrāk\-%
ṣīc caṇḍālī āyuṣmantam Ānandaṃ dūrād evāgacchantaṃ
dṛṣṭvā ca punaḥ Prakṛtiṃ duhitaram idam avocat \da{} ayam
asau putri śramaṇānanda āgacchati śayanaṃ prajñapaya \da{}
atha Prakṛtir mātaṅgadārikā hṛṣta-tuṣṭa-pramudita-manā
āyuṣmata Ānandasya śayyām prajñapayati \da{} atha āyuṣmān
Ānando yena caṇḍālī-gṛhaṃ tenopasaṃkrāntaḥ \da{} upasaṃ\-%
krāmya vedīm upaniśrityāsthāt \da{} ekānta-sthitaḥ sa punar
āyuṣmān Ānandaḥ praroditi aśrūṇi pravartayamāna evam
āha \da{} vyasana-prāpto 'ham asmi na ca me Bhagavān samanvā\-%
harati \da{} atha Bhagavān āyuṣmantam Ānandaṃ samanvā\-%
harati sma samanvāhṛtya saṃbuddha-mantraiś caṇḍāla\-%
mantrān pratihanti sma \da{} tatreyaṃ vidyā
\begin{addmargin}[2em]{0em}
sthitir acyutir anītiḥ svasti sarvaprāṇibhyaḥ \da{}\\
saraḥ prasannaṃ nirdoṣaṃ praśāntaṃ sarvato 'bhayam \da{}\\
ītayo yatra śāmyanti bhayāni calitāni ca \dd{}\\
taṃ vai deva namasyanti sarva-siddhāś ca yoginaḥ \da{}\\
etena satya-vākyena svasty Ānandāya bhikṣave \dd{}
\end{addmargin}
athāyuṣmān Ānandaḥ pratihata-caṇdāla-mantraś caṇḍāla\-%
gṛhān niṣkramya yena svako vihāras tenopasaṃkramitum
ārabdhaḥ \da{} adrākṣīt Prakṛtir mātaṅgadārikā Ānandam āyuṣ\-%
mantaṃ pratigacchantam dṛṣṭvā ca punaḥ svāṃ jananīm
idam avoat \da{} ayam asau mātaḥ śramaṇānandaḥ pratiga-

\newpage
\noindent
瞿答摩の憶念(原文訂正)せしならむ,その故に我が咒は破られて
あるならむ」。プラクリティ亦云へり。「母よ,如何に沙門瞿答摩の
咒は我等のそれより力强きや」。彼女へ母は云へり。「沙門瞿答摩
の咒は一層强力なり。それらの咒は娘よ,一切世界を超越す,そ
れらの咒を瞿答摩は欲するならば破る,されど世界は沙門瞿答摩
の咒を破る能はじ。かくの如く沙門瞿答摩の咒は一層强大なり。\wosfnt{%
  この一節はディヴャーヷダーナ(殊勝說話)の第三十三章から取つた。殊
  勝說話は全篇三十八章の說話から成る佛敎經典で,その内容も樣式も種々
  雜多で連絡統一を缺き,成立もさう古いとは云へぬが,然しその中に極め
  て古い要素を含んでゐることが注意される。文體は極めて簡潔明快なる純
  梵語であるが,まゝ從來の文學中に用例なき方言を發見する。三十八の說
  話中その三十までは支那譯有部律其の他の經中に散在する。第二十六章か
  ら第二十九章に至る阿育王物語は興味多いものであり,阿育王經や阿育王
  傳(大正藏第五十卷)等に相當する。

  此に出した阿難誘惑の說話は,虎耳說話の一部分をなすもので支那譯經
  典の摩登伽經,舎頭諫太子二十八宿經(大正藏第二十一卷)等に一致し,
  釋尊が賤族の女を得度せしめて敎團の一員とせしことに對し,市民の不滿
  激昂,王への愁訴,王の訪問,釋尊の種姓に何等の差別なしとの說示がこの
  一章の梗槪である。此には只最初の部分である阿難が賤族の女から戀せ
  られ,咒術によりて破戒の危機に臨みし一段を擧げた。阿難が釋尊に親し
  く仕へながら修道に於いては割合に進み得ず。未だ離慾の境地に到達し得
  なかつたと傳へられることは興味多いことである。この阿難を拉し來りて
  旃陀羅の少女を配し,凄愴なる咒法執行の一場面を描いた所は經典として
  やや特殊の手法を見る。尙ほ印度の井戸には釣瓶の如き汲水具が無い。水
  を汲むものは自身これを携へて行くのである。これ阿難の自ら飲み得ず,
  女に水を乞ふ所以。又種族を異にすれば下種のものから飲食を受けられな
  いのである。況んや最下の旃陀羅族に於いては猶更のことである。これら
  を注意して讀むべきである。

  尙文中に見える咒についてはその一部分に純粹梵語の文法では解き得ら
  れない語がある。アマレー,ヴィマレー,クンクメー,スマネーの四語の如き
  は若し純粹梵語として解釋せば於格であるかの如く見える。然しこれは決
  して於格ではない。俗語の中にはかうした用例が屢々あるのだが,これは呼
  格である。般若心經の中に出て來るガテー,ガテー等の咒も亦さうである。
  その意義も茫然として捕捉し難いのが常である。此の咒の如きはまだ比較
  的意義の判然してゐる部分が多い方と云はねばならぬ。この方面の研究に
  ついては學会に未だ完成した業績が擧つてゐない。アタルヷ吠陀あたりか
  らみつしり研究してかからねばならぬ相當重大な問題が課せられてゐる。
  }


\newpage
cchati \da{} tāṃ mātāha \da{} niyataṃ putri śramaṇena Gautamena
samanvāhṛto (gato) bhaviṣyati tena mama mantrāḥ pratihatā
bhaviṣyanti \da{} Prakṛtir āha \da{} kiṃ punar amba dalavattarāḥ
śramaṇasya Gautamasya mantrā nāsmākam \da{} tāṃ mātāha \da{}
balavattarāḥ śramaṇasya Gautamasya mantrā nāsmākaṃ
ye putri mantrāḥ sarva-lokasya prabhavanti tān mantrān
śramaṇo Gautama ākaṅksamāṇāḥ pratihanti na punar lokaḥ
prabhavati śramaṇasya Gautamasya mantrān pratihantum
evaṃ balavattarāḥ śramaṇasya Gautamasya mantrāḥ \dd{}

\rightline{(Divyāvadāna XXXIII)}

\newpage
\theendnotes

%%% Local Variables:
%%% mode: latex
%%% TeX-master: "IntroductionToSanskrit"
%%% End:
