
\chapter*{總説}
\label{cha:general}

\numberParagraph
梵語とは支那日本でのみ呼ぶ名稱で正しくはサンスクリタ
(saṃskṛta 完成されたもの)と云ふべく,卽ち歐洲人のサンスク
リット(sanskrit)と呼ぶものである。これは印度古代の文學上
の言語でパーニニ(Pāṇini)なる學匠の規定するところである。

印度最古の言語をヴェーダ(Veda 吠陀)と云ふ。これは知識
の意。印度否寧ろ世界で最古のまとまつた文學であり,大部分讃
歌の形で出來て居り,古代民族が天然の現象に對し或は畏怖し或
は讃美して唱詠せし宗敎味豐かな文献である。サンスクリットは
音韻學上大體この吠陀の言語と同一であるが若干變化した所があ
る。それで吠陀に對して後期サンスクリット或は典文サンスク
リット等と呼ぶこともある。語格から云はばそれは發達ではな
く,寧ろ退化の過程を取つたものと云へる。若干の語尾の曲りは
消滅し,幾多の新語新義が登場するに至つた。

支那で梵語と呼ぶやうになつたのは何時ごろからのことか詳か
ではないが,餘程古い時代からこの名稱が用ひられてゐる。印度で
は梵字(Brahmalipi)と云ふ名稱はあるが,梵語といふやうな稱
呼は見えない。梵字と云ふのも梵天なる神が製作せし文字と云ふ
ことで,やはりインドの古い傳説に根據がある。これらの關係から
言語に梵語と云ふ呼び方をするやうになつたのであらう。梵とは
淸淨の義と見做されてゐる。支那で飜譯された佛敎經典,卽ち大
藏經の大部分はその原語がこのサンスクリットであつたことは注
意すべきである。

\numberParagraph
吠陀の言語から幾多の方言が派生した。その最も古いもの
はパーリ(Pāli)語であり,西紀前三世紀に屬するアソカ(Asoka)
王の刻文に現はれ,現今印度佛敎徒の聖典語である。パーリ語で
出來,整備した三藏經典は佛敎研究者の必須の寶物であるが,パ
ーリ語と梵語との關係も極めて密接であり,梵語を一通り學べば
パーリ語は自ずから通ずるやうになつてゐる。

尚ほ印度現代の方言の主要なるものに Panjābī, Sindhī, Guja\-%
rātī, Marāṭhī, Hindī, Bihārī, Bengālī がある。この中 Hindī
は Arabia 語 Persia 語が混合して Urdū 又は Hindūstānī と
云はれ印度各地を通じて用いられる交通用語である。又南印度に
Dravida 系統の Telugu, Tamil, Canara, Malayāla がある。こ
れらは梵語系統卽ちアーリヤ系統ではないが多くの梵語要素は含
まれてゐる。又 Ceylon には Sinhala 語がある。其他各地方に
夥しい數の方言がある。

\numberParagraph
此に注意すべきは印度アーリヤ民族は古代に於いて決して
文字を使用しなかつたことである。而も多くの不朽の優秀な文書
を傳へ得たことは驚嘆に値する。文字が使はれるやうになつたの
は西紀前六世紀以後であらう。卽ち西紀前約七百年頃メソポタミ
ア(Mesopotamia)を介してセム民族の書體が西北印度方面へ傳
へられた。この書體を採用した最古のものは前三世紀頃の貨幣や
刻文に見られるがこれをブラフミー(Brahmī)と云ふ。これは左
から右へ書くが曾て一度は右から左へ書かれたやうな痕跡が認め
られる。このブラフミーから數多い印度の書體が派生したのであ
る。就中最も重要なるはナーガリー(Nāgarī)書體である。或は
デーヷーナーガリー(Devanāgarī)とも呼ばれ,大體紀元八世紀頃
から用ひられる書體である。現今 Sanskrit はこの書體で記すこ
とになつてゐる。

\numberParagraph
因みに少し遡つて五,六世紀頃に用ひられたものに,悉曇
(Siddhām)書體がある。これは佛敎と共に支那へ流傳し隨つて
日本へも傳來した。現在日本の寺院等に於いて或は圖像の上に或
は卒都婆の面に書かれてゐる梵字は卽ちこの悉曇書體である。日
本ではこの書體で書かれた古文書が相當に多く保存されてゐる。
尚ほ日本で出版された大藏經の中の梵字はこの書體で記されてゐ
る。正し黄蘗版の藏經は明藏の覆刻であつてその中の梵字に又別
なランツァ書體が見える。實際梵字の書體も夥しいもので
あることに注意すべきである。

かやうに澤山の書體ができるにはできてゐるが,梵語を寫すに
は別にどれに依らねばならぬといふ定りもない。只現在ではナー
ガリー文字が多く使われてゐてこれを使へば便利でもあるし,梵
語の文書を讀む場合これを知らずしてはどうも一寸工合がわる
い。梵語の學習者としては何としてもナーガリー文字を讀むこと
は勿論,これを自由に書けるだけの練習はしてほしい。併しさう
した餘裕に惠まれない人にはローマ字だけでも結構間に合ふであ
らう。若干の符號を附加してローマ字で梵語の全音を寫し得るや
うになつてゐる。日本の假名で梵語の音を寫す試みは從來いろい
ろに企てられたがどうもうまく成功しないのは遺憾である。それ
は日本の假名が總て母音を含んだ綴音になつてゐるために,純粹
子音を寫す場合に瞭然とこれを表はすよき方法がない。本書でも
最初ナーガリー字の知識を與へるだけの準備はしてあるが全巻の
用字としてはすべてローマ字を採用することとした。

これに就いて特に注意を促しておきたいのは,初學者が動もす
れば言語と文字の關係を的確に把握しないために,文字そのもの
を言語と誤認したりして,「梵語は一體どんな字ですか」などと云
ふ愚問を發するのだが,本書の學習者にはそんな考へ方があつて
はならぬ。

%%% Local Variables:
%%% mode: latex
%%% TeX-master: "IntroductionToSanskrit"
%%% End:
