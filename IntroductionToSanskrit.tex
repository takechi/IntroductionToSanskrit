\documentclass[12pt,vartwoside]{ltjsbook}
%\setlength{\textwidth}{0.68\fullwidth}
\setlength{\evensidemargin}{\oddsidemargin}

\setlength{\marginparwidth}{0.25\fullwidth}

\usepackage[pdfencoding=auto]{hyperref}
\usepackage{fontspec}
\usepackage[no-math]{luatexja-fontspec}

\usepackage{scrextend}

% font for Japanese
% https://github.com/adobe-fonts/source-han-serif
\setmainjfont{SourceHanSerifJP}
\setsansjfont{SourceHanSansJP}
\ltjsetparameter{jacharrange={-2}}

% font for Devanagari
% https://github.com/google/fonts/tree/master/ofl/lohitdevanagari
% \newfontfamily\dnf[Script=Devanagari]{Lohit Devanagari}
% https://www.oah.in/Sanskrit/itranslator2003.htm
% \newfontfamily\dnf[Script=Devanagari]{Sanskrit2003}
% http://svayambhava.blogspot.jp/p/siddhanta-devanagariunicode-open-type.html
\newfontfamily\dnf[Script=Devanagari]{siddhanta}
\newcommand{\repha}{\ {\dnf }}

% for variant glyph
% \usepackage{luatexja-otf}

% Hanazono font for some variantions
\newjfontfamily\HanazonoA{HanaMinA}

\usepackage{enumitem}
\usepackage{pxrubrica}
\usepackage{calc}
\usepackage{multirow, bigdelim}
\usepackage{umoline}
\usepackage{graphicx, color}
\usepackage{longtable}

%\usepackage{type1cm}

\usepackage{endnotes}
\renewcommand{\theendnote}{\#\arabic{endnote}}

% enumerate paragraphs
\newcounter{paragraphCounter}
\setcounter{paragraphCounter}{0}
\newcommand{\formatParagraphCounter}[1]{\textbf{#1}}
\newcommand{\printParagraphCounter}[0]{\formatParagraphCounter{\arabic{paragraphCounter}}.\hspace{1\zw}}
\newcommand{\numberParagraph}[1][0]{\refstepcounter{paragraphCounter}\printParagraphCounter}

\newcommand{\anunasikam}{\hspace{0.75em}{\dnf ̐}\hspace{-0.75em}m}
\newcommand{\anunasikamd}{\hspace{0.75em}{\dnf ̐}\hspace{-0.75em}ṃ}
\newcommand{\da}{{\dnf ।}{}}
\newcommand{\dd}{{\dnf ॥}{}}

\newcommand{\cellAlign}[2]{\multicolumn{1}{#1}{#2}}

\newcommand{\rt}[0]{$\sqrt{}$}

% 【註】
\newcommand{\nt}[1]{%
\noindent\hrulefill\linebreak
【註】\hspace{1\zw} #1 \par \vspace{-1\zw}
\noindent\hrulefill}

% 例:
\newcommand{\eg}[1]{%
\begin{description}[font=\normalfont, style=multiline, labelwidth=\widthof{されど}, align=right, labelindent=10mm, leftmargin=20mm]
\item[例:] #1
\end{description}}

% 演習
\newcommand{\ex}[1]{%
\begin{center}
\Underline{{\Overline{{\textbf{演習#1}}}}}
\end{center}\addcontentsline{toc}{subsection}{演習 #1}}


% 文抄タイトル
\newcommand\texttitle[1]{%
\begin{center}{\Large #1}\end{center}}

% シンボルなし脚注
\newcommand\wosfnt[1]{%
\begingroup%
\renewcommand\thefootnote{}\footnote{#1}%
\addtocounter{footnote}{-1}%
\endgroup}


% 傍注
\newcommand{\mymp}[1]{%
%  \marginpar{\footnotesize{#1}}
}

\title{入門サンスクリット}
\author{泉 芳璟 著}

\begin{document}
\maketitle
\setcounter{tocdepth}{3}
\tableofcontents

\chapter*{序言}
\label{cha:preface}
かれこれ三十年以上も梵語を敎へてゐる經驗から梵語の文典に
ついて種々意見もあるが,大體に於て文典學習の方法に二通りの
型がある。卽ち言語の組織の理解と言語そのものの實習である。
これは必ずしも梵語に限つたことではないが梵語のやうに複雜な
體系を有つてゐる言語では尚ほ更のことであらうと思ふ。言語の
組織に就ては名詞變化なり動詞變化なり順序を追うて秩序整然と
排列されたものが必要である。然るに實習となると必ずしも順序
は追はなくても可い。あちらからもこちらからも随時に必要なも
のを取り出さねばならぬ。言ひ換へれば名詞變化が濟まねば動詞
變化に移られぬといふやうなものでない。名詞を少しやつたと思
ふうちに動詞が必要となつて來るのが實狀である。

ところが從來多くの文典は組織の排列に重きを置くか實習に重
きを置くかの孰れかを狙つてゐるので,組織に重きを置けば實際
の運用が鈍るし,實習に重きを置けば組織の纒めが疎かになり易
い。この兩端を折衷することができぬものかと私は常に考へ來
つた所であつた。この文典が存在理由を要求し得るとせばこの點
に多少の考慮を拂つたといふことである。勿論非常に限られた頁
數で十分のことは望み得られないが,然しこれによつて初學入門
の人が相當複雜なこの語の要領を把握してくれるには十分役立つ
だらうといふ自信だけはもつてゐる。

本書は次の二書が資料となつてゐる。
\begin{itemize}
\item Richard Fick: Praktische Grammatik der Sanskrit-
Sprache für den Selbstunterricht. Wien und Leipzig.
\item Macdonell: A Sanskrit Grammar for Students. London. 1927
\end{itemize}

又参考書として次のものがある。
\begin{itemize}
\item Whiteney. W. D.: Sanskrit Grammar,including both the
classical language, and the older dialects of Veda and Brahmana. Leipzig, 1896.
\item Monier Williams, M.: Practical Grammar of the San\-skrit language. Oxford, 1887.
\item Bhandarkar, R.G.: First Book of Sanskrit, Bombay, 1926. Second Book of Sanskrit, Bombay, 1928.
\item Stenzler, A.F.---Pischel, R.---Geldner, K.F.: Elementar\-buch der Sanskrit-Sprache, Giessen, 1923.
\item Bühler G.: Leitfaden für den Elementarkursus des Sanskrit, Wien, 1883.
\end{itemize}

尚ほ邦語で書かれたものでは次の如きものがある。
\begin{itemize}
\item 榊亮三郎氏 解説梵語學 明治四十年\\
\hfil 京都 眞言宗高等中學校。
\item 荻原雲來氏 實習梵語學 昭和七年(十二版)\\
\hfil 東京 丙午社。
\item 阿滿得壽氏 梵語文法綱要 大正十四年\\
\hfil 京都 伏見 壽竹堂。
\end{itemize}

%%% Local Variables:
%%% mode: latex
%%% TeX-master: "IntroductionToSanskrit"
%%% End:

\chapter*{總説}\addcontentsline{toc}{chapter}{總説}
\label{cha:general}

\numberParagraph
梵語とは支那日本でのみ呼ぶ名稱で正しくはサンスクリタ
(saṃskṛta 完成されたもの)と云ふべく,卽ち歐洲人のサンスク
リット(sanskrit)と呼ぶものである。これは印度古代の文學上
の言語でパーニニ(Pāṇini)なる學匠の規定するところである。

印度最古の言語をヴェーダ(Veda 吠陀)と云ふ。これは知識
の意。印度否寧ろ世界で最古のまとまつた文學であり,大部分讚
歌の形で出來て居り,古代民族が天然の現象に對し或は畏怖し或
は讚美して唱詠せし宗敎味豐かな文献である。サンスクリットは
音韻學上大體この吠陀の言語と同一であるが若干變化した所があ
る。それで吠陀に對して後期サンスクリット或は典文サンスク
リット等と呼ぶこともある。語格から云はばそれは發達ではな
く,寧ろ退化の過程を取つたものと云へる。若干の語尾の曲りは
消滅し,幾多の新語新義が登場するに至つた。

支那で梵語と呼ぶやうになつたのは何時ごろからのことか詳か
ではないが,餘程古い時代からこの名稱が用ひられてゐる。印度で
は梵字(Brahmalipi)と云ふ名稱はあるが,梵語といふやうな稱
呼は見えない。梵字と云ふのも梵天なる神が製作せし文字と云ふ
ことで,やはりインドの古い傳説に根據がある。これらの關係から
言語に梵語と云ふ呼び方をするやうになつたのであらう。梵とは
淸淨の義と見做されてゐる。支那で飜譯された佛敎經典,卽ち大
藏經の大部分はその原語がこのサンスクリットであつたことは注
意すべきである。

\numberParagraph
吠陀の言語から幾多の方言が派生した。その最も古いもの
はパーリ(Pāli)語であり,西紀前三世紀に屬するアソカ(Asoka)
王の刻文に現はれ,現今印度佛敎徒の聖典語である。パーリ語で
出來,整備した三藏經典は佛敎硏究者の必須の寶物であるが,パ
ーリ語と梵語との關係も極めて密接であり,梵語を一通り學べば
パーリ語は自ずから通ずるやうになつてゐる。

尙ほ印度現代の方言の主要なるものに Panjābī, Sindhī, Guja\-%
rātī, Marāṭhī, Hindī, Bihārī, Bengālī がある。この中 Hindī
は Arabia 語 Persia 語が混合して Urdū 又は Hindūstānī と
云はれ印度各地を通じて用いられる交通用語である。又南印度に
Dravida 系統の Telugu, Tamil, Canara, Malayāla がある。こ
れらは梵語系統卽ちアーリヤ系統ではないが多くの梵語要素は含
まれてゐる。又 Ceylon には Sinhala 語がある。其他各地方に
夥しい數の方言がある。

\numberParagraph
此に注意すべきは印度アーリヤ民族は古代に於いて決して
文字を使用しなかつたことである。而も多くの不朽の優秀な文書
を傳へ得たことは驚嘆に値する。文字が使はれるやうになつたの
は西紀前六世紀以後であらう。卽ち西紀前約七百年頃メソポタミ
ア(Mesopotamia)を介してセム民族の書體が西北印度方面へ傳
へられた。この書體を採用した最古のものは前三世紀頃の貨幣や
刻文に見られるがこれをブラフミー(Brahmī)と云ふ。これは左
から右へ書くが曾て一度は右から左へ書かれたやうな痕跡が認め
られる。このブラフミーから數多い印度の書體が派生したのであ
る。就中最も重要なるはナーガリー(Nāgarī)書體である。或は
デーヷナーガリー(Devanāgarī)とも呼ばれ,大體紀元八世紀頃
から用ひられる書體である。現今 Sanskrit はこの書體で記すこ
とになつてゐる。

\numberParagraph
因みに少し溯つて五,六世紀頃に用ひられたものに,悉曇
(Siddhām)書體がある。これは佛敎と共に支那へ流傳し隨つて
日本へも傳來した。現在日本の寺院等に於いて或は圖像の上に或
は卒都婆の面に書かれてゐる梵字は卽ちこの悉曇書體である。日
本ではこの書體で書かれた古文書が相當に多く保存されてゐる。
尙ほ日本で出版された大藏經の中の梵字はこの書體で記されてゐ
る。正し黄蘗版の藏經は明藏の覆刻であつてその中の梵字に又別
なランツァ書體が見える。實際梵字の書體も夥しいもので
あることに注意すべきである。

かやうに澤山の書體ができるにはできてゐるが,梵語を寫すに
は別にどれに依らねばならぬといふ定りもない。只現在ではナー
ガリー文字が多く使われてゐてこれを使へば便利でもあるし,梵
語の文書を讀む場合これを知らずしてはどうも一寸工合がわる
い。梵語の學習者としては何としてもナーガリー文字を讀むこと
は勿論,これを自由に書けるだけの練習はしてほしい。併しさう
した餘裕に惠まれない人にはローマ字だけでも結構間に合ふであ
らう。若干の符號を附加してローマ字で梵語の全音を寫し得るや
うになつてゐる。日本の假名で梵語の音を寫す試みは從來いろい
ろに企てられたがどうもうまく成功しないのは遺憾である。それ
は日本の假名が總て母音を含んだ綴音になつてゐるために,純粹
子音を寫す場合に瞭然とこれを表はすよき方法がない。本書でも
最初ナーガリー字の知識を與へるだけの準備はしてあるが全卷の
用字としてはすべてローマ字を採用することとした。

これに就いて特に注意を促しておきたいのは,初學者が動もす
れば言語と文字の關係を的確に把握しないために,文字そのもの
を言語と誤認したりして,「梵語は一體どんな字ですか」などと云
ふ愚問を發するのだが,本書の學習者にはそんな考へ方があつて
はならぬ。

%%% Local Variables:
%%% mode: latex
%%% TeX-master: "IntroductionToSanskrit"
%%% End:


\part{文法}
\chapter{書法}
\numberParagraph
Devanāgarī 字母は四十八の文字にて書かれる。母音十三
子音三十五(この中に隨韻アヌスヷーラと止聲ヴィサルガを含
む)。これによつて梵語の音は總て殘る所なく寫されることにな
つてゐる。現時ローマ字に若干の符號を附加してこれを寫す。こ
の方法は曾て 1894 A.D. にゼネヷの東洋學會(Oriental Con\-%
gress)で規定したものに準據することになつてゐる。

\begin{center}
〔母音字母〕
\end{center}

\numberParagraph
デーヷーナーガリー字母は次の如くである。\endnote{底本は {\dnf अ, आ, ओ, औ} が異体字になっている。
また,{\dnf अ} の右側に {\dnf ◌} ではなく「━」のような文字が入っている。}

\begin{tabular}{cc}
  \begin{minipage}{0.46\hsize}
  \begin{tabular}{cccll}
  {\dnf अ} & {\dnf ◌} & a & ア & \rdelim\}{9}{*}[\textbf{單母音}] \\
  {\dnf आ} & {\dnf ा} & ā & アー & \\
  {\dnf इ} & {\dnf ि} & i & イ & \\
  {\dnf ई} & {\dnf ी} & ī & イー& \\
  {\dnf उ} & {\dnf ु} & u & ウ & \\
  {\dnf ऊ} & {\dnf ू} & ū & ウー & \\
  {\dnf ऋ} & {\dnf ृ} & ṛ & リ & \\
  {\dnf ॠ} & {\dnf ॄ} & ṝ & リー & \\
  {\dnf ऌ} & {\dnf ॢ} & ḷ & リ &
  \end{tabular}
  \end{minipage}
  &
  \begin{minipage}{0.46\hsize}
  \begin{tabular}{cccll}
  {\dnf ए} & {\dnf े} & e & エー & \rdelim\}{4}{*}[\textbf{複母音}] \\
  {\dnf ऐ} & {\dnf ै} & ai & アーイ & \\
  {\dnf ओ} & {\dnf ो} & o & オー & \\
  {\dnf औ} & {\dnf ौ} & au & アーウ &
  \end{tabular}
  \vspace{1\zh}

  \renewcommand{\arraystretch}{1.5}
  \begin{tabular}{ccll}
  {\dnf ः} & ḥ & フ(止聲)& \rdelim\}{2}{*}[\parbox{6\zw}{この二個は母音に隨ふ符號である。}] \\
  {\dnf ँ} & ṃ & ン(隨韻)&
  \end{tabular}
  \renewcommand{\arraystretch}{1}
  \end{minipage}
\end{tabular}

\begin{center}
〔子音字母〕
\end{center}

\begin{tabular}{cc}
  \begin{minipage}{0.46\hsize}
  \begin{tabular}{llp{\widthof{チュハ}}lll}
  {\dnf क} & ka & カ & 無氣 & \rdelim\}{2}{*}[\parbox{1\zw}{無響}] & \rdelim\}{5}{*}[喉音] \\
  {\dnf ख} & kha & クハ & 含氣 & & \\
  {\dnf ग} & ga & ガ & 無氣 & \rdelim\}{2}{*}[\parbox{1\zw}{有響}] & \\
  {\dnf घ} & gha & グハ & 含氣 & & \\
  {\dnf ङ} & ṅa & ンガ & 鼻音 &
  \end{tabular}
  \end{minipage}
  &
  \begin{minipage}{0.46\hsize}
  \begin{tabular}{llllll}
  {\dnf प} & pa & パ & 無氣 & \rdelim\}{2}{*}[\parbox{1\zw}{無響}] & \rdelim\}{5}{*}[\parbox{1\zw}{唇音}] \\
  {\dnf फ} & pha & プハ & 含氣 & & \\
  {\dnf ब} & ba & バ & 無氣 & \rdelim\}{2}{*}[\parbox{1\zw}{有響}] & \\
  {\dnf भ} & bha & ブハ & 含氣 & & \\
  {\dnf म} & ma & マ & 鼻音 &
  \end{tabular}
  \end{minipage}
  \\
  \begin{minipage}{0.46\hsize}
  \begin{tabular}{llp{\widthof{チュハ}}lll}
  {\dnf च} & ca & チャ & 無氣 & \rdelim\}{2}{*}[\parbox{1\zw}{無響}] & \rdelim\}{5}{*}[上顎音] \\
  {\dnf छ} & cha & チュハ & 含氣 & & \\
  {\dnf ज} & ja & ジャ & 無氣 & \rdelim\}{2}{*}[\parbox{1\zw}{有響}] & \\
  {\dnf झ} & jha & ジュハ & 含氣 & & \\
  {\dnf ञ} & ña & ニャ & 鼻音 &
  \end{tabular}
  \end{minipage}
  &
  \begin{minipage}{0.46\hsize}
  \begin{tabular}{llll}
  {\dnf य} & ya & ヤ(上顎音) & \rdelim\}{4}{*}[\parbox{3\zw}{有響音半母音}] \\
  {\dnf र} & ra & ラ(舌音) & \\
  {\dnf ल} & la & ラ(齒音) & \\
  {\dnf व} & va & ヷ(唇音) &
  \end{tabular}
  \end{minipage}
  \\
  \begin{minipage}{0.46\hsize}
  \begin{tabular}{llp{\widthof{チュハ}}lll}
  {\dnf ट} & ṭa & タ & 無氣 & \rdelim\}{2}{*}[\parbox{1\zw}{無響}] & \rdelim\}{5}{*}[舌音] \\
  {\dnf ठ} & ṭha & トハ & 含氣 & & \\
  {\dnf ड} & ḍa & ダ & 無氣 & \rdelim\}{2}{*}[\parbox{1\zw}{有響}] & \\
  {\dnf ढ} & ḍha & ドハ & 含氣 & & \\
  {\dnf ण} & ṇa & ナ & 鼻音 &
  \end{tabular}
  \end{minipage}
  &
  \begin{minipage}{0.46\hsize}
  \begin{tabular}{llll}
  {\dnf श} & śa & シャ(上顎音) & \rdelim\}{3}{*}[\parbox{3\zw}{無響音硬吹氣}] \\
  {\dnf ष} & ṣa & シャ(舌音) & \\
  {\dnf स} & la & サ(齒音) & \\
  {\dnf ह} & ha & ハ(喉音) & 軟吹氣
  \end{tabular}
  \end{minipage}
  \\
  \begin{minipage}{0.46\hsize}
  \begin{tabular}{llp{\widthof{チュハ}}lll}
  {\dnf त} & ta & タ & 無氣 & \rdelim\}{2}{*}[\parbox{1\zw}{無響}] & \rdelim\}{5}{*}[齒音] \\
  {\dnf थ} & tha & トハ & 含氣 & & \\
  {\dnf द} & da & ダ & 無氣 & \rdelim\}{2}{*}[\parbox{1\zw}{有響}] & \\
  {\dnf ध} & dha & ドハ & 含氣 & & \\
  {\dnf न} & na & ナ & 鼻音 &
  \end{tabular}
  \end{minipage}
  &
\end{tabular}

この順序は辭典の語彙排列等の標準となり一定不變のものである。

\begin{enumerate}[label=(\alph*)]
\item 子音に用ひられし文字は常に a 音を伴つてゐる。{\dnf क} = ka
{\dnf त} = ta 等。故に純粹の子音を表はすには Virāma なる符號,
卽ち格字の下に斜線を加ふ。例 {\dnf क्} = k,{\dnf त्} = t.
\item デーヷーナーガリー字を書くには各字の特殊なる部分
先づ書き次に垂直線,最後に水平線を引くを原則とす。\endnote{底本では {\dnf त्} の筆順が例示されているが,省略。}
\item 語首の {\dnf अ} a が聲音の規定により消失する時は(14 條)
Avagrapha と呼ばるる符號 {\dnf ऽ} を以て表はす。例 {\dnf तेऽपि} te 'pi
(彼等も亦)は {\dnf ते अपि} のことである。
\end{enumerate}

\nt{
\begin{enumerate}[label=(\arabic*)]
\item 五種の鼻音が夫々自己と同類の子音に先立つ時に簡便を尚
ぶより隨韻を以つて代用することあるも,これは正しい方法ではない。
{\dnf अङ्कित} aṇkita を {\dnf अंकित} aṃkita とし {\dnf कम्पित} kampita を {\dnf कंपित}
kaṃpita とする類である。勿論隨韻を代用しても發音は夫々五種
の鼻音通りになすべきである。又同樣に文章の終の {\dnf म} m が時として隨
韻で寫される。{\dnf अहम} aham を {\dnf अहं} ahaṃ とするが如きである。然し
これは誤謬を惹起し易いから避ける方が望ましい。甚だまぎらはしき方法である。隨韻
の使用は 28 條の範圍を以て大體の限界とし原則を立てるべきである。前
加語 sam の如きは特に獨立語の如く見做されるので saṃ とするは當然
である。所が印度の寫本などはとかく簡便をよいことにして代用隨韻を濫
用する傾向がある。學者はなるべくこれに隨はないやうにしたい。
\item 隨韻と止聲とはその字母中に於ける順序位置に關して初學を惑は
すことも少くない。辭典の檢索に於いて次のことは心得置くべきである。
\item 半母音吹氣音に先立つ隨韻は總ての子音に先立つ。故に saṃ\-%
vara, saṃśaya は sa-ka より先に置かれる。これを純粹隨韻と稱す。
\item 之に對して變化隨韻又は代用隨韻なるものがある。それは喉等五
類の子音の前にある場合は隨韻の形を取るもそのまゝその類の鼻音に代用
し得べきものである。これは夫々その鼻音の位置にあるものである。例せ
ば saṃkāśa は sa-ga の前に置かれる。saṃ とあるも saṅ であるが如
く見做すべきものである。
\item これと同樣に,硬喉唇音に先立つ純粹止聲は總ての子音に先立つ。
故に antaḥkaraṇa や antaḥpura は anta の後で anta-ka の前に置
かれる。然し硬吹氣音に先立つ止聲は變化又は代用止聲であつて夫々の硬
吹氣音と見做されねばならぬ。卽ち ḥś, ḥṣ, ḥs は,各々 ś, ṣ, s の後に
置かれる。
\end{enumerate}}

\begin{enumerate}[label=(\alph*), start=4]
\item a 以外の母音が子音に續く場合は夫々の母音の略符を附
加して文字を作る。今 {\dnf क} ka, を以て例すれば {\dnf का} kā, {\dnf कि} ki,
{\dnf की} kī, {\dnf कु} ku, {\dnf कू} kū, {\dnf कृ} kṛ, {\dnf कॄ} kṝ, {\dnf कॢ} kḷ, {\dnf कॆ} ke, {\dnf कै} kai, {\dnf कॊ} ko,
{\dnf कौ} kau. となす。其他の字も推知すべきである。

稍特殊の形を例示せば {\dnf रु} ru, {\dnf रू} rū. 等。
\item {\dnf र्} r は子音または {\dnf ऋ} ṛ 母音に先立つ時は c の如く記
され字の上に置かれる。例 {\dnf अर्क} arka {\dnf निरृतिः} nirṛtiḥ 等。
又 {\dnf र्} r が子音の後に來る時は右より左へ下に向ふ斜線にて
記さる。例 {\dnf क्र} kra 等。
\end{enumerate}

\printParagraphCounter
子音の結合の主要なるものは次の如くである。

% {\dnf क्क} k-ka, {\dnf क्ख} k-kha, {\dnf क्च} k-ca, {\dnf कण} k-ṇa, {\dnf क्त} k-ta, {\dnf क्त्य} k-t-ya,
(略)\endnote{底本のグリフ(リガチャ)を正確に再現できるフォントがないため,省略。}

數字は次の如くである。

\begin{center}
\begin{tabular}{cccccccccc}
  {\dnf १} & {\dnf २} & {\dnf ३} & {\dnf ४} & {\dnf ५} & {\dnf ६} & {\dnf ७} & {\dnf ८} & {\dnf ९} & {\dnf ०} \\
  1 & 2 & 3 & 4 & 5 & 6 & 7 & 8 & 9 & 0
\end{tabular}
\end{center}

句讀の符號は小段落に {\dnf ।},大段落に {\dnf ॥} を用ふ。

\newpage
\def\enotesize{\normalsize}
\theendnotes

%%% Local Variables:
%%% mode: latex
%%% TeX-master: "IntroductionToSanskrit"
%%% End:

\chapter{聲法}
\label{cha:pronunciation}

\section{發音}
\numberParagraph
以上の字母の發音に就て,大體近似せる音を假名を以て
示したが,二三の注意すべき點を下に擧げる。ṛ, ṝ は特殊の母
音である。リ リーと發音すれば大差なし。ḷ は li の音に近い。e,
o は夫々エー,オーと長く,ai, au もアーイ,アーウである。長音
符號はなきも必ず長く發音する。ṅ は ng の如く,kha, gha
等の h は聞えるやうに,クハ,グハであつて,カ,ガでない。ñ は
ny の如く,ṭ, ḍ, ṇ 等は舌端を巻き上顎齒根へ觸れて發音する。
ś は舌の根と上顎の間を氣息が擦過して發音せられる。具體的に
は舌の中央を高める心地にてシャを發音すれば可い。ṣ は sh の
如くシャである。h は喉音であつて全く口腔の聲音期關に關係せ
ぬ。ṃ も ḥ も必ず母音に隨ふものであつて ṃ は母音となるべ
き氣息が鼻腔に入つて外に出づるより生ずる音である。ḥ は母音
の發生に伴つて起る一種の摩擦音である。故に前に來る母音の影
響を受けて aḥ, iḥ, uḥ は夫々 アハ,イヒ,ウフ等となる。

\numberParagraph
子母音全部を發聲機關に隨つて喉,上顎,舌,齒,唇の五種に
分類し,又半母音,吹氣音の命名あること前述の如くである。更に
子音中含氣あり,無氣あり,硬あり,軟あり。含氣とは h 音を含む
ものであり,無氣とは含まざるものを云ふ。硬とは無響,軟とは
有響と見るべく,發音に當り聲帶の振動を伴ふか伴はざるかによ
りて次の如く有響無響を分つ。卽ち k, kh, c, ch, ṭ, ṭh, t, th, p,
ph, ś, ṣ, s の十三は無響であり其の他は子母音を通じて有響と
知るべきである。

\numberParagraph
單母音は a 音を添へて强められ,その質を變化して\textbf{複母音}
となる。複母音には二重の段階がある。第一重を重韻,グナ(guṇa)
と云ひ第二重を複重韻,ヴィリッド\textsubscript{ヒ}(vṛddhi)と云ふ。卽ち i, ī の
重韻は e であつて u, ū の重韻は o である。又 ai, au は夫々
その\textbf{複重韻}である a, ā の重韻は無く ā はその複重韻である。尚
ほ ṛ, ṝ の重韻は ar で\textbf{複重韻}は ār であり ḷ の重韻は al であ
つて複重韻は存在しない。又 ṛ の重韻は時に ra となり複重韻
は rā となることもある。

\begin{center}
\fbox{
\begin{tabular}{cccccc}
  平音   & a ā & $\underbrace{\text{i\hspace{1em} ī}}$ & $\underbrace{\text{u\hspace{1em} ū}}$ & $\underbrace{\text{ṛ\hspace{1em} ṝ}}$ & ḷ \\
  重韻   &     & e   & o   & ar(ra) & al \\
  複重韻 & ā   & ai  & au  & ār(rā) & \\
\end{tabular}}
\end{center}

造語の場合に於ける次の例を見てこれらの變化を知るべきである。

\begin{tabular}{lll}
  div (輝く) & deva (天) & daiva (運命) \\
  lul (轉ず) & lola (貪慾の) & \\
  bhṛ (\ruby{擔}{にな}ふ) & bhara (保持する) & bhāra (荷物) \\
  kṛ (作す) & kartṛ (作者) & kārya (義務) \\
  kḷp (適當にする) & \multicolumn{2}{l}{kalpate (彼は適當である)} \\
\end{tabular}

\section{連聲法(Saṃdhi)}
\numberParagraph
梵語には語と語の間に特殊の處置が施される。卽ち語末と
語首の音が相會してそこに性質を變ずるか,又は消失するか,又
は何か他の音が挿入せられるか等の變化がある。これを連聲法
(サンドヒ)と云ふ。これには母音と母音とに相會する場合,子音と
子音と相會する場合を分つて説くことにする。

\subsection{語と語の連聲法}
\subsubsection{母音相互間の連聲法}
\numberParagraph\label{np:11}
等しき單母音はその長韻となる。

\eg{
  adya api = adyāpi (今日も亦)\\
  yadi icchasi = yadīcchash (若しも汝が欲するならば)\\
  sādhu uktam = sādhūktam (よくも云はれた)。}

\numberParagraph
a, ā は等しからざる單母音と共に重韻となり e, o, ai, au
と共に複重韻となる。
\eg{
  ca iti = ceti (而して以上)\\
  tena uktam = tenoktam (彼は云へり)\\
  adhunā ṛṣir yajate = adhunarṣir yajate (今聖仙は
  祭祀す)\endnote{底本は「祀」ではなく「示己」。} \\
  kva eti = kvaiti (彼は何處に行く)\\
  atra oṣadhī tiṣṭhati = atrauṣadhī tiṣṭhati (此に藥
  草がある)。}

\numberParagraph
i, ī, u, ū, ṛ, ṝ は等しからざる母音の次前に在る時
各自相當の半母音 y, v, r となる。
\eg{
  yadi evam = yady evam (若し此の如くならば)\\
  astu etat = astv etat (これをしてあらしめよ)。}

\numberParagraph
a 以外の母音の次前に來る e, o は轉じて a となる(稀に
は ay, av)。
\eg{
  vane iha = vana iha (此に林に於て)\\
  prabho ehi = prabha ehi (王よ來れ)}

e 又は o の次後の a は省略せらる。而して省字 \,' を用ふ。\\
\hfill 6 條(c)
\eg{
  te atra = te 'tra (彼等はそこに)\\
  so api = so 'pi (彼も亦)。}

\numberParagraph
ai は母音の次前に來る時は ā となり,au は通常は āv となる。
\eg{
  bhāryāyai akathayat = bhāryāyā akathayat (彼は妻
  に語れり)\\
  putrau āgacchataḥ = putrāv āgacchataḥ (二人の子
  等は來れり)。}

\nt{格例法及活用法の兩數の語尾 ī, ū, e 及び amī (代名詞)の ī は
次後に母音來るも變化せず。}

\subsubsection{子音相互間の連聲法}
\numberParagraph
二個以上の子音が聲の終に來る時はその第一の聲のみを
存し,餘は總て除き去る。
\eg{
  prāṅks = prāṅ (東方の)\\
  bhavants = bhavan (bhū の現在分詞)〔主・男・單〕\\
  ahant = ahan (彼は打ちき)}
但し r が語根に屬する(又は語根に代用せらるる)k, ṭ, t, p
に先立つ場合は殘存せしむ。
\eg{
  ūrk (力)(k は j の代用)\\
  amārṭ (彼は拭へり)(語根 mṛj, ṭ は j の代用)。}

\numberParagraph \label{np:17}
連聲の法則は語の終に立つを許さるる次の九個の音の孰
れかに還元したる後適用せらるべきである。

\begin{center}
\begin{tabular}{cccccc}
  k & ṭ & t & p & & \\
  ṅ & ṇ & n & m & 並に & ḥ
\end{tabular}
\end{center}
三十四個の子音はこれらの九個の孰れかに還元せらる。卽ち語末
の音は硬無氣であらねばならぬ。隨て語末の音は必ずこれに變へ
られる。上顎音(ś を含む)及び h は k 又は ṭ に(ñ は ṅ
に),ṣ は ṭ に,s 及び r は ḥ に置き替へられる。y, l, v は決
して語末に來ない。

\subsubsection{同化の法則}
\numberParagraph
硬音は軟音の次前には軟音となり,軟子音は硬子音の次前
には硬となる。
\eg{
  vanāt āgacchati = vanād āgacchati (彼は林より來
  れり)\\
  āpad kāle = āpat kāle (不幸の時に)。}

\numberParagraph
硬軟共に鼻音の前には鼻音となる。
\eg{
  vāc-mātreṇa = vāṅ-mātreṇa (言葉のみを以て)\\
  saṭ-māsān = saṇ-māsān (六ケ月の間)\\
  yāvat na = yāvan na (乃至......無し)。}

\numberParagraph
語の始の h は次前の子音の軟含氣音と變ず。
\eg{
  samyak-huta = samyag-ghuta (正しく供へられたる)\\
  tak kasmāt hetoḥ = tat kasmād dhetoḥ (其の故は
  如何)。}

\numberParagraph
齒音は次下の上顎音下音又は l と同化し,次に來る ś と
共に cch なとる\endnote{「なとる」はママ。}。
\eg{
  yāvat jīvam = yāvaj jīvam (生命の限り)\\
  tad jalam = taj jalam (その水は)\\
  vidyut-latā = vidyul-latā (電蘿)\\
  tad śrutvā = tac chrutvā (それを聞きて)。}

\subsubsection{s と r に關する法則}
\numberParagraph
硬音の前の s.\\
硬齒音の前には變化せず。\\
次下の硬顎音舌音と同化して ś,時としては ṣ となる。\\
その他の硬音の前(並に文章の終末には)s は ḥ となる。
\eg{
  tatas tena bhaṇitam (かくて彼は云へり)\\
  tatas ca = tataś ca (而もかくて)\\
  tatas kṣipati = tataḥ kṣipati (彼は投ぐ)。}

\numberParagraph
a, ā 以外の母音の次後の s は軟音の前に r となる。r の前に
は消失して先立つ短母音は延長せらる。
\eg{
  agnis idhyate = agnir idhyate (火は點ぜられたり)\\
  tayos ekahḥ = tayor ekaḥ (二人のうちの一人)\\
  kavis racayati kāvyam = kavī racayati kāvyam (詩
  人は一の詩を作れり)。}

\nt{
  bhos 間投詞は軟音の前にその s を去る。
  \eg{
    bhos mitra = bho mitra (おゝ友よ)。}}

\numberParagraph
語の終の as は軟子音又は a の次前にある時は o となる。
\eg{
  devas jayatu = devo jayatu (王をして勝利あらしめよ)\\
  putras api = putro 'pi (子も亦)}
その餘の母音の次前に來る時は s 消失す。
\eg{
  siṃhas āha = siṃha āha (獅子は云へり)。}

\nt{
  代名詞の sas eṣas は總て sa eṣa となる。只 a の前には so
  eṣo となり。文章の結尾に saḥ eṣaḥ となるのみ。}

\numberParagraph
語末の r は語末の s の如く取り扱はれる。只軟音の前に變
化しない。
\eg{
  punaḥ punar upakuryāt (人は再三善をなすべし)}
r の前には消失して先立つ短母音は延長せられる(23條)。

\subsubsection{鼻音の變化}
\numberParagraph
語末の n は軟の上顎音下音及び ś の前に在る時これと
同類の鼻音となる。
\eg{
  vṛttāntān jānāmi = vṛttāntāñ jānāmi (我は出來事
  を知る)}
又 s は變じて ch となることもある。
\eg{
  tān śaśāpa = tāñ śaśāpa 又は tāñ chaśāpa (彼は彼
  等を\ruby{詛}{のろ}へり)}
l の次前の n は ṃl 又は \anunasikam{}l となる。
\eg{
  asmin loke = asmiṃl loke (この世界に於いて)。}

\numberParagraph
語末の n と次下の硬の上顎音下音及び齒音との間にはこ
れらの音に相當する硬吹氣音を挿入し且つ n は隨韻となる。
\eg{
  kasmin cid vane = kasmiṃś cid vane (とある林に於
  いて)\\
  agaman tataḥ = agamaṃs tataḥ (かくて彼は往けり)。}

\numberParagraph
語末の m は總ての子音の前に隨韻となる。母音の前には
變化しない。
\eg{
  uktam ca = uktaṃ ca (而してそれは云はれたり)\\
  kim karomi = kiṃ karomi (吾れは何を爲すべきか)\\
  sam-gacchati = saṃgacchati (彼は共に來れり)。}

\numberParagraph
m 以外の鼻音が語末にありて短母音に先立たれた時は重
複せられる。
\eg{
  tasmin adrau = tasminn adrau (その山の上に)\\
  smaran api = smarann api (彼は念じつゝも)。}
\nt{
  語の始の ch は短母音又は前置詞 ā 又は否定詞 mā の次後に來
  る時は cch となる。}

\subsection{一語中の連聲法}
格例法,活用法又は造語法に關し,後接字を加ふる場合に聲音
の變化を起こすことがある.大體上述の原則を適用すべきも若干の
注意すべきものがある。

\numberParagraph
一綴の名詞又は時としては動詞の語根及びその語幹 i,
ī, u, ū は母音にて始まる語尾の前には iy, uv と變化す。此等
が子音の次前に在るとき殊に然り。
\eg{
  bhū + i = bhuvi (地上に)\\
  su + e = suve (我は生む)\\
  śaknu + anti = śaknuvanti (彼らは能ふ)。}

\numberParagraph
複母音 e, ai, o, au は次後の母音並に y の前に ay, āy,
av, āv となる。
\eg{
  nī = ne + ana = nay + ana (眼)\\
  bhū + a + ti = bho = bhav + a + ti (彼は有り)}

\numberParagraph
語根に屬せる r 又は v の次前なる i または u は其語根の
次後に子音來る時は槪ね延長せられる。
\eg{
  div + yati = dīvyati (彼は賭事を遊ぶ)\\
  pur + bhiḥ = pūrbhiḥ (都邑によりて)。}

\numberParagraph
語根又は語基の末尾にある子音は母音,半母音又は鼻音に
て始まる後接字の次前にある時は變化しない。其の他の後接字の
次前にある時は \ref{np:11} 以下の諸條を適用する。
\eg{
  marut + am = marutam (風を)\\
  vāc + ya = vācya (言はるべき)
  \item[されど] marut + bhyas = marudbhyaḥ}

\numberParagraph
t 又は th は後接字の聲の始にして軟含氣音の次後に在る
時は軟音となり且つその吹氣音 h を己に移す。
\eg{
  labh + ta = labdha (得られたる)\\
  rundh + thaḥ = runddhaḥ (汝等兩人は拒む)。}

\numberParagraph
g, d, b で始まり gh, dh, bh, h で終る語根が \ref{np:17} 條によ
りてその含氣を失つた場合に始の音が含氣となる。これを代償の
法則と云ふ。例:duh (乳搾る)~dhuk, budh (賢なる)~bhut.

\numberParagraph
齒音は舌音の次後に來る時は通常は舌音となる。
\eg{
  iṣ + ta = iṣṭa (欲したる)\\
  dviṣ + dhi = dviḍḍhi (憎め)\\
  īḍ + te = īṭṭe (彼は\ruby{讚}{ほ}む)
  \item[されど] saṭsu (六に於て)。}

\numberParagraph
ś は t の次前に在る時は ṣ となる。
\eg{
  dṛś + ta = dṛṣṭa (見られたる)。}
この外の子音の次前にあるとき ś と ṣ は \ref{np:17} 條に準ず。

\numberParagraph
n は母音又は n, m, y, v にて從はれ、ṛ, ṝ, r, ṣ に先立たれ
た時には ṇ に變ず。但し母音喉音唇音 ṃ, y, v, h の中間に入り
來るを妨げない。
\eg{
  kar + ana = kāraṇa (因)\\
  brahman + ya = brāhmaṇya (信仰ある)\\
  pūṣan の屬單 pūṣṇāḥ \\
  grah + nāti = gṛhṇāti (彼は取る)\\
  Rāmāyaṇa (ラーマ物語)}

\nt{
  前接字 nis, parā, pari 又は pra にて先だたれた前接字 ni 並
  に大槪の語根の n は舌音化す。}

\numberParagraph
s は k, r, l 又は a, ā 以外の母音によりて直に先立たれた
時 ṣ に變ず。但し s が語尾に在るか又は r が直に從ふ場合を
除く。隨韻止聲の中間に入り來るを妨げない。
\eg{
  dhanus + aḥ = dhnuṣaḥ (弓)屬單 \\
  dhanus = dhanūṃṣi (複主)\\
  vac + syati = (vak + syati =) vakṣyati (彼は語るべし)\\
  sarpis = sarpiṣā (淨酪を持って)具單
\item[されど] sarpiḥ (語の終)\\
  manasā (a に先立たる)\\
  tamisram (r 從ふ)。}

\newpage
\theendnotes

%%% Local Variables:
%%% mode: latex
%%% TeX-master: "IntroductionToSanskrit"
%%% End:

\chapter{語尾曲法}\label{cha:inflection}
\section{格例法}
\numberParagraph
言語の各類に就て變化するものと變化しないものとがあ
る。變化しないものとは接續詞,間投詞,副詞の如きである。今
語尾曲法に於て取扱はうとするものはこれら変化せざるものを暫
らく除外して變化するものに就て論ずる。變化する詞にも二樣の
種類が分れる。一は卽ち名詞的變化であつて性と數と格によつて
語尾を異にするもの。この種に屬するものは名詞,代名詞,形容
詞,數詞,分詞である。他は卽ち動詞的變化であつて人稱と數と
時と法とに隨つて語尾を異にする。前者の取扱ひを格例法と云
ひ,後者の取扱ひを活用法と云ふ。

\numberParagraph
梵語では格例法に三性 三數 八格を認める。三性とは男
女 中であり,三數とは單 兩 複であり,八格とは主 業 具 爲
從 屬 於 呼である。活用法には三の人稱と 三數と 三時と 四
法を認める。曰く 一人稱 二人稱 三人稱,曰く單 兩 複,曰く
現在 過去 未來,曰く現實法,可能法,命令法,條件法である。
これらの意義は後節に説く。

\numberParagraph
梵語では名詞的變化は語基を以て,動詞的變化は語根を以
て單位とする。單位とは分解の終極を意味し,それ以上溯る必要
を見ない形である。この形で辭典に記録されるものと知るべきで
ある。嚴密に云はば如何なる詞もその語根まで溯るべきである
が,名詞,代名詞,形容詞,數詞,分詞は必ずしも語根の詮索の
手を延すを要しない。出來上つた語基の形に語尾が附けられる。
然し動詞は語根によりて論ぜられる。之を要するに語根は梵語の
あらゆる語彙を分析してこれを原始の狀態に還元したものであ
る。印度平原に花咲く言語から得た收穫である。萬を以て數へら
れる言語が數百の語根に還元せらるゝことは實に言語學の寄與貢
献の偉大を示すものである。

\numberParagraph
語根は多くは單綴の音であり,意義はあるも未だ言語では
ない。云はば言語のエキスである。今語根に若干の変化を施して
語基を作り,これに語尾を附加すれば始めて活用する言語とな
る。或は之を逆に考へて活用する言語を比較し分析して語尾を切
離し,語基を整理して最後の形を得たとすればそれが語根であ
る。

\numberParagraph
\textbf{名詞的變化}に就て\textbf{語基が母音}で終るものと\textbf{子音で終るも
  の}との二類を分つ。前者を母音語基,後者を子音語基と呼ぶ。母
音語基は語尾が附加せられて太だ不規則なる變化をなすから,稍
複雜ではあるが,語基に强弱を認むる必要あまりなきと,數に於
いて飛び離れて多いとの理由から,これを最初に取扱ふことゝす
る。先づ最初に a にて終る男性中性の語基を擧げる。女性は a
にて終る場合が無い。

\numberParagraph
\hfil 1) \fbox{a 語基,男性 deva(神)} \hfil\,\addcontentsline{toc}{subsection}{1) a 語基 男性}
\begin{center}
\begin{tabular}{c*{3}{p{0.2\hsize}}}
     & \cellAlign{c}{單} & \cellAlign{c}{兩}          & \cellAlign{c}{複} \\
  主 & devas             & \rdelim\}{3}{*}[devau]     & \rdelim\}{2}{*}[devās] \\
  呼 & deva              &                            & \\
  業 & devam             &                            & devān \vspace{0.5\zw} \\
  具 & devena            & \rdelim\}{3}{*}[devābhyām] & devais \\
  爲 & devāya            &                            & \rdelim\}{2}{*}[devebhyas] \\
  從 & devāt             &                            & \vspace{0.5\zw} \\
  屬 & devasya           & \rdelim\}{2}{*}[devayos]   & devānām \\
  於 & deve              &                            & deveṣu
\end{tabular}
\end{center}

\numberParagraph
\hfil 2) \fbox{a 語基,中性 dāna(施} \hfil\,\addcontentsline{toc}{subsection}{2) 同 中性}
\begin{center}
\begin{tabular}{c*{3}{p{0.2\hsize}}}
     & \cellAlign{c}{單}      & \cellAlign{c}{兩}          & \cellAlign{c}{複} \\
  主 & \rdelim\}{2}{*}[dānam] & \rdelim\}{3}{*}[dāne]      & \rdelim\}{3}{*}[dānāni] \\
  業 &                        &                            & \\
  呼 & dāna                   &                            &
\end{tabular}
\end{center}
餘は全く男性の如く變化する。

\numberParagraph
a 語基の男中性形容詞もこれに準じて變化する。例:基
pāpa (罪ある),主男 pāpas,中 pāpam 等。形容詞の女性は ā
又は ī に終る。格變化はその條を參照。形容詞の中性單數主格
の形は副詞の意義を有す。例:śīghra (速かなる),śīghram (速
かに)。

\numberParagraph
\hfil 3) \fbox{i 語基,男女中性} \hfil\,\addcontentsline{toc}{subsection}{3) i 語基}

男性 kavi (詩人),女性 mati (意),中性 vāri (水).

\begin{center}
\begin{tabular}{c*{3}{p{0.2\hsize}}}
  \multicolumn{4}{c}{單} \\
     & \cellAlign{c}{男}      & \cellAlign{c}{女}              & \cellAlign{c}{中} \\
  主 & kavis                  & matis                          & vāri \\
  呼 & kave                   & mate                           & vāri, vāre \\
  業 & kavim                  & matim                          & vāri \\
  具 & kavinā                 & matyā                          & vāriṇā \\
  爲 & kavaye                 & mataye, matyai                 & vāriṇe \\
  從 & \rdelim\}{2}{*}[kaves] & \multirow{2}{*}{mates, matyās} & \multirow{2}{*}{vāriṇas} \\
  屬 &                        &                                & \\
  於 & kavau                  & matau, matyām                  & vāriṇi
\end{tabular}
\end{center}
\begin{center}
\begin{tabular}{c*{3}{p{0.2\hsize}}}
  \multicolumn{4}{c}{兩} \\
  主 & \rdelim\}{3}{*}[kavī]      & \multirow{3}{*}{matī}      & \multirow{3}{*}{vāriṇī} \\
  呼 &                            &                            & \\
  業 &                            &                            & \\
  具 & \rdelim\}{3}{*}[kavibhyām] & \multirow{3}{*}{matibhyām} & \multirow{3}{*}{vāribhyām} \\
  爲 &                            &                            & \\
  從 &                            &                            & \\
  屬 & \rdelim\}{2}{*}[kavyos]    & \multirow{2}{*}{matyos}    & \multirow{2}{*}{vāriṇos} \\
  於 &                            &                            &
\end{tabular}
\end{center}
\begin{center}
\begin{tabular}{c*{3}{p{0.2\hsize}}}
  \multicolumn{4}{c}{複} \\
  主 & \rdelim\}{2}{*}[kavayas]   & \multirow{2}{*}{matayas}   & \rdelim\}{3}{*}[vārīṇi] \\
  呼 &                            &                            & \\
  業 & kavīn                      & matīs                      & \\
  具 & kavibhis                   & matibhis                   & vāribhis \\
  爲 & \rdelim\}{2}{*}[kavibhyas] & \multirow{2}{*}{matibhyas} & \multirow{2}{*}{vāribhyas} \\
  從 &                            &                            & \\
  屬 & kavīnām                    & matīnām                    & vārīṇām \\
  於 & kaviṣu                     & matiṣu                     & vāriṣu
\end{tabular}
\end{center}

\numberParagraph \label{np:49}
\hfil 4) \fbox{若干の不規則なるもの} \hfil\,\addcontentsline{toc}{subsection}{4) 若干の不規則なるもの}
\begin{enumerate}[label=(\alph*)]
\item sakhi (男性,友)。
\begin{description}[font=\normalfont]
\item[單] sakhā, sakhe, sakhāyam, sakhyā, sakhye, sakhyus,
sakhyou.
\item[兩] sakhāyau, sakhibhyām, sakhyos.
\item[複] sakhāyaś, sakhīn, sakhibhis. 餘は kavi に準ず。
\end{description}
\item pati は「夫」の意味の場合,單具 patyā,爲 patye,從屬
patyus,於 patyau. 又「主」の意味に用ひらるゝか又は bhū-pati
(地主)の如く合成語の終にある時は kavi に準じて變化す。
\item \label{item:49c} akṣi (眼),asthi (骨),dadhi (酸乳),sakthi (腿)は an
語基(86條參照)に準じて變化する部分を有す。單具 akṣṇā 等
(87條)。
\end{enumerate}

\numberParagraph
\hfil 5) \fbox{u 語基、男性 guru (師),女性 dhenu (牝牛),中性 madhu (蜜)} \hfil\,\addcontentsline{toc}{subsection}{5) u 語基}
\begin{center}
\begin{tabular}{c*{3}{p{0.2\hsize}}}
  \multicolumn{4}{c}{單} \\
     & \cellAlign{c}{男}      & \cellAlign{c}{女}                & \cellAlign{c}{中} \\
  主 & gurus                  & dhenus                           & madhu \\
  呼 & guruo                  & dheno                            & madhu, madho \\
  業 & gurum                  & dhenum                           & madhu \\
  具 & guruṇā                 & dhenvā                           & madhunā \\
  爲 & gurave                 & dhenave, dhenvai                 & madhune \\
  從 & \rdelim\}{2}{*}[guros] & \multirow{2}{*}{dhenos, dhenvās} & \multirow{2}{*}{madhunas} \\
  屬 &                        &                                  & \\
  於 & gurau                  & dhenau, dhenvām                  & madhuni
\end{tabular}
\end{center}
\begin{center}
\begin{tabular}{c*{3}{p{0.2\hsize}}}
  \multicolumn{4}{c}{兩} \\
  主 & \rdelim\}{3}{*}[gurū]      & \multirow{3}{*}{dhenū}      & \multirow{3}{*}{madhum} \\
  呼 &                            &                             & \\
  業 &                            &                             & \\
  具 & \rdelim\}{3}{*}[gurubhyām] & \multirow{3}{*}{dhenubhyām} & \multirow{3}{*}{madhubbyām} \\
  爲 &                            &                             & \\
  從 &                            &                             & \\
  屬 & \rdelim\}{2}{*}[gurvos]    & \multirow{2}{*}{dhenvos}    & \multirow{2}{*}{madhunos} \\
  於 &                            &                             &
\end{tabular}
\end{center}
\begin{center}
\begin{tabular}{c*{3}{p{0.2\hsize}}}
  \multicolumn{4}{c}{複} \\
  主 & \rdelim\}{2}{*}[guravas]   & \multirow{2}{*}{dhenavas}   & \rdelim\}{3}{*}[madhūni] \\
  呼 &                            &                             & \\
  業 & gurūn                      & dhenūs                      & \\
  具 & gurubhyas                  & dhenubhis                   & madhubhis \\
  爲 & \rdelim\}{2}{*}[gurubhyas] & \multirow{2}{*}{dhenubhyas} & \multirow{2}{*}{madhubhyas} \\
  從 &                            &                             & \\
  屬 & gurūṇām                    & dhenūnām                    & madhūnām \\
  於 & guruṣu                     & dhenuṣu                     & madhuṣu
\end{tabular}
\end{center}

\numberParagraph
i, u 語基形容詞男女中性これに準ず。例:語基 śuci (淸
き)主,男,女 śucis 中,śuci. 語基 tanu (痩せたる)主,男,女
tanus, 中 tanu. 中性にてはこの語尾以外爲從屬於の單,屬於
の兩には之に相當せる男性の語尾を用ふるを得。

\hfil 6) \fbox{ā, ī, ū 語基} \hfil\,\addcontentsline{toc}{subsection}{6) ā, ī, ū 語基等}

\numberParagraph \label{np:52}
ā, ī, ū 語基は女性である。その中 ī, ū 語基にありては
多綴のものと單綴のものと稍變化を異にす。例:kanyā (少女),
nadī (河),vadhū (婦).
\begin{center}
\begin{tabular}{c*{3}{p{0.2\hsize}}}
  \multicolumn{4}{c}{單} \\
  主 & kanyā                     & nadī                    & vadhū \\
  呼 & kanye                     & nadi                    & vadhu \\
  業 & kanyām                    & nadīm                   & vadhūm \\
  具 & kanyayā                   & nadyā                   & vadhvā \\
  爲 & kanyāyai                  & nadyai                  & vadhvai \\
  從 & \rdelim\}{2}{*}[kanyāyās] & \multirow{2}{*}{nadyās} & \multirow{2}{*}{vadhvās} \\
  屬 &                           &                         & \\
  於 & kanyāyām                  & nadyām                  & vadhvām
\end{tabular}
\end{center}
\begin{center}
\begin{tabular}{c*{3}{p{0.2\hsize}}}
  \multicolumn{4}{c}{兩} \\
  主 & \rdelim\}{3}{*}[kanye]      & \multirow{3}{*}{nadyau}    & \multirow{3}{*}{vadhvau} \\
  呼 &                             &                            & \\
  業 &                             &                            & \\
  具 & \rdelim\}{3}{*}[kanyābhyām] & \multirow{3}{*}{nadībhyām} & \multirow{3}{*}{vadhūbhyām} \\
  爲 &                             &                            & \\
  從 &                             &                            & \\
  屬 & \rdelim\}{2}{*}[kanyayos]   & \multirow{2}{*}{nadyos}    & \multirow{2}{*}{vadhvos} \\
  於 &                             &                            &
\end{tabular}
\end{center}
\begin{center}
\begin{tabular}{c*{3}{p{0.2\hsize}}}
  \multicolumn{4}{c}{複} \\
  主 & \rdelim\}{3}{*}[kanyās]     & \rdelim\}{2}{*}[nadyas]    & \multirow{2}{*}{vadhvas} \\
  呼 &                             &                            & \\
  業 &                             & nadīs                      & vadhūs \\
  具 & kanyābhis                   & nadhībhis                  & vadhūbhis \\
  爲 & \rdelim\}{2}{*}[kanyābhyas] & \multirow{2}{*}{nadibhyas} & \multirow{2}{*}{vadhūbhyas} \\
  從 &                             &                            & \\
  屬 & kanyānām                    & nadīnām                    & vadhūnām \\
  於 & kanyāsu                     & nadīṣu                     & vadhuṣu
\end{tabular}
\end{center}

\numberParagraph
單綴の ī, ū 語基。śrī (幸福), bhū (地).
\begin{center}
\begin{tabular}{c*{2}{p{0.3\hsize}}}
  \multicolumn{3}{c}{單} \\
  主 & \rdelim\}{2}{*}[śrīs]           & bhūs \\
  呼 &                                 & \\
  業 & śriyam                          & bhuvam \\
  具 & śriyā                           & bhuvā \\
  爲 & śriye, śriyai                   & bhuve, bhuvai \\
  從 & \rdelim\}{2}{*}[śriyas, śriyās] & \multirow{2}{*}{bhuvas, bhuvās} \\
  屬 &                                 & \\
  於 & śriyi, śriyām                   & bhuvi, bhuvām
\end{tabular}
\end{center}
\begin{center}
\begin{tabular}{c*{2}{p{0.3\hsize}}}
  \multicolumn{3}{c}{兩} \\
  主 & \rdelim\}{3}{*}[śriyau]   & \multirow{3}{*}{bhūs} \\
  呼 &                           & \\
  業 &                           & \\
  具 & \rdelim\}{3}{*}[śribhyām] & \multirow{3}{*}{bhūbhyām} \\
  爲 &                           & \\
  從 &                           & \\
  屬 & \rdelim\}{2}{*}[śriyos]   & \multirow{2}{*}{bhuvos} \\
  於 &                           &
\end{tabular}
\end{center}
\begin{center}
\begin{tabular}{c*{2}{p{0.3\hsize}}}
  \multicolumn{3}{c}{複} \\
  主 & \rdelim\}{3}{*}[śriyas]   & \multirow{3}{*}{bhuvas} \\
  呼 &                           & \\
  業 &                           & \\
  具 & śrībhis                   & bhūbhyas \\
  爲 & \rdelim\}{2}{*}[śribhyas] & \multirow{2}{*}{bhūbhyas} \\
  從 &                           & \\
  屬 & śriyām, śrīṇām            & bhuvām, bhūnām \\
  於 & śrīṣu                     & bhūṣu
\end{tabular}
\end{center}

\nt{
  strī は不規則にして單主 strī, 呼 stri, 業 striyam 又は strīm,
  爲 striyai, 從屬 striyās, 具 striyām, 複業 strīs 又は striyas, 屬
  strīṇām.}

\numberParagraph
複母音語基,nau (舟),go (牛)。\endnote{底本では両数形について主・呼・業ではなく
  呼・業が \} でまとめられている(主格の欄は空欄)。また具・為をまとめる \} が
  複数形ではなく単数形の列に付いている。}
\begin{center}
\begin{tabular}{c*{6}{p{0.12\hsize}}}
     & \multicolumn{2}{c}{單}                         & \multicolumn{2}{c}{兩}                               & \multicolumn{2}{c}{複} \\
  主 & \rdelim\}{2}{*}[naus]  & \multirow{2}{*}{gaus} & \rdelim\}{3}{*}[nāvau]    & \multirow{3}{*}{gāvau}   & \rdelim\}{3}{*}[nāvas]   & \rdelim\}{2}{*}[gāvas] \\
  呼 &                        &                       &                           &                          &                          & \\
  業 & nāvam                  & gām                   &                           &                          &                          & gās \\
  具 & nāvā                   & gavā                  & \rdelim\}{3}{*}[naubhyām] & \multirow{3}{*}{gobhyām} & \rdelim\}{2}{*}[naubhis] & \multirow{2}{*}{gobhis} \\
  爲 & nāve                   & gave                  &                           &                          &                          & \\
  從 & \rdelim\}{2}{*}[nāvas] & \multirow{2}{*}{gos}  &                           &                          & nāvām                    & gavām \\
  屬 &                        &                       & \rdelim\}{2}{*}[nāvos]    & \multirow{2}{*}{gavos}   & \multirow{2}{*}{nauṣu}   & \multirow{2}{*}{goṣu} \\
  於 & nāvi                   & gavi                  &                           &                          &                          &
\end{tabular}
\end{center}

\numberParagraph \label{np:55}
ṛ 語基 これは一部分 作者名詞であり,一部分は 親族名
詞である。男 dātṛ (施者),女 svasṛ (姉妹)。
\begin{center}
\begin{tabular}{c*{4}{p{0.12\hsize}}}
     & \multicolumn{2}{c}{單}                           & \multicolumn{2}{c}{複} \\
  主 & dātā                   & svasā                   & \rdelim\}{2}{*}[dātāras]   & \multirow{2}{*}{svasāras} \\
  呼 & dātar                  & svasar                  &                            & \\
  業 & dātāram                & svasāram                & dātrīn                     & svasṝs \\
  具 & dātrā                  & svasrā                  & dātṛbhis                   & svasṛbhis \\
  爲 & dātre                  & svasre                  & \rdelim\}{2}{*}[dātṛbhyas] & \multirow{2}{*}{svasṛbhyas} \\
  從 & \rdelim\}{2}{*}[dātur] & \multirow{2}{*}{svasur} &                            & \\
  屬 &                        &                         & dātṝṇām                    & svasṝṇām \\
  於 & dātari                 & svasari                 & dātrṣu                     & svasṛṣu
\end{tabular}
\end{center}

\begin{center}
\begin{tabular}{c*{2}{p{0.24\hsize}}}
     & \multicolumn{2}{c}{兩} \\
  主 & \rdelim\}{3}{*}[dātārau]   & \multirow{3}{*}{svasārau} \\
  呼 &                            & \\
  業 &                            & \\
  具 & \rdelim\}{3}{*}[dātṛbhyām] & \multirow{3}{*}{svasṛbhyām} \\
  爲 &                            & \\
  從 &                            & \\
  屬 & \rdelim\}{2}{*}[dātros]    & \multirow{2}{*}{svasros} \\
  於 &                            &
\end{tabular}
\end{center}
中性の變化は ī, ū の中性に準ず。卽ち單主業 dātṛṇā,
爲 dātṛṇe, 兩主呼業 dātṛṇī, 複主呼業 dātṝṇī.

\numberParagraph
親族名詞〔naptṛ (孫),svasṛ (姉妹)を除く〕は單,業,
兩主業呼,及複主に於て異る。a が短音である。

\begin{tabular}{cl}
  語基 & pitṛ (父) pitaram, pitarau, pitaras. \\
  〃   & mātṛ (母) mātram, mātarau, mātaras. \\
  〃   & duhitṛ (娘) duhitaram, huhitarau, huhitaras.
\end{tabular}

(注意)格例法は次 \ref{np:78} 條に連續する。

\ex{第一}

\begin{longtable}{c*{2}{p{0.45\hsize}}}
   1. & andhasya dīpo vidyā.             & 知識は盲冥の燈火なり。\\
   2. & ācāraḥ pradhāno dharmaḥ.         & 善行は最上の法なり。\\
   3. & kālasya kuṭilā gatiḥ.            & 時の路は曲れり。\\
   4. & śīlaṃ narasya bhūṣaṇam.          & 戒は人の莊嚴なり。\\
   5. & alpaḥ kālo bahavaś ca vigh\-nāḥ. & 短い時と多くの障害。\\
   6. & yathā vṛkṣās tathā phalāni.      & 樹の如くかくの如く果(あり)。\\
   7. & aputrasya gṛhaṃ śūnyam.          & 子なき者の家は空虚なり。\\
   8. & duḥkhaṃ kadāpi sukhasya          & 苦は往々樂の因なり。\\
   9. & yatra yatra dhūmas tatra tatra vahniḥ. & 烟ある所其處に火あり。\\
  10. & pādapānāṃ bhayaṃ vātaḥ padmānāṃ śiśiro bhayam. & 樹の恐は風,蓮花の恐は冷氣。\\
      & parvatānāṃ bhayaṃ vajraḥ sādhūnāṃ durjano bhayam. & 山の恐は雷,善人の恐は悪人なり。\\
  11. & upadeśo hi mūrkhāṇāṃ pra\-kopāya na śāntaye. & 論議は愚者の怒を招く,鎭靜に役立たない。\\
  12.& nagaraṃ devena jitam. & 都城は王によりて打ち勝たれた。\\
  13. & naraḥ sarpeṇa daṣṭo mṛtaś ca. & 人は蛇によりて嚙まれて而して死せり。\\
  14. & siṃho vyādhasya śareṇa hataḥ. & 獅子は獵師の\ruby{箭}{や}によりて殺された。\\
  15. & mūṣikā śyenena gṛhītā bhak\-ṣitā ca. & 鼠は鷹によりて捉えられ而して食はれた。\\
  16. & raṇe nṛpasya senayārayo jitāḥ. & 戰鬪に於て王の軍隊のために敵は征服された。\\
  17. & nirdhanasya kutaḥ sukham. & 貧しきものには何處にか幸福あらむ。\\
  18  & vidyā mitraṃ pravāse ca & 旅行に於て知識は友であり \\
      & bhāryā mitraṃ gṛheṣu ca. & 妻は家に於て友であり,\\
      & vyādhitasyauṣadhaṃ mitram dharmo mitraṃ mṛtasya ca & 病めるものに藥は友であり,而して死者にとりては正義は友である。
\end{longtable}

%%% Local Variables:
%%% mode: latex
%%% TeX-master: "IntroductionToSanskrit"
%%% End:

\section{活用法}
\numberParagraph
梵語の活用法は語根に若干の變化を加へてこれに語尾を
附するにある。卽ち yaj (祭る)+ a = yaja + ti (二人稱單數,現
實法語尾),yajati (彼は祭る)。

梵語にては能動調受動調ある外に能動調を更に爲他,爲自の二
類に分ち,語尾を異にす。されどその意義に至つては爲他も爲自
も同樣である。

三數は格例法の如くである。

\numberParagraph
時法に關しては異れる四類がある。1. 現在組織は現實
法,第一過去,可能法,命令法である(これら四類は共通の語基
を有す)。2. 未來時。3. 第二過去時,4. 第三過去時。この中
2, 3, 4 類は 1 類が特定の語基を有するに反し,語根へ直接に語尾
が附けられる。受動も亦さうである。只第十類動詞(\ref{np:59}, \ref{np:73} 條)の
みは大抵の形が現在語基から作られる。

\numberParagraph \label{np:59}
現在語基から作られる形,現實法,第一過去,可能法,命
令法。語基の作られる樣式によりて語基は二種十類に分たれる。

\fbox{(A) 第一種變化} 現在語基がすべて a にて終る。第一類,
第六類,第四類,第十類がこれに屬す。

第一類。語根に a を附加して語基を作る。その母音は重韻と
なる。bhū (ある)~ bhavati (彼はあり)。

第六類 語根に語勢ある a を附加して語基を作る。語根の母
音は變化せず。tub (打つ)~ tudati (彼は打つ)。

第四類。語根に ya を附加して語基を作る。div (博戯す)~
dīvyati (彼は博戯す)。

第十類。語根に aya を附加して語基を作る。語根の母音は通
常重韻化せらる。cur (盗む)~ corayati (彼は盗む)。

\fbox{(B) 第二種變化} 語根に强弱を分つ。これに屬するものは
第二類,第三類,第五類,第七類,第八類,第九類である。

第二類は語根に直に語尾を附加する。ad (食ふ)~ atti (彼は
食ふ)。

第三類。語根が重複せらる。hu (供ふ)~ juhoti (彼は供ふ),
juhumas (我々は供ふ)。

第七類。强語基に na 弱語基に n を挿んで現在語基を作る。
bhid (破る)~ bhinatti (彼は破る),bhindmas (我々は破る)。

第五類。强語基に no, 弱語基に nu を加へる。su (搾る)~
sunoti (彼は搾る),sunumas (我々は搾る)。

第八類。强語基に o, 弱語基に u を附加する。tan (擴ぐ)~
tanoti (彼は擴ぐ),tanumas (我々は擴ぐ)。

第九類。語根に nā を附加して强語基,子音語尾の前には nī,
母音語尾の前には n を附加して弱語基を作る。aś (食ふ)~
aśnāti (彼は食ふ),aśnīmas (我々は食ふ),aśnate (彼等は食
ふ)。

その他,受動並に派生動詞,卽ち催起動詞,重複動詞,求欲動詞
名稱動詞がある。

\begin{center}現在組織\end{center}
\subsection{第一種變化}
\subsubsection[第一類 a 級 現在]{第一類 a 級}
\numberParagraph \label{np:60}
a 級の構成。語根に a を附加す。語根の母音は重韻化す
る。聲中に在りて本來長きと位置によりて長きは重韻化せず。nī
(導く)現在語基 ne + a = naya (\ref{np:31}條)。nind (嘲る)~ ninda.

\numberParagraph \textbf{現實法の語尾。}

爲他 mi, si, ti; vas, thas, tas; mas, tha, anti.

爲自 e, se, te; vahe, āthe, āte; mahe, dhve, ante.

語基の a は m, v にて始まる語尾の前に延長せられ,anti,
ante の前に省略せられ,āthe, āte の ā と合して e となる。

\numberParagraph
a 級の語根 bhū (ある),現在語基,bho + a = bhava.
\begin{center}
\begin{tabular}{c*{3}{p{0.15\hsize}}}
  \multicolumn{4}{c}{\textbf{現實法}} \\
  \multicolumn{4}{c}{爲他} \\
     & 單       & 兩        & 複 \\
  1. & bhavāmi  & bhavāvas  & bhavāmas \\
  2. & bhavasi  & bhavathas & bhavatha \\
  3. & bhavati  & bhavatas  & bhavanti \\
  \multicolumn{4}{c}{爲自} \\
  1. & bhave   & bhavāvahe & bhavāmahe \\
  2. & bhavase & bhavethe  & bhavadhve \\
  3. & bhavate & bhavete   & bhabante
\end{tabular}
\end{center}

\numberParagraph
若干の不規則なる語根を擧ぐれば
\begin{enumerate}[label=(\alph*)]
\item guh (覆ふ)は語根の母音を重韻化せず。只延長す。
gūhati.
\item kram (歩む)は ya 級によりても變化せられ(\ref{np:69}條),
爲他にありて母音を延長し爲自に於ては延長せず。
\item 或る語根は鼻音を失ふ。

\indent daṃś (咬む)~ daśati.

\indent rañj (染める)~ rajati.

\indent sañj (着く)~ sajati.

\indent svañj (抱く)~ svajati.
\item gam (行く),yam (與ふ),iṣ (欲す),ṛ (達す)は夫々
語基 gaccha, yaccha, iccha, ṛccha に作る。
\item sad (坐す)~ sīdati, sthā (立つ)~ tiṣṭhati, pā (飲む)
~ pibati, ghrā (嗅ぐ)~ jighrati.
\end{enumerate}

\ex{第二}

\begin{longtable}{c*{2}{p{0.45\hsize}}}
 1. & kva gacchasi. & 汝は何處に行くか。\\
 2. & gacchāmi grāmam. & 我は村へ行く。\\
 3. & mitraṃ hvayāmi. & 我は友を喚ぶ。\\
 4. & kuto dhāvathaḥ. & 何故に汝等兩人は走るか。\\
 5. & gajasya bhayād ghāvāvaḥ. & 象の恐怖の故に我等兩人は
走る。\\
 6. & ghaṭas tale patati. & 甕は地上へ落ちたり。\\
 7. & pāpā janāḥ svargaṃ na gacchanti. & 罪ある人々は天國へ行かぬ。\\
 8. & agnis tiṣṭhati gūḍho dāruṣu. & 火は薪の中に隱れてあり。\\
 9. & mitrasya phalaṃ prayacchāmi. & 我は友に果物を與ふ。\\
10. & Devadatto 'nnaṃ pacati. & デーヷダッタは果物を煮る。\\
11. & vane vṛkam īkṣāmahe. & 我々は林の中に狼を見る。\\
12. & ācāryaḥ śiṣyaṃ nindati. & 師は弟子を責む。\\
13. & prāsādasya samīpe hrado bhavati. & 宮殿の側に池がある。\\
14. & nṛpaḥ sainyena Pāṭaliputraṃ praṭiṣṭhati. & 王は軍隊と共にパータリプトラへ出發せり。\\
15. & Godāvaryā jale gajau viha\-rataḥ. & ゴーダーヷリー河の水に於て二つの象が遊ぶ。\\
16. & padmasya pattreṣu jalaṃ na sajati. & 蓮華の葉に於て水は着かぬ。\\
17. & anilasya vaśena vṛkṣāḥ kam\-pante. & 風の力によりて樹々は動
く。\\
18. & puṣpāṇi vasante prasphoṭanti. & 花は春に於いて開く。\\
19. & na niścayād viramanti dhīrāḥ. & 勇者は計劃を中止せぬ。\\
20. & adya brāhmaṇau nagaraṃ tyajataḥ. & 今日二人の婆羅門が市城
を去つた。\\
21. & candrasyāloke kumudaṃ vika\-sati. & 月の出現に於いて蓮華は開
く。\\
22. & lobhāt krodhaḥ prabhavati & 貪慾より憤怒は生ず。\\
    & lobhāt kāmaḥ pravartate, & 貪慾より愛慾は起る。\\
    & lobhān mohaś ca nāśaś ca & 貪慾より愚痴と破滅と
(あり)。\\
    & lobhaḥ pāpasya kāraṇam. & 貪慾は罪惡の原因なり。
\end{longtable}


\subsubsection{第六類 á 級}
\begin{center}第一過去變化\end{center}

\numberParagraph \label{np:64}
第六類では語根の母音は變化しない。附加せらるる a は
語勢を有するのが特徴である。

\numberParagraph 第一過去の語尾。

爲他 am, s, t; va, tam, tām; ma, ta, an.

爲自 i, thās, ta; vahi, āthām, ātām; mahi, dhvam, anta.

\numberParagraph
第一過去はこれらの語尾を附加する外に過去符 a を語根
の前に加ふ。又 m, v にて始まる語尾の前の語基の a は延長せら
れ,語尾 an 並に anta の前に a は省略せられ,āthām, ātām
の ā と合して e に變ず。

過去符 a は前置詞の後語根の前に加ふ。卽ち tyaj (捨)+ pari
(斷念す)三單.第一過去 paryatyajat (\ref{np:13} 條參照)。

\numberParagraph
語根 tud (打つ),現在語基 tuda.
\begin{center}
\begin{tabular}{c*{3}{p{0.15\hsize}}}
  \multicolumn{4}{c}{\textbf{第一過去}} \\
  \multicolumn{4}{c}{爲他} \\
     & 單      & 兩       & 複 \\
  1. & atudam  & atudāva  & atudāma \\
  2. & atudas  & atudatam & atudata \\
  3. & atudat. & atudatām & atudan \\
  \multicolumn{4}{c}{爲自} \\
  1. & atude(a+i) & atudāvahi & atudāmahi \\
  2. & atudathās  & atudethām & atudata \\
  3. & atudata    & atudetām  & atudan
\end{tabular}
\end{center}

\numberParagraph
á 級の不規則なる語根。
\begin{enumerate}[label=(\alph*)]
\item 或る語根は結尾の子音の前に微韻を挿入する。

\indent muc (解く)muñcati.

\indent lip (塗る)limpati.

\indent sic (\ruby{灌}{そそ}ぐ)siñcati.

\indent kṛt (斷つ)kṛntati.

\indent vid (見出す)vindati.

\indent lup (碎く)lumpati.
\item prach (問ふ)はpṛcchati, iṣ (欲す)は icchati とな
る。(第一類の gam 等と比較)。
\end{enumerate}

\ex{第三}

\begin{longtable}{c*{2}{p{0.45\hsize}}}
 1. & vṛkṣasya cchāyāyāṃ munir asīdat. & 樹の蔭に於て聖者は坐したり。\\
 2. & Kālidāsaṃ kaviṃ sevāmahe. & カーリダーサなる詩人を我々は尊敬する。\\
 3. & kanyā Gaṅgāyās tīre 'krī\-ḍan. & 少女等は恒河の岸に於て遊戯した。\\
 4. & gajasya siṃhena saha yud\-dham abhavat. & 獅子と共に象の爭ひがあつた。\\
 5. & tṛṣṇā pathikam abādhata. & 渇が旅人を苦しめた。\\
 6. & putrasya śokād Daśaratho nṛpo jīvitaṃ paryatyajat. & 子を悲しみて十車王は命を捨てた。\\
 7. & śiṣyau gṛhasthasya bhāryāṃ bhikṣāṃ ayācetām. & 二人の弟子は長者の妻に施物を請へり。\\
 8. & Prayāge Gaṅgā Yamunayā saha saṃgacchate. & プラヤーガに於て恒河はヤムナーと出會ふ。\\
 9. & dāsyo 'nnam ānayan. & 婢女等は食物を持ち來つた。\\
10. & saṃkaṭe dhīro dhṛtiṃ na muñcati. & 危難に際し勇者は堅持を捨てない。\\
11. & lajjayā kanyā na pratyabhā\-ṣata. & 羞恥を以て少女は答へなかつた。\\
12. & kīrtiṃ labhante kavayaḥ. & 詩人等は名譽を得る。\\
13. & bubhukṣayā pīḍitaḥ śṛgālo vānān nagaram adhāvat. & 飢に苦しめられ\ruby{野干}{や|かん}は林から町へ走つた。\\
14. & ācāryasya gṛhe Śūdrakeṇa ka\-vinā kṛtāṃ Mṛcchakaṭikām apaṭhāva. & 師の家に於てシュードラカなる詩人によつて作られ
しムリッチュㇵカティカーを我々二人は讀めり。\\
15. & hastena śilām akṣipan naraḥ. & 人は手を以て石を投げた。\\
16. & siṃhaḥ Pāṇineḥ priyān prā\-ṇān aharat. & 獅子はパーニニの愛する生命を奪へり。\\
17. & lubdho naro na visṛjaty & 貪慾の人は貧窮(となる)\\
    & arthaṃ daridratāyāḥ śaṅkayā. & 心配のために富を施さない。\\
18. & bālo vāri pāṇinā kūpād ud\-dharati. & 小兒は手を以て井より水を汲む。\\
19. & lobhena buddhiś calati. & 貪慾によりて思慮は動搖する。\\
20. & makṣikā vraṇam icchanti, & 蠅は瘡傷を欲し,\\
    & dhanam icchanti pārthivāḥ, & 王者は富を欲し,\\
    & nīcāḥ kalaham icchanti, & 下賤なものは爭を欲し,\\
    & śāntim icchanti sādhavaḥ. & 善人は寂靜を欲す。
\end{longtable}

\subsubsection[第四類 ya 級 可能法]{第四類 ya 級}
\begin{center}可能法\end{center}

\numberParagraph \label{np:69}
第四類の動詞は語根に ya を附加して語基を作る。語勢は
語根にあり。語根 sidh (成就す)~ sidhya.

\numberParagraph
可能法の語尾は第一種變化現在語基の結尾の a と合して
次の如くなる。

爲他 eyam, es, et; eva, etam, etām; ema, eta, eyus.

爲自 eya, ethās, eta; evahi, eyāthām, eyātām; emahi,
edhvam, eran.

\numberParagraph
ya 級の變化。語根 div (博戯する)~ dīvya.

\begin{center}
\begin{tabular}{c*{3}{p{0.15\hsize}}}
  \multicolumn{4}{c}{\textbf{可能法}} \\
  \multicolumn{4}{c}{爲他} \\
     & 單       & 兩       & 複 \\
  1. & dīvyeyam & dīvyeva  & dīvyema \\
  2. & dīvyes   & dīvyetam & dīvyeta \\
  3. & dīvyet   & dīvyetām & dīvyeyus \\
  \multicolumn{4}{c}{爲自} \\
  1. & dīvyeya   & dīvyevahi   & dīvyemahi \\
  2. & dīvyethās & dīvyeyāthām & dīvyedhvam \\
  3. & dīvyeta   & dīvyeyātām  & dīvyeran
\end{tabular}
\end{center}

\numberParagraph
若干の不規則なる語根。
\begin{enumerate}[label=(\alph*)]
\item am に終る語根及び mad (醉ふ)はその a を延長す。dam
(馴らす) dāmyati, śam (靜める) śāmyati, kram (歩む)
krāmyati, bhram (彷徨す)には bhrāmyati と bhramyati
の兩形があり。又第一類の bhramati にも變化せらる。
\item jan (生る)は jāyate に作る。
\end{enumerate}

\ex{第四}
\begin{longtable}{c*{2}{p{0.45\hsize}}}
 1. & sevakaḥ prabhuṃ praṇamati. & 僕は主人に\ruby{敬禮}{けい|れい}する。\\
 2. & satyena jayed anṛtam. & 人は正義を以て虚僞に勝つべきである。\\
 3. & sādhavaḥ pratijñāyā na ca\-lanti kadācana. & 善人は何時も約束から動かない。\\
 4. & śatror api guṇān vaded doṣāṃś ca guror api. & 人は敵に對しても功績を,
 而して師に對しても過失を語るべきだ。\\
 5. & parasya duḥkhena sādhur nityaṃ duḥkhito bhavati. & 善人は他の苦によりて常に
 苦しめられてある。\\
 6. & pratyahaṃ pratyavekṣeta na\-raś caritam. & 日日人は行を檢察すべきだ。\\
 7. & asādhuḥ sādhur vā bhavati khalu jātyaiva puruṣaḥ. & 人は生れながらに惡人若く
 は善人である。\\
 8. & astamite sūrye vihagā vilī\-yante, paṅkajā nimīlanti, māla\-tyaś ca vikasanti. & 日沒する時鳥は隱れ,蓮花
は閉ぢ,素馨は咲く。\\
 9. & janakaḥ prītyā putram āśliṣyati. & 父は喜を以て子を抱擁せり。\\
10. & mitrair jñātibhiś ca parihīṇo daridro drutam anaśyat. & 友と親族に捨てられし貧人は速かに亡びた。\\
11. & vasantasya kāle 'layo bhrām\-yanti mukhena ca madhu pibanti. & 春の時に於て蜜蜂は飛び回
り口を以て蜜を吸ふ。\\
12. & sūtasya daṇḍena praṇuditā aśvā na śrāmyanti. & 御者の笞に勵まされて馬は
疲勞しない。\\
13. & grāmasyārthe kulaṃ tyajet. & 人は村邑のために家族を捨てるべきだ。\\
14. & udyamena hi sidhyanti kār\-yāṇi na manorathaiḥ. & 勤勉によりて事は成就す。
希望によりてに非ず。\\
& na hi suptasya siṃhasya pra\-viśanti mukhe mṛgāḥ. & 蓋し眠れる獅子の口に鹿
は入るものでないから。
\end{longtable}

\subsubsection[第十類 aya 級 命令法]{第十類 aya 級}
\begin{center}命令\end{center}

\numberParagraph \label{np:73}
第十類の構成は語根に aya を附加す。語勢は前の a に
在り。聲中にある i, u, ṛ は單子音の前には重韻化し,聲の終に
在る。i, u, ṛ 又は單子音の中間にある a は複重韻化し,多子音
の前にあると長母音の場合は變化しない。cur (盗む)~ coraya,
cint (考ふ)~ cintaya. この第十類は實際 催起動詞,名稱動詞と
同類である。只現在語基が大體に於て變化しないで用ひられるの
を異とする。

\numberParagraph
命令法の語尾。

爲他 āni, dhi, tu; āva, tam, tām; āma, ta, antu.

爲自 ai, sva, tām: āvahai, āthām, ātām; āmahai,
dhvam, antām.

\numberParagraph
二人稱の dhi は第一種變化に於ては通じて附加しない。
現在語基の結尾 a は antu, antām の前に消滅し。āthām, ātām
の ā と合して e に變ず。

\numberParagraph
aya 級。命令法の語尾。

\begin{center}
\begin{tabular}{c*{3}{p{0.15\hsize}}}
  \multicolumn{4}{c}{爲他} \\
     & 單       & 兩        & 複 \\
  1. & corayāṇi & corayāva  & corayāma \\
  2. & coraya   & corayatam & corayata \\
  3. & corayatu & corayatām & corayantu \\
  \multicolumn{4}{c}{爲自} \\
  1. & corayai   & corayāvahai & corayāmahai \\
  2. & corayasva & corayethām  & corayadhvam \\
  3. & corayatām & corayetām   & corayantām
\end{tabular}
\end{center}

\ex{第五}
\begin{longtable}{c*{2}{p{0.45\hsize}}}
 1. & adhunā muñca śayyām. & 今臥床を離れよ。\\
 2. & vaidyo rogārtasyauṣadhaṃ yacchatu. & 醫師をして病に苦しむもの
 に藥を與へしめよ。\\
 3. & adya pitur gṛham āgaccha\-tam. & 今汝等兩人は父の家に歸れかし。\\
 4. & bhadre, maivaṃ vada. & 善き婦人よ是の如く語る勿れ。\\
 5. & Pāṇiner vyākaraṇasya karur buddhiṃ śaṃsāmaḥ. & 我々は文法作家パーニニの
 聰明を讚む。\\
 6. & bhāryā patyuḥ snihyatu. & 妻をして夫を愛せしめよ。\\
 7. & tyaja nīcānāṃ saṃsargam. & 下劣なるものとの交際を避けよ。\\
 8. & kṣamāṃ bhaja. & 堪忍せよ。\\
 9. & dharme ratir bhavatu. & 法に於て快樂があれかし。\\
10. & āpṛcchasva priyaṃ sakhāyam. & 愛する共に吿別せよ。\\
11. & durjaneṣv api mā pāpaṃ cin\-tayasva kadācana. & 何時にても惡人に對してす
らも惡を考ふること勿れ。\\
12. & bhūpatayaḥ sarvadā prajā dharmeṇa rakṣantu. & 常に王をして人民を法を以て守らしめよ。\\
13. & mitrasya dhanaṃ prayaccha. & 共に財物を與へよ。\\
14. & kumbhakāro daṇḍena ghaṭam akhaṇḍayat. & 甕作りは杖にて甕を粉碎せり。\\
15. & bho sakhe kṣaṇam atra tiṣṭha. & おゝ友よ暫しそこに立て。\\
16. & śokasya hetuṃ vada. & 悲哀の原因を語れ。\\
17. & śiṣyā guroḥ pādau pūjayanti. & 弟子等は師の兩足を尊敬せり。\\
18. & stenā rātrau gṛhaṃ pravi\-śanti janānāṃ ca dhanaṃ cora\-yanti. & 盗人は夜に於て家に入り,
而して人々の富を盗む。\\
19. & kauliko nṛpasya duhitaraṃ paryaṇayat. & 織師は王女に婚せり。\\
20. & muktim icchasi ced viṣam iva viṣayāṃs tyaja. & 若しも汝が解脱を欲するな
らば毒の如くに欲境を捨てよ。
\end{longtable}

\numberParagraph \addcontentsline{toc}{subsubsection}{現在組織總覽}
以上第一種變化の動詞に現在組織の語尾を附加せるもの
を擧げた。勿論この四種の法は各類に應用せらるべきもの。故に
今試みに次に bhū (ある)に就きて全部の語尾表を擧げる。


\begin{center}
\begin{tabular}{c*{3}{p{0.15\hsize}}}
  \multicolumn{4}{c}{\textbf{現實法}} \\
  \multicolumn{4}{c}{爲他} \\
     & 單       & 兩       & 複 \\
  1. & bhavāmi  & bhavāvas  & bhavāmas \\
  2. & bhavasi  & bhavathas & bhavatha \\
  3. & bhavatai & bhavantas & bhavanti \\
  \multicolumn{4}{c}{爲自} \\
  1. & bhave   & bhavāvahe & bhavāmahe \\
  2. & bhavase & bhavethe  & bhavadhve \\
  3. & bhavate & bhavete   & bhavante
\end{tabular}
\end{center}
\begin{center}
\begin{tabular}{c*{3}{p{0.15\hsize}}}
  \multicolumn{4}{c}{\textbf{第一過去}} \\
  \multicolumn{4}{c}{爲他} \\
     & 單       & 兩       & 複 \\
  1. & abhavam & abhavāva  & abavāma \\
  2. & abhavas & abhavatam & abhavata \\
  3. & abhavat & abhavatām & abhavan \\
  \multicolumn{4}{c}{爲自} \\
  1. & abhave     & abhavāvahi & abhavāmahi \\
  2. & abhavathās & abhavethām & abhadhvam \\
  3. & abhavata   & abhavetām  & abhavanta
\end{tabular}
\end{center}
\begin{center}
\begin{tabular}{c*{3}{p{0.15\hsize}}}
  \multicolumn{4}{c}{\textbf{可能法}} \\
  \multicolumn{4}{c}{爲他} \\
     & 單       & 兩       & 複 \\
  1. & bhaveyam & bhaveva  & bhavema \\
  2. & bhaves   & bhavetam & bhaveta \\
  3. & bhavet   & bhavetām & bhaveyus \\
  \multicolumn{4}{c}{爲自} \\
  1. & bhaveya   & bhavevahi   & bhavemahi \\
  2. & bhavethās & bhaveyāthām & bhavedhvam \\
  3. & bhaveta   & bhaveyātām  & bhaveran
\end{tabular}
\end{center}
\begin{center}
\begin{tabular}{c*{3}{p{0.15\hsize}}}
  \multicolumn{4}{c}{\textbf{命令法}} \\
  \multicolumn{4}{c}{爲他} \\
     & 單      & 兩       & 複 \\
  1. & bhavāni & bhavāva  & bhavāma \\
  2. & bhava   & bhavatam & bhavata \\
  3. & bhavatu & bhavatām & bhavantu \\
  \multicolumn{4}{c}{爲自} \\
  1. & bhavai   & bhavāvahai & bhavāmahai \\
  2. & bhavasva & bhavethām  & bhavadhvam \\
  3. & bhavatām & bhavetām   & bhavantām
\end{tabular}
\end{center}

(活用法は次 \ref{np:136} 條に連續す)。

%%% Local Variables:
%%% mode: latex
%%% TeX-master: "IntroductionToSanskrit"
%%% End:

\section{格例法(續き)}
\subsection{子音語基}
\numberParagraph \label{np:78}
子音に終る語基は次の如きものである。

\begin{itemize}[label=\hspace{2\zw}]
\item 語根語基
\item as, is, us に終るもの
\item in に終るもの
\item ac に終るもの
\item an, man, van に終るもの
\item at に終るもの
\item mat, vat に終るもの
\item vas に終る過去分詞
\item yas に終る比較級
\end{itemize}

\numberParagraph
語尾は各語基を通じて規則正しく次のものが用ひられる。
この點母音語基の如く不規則ではない。男性,女性は次の如くで
ある。

\begin{center}
\begin{tabular}{c*{3}{p{0.24\hsize}}}
     & 單                  & 兩                     & 複 \\
  主 & s                   & \rdelim\}{2}{*}[au]    & \multirow{2}{*}{as} \\
  業 & am                  &                        & \\
  具 & ā                   & \rdelim\}{3}{*}[bhyām] & bhis \\
  爲 & e                   &                        & \rdelim\}{2}{*}[bhyas] \\
  從 & \rdelim\}{2}{*}[as] &                        & \\
  屬 &                     & \rdelim\}{2}{*}[os]    & ām \\
  於 & i                   &                        & su
\end{tabular}
\end{center}

呼格は兩複に於て必ず主格に同じ。單に於ては大抵語基の形で
あるが,或は主格の形であることもある。中性は主業格が單に於
て語尾を有せず。兩に於て ī, 複に於て i となる外,男女性と同じ
である。

\subsubsection{語根語基}

\numberParagraph
語根語基及それに準ずるもの。男性,語基 marut (風),
女性 vāc (言語)。

\begin{center}
\begin{tabular}{c*{4}{p{0.15\hsize}}}
     & \multicolumn{2}{c}{單}                           & \multicolumn{2}{c}{複} \\
  主 & \rdelim\}{2}{*}[marut]   & \multirow{2}{*}{vāk (\ref{np:17}條)} & \rdelim\}{3}{*}[marutas]    & \multirow{3}{*}{vācas} \\
  呼 &                          &                                        &                             & \\
  業 & marutam                  & vācam                                  &                             & \\
  具 & marutā                   & vācā                                   & marudbhis (\ref{np:18}條) & vāgbhis \\
  爲 & marute                   & vāce                                   & marudbhyas                  & vāgbhyas \\
  從 & \rdelim\}{2}{*}[marutas] & \multirow{2}{*}{vācas}                 & \rdelim\}{2}{*}[marutām]    & \multirow{2}{*}{vācām}\\
  屬 &                          &                                        &                             & \\
  於 & maruti                   & vāci                                   & marutsu                     & vākṣu \\
     &                          &                                        &                             & (\ref{np:17} 條,\ref{np:39}條)
\end{tabular}
\end{center}

\begin{center}
\begin{tabular}{c*{2}{p{0.24\hsize}}}
     & \multicolumn{2}{c}{兩} \\
  主 & \rdelim\}{3}{*}[marutau]    & \multirow{3}{*}{vācau} \\
  呼 &                             & \\
  業 &                             & \\
  具 & \rdelim\}{3}{*}[marudbhyām] & \multirow{3}{*}{vāgbhyām} \\
  爲 &                             & \\
  從 &                             & \\
  屬 & \rdelim\}{2}{*}[marutos]    & \multirow{2}{*}{vācos} \\
  於 &                             &
\end{tabular}
\end{center}
中性語の主業複は最後の音の次前にその音に相當する鼻音を挿
む。hṛd (心)は hṛndi, asṛj (血)は asṛñji, jagat (世界), jaganti.
これは語基の强弱あるものと見做しても不可でない。

\numberParagraph
\textbf{この種類に屬する若干の語基。}

suhṛd 男(友)。

主 suhṛt, 業 suhṛdam, 具複 suhṛdbhis

dharmabudh 男,女,中(法に明かなる)。

主 dharmabhut, 業 dharmabudham, 具複 dharmabhudbhi (\ref{np:35}條)

vaṇij 男(商人)。

主 vaṇik, 業 vaṇijam, 具複 vaṇigbhis.

diś 女(方)。

主 dik, 業 diśam, 具複 digbhis.

tviṣ 女(光)。

主 tviṭ (\ref{np:17}條),業 tviṣam, 具複 tviḍbhis, 於 tviṭsu 又は
tviṭtsu.

ir, ur に終る語基は單の主呼と並に子音が次に來る場合 i, u を
延長す。

語基 gir 女(歌),主 gīr, 業 giram, 具複 gīrbhis, 於 gīrṣu.

pur 女(城)。主 pūr, 業 puram, 具複 pūrbhis, 於 pūrṣu.

ap 女(水)は常に複數にして,主 āpas, 業 apas, 具 abdhis,
爲從 adbhyas, 屬 apām, 於 apsu である。

puṃs 男(人)は全く不規則で,單 pumān puman, pumāṃsam,
puṃsā puṃse, puṃsas, puṃsi, 兩 pumāṃsau, pumbhyās,
puṃsos, 複 pumāṃsas, puṃsas, pumbhis, pumbhyas, puṃ\-%
sām, puṃsu.

\subsubsection{as, is, us に終る語基}

\numberParagraph
男女性にあつては主單に as の a を延長する。中性では
主呼業の複に a, i, u を延長し,且つ隨韻が挿入せられる。

語基 sumanas 男,女,中(善意ある),cakṣus 中(眼)。

\begin{center}
\begin{tabular}{c*{3}{p{0.2\hsize}}}
  \multicolumn{4}{c}{單} \\
     & 男女                       & \multicolumn{2}{c}{中} \\
  主 & sumanās                    & \rdelim\}{3}{*}[sumanas]   & \multirow{3}{*}{cakṣus} \\
  呼 & sumanas                    &                            & \\
  業 & sumanasam                  &                            & \\
  具 & sumanasā                   & sumanasā                   & cakṣuṣā \\
  爲 & sumanase                   & sumanase                   & cakṣuṣe \\
  從 & \rdelim\}{2}{*}[sumanasas] & \multirow{2}{*}{sumanasas} & \multirow{2}{*}{cakṣuṣas} \\
  屬 &                            &                            & \\
  於 & sumanasi                   & sumanasi                   & cakṣuṣi
\end{tabular}
\end{center}
\begin{center}
\begin{tabular}{c*{3}{p{0.2\hsize}}}
  \multicolumn{4}{c}{兩} \\
     & 男女                                                                                            & \multicolumn{2}{c}{中} \\
  主 & \rdelim\}{3}{*}[sumanasau]                                                                      & \multirow{3}{*}{sumanasī}    & \multirow{3}{*}{cakṣuṣī} \\
  呼 &                                                                                                 &                              & \\
  業 &                                                                                                 &                              & \\
  具 & \rdelim\}{3}{*}[\parbox{8cm-\tabcolsep-\widthof{$\Bigg\}$}}{sumananobhyām \\(\ref{np:24}條)}] & \multirow{3}{*}{sumanobhyām} & \multirow{3}{*}{\parbox{8cm-\tabcolsep-\widthof{$\Bigg\}$}}{cakṣurbhyām \\(\ref{np:23}條)}} \\
  爲 &                                                                                                 &                              & \\
  從 &                                                                                                 &                              & \\
  屬 & \rdelim\}{2}{*}[sumananasos]                                                                    & \multirow{2}{*}{sumanasos}   & \multirow{2}{*}{cakṣuṣos} \\
  於 &                                                                                                 &                              &
\end{tabular}
\end{center}
\begin{center}
\begin{tabular}{c*{3}{p{0.2\hsize}}}
  \multicolumn{4}{c}{複} \\
     & 男女                         & \multicolumn{2}{c}{中} \\
  主 & \rdelim\}{3}{*}[sumanasas]   & \multirow{3}{*}{sumanāṃsi}   & \multirow{3}{*}{cakṣūṃṣi} \\
  呼 &                              &                              & \\
  業 &                              &                              & \\
  具 & sumanobhis                   & sumanobhis                   & cakṣurbhis \\
  爲 & \rdelim\}{2}{*}[sumanobhyas] & \multirow{2}{*}{sumanobhyas} & \multirow{2}{*}{cakṣurbhyas} \\
  從 &                              &                              & \\
  屬 & sumanasām                    & sumanasām                    & cakṣuṣām \\
  於 & sumanaḥsu (\ref{np:22}條)  & sumanaḥsu                    & cakṣuḥṣu (\ref{np:39}條)
\end{tabular}
\end{center}

\begin{center}\textbf{(3) in に終る語基}\end{center}

\numberParagraph
in に終る語基は所有を表はすもので,多くは形容詞であり,
男性中性に變化する。子音で始まる語尾の前に n は消失する。
又同樣に單主,及び中單主業も然り。中單呼にあつては消失せざ
ることもあり,この i は男單主,並に中複主業にあつては延長せ
らる。

語基 balin (力ある)。

\begin{center}
\begin{tabular}{c*{6}{p{0.12\hsize}}}
     & \multicolumn{2}{c}{單}     & \multicolumn{2}{c}{兩}     & \multicolumn{2}{c}{複} \\
     & 男      & 中                                 & 男                       & 中                      & 男                       & 中 \\
  主 & balī    & bali                               & \rdelim\}{3}{*}[balinau] & \multirow{3}{*}{balinī} & \multirow{3}{*}{balinas} & \multirow{3}{*}{balīni} \\
  呼 & balin   & balin, bali                        &                                                    & \\
  業 & balinam & bali                               &                                                    & \\
     & \multicolumn{2}{c}{\upbracefill}             &                          &                         & \multicolumn{2}{c}{\upbracefill} \\
  具 & \multicolumn{2}{c}{balinā}                   & \multicolumn{2}{l}{\rdelim\}{3}{*}[balibhyām]}     & \multicolumn{2}{c}{balibhis} \\
  爲 & \multicolumn{2}{c}{baline}                   &                          &                         & \multicolumn{2}{l}{\rdelim\}{2}{*}[balibhyas]} \\
  從 & \multicolumn{2}{l}{\rdelim\}{2}{*}[balinas]} &                          &                         & \\
  屬 &         &                                    & \multicolumn{2}{l}{\rdelim\}{2}{*}[balinos]}       & \multicolumn{2}{c}{balinām} \\
  於 & \multicolumn{2}{c}{balini}                   &                          &                         & \multicolumn{2}{c}{baliṣu}
\end{tabular}
\end{center}

この語基の女性形容詞は語基に ī を附加して作り nadī (\ref{np:52}條)
に準じて變化せらる。

\ex{第六}
\begin{longtable}{c*{2}{p{0.45\hsize}}}
 1. & sarvaḥ padasthasya suhṛd bandhur āpadi durlabhaḥ. & 顯要の位置にあるものは一
切が友であり不幸に於ては親族が得難い。\\
 2. & yathaā cittaṃ tathā vāco yathā vācas tathā kriyāḥ. & 心の如く此の如く語あり,
語の如く此の如く行爲あり。\\
 3. & durgrāhyḥ pāṇinā vāyur duḥsparśaḥ pāṇinā śikhī. & 風は手を以て捉へ難く,火は手を以て觸れ難い。\\
 4. & kṣamā rūpaṃ tapasvinaḥ. & 忍耐は苦行者の美貌である。\\
 5. & niyato dehināṃ mṛtyur ani\-tyaṃ khalu jīvitam. & 人にとつて死は定まつてゐるが命は不定である。\\
 6. & namanti phalino vṛkṣāḥ na\-manti guṇino janāḥ. & 果實ある樹は曲り德ある人も曲る。\\
 7. & śrutiḥ smṛtiś ca dvijānāṃ cakṣuṣī. & 吠陀と法典(天啓と傳說)とは婆羅門の兩眼である。\\
 8. & vaidyo na prabhur āyuṣaḥ. & 醫は壽命の主でない。\\
 9. & snānāya sarasas tīraṃ sa gacchati. & 沐浴のために湖の岸へ彼は行く。\\
10. & āyur eva paraṃ nidhānam. &  壽命こそ最高の寶なれ。\\
11. & saṃpadas tasya yasya saṃ\-tuṣṭaṃ mānasam. & 滿足せる心ある人には幸福がある。\\
12. & āpadas tasya yasya vittaṃ na vidyate. & 財物なき人には不幸がある。\\
13. & ākiṃcanyaṃ nidhānaṃ vi\-duṣām. & 無一物の狀態は學者の寶である。\\
14. & bhāryāyāḥ sundaraḥ snigdho veśyāyāḥ sundaro dhanī, Śrī\-devyāḥ sundaraḥ śūro Bhāra\-tyāḥ sundaraḥ sundhī. & 妻の好むは情ある人,娼婦の好むは富人,吉祥天の
好むは勇士,辯才天の好むは賢人である。\endnote{底本では「好むは賢人」ではなく「好むはは賢人」。}
\end{longtable}

\numberParagraph
語基には强弱の二語基の場合或は强弱中の三語基の場合
がある。强弱と云ふのは語勢等の關係で音量の多いのが强と名け
られ,少いのが弱と名けられる。中はその中間のものである。而
してこの强弱中の語基は使用せらるゝ場所も一定してゐて決して
混亂はない。卽ち男性,女性にして二語基の場合は主呼業の單兩
及び主呼の複に强語基その他には弱語基を用ふ。三語基の場合は
前の弱語基を用ふる場所の中,大體子音にて始まる語尾の前には
中語基母音にて始まる語尾の前には弱語基が用ひられる。中性で
は複の主業が强語基となる。三語基の場合には單の主業呼が中語
基となる。兩の主業は常に弱語基である。其の餘は男女性に同
じ。

\begin{enumerate}[label=(\alph*)]
\item 二語基の例:强 tudant, 弱 tudat (打ちつゝ)。
\item 三語基の例:强 vidvāṃs, 中 vidvat, 弱 vidus (賢者)。
\end{enumerate}

\subsubsection{ac に終る語基}

\numberParagraph
ac に終る語基は形容詞であつて一部分二語基,一部分は
三語基を有す。卽ち二語基 prāc (東方の)强基 prāñc, 弱基
prāc であり,三語基は pratyac (西方の)强基 pratyañc, 中基
pratyac, 弱基 pratīc.

\begin{center}
\begin{tabular}{c*{4}{p{0.15\hsize}}}
     & \multicolumn{4}{c}{單} \\
     & \cellAlign{c}{男}                       & \cellAlign{c}{中}     & \cellAlign{c}{男}           & \cellAlign{c}{中} \\
  主 & \rdelim\}{2}{*}[prāṅ (\ref{np:17}條)] & \multirow{2}{*}{prāk} & \multirow{2}{*}{pratyaṅ}    & \multirow{2}{*}{pratyak} \\
  呼 &                                         &                       &                             & \\
  業 & prāñcam                                 & prāk                  & pratyañcam                  & pratyak \\
     & \multicolumn{2}{c}{\upbracefill}                                & \multicolumn{2}{c}{\upbracefill} \\
  具 & \multicolumn{2}{c}{prācā}                                       & \multicolumn{2}{c}{pratīcā} \\
  爲 & \multicolumn{2}{c}{prāce}                                       & \multicolumn{2}{c}{pratīce} \\
  從 & \multicolumn{2}{l}{\rdelim\}{2}{*}[prācas]}                     & \multicolumn{2}{c}{\multirow{2}{*}{pratīcas}} \\
  屬 &                                                                 & \\
  於 & \multicolumn{2}{c}{prāci}                                       & \multicolumn{2}{c}{pratīci}
\end{tabular}
\end{center}

\begin{center}
\begin{tabular}{c*{4}{p{0.15\hsize}}}
     & \multicolumn{4}{c}{兩} \\
     & \cellAlign{c}{男}        & \cellAlign{c}{中}      & \cellAlign{c}{男}           & \cellAlign{c}{中} \\
  主 & \rdelim\}{3}{*}[prāñcau] & \multirow{3}{*}{prācī} & \multirow{3}{*}{pratyañcau} & \multirow{3}{*}{pratīcī} \\
  呼 &                          &                        &                             & \\
  業 &                          &                        &                             & \\
     & \multicolumn{2}{c}{\upbracefill}                  & \multicolumn{2}{c}{\upbracefill} \\
  具 & \multicolumn{2}{l}{\rdelim\}{3}{*}[prāgbhyām]}    & \multicolumn{2}{c}{\multirow{3}{*}{pratyagbhyām}} \\
  爲 &                                                   & \\
  從 &                                                   & \\
  屬 & \multicolumn{2}{l}{\rdelim\}{2}{*}{prācos}}       & \multicolumn{2}{c}{\multirow{2}{*}{pratīcos}} \\
  於 &                                                   &
\end{tabular}
\end{center}

\begin{center}
\begin{tabular}{c*{4}{p{0.15\hsize}}}
     & \multicolumn{4}{c}{複} \\
     & \cellAlign{c}{男}        & \cellAlign{c}{中}       & \cellAlign{c}{男}           & \cellAlign{c}{中} \\
  主 & \rdelim\}{2}{*}[prāñcas] & \multirow{2}{*}{prāñci} & \multirow{2}{*}{pratyañcas} & \multirow{2}{*}{pratyañci} \\
  呼 &                          &                         &                             & \\
  業 & prācas                   & prāñci                  & pratīcas                    & pratyañci \\
     & \multicolumn{2}{c}{\upbracefill}                   & \multicolumn{2}{c}{\upbracefill} \\
  具 & \multicolumn{2}{c}{prāgbhis}                       & \multicolumn{2}{c}{pratyagbhis} \\
  爲 & \multicolumn{2}{l}{\rdelim\}{2}{*}[prāgbhyas]}     & \multicolumn{2}{c}{pratyagbhyas} \\
  從 &                                                    & \\
  屬 & \multicolumn{2}{c}{prācām}                         & \multicolumn{2}{c}{pratīcām} \\
  於 & \multicolumn{2}{c}{prākṣu}                         & \multicolumn{2}{c}{pratyakṣu}
\end{tabular}
\end{center}

女性語基は弱語基に ī を附加して作られる。卽ち prācī,
pratīcī であつて nadī (\ref{np:52}條)に準じて變化せられる。

\subsubsection{an, man, van に終る語基}

\numberParagraph \label{np:86}
これは皆三語基を有す。强語基では後接字の a は延長せ
られ,中語基では n が省かれ,弱語基では a が除去せらる。
man, van の語基では弱語基にても m 又は v の前に子音が來る
場合 a は保存せらる。單主,男は ā, 中は a にて終る。

語基 rājan 男(王),强語基 rājān, 中語基 rāja, 弱語基 rājñ.
adhvan 男(路),强語基 adhvān, 中語基 adhva, 弱語基 adhvan.
語基 karman 中(業)同樣。

\begin{center}
\begin{tabular}{c*{3}{p{0.15\hsize}}}
     & \multicolumn{3}{c}{單} \\
  主 & rājā                    & adhvā                     & karma \\
  呼 & rājan                   & adhvan                    & karman, karma \\
  業 & rājānam                 & adhvānam                  & karma \\
  具 & rājñā                   & adhvanā                   & karmaṇā \\
  爲 & rājñe                   & adhvane                   & karmaṇe \\
  從 & \rdelim\}{2}{*}[rājñas] & \multirow{2}{*}{adhvanas} & \multirow{2}{*}{karmaṇas} \\
  屬 &                         &                           & \\
  於 & rājñi                   & adhvani                   & karmaṇi
\end{tabular}
\end{center}

\begin{center}
\begin{tabular}{c*{3}{p{0.15\hsize}}}
     & \multicolumn{3}{c}{兩} \\
  主 & \rdelim\}{3}{*}[rājānau]   & \multirow{3}{*}{adhvānau}   & \multirow{3}{*}{karmaṇī} \\
  呼 &                            &                             & \\
  業 &                            &                             & \\
  具 & \rdelim\}{3}{*}[rājabhyām] & \multirow{3}{*}{adhvabhyām} & karmabhyām \\
  爲 &                            &                             & \\
  從 &                            &                             & \\
  屬 & \rdelim\}{2}{*}[rājños]    & \multirow{2}{*}{adhvanos}   & \multirow{2}{*}{karmaṇos} \\
  於 &                            &                             &
\end{tabular}
\end{center}

\begin{center}
\begin{tabular}{c*{3}{p{0.15\hsize}}}
     & \multicolumn{3}{c}{複} \\
  主 & \rdelim\}{2}{*}[rājānas]   & \multirow{2}{*}{adhvānas}   & \rdelim\}{3}{*}[karmāṇi] \\
  呼 &                            &                             & \\
  業 & rājñas                     & adhvanas                    & \\
  具 & rājabhis                   & adhvabhis                   & karmabhis \\
  爲 & \rdelim\}{2}{*}[rājabhyas] & \multirow{2}{*}{adhvabhyas} & \multirow{2}{*}{karmabhyas} \\
  從 &                            &                             & \\
  屬 & rājñām                     & adhvanām                    & karmaṇām \\
  於 & rājasu                     & adhvasu                     & karmasu
\end{tabular}
\end{center}

\nt{三性を通じ單於,及び中性兩の主業には a を保存するを得。卽
ち rājñi と並んで rājani, nāmnī 中(名)と並んで nāmanī.}

\numberParagraph \label{np:87} \textbf{若干の不規則なる語基。}
\begin{enumerate}[label=(\alph*)]
\item śvan 男(犬)及び yuvan (若き)は śun yūn なる弱
  語基を有す。
  \begin{description}[font=\normalfont]
  \item[單] śvā, śvānam, śunā, śune 等。
  \item[兩] śvānau, śvabhyām, śunos.
  \item[複] śvānas, śunas, śvabhis, śvabhyas, śunām, śvasu.
  \end{description}
\item panthan 男(路)は强語基 panthān, 中語基 pathi,
  弱語基 path.
  \begin{description}[font=\normalfont]
  \item[單] panthās, panthānam, pathā, pathe, pathas, pathi.
  \item[兩] panthānau, pathibhyām, pathos.
  \item[複] panthānas, pathas, pathibhis, pathibhyas, pathām, pathiṣu.
  \end{description}
\item athan 中(日)は中語基 ahar 若くは ahas.
  \begin{description}[font=\normalfont]
  \item[單] 主呼業 ahar, 具 ahnā 等。
  \item[兩] ahni, ahobhyām, ahnos.
  \item[複] ahāni, 具 ahobhis 等。
  \end{description}
\item 中性 akṣan (眼),asthan (骨)は弱語基の格のみを有
  す。akṣṇā, akṣṇe, akṣṇas, akṣṇi 等,其の餘の格は akṣi, asthi
  の如く i 語基に準じて變化す。\ref{np:49}條\ref{item:49c}參照。
\end{enumerate}

\subsubsection{at に終る語基}

\numberParagraph
これらの語基は殆んどすべて現在又は未來分詞である。
而して强語基は ant 弱語基は at である。

語基 tudat (打ちつゝ),强 tudant, 弱 tudat.
\begin{center}
\begin{tabular}{c*{4}{p{0.15\hsize}}}
     & \multicolumn{2}{c}{單}                            & \multicolumn{2}{c}{複} \\
     & \cellAlign{c}{男}        & \cellAlign{c}{中}      & \cellAlign{c}{男}             & \cellAlign{c}{中} \\
  主 & \rdelim\}{2}{*}[tudan]   & \rdelim\}{3}{*}[tudat] & \multirow{2}{*}{tudantas}     & \multirow{3}{*}{tudanti} \\
  呼 &                          &                        &                               & \\
  業 & tudantam                 &                        & tudatas                       & \\
     & \multicolumn{2}{c}{\upbracefill}                  & \multicolumn{2}{c}{\upbracefill} \\
  具 & \multicolumn{2}{c}{tudatā}                        & \multicolumn{2}{c}{tudadbhis} \\
  爲 & \multicolumn{2}{c}{tudate}                        & \multicolumn{2}{c}{tudadbhyas} \\
  從 & \multicolumn{2}{c}{\rdelim\}{2}{*}[tudatas]}      & \multicolumn{2}{c}{\multirow{2}{*}{tudatām}} \\
  屬 &                                                   & \\
  於 & \multicolumn{2}{c}{tudati}                        & \multicolumn{2}{c}{tudatsu}
\end{tabular}
\end{center}

\begin{center}
\begin{tabular}{c*{2}{p{0.24\hsize}}}
     & \multicolumn{2}{c}{兩} \\
     & \cellAlign{c}{男}           & \cellAlign{c}{中} \\
  主 & \rdelim\}{3}{*}[tudantau]   & \multirow{3}{*}{tudantī, tudatī} \\
  呼 &                             & \\
  業 &                             & \\
     & \multicolumn{2}{c}{\upbracefill} \\
  具 & \multicolumn{2}{c}{\rdelim\}{3}{*}[tudadbhyām]} \\
  爲 &                             & \\
  從 &                             & \\
  屬 & \multicolumn{2}{c}{\rdelim\}{2}{*}[tudatos]} \\
  於 &                             &
\end{tabular}
\end{center}

\numberParagraph \label{np:89}
これらの語基の女性は\ref{np:90}條に準じ弱基又は强基に ī を附
加して作る。例:tudatī 又は tudantī.

\numberParagraph \label{np:90}
中性の兩主業に於て並に女性の構成に於て强基弱基の孰
れを取るかに關しては次の規定がある。
\begin{enumerate}[label=(\alph*)]
\item a 級 ya 級 aya 級並びに派生動詞に關しては强形(ant)
  が作られねばならぬ。卽ち bhavat, 女性 bhavantī.
\item á 級,語根級の ā に終る語根例へば yā, 並に爲他未
  來分詞に關しては强弱兩形が作られる。卽ち tudat, 女性
  tudatī 又は tudantī.
\item 餘の級の語根からは必ず弱形(at)が作られねばなら
  ぬ。卽ち kurvat, 女性 kurvatī.
\end{enumerate}

\numberParagraph
重複語根(\ref{np:145}條)は弱語基で總ての格を構成する。卽ち
datat (與へつゝ)男主 dadat, 業 dadatām 等。只中性複の主
業には兩形並び行はれる。

\numberParagraph
mahat (大なる)は强基 ānt である。
\begin{description}[font=\normalfont]
\item[男,單] mahān, mahāntam, mahatā 等。
\item[兩] mahāntau.
\item[複] mahāntas, mahatas, mahadbhis 等。
\item[中] mahat, mahatī, mahānti.
\end{description}

\subsubsection{mat 及び vat に終る語基}

\numberParagraph
mat, vat に終る所有を表はす形容詞は at に終る分詞と
同樣に變化する。但し主單男は mān 及び vān となる。卽ち
balavat (强き),主 balavān.

\subsubsection{vas に終る過去能動分詞}

\numberParagraph \label{np:94}
過去能動分詞(\ref{np:177}條)は三語基を有す。强基は vāṃs,
中基 vat, 弱基は us である。主單は vān, 呼單は van.

語基 vidvas (知れる),强形 vidvāṃs, 中形 vidvat, 弱形
vidus.
\begin{center}
\begin{tabular}{c*{4}{p{0.15\hsize}}}
     & \multicolumn{2}{c}{單}                       & \multicolumn{2}{c}{複} \\
     & \cellAlign{c}{男} & \cellAlign{c}{中}        & \cellAlign{c}{男}          & \cellAlign{c}{中} \\
  主 & vidvān            & \rdelim\}{3}{*}[vidvat]  & \rdelim\}{2}{*}[vidvāṃsas] & \rdelim\}{3}{*}[vidvāṃsi] \\
  呼 & vidvan            &                          &                            & \\
  業 & vidvāṃsam         &                          & viduṣas                    & \\
     & \multicolumn{2}{c}{\upbracefill}             & \multicolumn{2}{c}{\upbracefill} \\
  具 & \multicolumn{2}{c}{viduṣā}                   & \multicolumn{2}{c}{vidvadbhis} \\
  爲 & \multicolumn{2}{c}{viduṣe}                   & \multicolumn{2}{c}{vidvadbhyas} \\
  從 & \multicolumn{2}{c}{\rdelim\}{2}{*}[viduṣas]} & \multicolumn{2}{c}{\multirow{2}{*}{viduṣām}} \\
  屬 &                                              & \\
  於 & \multicolumn{2}{c}{viduṣi}                   & \multicolumn{2}{c}{vidvatsu}
\end{tabular}
\end{center}

\begin{center}
\begin{tabular}{c*{2}{p{0.24\hsize}}}
     & \multicolumn{2}{c}{兩} \\
     & \cellAlign{c}{男}          & \cellAlign{c}{中} \\
  主 & \rdelim\}{3}{*}[vidvāṃsau] & \multirow{3}{*}{tudantī, tudatī} \\
  呼 &                            & \\
  業 &                            & \\
     & \multicolumn{2}{c}{\upbracefill} \\
  具 & \multicolumn{2}{c}{\rdelim\}{3}{*}[vidvadbhyām]} \\
  爲 &                            & \\
  從 &                            & \\
  屬 & \multicolumn{2}{c}{\rdelim\}{2}{*}[viduṣos]} \\
  於 &                            &
\end{tabular}
\end{center}

女性語基は弱基に ī を附加して作る viduṣī.

\subsubsection{yas に終る比較級形容詞}

\numberParagraph
yas に終る比較級(\ref{np:98}條)は二語基を有す卽ち强形 yāṃs
及び弱形 yas.

語基 śreyas (よりよき),强形 śreyāṃs. 弱形 śreyas.
\begin{center}
\begin{tabular}{c*{4}{p{0.15\hsize}}}
     & \multicolumn{2}{c}{單}                        & \multicolumn{2}{c}{複} \\
     & \cellAlign{c}{男} & \cellAlign{c}{中}         & \cellAlign{c}{男}          & \cellAlign{c}{中} \\
  主 & śreyān            & \rdelim\}{3}{*}[śreyas]   & \rdelim\}{2}{*}[śreyāṃsas] & \rdelim\}{3}{*}[śreyāṃsi] \\
  呼 & śreyan            &                           &                            & \\
  業 & śreyāṃsam         &                           & śreyasas                   & \\
     & \multicolumn{2}{c}{\upbracefill}              & \multicolumn{2}{c}{\upbracefill} \\
  具 & \multicolumn{2}{c}{śreyasā}                   & \multicolumn{2}{c}{śreyobhis} \\
  爲 & \multicolumn{2}{c}{śreyase}                   & \multicolumn{2}{c}{\rdelim\}{2}{*}[śreyobhyaas]} \\
  從 & \multicolumn{2}{c}{\rdelim\}{2}{*}[śreyasas]} & \\
  屬 &                   &                           & \multicolumn{2}{c}{śreyasām} \\
  於 & \multicolumn{2}{c}{śreyasi}                   & \multicolumn{2}{c}{śreyaḥsu}
\end{tabular}
\end{center}

\begin{center}
\begin{tabular}{c*{2}{p{0.24\hsize}}}
     & \multicolumn{2}{c}{兩} \\
     & \cellAlign{c}{男}          & \cellAlign{c}{中} \\
  主 & \rdelim\}{3}{*}[śreyāṃsau] & \multirow{3}{*}{śreyasī} \\
  呼 &                            & \\
  業 &                            & \\
     & \multicolumn{2}{c}{\upbracefill} \\
  具 & \multicolumn{2}{c}{\rdelim\}{3}{*}[śreyobhyām]} \\
  爲 &                            & \\
  從 &                            & \\
  屬 & \multicolumn{2}{c}{\rdelim\}{2}{*}[śreyasos]} \\
  於 &                            &
\end{tabular}
\end{center}

女性語基は弱基に ī を附加して作る。卽ち śreyas, 女性
śreyasī; garīyas (より重き),女性 garīyasī.

\ex{第七}
\begin{longtable}{c*{2}{p{0.45\hsize}}}
 1. & rājño gṛhe mahān utsavo 'bhavat. & 王の家に於て大なる祭があつた。\\
 2. & vāyur ambhasi nāvaṃ ha\-rati. & 風は水に於て舟を運び行く。\\
 3. & vaṇijaḥ sutā sakhībhiḥ sa\-hārāme krīḍanāya gacchati sma & 商人の娘は友達と共に國へ遊戯のために行けり。\\
 4. & mahati vipadaḥ sāgare pra\-mādena nāpito 'patat. & 理髪師は不注意によつて不幸の大海に落ちた。\\
 5. & vṛddho dhanī vaṇig yuvatiṃ nirdhanasya duhitaraṃ parya\-ṇayat. & 老いたる富める商人は若き貧人の娘と結婚せり。\\
 6. & asamarthānāṃ puṃsāṃ kopa ātmana upadravāya bhavati. & 無能力なる人々の怒は自らの危難を招く。\\
 7. & ravir niśāyās tamo 'paharati. & 太陽は夜の暗を拂ひ去る。\\
 8. & sūryasya tejasā saṃtaptaḥ pānthaś chāyām āśrayate. & 日の熱に熱せられ旅人は蔭に近寄る。\\
 9. & medhāvī śuddhaṃ jīvitam ācaret. & 知者は淨き生を送るべきだ。\\
10. & śuṣkavat saras tyajanti sā\-rasāḥ. & 鶴は涸れたる池を去る。\\
11. & balavatī dantānāṃ vedanā brāhmaṇaṃ bādhate. & 太だしい齒痛は婆羅門を苦しめる。\\
12. & hariṇo lubdhakasya śarāṇāṃ prahārād ubbhāritaḥ kṛcchreṇa saraḥ praviṣṭaḥ. & 鹿は獵師の箭の打擊から逃
れてやつとのことで池に入つた。\\
13. & munis tapasi rato vane tiṣṭhati. & 聖者は苦行を樂みて林に住す。\\
14. & kāvyānāṃ śāstrāṇāṃ ca vino\-dena kālo gacchati dhīmatām. & 詩書の樂に賢者の時は過ぎ行く。\\
15. & śunaḥ puccham iva vyarthaṃ jīvitaṃ vidyayā vinā. & 知識なくしては生は犬の尾の如く價値無し。\\
16. & śriyā striyo haranti puṃsāṃ manāṃsi ca cakṣūṃsi ca. & 婦女は美によりて人々の心と目とを奪ふ。\\
17. & balaṃ vidyā ca viprāṇāṃ rājñāṃ sainyaṃ balaṃ tathā. &知識は婆羅門の力,此の如く王者の力は軍隊。\\
& balaṃ vittaṃ ca vaiśyānāṃ śūdrāṇāṃ ca kaniṣṭhatā. & 商人にとつては財は力,シュードラにとつては下賤な
ることが(力である)。
\end{longtable}

%%% Local Variables:
%%% mode: latex
%%% TeX-master: "IntroductionToSanskrit"
%%% End:

\section{比較法}
\numberParagraph
形容詞の比較級は男性の語基に tara を,最上級は tama
を附加して作る。變化は a 語基に準ず。女性は終の母音を ā に
作り kanyā の變化に準ず。priya (愛する)~ priyatara (より
愛する),priyatama (最も愛する)。

\numberParagraph
二語基ある場合は弱基に,三語基ある場合は中基に附加せ
らる。例:prāc (東方の)~ prāktara, prāktama.

\numberParagraph \label{np:98}
他の方法は後接字 īyas, iṣṭha を語基に附加して作る。
語基の母音は重韻化又は延長によりて强められる。kṣipra (速か
なる)は語根 kṣip から比較級 kṣepīyas, 最上級 kṣepiṣṭha が
作られ,mṛdu (軟かき)は mradīyas, mradiṣṭha; dūra (遠き)
は davīyas, daviṣṭha.

時には原級の形が比較級最上級の形と異るものがある。例せば
śreyas, śreṣṭha の原級は śrī にあらずして praśasya (勝れた
る)であり,kanīyas, kaniṣṭha の原級は alpa (小なる)である。

%%% Local Variables:
%%% mode: latex
%%% TeX-master: "IntroductionToSanskrit"
%%% End:

\section{代名詞}
\subsection{人稱代名詞}
\numberParagraph
一人稱代名詞(吾)語基は 單 mad, 複 asmad.

\begin{center}
\begin{tabular}{c*{3}{p{0.2\hsize}}}
     & 單                    & 兩                        & 複 \\
  主 & \rdelim\}{2}{*}[aham] & \rdelim\}{3}{*}[āvām]     & \rdelim\}{2}{*}[vayam] \\
  呼 &                       &                           & \\
  業 & mām, mā               &                           & asmān, nas \\
  具 & mayā                  & \rdelim\}{3}{*}[āvābhyām] & asmābhis \\
  爲 & mahyam, me            &                           & asmabhyam, nas \\
  從 & mat                   &                           & asmat \\
  屬 & mama, me              & \rdelim\}{2}{*}[āvayos]   & asmākam, nas \\
  於 & mayi                  &                           & asmāsu
\end{tabular}
\end{center}

兩數の業爲屬に nau を用ふることがある。

\numberParagraph
二人稱代名詞(汝)語基は 單 tvad, 複 yuṣmad.

\begin{center}
\begin{tabular}{c*{3}{p{0.2\hsize}}}
     & 單                    & 兩                         & 複 \\
  主 & \rdelim\}{2}{*}[tvam] & \rdelim\}{3}{*}[yuvām]     & \rdelim\}{2}{*}[yūyam] \\
  呼 &                       &                            & \\
  業 & tvām, tvā             &                            & yuṣmān, vas \\
  具 & tvayā                 & \rdelim\}{3}{*}[yuvābhyām] & yuṣmābhis \\
  爲 & tubhyam, te           &                            & yuṣmabhyam, vas \\
  從 & tvat                  &                            & yuṣmat \\
  屬 & tava, te              & \rdelim\}{2}{*}[yuvayos]   & yuṣmākam, vas \\
  於 & tvayi                 &                            & yuṣmāsu
\end{tabular}
\end{center}

兩數の業爲屬に vām を用ふることがある。

\subsection{指示代名詞}
\numberParagraph
語基 tad. これは同時に三人稱代名詞(彼,夫)でもあり得
る,\endnote{底本では女性複数具格(tābhis)が空欄。}

\begin{center}
\begin{tabular}{c*{9}{p{0.085\hsize}}}
     & \multicolumn{3}{c}{單}                                                                                & \multicolumn{3}{c}{兩}                                           & \multicolumn{3}{c}{複} \\
     & \multicolumn{3}{c}{\downbracefill}                                                                    & \multicolumn{3}{c}{\downbracefill}                               & \multicolumn{3}{c}{\downbracefill} \\
     & 男                                & 中                   & 女                      & 男                   & 中                  & 女                  & 男                  & 中                     & 女 \\
  主 & \rdelim\}{2}{*}[sas\footnotemark] & \rdelim\}{3}{*}[tat] & \rdelim\}{2}{*}[sā]     & \rdelim\}{3}{*}[tau] & \multirow{3}{*}{te} & \multirow{3}{*}{te} & \rdelim\}{2}{*}[te] & \rdelim\}{3}{*}[tāni]  & \multirow{3}{*}{tās} \\
  呼 &                                   &                      &                         &                      &                     &                     &                     &                        & \\
  業 & tam                               &                      & tām                     &                      &                     &                     & tān                 &                        & \\
     & \multicolumn{2}{c}{\upbracefill}                         &                         & \multicolumn{3}{c}{\upbracefill}                                 & \multicolumn{2}{c}{\upbracefill}             & \\
  具 & \multicolumn{2}{c}{tena}                                 & tayā                    & \multicolumn{3}{c}{\rdelim\}{3}{*}[tābhyām]}                     & \multicolumn{2}{c}{tais}                     & tābhis \\
  爲 & \multicolumn{2}{c}{tasmai}                               & tasyai                  &                      &                     &                     & \multicolumn{2}{c}{\rdelim\}{2}{*}[tebhyas]} & \multirow{2}{*}{tābhyas} \\
  從 & \multicolumn{2}{c}{tasmāt}                               & \rdelim\}{2}{*}[tasyās] &                      &                     &                     &                   &                          & \\
  屬 & \multicolumn{2}{c}{tasya}                                &                         & \multicolumn{3}{c}{\rdelim\}{2}{*}[tayos]}                       & \multicolumn{2}{c}{teṣām}                    & tāsām \\
  於 & \multicolumn{2}{c}{tasmin}                               & tasyām                  &                      &                     &                     & \multicolumn{2}{c}{teṣu}                     & tāsu
\end{tabular}
\footnotetext{{\ref{np:24}條註參照。}}
\end{center}

\numberParagraph
語基 etad (此れ)の變化は tad に準ず。

主,單 eṣas, eṣā, etad.

\numberParagraph
語基 enad (彼)は只業の三數と具の單,屬於の兩がある
のみである。卽ち:

\begin{center}
\begin{tabular}{c*{3}{p{0.15\hsize}}}
     & \multicolumn{3}{c}{單} \\
     & 男   & 中                        & 女 \\
  業 & enam & enat                      & enām \\
     & \multicolumn{2}{c}{\upbracefill} & \\
  具 & \multicolumn{2}{c}{enena}        & enayā
\end{tabular}
\end{center}
\begin{center}
\begin{tabular}{c*{3}{p{0.15\hsize}}}
     & \multicolumn{3}{c}{兩} \\
  業 & enau & ene                      & ene \\
     & \multicolumn{3}{c}{\upbracefill} \\
  屬 & \multicolumn{3}{c}{\rdelim\}{2}{*}[enayos]} \\
  於 & \\
\end{tabular}
\end{center}
\begin{center}
\begin{tabular}{c*{3}{p{0.15\hsize}}}
     & \multicolumn{3}{c}{複} \\
  業 & enān & enāni & enās
\end{tabular}
\end{center}

\numberParagraph
語基 idam (これ),adas (其れ)。

\begin{center}
\begin{tabular}{c*{9}{p{0.1\hsize}}}
     & \multicolumn{6}{c}{單數} \\
     & 男                    & 中                    & 女                     & 男                    & 中                    & 女 \\
  主 & \rdelim\}{2}{*}[ayam] & \rdelim\}{3}{*}[idam] & \rdelim\}{2}{*}[iyam]  & \multirow{2}{*}{asau} & \rdelim\}{3}{*}[adas] & \rdelim\}{2}{*}[asau] \\
  呼 &                       &                       &                        &                       &                       & \\
  業 & imam                  &                       & imām                   & amum                  &                       & amūm \\
     & \multicolumn{2}{c}{\upbracefill}              &                        & \multicolumn{2}{c}{\upbracefill}              & \\
  具 & \multicolumn{2}{c}{anena}                     & anayā                  & \multicolumn{2}{c}{amunā}                     & amuyā \\
  爲 & \multicolumn{2}{c}{asmai}                     & asyai                  & \multicolumn{2}{c}{amuṣmai}                   & amuṣyai \\
  從 & \multicolumn{2}{c}{asmāt}                     & \rdelim\}{2}{*}[asyās] & \multicolumn{2}{c}{amuṣmāt}                   & \rdelim\}{2}{*}[amuṣyās] \\
  屬 & \multicolumn{2}{c}{asya}                      &                        & \multicolumn{2}{c}{amuṣya}                    & \\
  於 & \multicolumn{2}{c}{asmin}                     & asyām                  & \multicolumn{2}{c}{amuṣmin}                   & amuṣyām
\end{tabular}
\end{center}
\begin{center}
\begin{tabular}{c*{4}{p{0.1\hsize}}}
     & \multicolumn{4}{c}{兩數} \\
     & 男                    & 中                   & 女                    & 男中女 \\
  主 & \rdelim\}{3}{*}[imau] & \multirow{3}{*}{ime} & \multirow{3}{*}{ime}  & \multirow{3}{*}{amū} \\
  呼 &                       &                      &                       & \\
  業 &                       &                      &                       & \\
     & \multicolumn{3}{c}{\upbracefill}                                     & \\
  具 & \multicolumn{3}{c}{\rdelim\}{3}{*}[ābhyām]}                          & \multirow{3}{*}{amūbhyām} \\
  爲 &                                                                      & \\
  從 &                                                                      & \\
  屬 & \multicolumn{3}{c}{\rdelim\}{2}{*}[ābhyām]}                          & \multirow{2}{*}{amuyos} \\
  於 &                                                                      &
\end{tabular}
\end{center}
\begin{center}
\begin{tabular}{c*{9}{p{0.1\hsize}}}
     & \multicolumn{6}{c}{複數} \\
     & 男                   & 中                     & 女                      & 男                    & 中                     & 女 \\
  主 & \rdelim\}{2}{*}[ime] & \rdelim\}{3}{*}[imāni] & \multirow{3}{*}{imās}   & \rdelim\}{2}{*}[amī]  & \rdelim\}{3}{*}[amūni] & \multirow{3}{*}{amūs} \\
  呼 &                      &                        &                         &                       &                        & \\
  業 & imān                 &                        &                         & amūn                  &                        & \\
     & \multicolumn{2}{c}{\upbracefill}              &                         & \multicolumn{2}{c}{\upbracefill}               & \\
  具 & \multicolumn{2}{c}{ebhis}                     & ābhis                   & \multicolumn{2}{c}{amībhis}                    & amūbhis \\
  爲 & \multicolumn{2}{c}{\rdelim\}{2}{*}[ebhyas]}   & \multirow{2}{*}{ābhyas} & \multicolumn{2}{c}{\rdelim\}{2}{*}[amībhyas]}  & \multirow{2}{*}{amūbhyas} \\
  從 &                                               &                         &                                                & \\
  屬 & \multicolumn{2}{c}{eṣām}                      & āsām                    & \multicolumn{2}{c}{amīṣām}                     & amūṣām \\
  於 & \multicolumn{2}{c}{eṣu}                       & āsu                     & \multicolumn{2}{c}{amīṣu}                      & amūṣu
\end{tabular}
\end{center}

\subsection{關係代名詞}
\numberParagraph
語基 yad (所のそれは),變化は tad に準ず。

\begin{center}
\begin{tabular}{c*{9}{p{0.085\hsize}}}
     & \multicolumn{3}{c}{單}                                                               & \multicolumn{3}{c}{兩}                                           & \multicolumn{3}{c}{複} \\
     & \multicolumn{3}{c}{\downbracefill}                                                   & \multicolumn{3}{c}{\downbracefill}                               & \multicolumn{3}{c}{\downbracefill} \\
     & 男                               & 中                      & 女                      & 男                   & 中                  & 女                  & 男                  & 中                     & 女 \\
  主 & \rdelim\}{2}{*}[yas]             & \rdelim\}{3}{*}[yat]    & \rdelim\}{2}{*}[yā]     & \rdelim\}{3}{*}[yau] & \multirow{3}{*}{ye} & \multirow{3}{*}{ye} & \rdelim\}{2}{*}[ye] & \rdelim\}{3}{*}[yāni]  & \multirow{3}{*}{yās} \\
  呼 &                                  &                         &                         &                      &                     &                     &                     &                        & \\
  業 & yam                              &                         & yām                     &                      &                     &                     & yān                 &                        & \\
     & \multicolumn{2}{c}{\upbracefill}                           &                         & \multicolumn{3}{c}{\upbracefill}                                 & \multicolumn{2}{c}{\upbracefill}             & \\
  具 & \multicolumn{2}{l}{yena}                                   & yayā                    & \multicolumn{3}{l}{\rdelim\}{3}{*}[yābhyām]}                     & \multicolumn{2}{l}{yais}                     & yābhis \\
  爲 & \multicolumn{2}{l}{yasmai}                                 & yasyai                  &                      &                     &                     & \multicolumn{2}{l}{\rdelim\}{2}{*}[yebhyas] \hfill} & \multirow{2}{*}{yābhyas} \\
  從 & \multicolumn{2}{l}{yasmāt}                                 & \rdelim\}{2}{*}[yasyās] &                      &                     &                     &                   &                          & \\
  屬 & \multicolumn{2}{l}{yasya}                                  &                         & \multicolumn{3}{l}{\rdelim\}{2}{*}[yayos]}                       & \multicolumn{2}{l}{yeṣām}                    & yāsām \\
  於 & \multicolumn{2}{l}{yasmin}                                 & yasyām                  &                      &                     &                     & \multicolumn{2}{l}{yeṣu}                     & yāsu
\end{tabular}
\end{center}

\subsection{疑問代名詞}
\numberParagraph
語基 kim (誰,何)亦 tad に準ず。

\begin{center}
\begin{tabular}{c*{9}{p{0.085\hsize}}}
     & \multicolumn{3}{c}{單}                                                               & \multicolumn{3}{c}{兩}                                           & \multicolumn{3}{c}{複} \\
     & \multicolumn{3}{c}{\downbracefill}                                                   & \multicolumn{3}{c}{\downbracefill}                               & \multicolumn{3}{c}{\downbracefill} \\
     & 男                               & 中                      & 女                      & 男                   & 中                  & 女                  & 男                  & 中                     & 女 \\
  主 & \rdelim\}{2}{*}[kas]             & \rdelim\}{3}{*}[kim]    & \rdelim\}{2}{*}[kā]     & \rdelim\}{3}{*}[kau] & \multirow{3}{*}{ke} & \multirow{3}{*}{ke} & \rdelim\}{2}{*}[ke] & \rdelim\}{3}{*}[kāni]  & \multirow{3}{*}{kās} \\
  呼 &                                  &                         &                         &                      &                     &                     &                     &                        & \\
  業 & kam                              &                         & kām                     &                      &                     &                     & kān                 &                        & \\
     & \multicolumn{2}{c}{\upbracefill}                           &                         & \multicolumn{3}{c}{\upbracefill}                                 & \multicolumn{2}{c}{\upbracefill}             & \\
  具 & \multicolumn{2}{l}{kena}                                   & kayā                    & \multicolumn{3}{l}{\rdelim\}{3}{*}[kābhyām]}                     & \multicolumn{2}{l}{kais}                     & kābhis \\
  爲 & \multicolumn{2}{l}{kasmai}                                 & kasyai                  &                      &                     &                     & \multicolumn{2}{l}{\rdelim\}{2}{*}[kebhyas] \hfill} & \multirow{2}{*}{kābhyas} \\
  從 & \multicolumn{2}{l}{kasmāt}                                 & \rdelim\}{2}{*}[kasyās] &                      &                     &                     &                   &                          & \\
  屬 & \multicolumn{2}{l}{kasya}                                  &                         & \multicolumn{3}{l}{\rdelim\}{2}{*}[kayos]}                       & \multicolumn{2}{l}{keṣām}                    & kāsām \\
  於 & \multicolumn{2}{l}{kasmin}                                 & kasyām                  &                      &                     &                     & \multicolumn{2}{l}{keṣu}                     & kāsu
\end{tabular}
\end{center}

\numberParagraph
疑問代名詞に cit, api, cana を附加すれば不定の意味を
有す。kaścit (誰にても),kimapi (何にても),na kaścit (決し
て何人も……せぬ)。

\subsection{代名詞的形容詞}
\numberParagraph
或る形容詞は代名詞的の變化をなす。
\begin{enumerate}[label=(\alph*)]
\item anyatara (他の)(主 anyas, anyā, anyat),katara (二
つの中誰,孰れ),katama (多くの中誰,孰れ),anyatara
(二つの中隨一),yatara (二の中の孰れ),yatama (多くの
中の孰れ),itara (他の)。
\item \label{item:108b}同樣に sarva (總ての),viśva (總ての),eka (一の),
ekatara (二の中の隨一の)は tad に準ず。只單中性主呼
業が m の語尾を附加せらるる差異に注意せねばならぬ。
\item adhara (劣れる,西方の),antara (外の),apara (他
の),uttara (上の),avara (後の,西方の),dakṣiṇa (右方
の,南の),para (後の),pūrva (前の),sva (自の)は \ref{item:108b}
に準じて變化す。只男中單,從,於と主複に於ては名詞變化
をなすを得。
\end{enumerate}

\subsection{代名詞的名詞}
\numberParagraph
名詞の或るものは屢々意義上代名詞の如く見做される。
\begin{enumerate}[label=(\alph*)]
\item ātman (我)は再歸代名詞として用ひられる。ātmānaṃ
naraḥ parīkṣeta (人は自らを省るべきだ)。
\item bhavat (君)なる語基は二人稱の意義に用ひられた代名
詞である。kva gacchati (又は gacchasi)bhavān (君は
何處に行く)。
\end{enumerate}

%%% Local Variables:
%%% mode: latex
%%% TeX-master: "IntroductionToSanskrit"
%%% End:

\section{數詞}
\subsection{順數}
\numberParagraph
\begin{longtable}{rlrl}
 1 & eka     & 13 & trayodaśa \\
 2 & dvi     & 14 & caturdaśa \\
 3 & tri     & 15 & pañcadaśa \\
 4 & catur   & 16 & ṣoḍaśa \\
 5 & pañca   & 17 & saptadaśa \\
 6 & ṣaṣ     & 18 & aṣṭādaśa \\
 7 & sapta   & 19 & navadaśa\footnotemark \\
 8 & aṣṭa    & 20 & viṃśati \\
 9 & nava    & 21 & ekaviṃśati \\
10 & daśa    & 22 & dvāviṃśati \\
11 & ekādaśa & 23 & trayoviṃśati \\
12 & dvādaśa & 24 & caturviṃśati \\
\\
25 & pañcaviṃśati              &   60 & ṣaṣṭi \\
26 & śaḍviṃśati                &   70 & saptati \\
27 & saptaviṃśati              &   80 & aśīti \\
28 & aṣṭāviṃśati               &   90 & navati \\
29 & navaviṃśati\footnotemark  &  100 & śata \\
30 & triṃśat                   &  200 & dve śate 又は dviśata\\
40 & catvāriṃśat               &  300 & trīṇi śatāni, 又は triśata \\
50 & pañcāśat                  & 1000 & sahasra \\
\end{longtable}
\addtocounter{footnote}{-2}
\stepcounter{footnote}\footnotetext{又は ūnaviṃśati, ekonaviṃśati.}
\stepcounter{footnote}\footnotetext{ūnatriṃśat.}

\numberParagraph
2, 3, 8 の數は 20, 30 に結合せられる時は夫々 dvā,
trayas, aṣṭā となり,80 に對しては dvi, tri, aṣṭa である。40
から 70 までと 90 とに對しては兩形が用ひられる。

19, 20 等は ūna (減ぜられし)又は ekona (一を減ぜし)を添へ
て呼ぶことがある。101, 102 等は adhika (越えたる,加へたる)
の語を添へて呼ぶ。ekādhikaṃ śatam 又は ekādhikaśatam.

\numberParagraph
eka (一)は單數並に複數(eke 或るもの)に於て三性に
變化し,代名詞的變化をなす。dvi (二)は規則的に dva の兩數
に變化せられ,tri と catur は次の如く變化する。

\begin{center}
\begin{tabular}{c*{9}{p{0.12\hsize}}}
     & 男                      & 中                     & 女                         & 男                        & 中                       & 女 \\
  主 & \rdelim\}{2}{*}[trayas] & \rdelim\}{3}{*}[trīṇi] & \multirow{3}{*}{tisras}    & \rdelim\}{2}{*}[catvāras] & \rdelim\}{3}{*}[catvāri] & \multirow{3}{*}{catasras} \\
  呼 &                         &                        &                            &                           &                          & \\
  業 & trīn                    &                        &                            & caturas                   &                          & \\
     & \multicolumn{2}{c}{\upbracefill}                 &                            & \multicolumn{2}{c}{\upbracefill}                     & \\
  具 & \multicolumn{2}{c}{tribhis}                      & tisṛbhis                   & \multicolumn{2}{c}{caturbhis}                        & catasṛbhis \\
  爲 & \multicolumn{2}{c}{\rdelim\}{2}{*}[tribhyas]}    & \multirow{2}{*}{tisṛbhyas} & \multicolumn{2}{c}{\rdelim\}{2}{*}[caturbhyas]}      & \multirow{2}{*}{catasṛbhyas} \\
  從 &                                                  &                            &                                                      & \\
  屬 & \multicolumn{2}{c}{trayāṇām}                     & tisṛṇām                    & \multicolumn{2}{c}{caturṇām}                         & catasṛṇām \\
  於 & \multicolumn{2}{c}{triṣu}                        & tisṛṣu                     & \multicolumn{2}{c}{caturṣu}                          & catasṛṣu
\end{tabular}
\end{center}

\numberParagraph
5 から 19 に至る迄は性の區別なく、若干の不規則はあ
るも複數として變化する。只主業は語基のまゝである。

主業呼 pañca, 具 pañcabhis, 爲從 pañcabhyas, 屬 pañcā\-%
nām, 於 pañcasu.

主業呼 ṣaṭ, 具 ṣaḍbhis, 屬 ṣaṇṇām, 於 ṣaṭsu.

主業呼 aṣṭa 又は aṣṭau, 具 aṣṭabhis 又は aṣṭābhis, 爲從
aṣṭabhyas, 又は aṣṭābhyas, 屬 aṣṭānām, 於 aṣṭasu 又は
aṣṭāsu.

\numberParagraph
20 より以上の數詞は名詞である。20 より 99 までは女
性單數,100, 1,000, 10,000 及び 100,000 は中性單數である。
數へられたる事物は或は複數となして數詞と同格にし,或は屬格
複數となして並べ,或は合成語となす。「二十人の婦女に圍まれ
たる王」と云ふ時 rājā viṃśatyā nārībhiḥ parivṛtaḥ 又は
rājā viṃśatyā nārīṇāṃ parivṛtaḥ; ṣaṣṭyāṃ varṣeṣu (六十年
の間),catvāri sahasrāṇi varṣāṇām (四千年),varṣa-śatam
(百年)。

\subsection{序數}
\numberParagraph
\begin{longtable}{rlrl}
 1. & prathama &  7. & saptama \\
 2. & dvitīya  &  8. & aṣṭama \\
 3. & tṛtīya   &  9. & navama \\
 4. & caturtha & 10. & daśama \\
 5. & pañcama  & 11. & ekādaśa \\
 6. & ṣaṣṭha   & 12. & dvādaśa \\
\\
20. & viṃśatitama, viṃśa          &   70. & saptatitama \\
30. & triṃśattama, triṃśa         &   80. & aśītitama \\
40. & catvāriṃśattama, catvāriṃśa &   90. & navātitama \\
50. & pañcāśattama, pañcaśa       &  100. & śatatama \\
60. & ṣaṣṭitama                   & 1000. & sahasratama \\
61. & ekaṣaṣṭitama, ekaṣaṣṭa      &       & \\
\end{longtable}

\numberParagraph
序數は a 語基男中性の變化に準ず。女性は ī を附加す
(caturthī, pañcamī)。但し最初の三は ā 語基(prathamā,
dvitīyā, tṛtīyā)なるを異とする。

\numberParagraph
prathama, dvitīya, tṛtīya は或る場合に代名詞的變化を
なし得。但し prathama は主複,dvitīya, tṛtīya は爲從,於,單
に於いてである。

\subsection{數の副詞}
\numberParagraph \label{np:118}
\begin{enumerate}[label=(\alph*)]
\item sakṛt (一度),dvis (二度),tris (三度),catus (四
度),pañcakṛtvas (五度),ṣaṭkṛtvas (六度)等。
\item ekadhā (一重に),dvidhā 又は dvedhā (二重に),
tridhā 又は tredhā (三重に),caturdhā (四重に)等。
\item ekaśas (單獨に),dviśas (一對づゝに),triśas (三つ
宛で)等。
\end{enumerate}

\ex{第八}
\begin{longtable}{c*{2}{p{0.45\hsize}}}
 1. & yasyārthās tasya mitrāṇi, & 富める彼には友あり,\\
    & yasyārthās tasya bāndhavāḥ, & 富める彼には親族あり,\\
    & yasyārthaḥ sa pumā\anunasikamd{}l loke, & 富める彼は世間に於て人なり,\\
    & yasyārthāḥ sa hi paṇḍitaḥ. & 富める彼は實に學者なり。\\
 2. & bho, ko bhavān. & おゝ君は誰なるか。\\
 3. & na bhavati mad dhanyataraḥ. & 吾れよりも幸福なるものはあらず。\\
 4. & siddhāḥ sarve 'smākaṃ ma\-norathāḥ. & 我等の希望の總ては成就した。\\
 5. & bho vañcitā vayam anena. & おゝ我々は彼によりて欺かれた。\\
 6. & mitra, kiyatā mūlynaitat pustakaṃ gṛhītam. rūpakānāṃ śatena. & 友よ\ruby[g]{幾何}{いくばく}にてこの書が得ら
 れしか。百ルーピー。\\
 7. & asminn eva latāgṛhe tvam abhavaḥ. & その蔓草の家に於て汝はありき。\\
 8. & kṛtaṃ kim ebhis tava vipri\-yaṃ, yad aniṣṭam eṣāṃ cinta\-yasi. & 汝が彼等に就て禍と考へる
 所の汝に不快な何ものが是等によりて爲されたるか。\\
 9. & tvaṃ me jīvitaṃ tvaṃ me hṛdayaṃ dvitīyaṃ tvaṃ kaumu\-dī nayanayor amṛtaṃ tvam aṅge. & 汝は吾が生命である,汝は
 吾が第二の心である,汝は兩眼に於て月光なり。身に於て汝は甘露なり。\\
10. & sarvasyātithir guruḥ. & 一切にとつて賓客は尊敬すべきだ。\\
11. & kā sā purī, ko vā deśaḥ. & かの町は何ぞ,又如何なる地方なるか。\\
12. & nīrasāny api rocante naḥ kar\-pāsasya phlāni. & 味は無きも綿の實は我々に願はしくある。\\
13. & caturdaśa sahasrāṇi śatrūṇāṃ raṇe hatāni. & 一萬四千の敵は戰に於いて殺された。\\
14. & ete trayaḥ puruṣaya gariṣṭhā bhavanti: ācāryaḥ, pitā, mā\-tā ca. & これら三人は人の中で最も
重んぜられてある。卽ち師,父及び母である。\\
15. & te putrā ye pitur bhaktāḥ, & 父に從順なるは子であり,\\
    & sa pitā yas tu poṣakaḥ, & 養育するものは父であり,\\
    & tan mitraṃ yatra viśvāsaḥ, & 信賴あるものは友であり,\\
    & sā bhāryā yasya nirvṛtiḥ. & 滿足あるものは妻である。
\end{longtable}

%%% Local Variables:
%%% mode: latex
%%% TeX-master: "IntroductionToSanskrit"
%%% End:

\section{合成語法}
\numberParagraph
二個以上の語を合成することは梵語には極めて著しい特
徴である。Veda や Brāhmaṇa ではせいぜい二語ぐらゐの合成
語が現はれるに過ぎないが,後期梵語ではこの傾向が次第に加は
つて來る。合成語は語尾變化の繁雜を簡易にすることがその發生
の動機の一である。詩人 Kālidāsa の「雲の使ひ」(I, 30) に
は河の流を形容して vīci-kṣobha-stanita-vihaga-śreṇi-kāñcī-
guṇāyāḥ と云ふ語があるが,これは「波-亂-囀-鳥-列-帶-
絲」と云ふ七語から成る合成語で,その意味は「波の亂れが(恰
うど)囀る鳥の一列(のやうなの)を一筋の帶とした所の」と云
ふことである。語と語の間の關聯を適當に理解するにはこの合成
語を正しく讀むことが重要な問題の一である。

\numberParagraph
印度の文典家は合成語を六種に分つて說く。卽ち六合釋
(ṣaṭ-samāsa 殺三磨娑)である。佛敎經典にも相當合成語が見え
る。而してその讀み方の如何が敎義重大の関係を有する。印度
佛敎々學の華やかなりし那爛陀寺時代の學匠達は經典の釋義の上
に常に六合釋を以て鎬を削つて諍論した形迹がある。支那に經典
が飜譯されても漢文の上に現はれた合成語の解釋が何等かの規範
なしには不便を感ずるので,之を看て取つた玄奘は印度文典の方
規を採用して經典解釋に新機軸を出した。これが六合釋である。
かくしてこれは久しく佛典硏究家にも規矩準繩となつたものであ
る。

\numberParagraph
六合釋とは相違,依主,持業,帶數,隣近,有財である。
然しこれらは性質上三種として說明せられる。卽ち 1. 並列合成
語(相違),2. 決定合成語(この中に依主,持業,帶數,有財を含
む),3. 副詞合成語(隣近)である。

\numberParagraph
合成語の各個の間には連聲法の規定(\ref{np:10}--\ref{np:29}條)が適用
せられる。又最後の語の外,各語はみな語基の形を取る。三語基
あるものは中,二語基あるものは弱を用ふ。rājan の如き n に
終るものはその n を除去して用ふ。

\numberParagraph
mahat は前分として mahā となる。akṣi (眼)は後分と
して akṣa, ahan (日)は aha 又は ahna に作る。sakhi (友)は
sakha, rātri (夜)は rātra, path (路)は patha, manas (意)は
manasa, varcas (輝)は varcasa となる。又時として反對に變化
することもある。gandha (香)が gandhi, go (牝牛)は母音の以
前に gava, 語末には gava 又は gu.

\subsection{並列合成語(相違釋 dvaṃdva)}
\numberParagraph
この合成語は各部分が互に等しき位にあるもので,各語
が連接し(copulative)又はその孰れかを選び取る(alternative)
意義を有す。
\begin{enumerate}[label=(\alph*)]
\item 語が二個より成るか三個以上より成るかで兩數又は複
數の語尾を追加する。candrādityau (月と日),kākākhu-%
mṛga-kūrmāḥ (鴉と鼠と鹿と龜と)。
\item 若し合成語が個々のものを意味しないでそのまゝで一
つの槪念を表はすやうな時には集合名詞として中性單數の
形を取る。pāṇi-pādam (手足)。gavāśvam (牛馬)。
\end{enumerate}

\subsection{決定合成語}
\subsubsection{依主釋(tatpuruṣa)}
\numberParagraph
依主とは後分が前分に依つて限定せらるゝものである。
前分は後分に對して格の關係を有す。例:grāma-gata (村へ行
きたる)の前分は業格,Indra-gupta (インドラに護られたる)の
前分は具格,svarga-patita (天界から墮ちたる)の前分は從格,
rāja-putra (王子)の前分は屬格である。

\numberParagraph
依主の前分が格の形を有することもある。vācaṃ-yama
(聲を制して),Gavāṃ-pati (牛主,憍梵波提),padme śaya (蓮華
の上に橫はれる)。

\numberParagraph
合成語の最後分として動詞の語根が用ひられる。veda-%
vid (吠陀に通ぜる),短母音の語根には t を加ふ。Aśva-jit (馬
勝)。語根の母音 ā は男中性の語として短縮せらる。abhyāsa-%
stha (近くにある)。

\subsubsection{持業釋(Karmadhāraya)}
\numberParagraph
依主の前分が形容詞副詞にして後分を限定する時は持業
釋と云ふ。nīlotpala (靑き蓮花),paramānanda (最上の歡喜),
ati-dīrgha (極めて長き)。前分が名詞であることもある。kusu-%
ma-sukumāra (花の如く軟かなる),puruṣa-siṃha (人獅子),か
くの如き場合は獅子の如き人の意味で前分が主體後分が譬喩であ
る。

\subsubsection{帶數釋(dvigu)}
\numberParagraph
依主の前分が數詞にして其の語形は中性又は ī にて終る
女性名詞である場合を帶數釋と云ふ。tri-rātra (三夜),pañca-%
gava 又は pañca-gavī (五牛)。

\subsection{所有合成語(有財釋 bahuvrīhi)}
\numberParagraph
これは最後分が名詞又は名詞の義に用ひられた形容詞で
あつて,全體が一つの形容詞として用ひられたる決定合成語であ
る。この合成語は「……を持てる」と云ふ意義を現はす。dīrgha-%
bāhu は「長き臂」でなくして「長き臂を持てる」の意。mauna-%
vrata は「沈默の戒を持てる」で「沈默の戒を受けたる」といふ
こと。manda-mati「劣れる智慧をもてる」卽ち「智慧劣りたる」
の意。cintā-para「思惟を最上とせる」卽ち「思惟に專一なる」。

\numberParagraph
有財は形容詞の作用を有するから,形容される名詞によ
りて性を定める。故に ā に終る最後分は男性又は中性の名詞に關
係する時は a となる。vidyā から alpha-vidya (少しく知れる),
jihvā から dvi-jihva (二舌を持てる),dhāryā から sa-bhārya
(妻を伴へる)が作られる。

\numberParagraph
hasta, pāṇi 等「手」と云ふ意味の語は最後分となつて
「……を手に持てる」の意味となる。pātra-hasta (器を手にせる),
padmapāṇi (蓮花を手にせる)等。

\subsection{副詞合成語即隣近釋(avyayībhāva)}
\numberParagraph
前分が不變詞であつて後分が名稱詞であり,而して副詞
的に用ひられた合成語である。この合成語は中,單,業の語尾を
取る。例:yatheccham (欲するまゝに),yāvaj-jīvam (生涯の
間),pratidinam (每日)。

支那の註釋家が六合釋を云々する中にこの隣近釋は全く誤解さ
れてゐるやうである。大抵,例には「長安住」と云ふのが出る。
長安に住してゐないでも近くに住んでゐるからそれを長安住と云
ふのだと云つていゐる。これでは何のことかわからぬ。これは「近
長安」(upanagaram) とでもあつた副詞的の句であつたのだら
う。これならば「都城の近くに」の意味で例として當てはまる。
隣近の意味は慈恩の義林章に「俱時の法義用增勝なるを以て自體
を彼に從へて而も其の名を立つ」とあつて,俱時の法とは文典的
に云ふならば主辭と賓辭である。これは同時に存在する二個の槪
念,俱時の法である。義は意味。用は働らき。增勝とは一層說明
を要する場合といふこと。自體を彼に從へるとは語そのものを彼
の主辭賓辭に從屬せしめてこれを設けたのであると云ふ。主辭賓
辭の意味を更に說明するもので,その場合その傍に隣近せしめる
ものである。卽ち副詞的の用法のことである。慈恩ではよくわか
つてゐたものが,後代の人がわからぬために何のことかわからぬ
儘に傳へられて來たものらしい。

\subsection{動詞合成法}
\numberParagraph
動詞は前接字又は副詞と合成せられその語根の意義は少
しく變化する。加へらるゝ前接字は

\begin{center}
\begin{tabular}{*{3}{p{0.3\hsize}}}
  ati (超えて)  & ava (下に) & parā (彼方に) \\
  adhi (上に)   & ā (まで)   & pari (回りに) \\
  anu (隨つて)  & ud (上に)  & pra (前に) \\
  antar (間に)  & upa (近く) & prati (反對に) \\
  apa (外に)    & ni (下に)  & vi (別に) \\
  abhi (對して) & nis (外に) & sam (共に)
\end{tabular}
\end{center}

副詞の例:

alam (十分に) + kṛ (作る) = alaṃkṛ (飾る)。

astam (下に) + gam (行く) = astaṃgam (沈む)。

āvis (明に) + bhū (ある) = āvirbhū (現はる)。

\numberParagraph
語根 as (有る),bhū (なる),kṛ (作す)を名詞の後に
加へて「……がある」,「……となる」,「……と作す」の意とする。
a 語基は ī に變じ,i, u はこれを延長し,ṛ は rī に變ず。śukla
(白き)は śuklīkṛ (白くす),śuci (淨き)は śucībhū (淨くなる),
mṛdu (軟らかき)は mṛdū-syāt [as の三單可能法] (軟らかくあるべ
し),mātṛ (母)は mātrīkṛ (母となす),又 bhasman (灰)は
bhasmīkṛ (灰と化す)に作る。

\ex{第九}
\begin{longtable}{c*{2}{p{0.45\hsize}}}
 1. & puruṣā api baṇā api guṇa\-cyutāḥ kasya na bhayāya. & グナ(德,弦)を離れた人も
箭も誰に取つてか恐怖を起さざらむ。\\
 2. & gajo gharmārtaś chāyārthī tamāla-vṛkṣaṃ samāśritaḥ. & 熱に苦められ蔭を欲する象
はタマーラ樹に椅れり。\\
 3. & Mumbā-pura-nivāso mamāro\-gyāya na kalpate. & ボンベイ市の滯留は私の健
康に適せず。\\
 4. & brāhmaṇā vidyopārjanārthaṃ Kanyakubjaṃ gatāḥ. & 婆羅門等は知識獲得のため
にカニヤクブジヤへ行けり。\\
 5. & jalada-ninada-muditāḥ śikhino nṛtyanti. & 雲の響を喜んで孔雀は舞ふ。\\
 6. & bālā kāntam indīvara-dala-prabhā-caura-cakṣuḥ kṣipati. & 少女は愛人に靑蓮の瓣
の光を盗む眼ざしを投げたり。\\
 7. & asāraḥ saṃsāro 'yam, giri\-nadī-vegopamaṃ yauvanaṃ, tṛṇāgni-samaṃ, śarad\-%
abhra-cchātā-sadṛśā bhogāḥ, svapna-sadṛśo mitra-putra-kala-tra-bhṛtya-varga-saṃyogaḥ:
evaṃ mayā samyak parijñātam. & この世界は空虛である。靑春は山より落つる\ruby{駃}{はや}き流
の如く,生命は枯草を火に投ずる如く,享樂は秋雲の影に似,友人,子,妻,奴僕の群の相會する
ことは夢の如し。かくわれによりて正しく認識せられたり。\\
 8. & arthaḥ puruṣāṇāṃ ṣaḍbhir upāyair bhavati bhikṣayā nṛpa\-sevayā kṛṣi-karmaṇā vidyopārja\-nayā
vyavahāreṇa vaṇik-karma\-ṇā vā. & 富は人にとりては六種の方法によりて存在す。乞食によりて,王に對する奉
仕によりて農耕の業によりて,知識の獲得によりて金融業又は商業によりて。\\
 9. & brāhmaṇī paruṣatara-vacanaiḥ patiṃ bhartsayamānābhāṣata bho na mayā tava hasta-lagna-yā
kvāpi labdhaṃ sukhaṃ na miṣṭānnasyādanaṃ na hasta-pāda-kaṇṭha-bhūṣaṇam. & 婆羅門の妻は一層\ruby{麤}{あ}らき言
葉で夫を叱りつゝ云へり おゝ汝の手に椅りすがる私は何處にも幸福を得ず。舐める食物のうまさ
もなし。手足頸の裝飾もなし。\\
10. & eko doṣo guṇa saṃnipāte nimajjatīndoḥ kiraṇeṣv ivāṅkaḥ. & 德の積聚に於ける一の過失
は月の光線の中に於ける斑點の如く影を消す。\\
11. & bho, svalpa-kāyo bhavān & おゝ汝は小なる身體をもてることよ。\\
12. & kaulikaḥ kṛta-maraṇa-niścayaś cāpa-pāṇir garuḍārūḍho yuddhā\-ya prasthitaḥ. & 織匠は死の決心をなして弓
を手にしガルダに駕して戰鬪のために出發せり。\\
13. & brāhmaṇena mitrasyātīva-pīvara-tanuḥ paśuḥ pradattaḥ. & 一人の婆羅門によりて友に
極めて身體肥へたる供犧の家畜が與へられたり。\\
14. & śubhe 'hani prahṛṣṭa-manāḥ kumāro mitreṇa saha gurujanā-nujñāto deśāntaraṃ gataḥ. & 美はしき日に心悅べる王子
は友と共に長上の命によりて國外に行けり。\\
15. & kṣapaṇakāḥ śrāvakasya gṛhe gacchanti prāṇa-dharaṇa-mā\-trāṃ cāśana-kriyāṃ kurvanti. & 乞食僧等は俗人の家に行け
り,而して生命の保持だけの食事をなせり。\\
16. & te dhanyā ye vīta-rāgā guru-vacana-ratās tyakta-saṃsāra-sangā veda-jñāne vilīnā vane
yauvanaṃ nayante. & 貪慾を離れ,師の語を樂み,輪廻の執着を捨て吠陀の智に沈潛し,森林に靑春
を過す彼等は幸福なり。\\
17. & varaṃ prāṇa-parityāgo na māna-parikhaṇḍanaṃ prāṇa-tyāgaḥ kṣaṇaṃ caiva māna-bhaṅgo dine dine & 矜持を失はんよりは寧ろ生
命を捨てむこと勝る。生命の捨は瞬間なり。矜持の破壞は日日のことなり。
\end{longtable}

%%% Local Variables:
%%% mode: latex
%%% TeX-master: "IntroductionToSanskrit"
%%% End:

\section{活用法(續き)}
\subsection{第二種變化}
\numberParagraph \label{np:136}
第二種變化の現在語基は强弱の二語基を有し,强語基は
次の如く用ひられる。
\begin{enumerate}[label=(\alph*)]
\item 現實法爲他單數の第一二三人稱
\item 命令法爲他の第一人稱全部並に第三人稱單數。
\end{enumerate}

\numberParagraph
第二種變化の現在組織の語尾は第一種變化のそれと稍異
る。その特徴次の如し。
\begin{enumerate}[label=(\alph*)]
\item \textbf{可能法の語尾。}

爲他 yām, yās, yāt; yāva, yātam, yātām; yāma,
yāta, yus.

爲自 īya, īthās, īta; īvahī, īyāthām, īyātām; īmahi,
īdhvam, īran.
\item \textbf{命令法の二人稱單數} 爲他に子音の後に dhi, 母音の後
に hi を加ふ。但し若干の例外がある。
\item 命令法第三人稱複數,爲他が重複級の動詞並に語根級の
あるものには an の代りに us である。又語根級の ā に
終る語根並に dvis は us 又は an の孰れにてもよい。
\item 三人稱複數爲自は ante, anta, antām の代りに ate,
ata, atām となる。重複級並に或る語根級の動詞は亦現實
法爲他 anti の代りに ati, 命令法,爲他 antu の代りに
atu の形を取る。
\end{enumerate}

\subsubsection{第二類 語根級}
\numberParagraph \label{np:138}
語根級の構造。語根のまゝに直接語尾を附加する。强形
に於て語根の母音は重韻化す。第一過去 三,複の us の前に語根
の終りの ā は消失する。例:語根 yā ~ ayus (彼等は行けり)。

\numberParagraph
語根 dviṣ (憎む)强基 dveṣ, 弱基 dviṣ.

\begin{center}
\begin{tabular}{c*{3}{p{0.23\hsize}}}
  \multicolumn{4}{c}{\textbf{現實法}} \\
  \multicolumn{4}{c}{爲他} \\
     & 單                                & 兩       & 複 \\
  1. & dveṣmi                            & dviṣvas  & dviṣmas \\
  2. & dvekṣi (\ref{np:17}, \ref{np:39}) & dviṣṭhas & dviṣṭha \\
  3. & dveṣṭi (\ref{np:36})              & dviṣṭas  & dviṣanti \\
  \multicolumn{4}{c}{爲自} \\
  1. & dviṣe  & dviṣvahe & dyiṣmahe \\
  2. & dvikṣe & dviṣāthe & dviḍḍhve (\ref{np:17}, \ref{np:36}) \\
  3. & dviṣṭe & dviṣāte  & dviṣate
\end{tabular}
\end{center}
\begin{center}
\begin{tabular}{c*{3}{p{0.23\hsize}}}
  \multicolumn{4}{c}{\textbf{第一過去}} \\
  \multicolumn{4}{c}{爲他} \\
     & 單                  & 兩       & 複 \\
  1. & adveṣam             & adviṣva  & adviṣma \\
  2. & adveṭ (\ref{np:17}) & adviṣṭam & adviṣṭa \\
  3. & adveṭ (\ref{np:17}) & adviṣṭām & adviṣan 又は adviṣus \\
  \multicolumn{4}{c}{爲自} \\
  1. & adviṣi    & adviṣvahi  & adviṣmahi \\
  2. & adviṣṭhās & adviṣāthām & adviḍḍhvam \\
  3. & adviṣṭa   & adviṣātām  & adviṣata
\end{tabular}
\end{center}
\begin{center}
\begin{tabular}{c*{3}{p{0.23\hsize}}}
  \multicolumn{4}{c}{\textbf{可能法}} \\
  \multicolumn{4}{c}{爲他} \\
     & 單      & 兩        & 複 \\
  1. & dviṣyām & dviṣyāva  & dviṣyāma \\
  2. & dviṣyās & dviṣyātam & dviṣyāta \\
  3. & dviṣyāt & dviṣyātām & dviṣyus \\
  \multicolumn{4}{c}{爲自} \\
  1. & dviṣīya   & dviṣīvahi   & dviṣīmahi \\
  2. & dviṣīthās & dviṣīyāthām & dviṣīdhvam \\
  3. & dviṣīta   & dviṣīyātām  & dviṣīran
\end{tabular}
\end{center}
\begin{center}
\begin{tabular}{c*{3}{p{0.23\hsize}}}
  \multicolumn{4}{c}{\textbf{命令法}} \\
  \multicolumn{4}{c}{爲他} \\
     & 單      & 兩      & 複 \\
  1. & dveṣāni & dveṣāva & dveṣāma \\
  2. & dviḍḍḥi & dviṣṭam & dviṣṭa \\
  3. & dveṣṭu  & dviṣṭām & dviṣantu \\
  \multicolumn{4}{c}{爲自} \\
  1. & dveṣai  & dveṣāvahai & dveṣāmahai \\
  2. & dvikṣva & dviṣāthām  & dviḍḍhvam \\
  3. & dviṣṭām & dviṣātām   & dviṣatām
\end{tabular}
\end{center}

\begin{center}\textbf{≪若干の不規則なる語根≫}\end{center}

\numberParagraph
語根 rud (泣く),svap (眠る),an (呼吸す),śvas (溜
息する),jakṣ (食ふ)に加へられる語尾が子音又は y 以外の半母
音で始まる場合,語基の尾を i に作り,二並に三の單,第一過,爲
他の時はそれを a 又は ī に作る。rodimi, rodiṣi, roditi; rudivas;
rudanti; 第一過 arodam, arodīs 又は arodas, arodīt 又は
arodat; arudiva; arudan; 可能法 rudyām 等。命令法 rodāni,
rudihi, roditu 等。

\numberParagraph
語根 as (あり)はその弱語基に於て a を失ふ。第一過去
は次の如くである。
\begin{center}
\begin{tabular}{c*{6}{p{0.15\hsize}}}
     & \multicolumn{3}{c}{\textbf{現實法}} & \multicolumn{3}{c}{\textbf{第一過去}} \\
     & 單    & 兩    & 複    & 單   & 兩    & 複 \\
  1. & asini & svas  & smas  & āsam & āsva  & āsma \\
  2. & asi   & sthas & stha  & āsīs & āstam & āsta \\
  3. & asti  & stas  & santi & asīt & āstām & āsan \\
     & \multicolumn{3}{c}{\textbf{命令法}} & \multicolumn{3}{c}{\textbf{可能法}} \\
  1. & asāni & asāva & asāma & syām & syāva  & syāma \\
  2. & edhi  & stam  & sta   & syās & syātam & syāta \\
  3. & astu  & stām  & santu & syāt & syātām & syus \\
\end{tabular}
\end{center}

\numberParagraph
語基 brū (言ふ)は强形の時子音に始まる語尾の前に
bravī に作る。現實法,現,爲他 bravīmi, bravīṣi, bravīti;
brūvas; bruvanti; 第一過去 abravam, avravīs, abravīt;
abrūva; abruvan; 可能法 brūyām 等。命令法 bravāni, brūhi,
bravītu 等。爲自は bruve, brūṣe, brūte; 三複 bruvate.

\numberParagraph
u にて終る語根は强形の時子音にて始まる語尾の前には
重韻にあらずして複重韻となる。stu (讚む)の現實,現,爲他は
staumi, stauṣi, stauti; 命令 stavāni, stuhi, stautu; 第一過
astavam, astaus, astaut; 三複 astuvan. 尙ほ stu は强語形
を stavī に作ることもあるからその時は三單現爲他は stavīti で
ある。

\numberParagraph
語根 han (殺す)の弱語形は m, y, v を除き子音に始ま
る語尾の前にその n を消失し,母音に始まる語尾の前には a を
除去して h は gh となる。現 hanmi, haṃsi, hanti; hanvas,
hathas, hatas; hanmas, hatha, ghnanti; 第一過去 ahanam,
ahan, ahan; ahanva, ahatām; ahanma, ahata,
aghnan. 二單命令爲他は jahi.

\subsubsection{第三類 重複級}
\numberParagraph \label{np:145}
第三類の現在組織は語根を重複して作る。强形に於て語
根の母音,第一過去の三複爲他の語尾 us の前にも終りの母音は
重韻化し,ā に終る語根はこの us の前にその ā を消失する。

\numberParagraph \label{np:146} \textbf{重複に關する原則。}
\begin{enumerate}[label=(\alph*)]
\item 重複とは語根の一部分通例は第一の子音とその次の母
音とが語根の前に置かれることである。tud (打つ)~
tutud, budh (知る)~ bubudh.
\item 含氣音はそれに相當する無氣音にて重複する。dhā (置
く)~ dadhā, bhuj (受用す)bubhuj.
\item 喉音は同種類の顎音にて,h は j にて重複せられる。
kṛ (作る)~cakṛ, gam (行く)~ jagam, hu (供ふ)~juhu,
khan (掘る)~ cakhan.
\item 連續せる子音が語根の始にある時はその第一のものを
重複す。śru (聞く)~śuśru, kram (歩む)~ cakram. 連續
せる子音の始が硬含氣音なる時は後の音又はそれに代るべ
き音で重複を行う。stu (讚む)~ tuṣtu, spṛś (觸る)~ pas\-%
parś, sthā (立つ)~ tiṣṭha.
\item 語根の母音は重複音中に現はるべきであるが若し長母
音ならば短母音が重複に用らる。dhā (置く)~ dadhā. 又
ṛ には i を用ふ。bhṛ (擔ふ)~ bibhṛ.
\end{enumerate}

\numberParagraph
語根 hu (供ふ),强形 juho, 弱形 juhu.
\begin{center}
\begin{tabular}{c*{3}{p{0.23\hsize}}}
  \multicolumn{4}{c}{\textbf{現實法}} \\
  \multicolumn{4}{c}{爲他} \\
     & 單     & 兩       & 複 \\
  1. & juhomi & juhuvas  & juhumas \\
  2. & juhoṣi & juhuthas & juhutha \\
  3. & juhoti & juhutas  & juhvati \\
  \multicolumn{4}{c}{爲自} \\
  1. & juhve  & juhuvahe & juhumahe \\
  2. & juhuṣe & juhvāthe & juhudhve \\
  3. & juhute & juhvāte  & juhvate
\end{tabular}
\end{center}
\begin{center}
\begin{tabular}{c*{3}{p{0.23\hsize}}}
  \multicolumn{4}{c}{\textbf{第一過去}} \\
  \multicolumn{4}{c}{爲他} \\
     & 單       & 兩       & 複 \\
  1. & ajuhavam & ajuhuva  & ajuhuma \\
  2. & ajuhos   & ajuhutam & ajuhuta \\
  3. & ajuhot   & ajuhutām & ajuhavus \\
  \multicolumn{4}{c}{爲自} \\
  1. & ajuhvi    & ajuhuvahi  & ajuhumahi \\
  2. & ajuhuthās & ajuhvāthām & ajuhudhvam \\
  3. & ajuhuta   & ajuhvātām  & ajuhvata
\end{tabular}
\end{center}
\begin{center}
\begin{tabular}{c*{3}{p{0.23\hsize}}}
  \multicolumn{4}{c}{\textbf{命令法}} \\
  \multicolumn{4}{c}{爲他} \\
     & 單       & 兩       & 複 \\
  1. & juhavāni & juhavāva & juhavāma \\
  2. & juhudhi  & juhutam  & juhuta \\
  3. & juhotu   & juhutām  & juhvatu \\
  \multicolumn{4}{c}{爲自} \\
  1. & juhavai & juhavāvahai & juhavāmahai \\
  2. & juhuṣva & juhvāthām   & juhudhvam \\
  3. & juhutām & juhvātām    & juhvatām
\end{tabular}
\end{center}
\begin{center}
\begin{tabular}{c*{3}{p{0.23\hsize}}}
  \multicolumn{4}{c}{\textbf{可能法}} \\
     & \multicolumn{3}{l}{爲他:juhuyām 等。 爲自:juhvīya 等。}
\end{tabular}
\end{center}

\begin{center}\textbf{≪重複級の若干の不規則なる語根≫}\end{center}

\numberParagraph
語根 dā (與ふ)並に dhā (置く)は序の如く弱語形を
dad 及び dadh に作る。而してこの dadh は t, th に始まる語
尾と共に dhatt, dhatth となる(\ref{np:35}條)。二單,命,爲他は dehi,
dhehi である。

\begin{center}
\begin{tabular}{c*{3}{p{0.23\hsize}}}
  \multicolumn{4}{c}{\textbf{dā の命令法,爲他}} \\
  1. & dadāni & dadāva & dadāma \\
  2. & dehi   & dattam & datta \\
  3. & dadātu & dattām & dadatu \\
  \multicolumn{4}{c}{\textbf{dhā の現實法,爲他}} \\
     & 單      & 兩       & 複 \\
  1. & dadhāmi & dadhvas  & dadhmas \\
  2. & dadhāsi & dhatthas & dhattha \\
  3. & dadhāti & dhattas  & dadhati
\end{tabular}
\end{center}

\numberParagraph
ā に終る或る語根は重複に母音 ī を以てす。弱語基は子
音語尾の前に ī となり母音語尾の前に消失す。mā (量る)三,單,
現,爲自 mimīte, 三,複 mimate. 又語根 hā 爲他(去る)の弱
語基は子音語尾の前に jahi 若くは jahī となる。母音語尾の前
若くは可能法の場合には jah となる。兩,現 jahivas 又は jahī\-%
vas, 二,單,命令 jahāhi, jahīhī 又は jahihī.

\subsubsection{第五類 nu 級}
\numberParagraph
第五類は語根に no を加へて强語基,nu を加へ弱語基を
作る。su (搾る)の强基 suno, 弱基 sunu. 母音に終る語根は v
又は m にて始まる語尾の前にその u を省くことがある。而して
二,單,命,他は語尾を附けない。sunuvas 又は sunvas; sunu\-%
mahe 又は sunmahe; 命 sunu. 子音に終る語根は u を省き得
ないし,又三複の語尾 anti の前にその u は uv に變る。而して
二,單,命,他は hi を附ける。āp (得る)は āpnuvas, āpnumas;
āpnuvanti; 命 āpnuhi.

\begin{center}
\begin{tabular}{cp{0.2\hsize}p{0.25\hsize}p{0.25\hsize}}
  \multicolumn{4}{c}{\textbf{現實法}} \\
  \multicolumn{4}{c}{爲他} \\
     & 單     & 兩               & 複 \\
  1. & sunomi & sunuvas (sunvas) & sunumas (sunmas) \\
  2. & suuoṣi & sunuthas         & suntha \\
  3. & sunoti & sunutas          & sunvanti \\
  \multicolumn{4}{c}{爲自} \\
  1. & sunve  & sunuvahe (sunvahe) & sunumahe (sunmahe) \\
  2. & sunuṣe & sunvāthe           & sunudhve \\
  3. & sunute & sunvāte            & sunvate
\end{tabular}
\end{center}
\begin{center}
\begin{tabular}{cp{0.18\hsize}p{0.26\hsize}p{0.26\hsize}}
  \multicolumn{4}{c}{\textbf{第一過去}} \\
  \multicolumn{4}{c}{爲他} \\
     & 單       & 兩                & 複 \\
  1. & asunavam & asunuva (asunva)  & asunuma (asunma) \\
  2. & asunos   & asunutam          & asunuta \\
  3. & asunot   & asunutām          & asunvan \\
  \multicolumn{4}{c}{爲自} \\
  1. & asunvi    & asunuvahi (asunvahi) & asunumahi (asunmahi) \\
  2. & asunuthās & asunvāthām           & asunudhvam \\
  3. & asunuta   & asunvātām            & asunvata
\end{tabular}
\end{center}
\begin{center}
\begin{tabular}{c*{3}{p{0.23\hsize}}}
  \multicolumn{4}{c}{\textbf{命令法}} \\
  \multicolumn{4}{c}{爲他} \\
     & 單       & 兩       & 複 \\
  1. & sunavāni & sunavāva & sunavāma \\
  2. & sunu     & sunutam  & sunuta \\
  3. & sunotu   & sunutām  & sunvantu \\
  \multicolumn{4}{c}{爲自} \\
  1. & sunavai & sunavāvahai & sunavāmahai \\
  2. & sunuṣva & sunvāthām   & sunudhvam \\
  3. & sunutām & sunvātām    & sunvatām
\end{tabular}
\end{center}
\begin{center}
\begin{tabular}{c*{3}{p{0.23\hsize}}}
  \multicolumn{4}{c}{\textbf{可能法}} \\
     & \multicolumn{3}{l}{爲他:sunuyām 等。 爲自:sunvīya 等。}
\end{tabular}
\end{center}

\numberParagraph
śru (聞く)の現在語基は śṛṇu, 隨つて śṛṇo. 現,爲他
śṛṇomi, śṛṇosi, śṛṇoti; śṛṇuvas 又は śṛṇvas, śṛṇuthas,
śṛṇutas; śṛṇumas 又は śṛṇmas, śṛṇutha, śṛṇvanti.

\subsubsection{第七類 鼻音級}
\numberParagraph
第七類鼻音級。强語基は na を挿入,弱語基はその終り
の子音に同類の鼻音を挿入して作る。bhid (破る),强語基
bhinad, 弱語基 bhind.

\begin{center}
\begin{tabular}{c*{3}{p{0.23\hsize}}}
  \multicolumn{4}{c}{\textbf{現實法}} \\
  \multicolumn{4}{c}{爲他} \\
     & 單       & 兩        & 複 \\
  1. & bhinadmi & bhindvas  & bhindmas \\
  2. & bhinatsi & bhintthas & bhinttha \\
  3. & bhinatti & bhinttas  & bhindanti \\
  \multicolumn{4}{c}{爲自} \\
  1. & bhinde  & bhindvahe & bhindmahe \\
  2. & bhintse & bhindāthe & bhinddhve \\
  3. & bhintte & bhindāte  & bhindate
\end{tabular}
\end{center}
\begin{center}
\begin{tabular}{c*{3}{p{0.23\hsize}}}
  \multicolumn{4}{c}{\textbf{第一過去}} \\
  \multicolumn{4}{c}{爲他} \\
     & 單        & 兩        & 複 \\
  1. & abhinadam & abhindva  & abhindma \\
  2. & abhinat   & abhinttam & abhintta \\
  3. & abhinat   & abhinttām & abhindan \\
  \multicolumn{4}{c}{爲自} \\
  1. & abhindi    & abhindvahi  & abhindmahi \\
  2. & abhintthās & abhindāthām & abhinddhvam \\
  3. & abhintta   & abhindātām  & abhindata
\end{tabular}
\end{center}
\begin{center}
\begin{tabular}{c*{3}{p{0.23\hsize}}}
  \multicolumn{4}{c}{\textbf{命令法}} \\
  \multicolumn{4}{c}{爲他} \\
     & 單        & 兩        & 複 \\
  1. & bhinadāni & bhinadāva & bhinadāma \\
  2. & bhinddhi  & bhinttam  & bhintta \\
  3. & bhinattu  & bhinttām  & bhindantu \\
  \multicolumn{4}{c}{爲自} \\
  1. & bhinadai & bhinadāvahai & bhinadāmahai \\
  2. & bhintsva & bhindāthām   & bhinddhvam \\
  3. & bhinttām & bhindātām    & bhindatām
\end{tabular}
\end{center}
\begin{center}
\begin{tabular}{c*{3}{p{0.23\hsize}}}
  \multicolumn{4}{c}{\textbf{可能法}} \\
     & \multicolumn{3}{l}{爲他:bhindyām 等。 爲自:bhindīya 等。}
\end{tabular}
\end{center}

\numberParagraph
是の如く yuj (結合す)~ yunajmi, yunakṣi, yunakti;
yuñjumas, yuṅktha, yuñjanti. piṣ (碎く)~ pinaṣmi, pinakṣi,
pinaṣṭi; piṃṣmas, piṃṣtha, piṃṣanti. rundh (止める)~
runadhmi, runatsi, runaddhi; rundhmas, runddha, rundhanti.

\subsubsection{第八類 u 級}
\numberParagraph
第八類 u 級。强語基は語根に o を加へ,弱語基は u を
加ふ。その u は m 又は v にて始まる語尾の前には省くことが
ある。その部類に屬する語根は kṛ (作る)の例外は別として皆 n
に終るから,結局は第五類の su と同樣となる。例せば tan (擴
ぐ)~强基 tano, 弱基 tanu. 尙ほその數も極めて少い。但し kṛ
は例外とは云へ,用ひらるゝこと極めて多いから次に表示する。
kṛ の强基は重韻の形を取り kar + o = karo であり,弱基は
kur となり u を加へて kuru に作る。この u はm, v を以て始
まる語尾の前には省かれねばならぬ。

\begin{center}
\begin{tabular}{c*{3}{p{0.23\hsize}}}
  \multicolumn{4}{c}{\textbf{現實法}} \\
  \multicolumn{4}{c}{爲他} \\
     & 單     & 兩       & 複 \\
  1. & karomi & kurvas   & kurmas \\
  2. & karoṣi & kuruthas & kurutha \\
  3. & karoti & kurutas  & kurvanti \\
  \multicolumn{4}{c}{爲自} \\
  1. & kurve  & kurvahe  & kurmahe \\
  2. & kuruṣe & kurvāthe & kurudhve \\
  3. & kurute & kurvāte  & kurvate
\end{tabular}
\end{center}
\begin{center}
\begin{tabular}{c*{3}{p{0.23\hsize}}}
  \multicolumn{4}{c}{\textbf{第一過去}} \\
  \multicolumn{4}{c}{爲他} \\
     & 單       & 兩       & 複 \\
  1. & akaravam & akurva   & akurma \\
  2. & akaros   & akurutam & akuruta \\
  3. & akarot   & akurutām & akurvan \\
  \multicolumn{4}{c}{爲自} \\
  1. & akurvi    & akurvahi   & akurmahi \\
  2. & akuruthās & akurvāthām & akurudhvam \\
  3. & akuruta   & akurvātām  & akurvata
\end{tabular}
\end{center}
\begin{center}
\begin{tabular}{c*{3}{p{0.23\hsize}}}
  \multicolumn{4}{c}{\textbf{命令法}} \\
  \multicolumn{4}{c}{爲他} \\
     & 單       & 兩       & 複 \\
  1. & karavāni & karavāva & karavāma \\
  2. & kuru     & kurutam  & kuruta \\
  3. & karotu   & kurutām  & kurvantu \\
  \multicolumn{4}{c}{爲自} \\
  1. & karavai & karavāvahai & karavāmahai \\
  2. & kuruṣva & kurvāthām   & kurudhvam \\
  3. & kurutām & kurvātām    & kurvatām
\end{tabular}
\end{center}
\begin{center}
\begin{tabular}{c*{3}{p{0.23\hsize}}}
  \multicolumn{4}{c}{\textbf{可能法}} \\
     & \multicolumn{3}{l}{爲他:kuryām 等。 爲自:kurvīya 等。}
\end{tabular}
\end{center}

\subsubsection{第九類 nā 級}
\numberParagraph
第九類 nā 級。强基は nā を加へ,弱基は子音にて始ま
る語尾の前には nī, 母音にて始まる語尾の前には n を加へて作
る。krī (買ふ)の强基 krīṇā, 弱基 krīṇī 又は krīn.

\begin{center}
\begin{tabular}{c*{3}{p{0.23\hsize}}}
  \multicolumn{4}{c}{\textbf{現實法}} \\
  \multicolumn{4}{c}{爲他} \\
     & 單      & 兩        & 複 \\
  1. & krīṇāmi & krīṇīvas  & krīṇīmas \\
  2. & krīṇāsi & krīṇīthas & krīṇītha \\
  3. & krīṇāti & krīṇītas  & krīṇanti \\
  \multicolumn{4}{c}{爲自} \\
  1. & krīne   & krīṇīvahe & krīṇīmahe \\
  2. & krīṇīṣe & krīṇāthe  & krīṇīdhve \\
  3. & krīṇīte & krīṇāte   & krīṇate
\end{tabular}
\end{center}
\begin{center}
\begin{tabular}{c*{3}{p{0.23\hsize}}}
  \multicolumn{4}{c}{\textbf{第一過去}\endnote{底本では爲自二人称複数が ``akrīṇīdhvam'' ではなく ``akriṇīdhvam''.}} \\
  \multicolumn{4}{c}{爲他} \\
     & 單      & 兩        & 複 \\
  1. & akrīṇām & akīṇīva   & akrīṇīma \\
  2. & akrīnās & akrīṇītam & akrīṇīta \\
  3. & akrīṇāt & akrīṇītām & akrīṇan \\
  \multicolumn{4}{c}{爲自} \\
  1. & akrīṇi     & akrīṇīvahi & akrīṇīmahi \\
  2. & akrīṇīthās & akrīṇāthām & akrīṇīdhvam \\
  3. & akrīṇīta   & akrīṇātām  & akrīṇata
\end{tabular}
\end{center}
\begin{center}
\begin{tabular}{c*{3}{p{0.23\hsize}}}
  \multicolumn{4}{c}{\textbf{命令法}} \\
  \multicolumn{4}{c}{爲他} \\
     & 單       & 兩       & 複 \\
  1. & krīṇāni  & krīṇāva  & krīṇāma \\
  2. & krīṇīhi  & krīṇītam & krīṇīta \\
  3. & krīṇātu  & krīṇītām & krīṇantu \\
  \multicolumn{4}{c}{爲自} \\
  1. & krīṇai   & krīṇāvahai & krīṇāmahai \\
  2. & krīṇīṣva & krīṇāthām  & krīṇīdhvam \\
  3. & krīṇītām & krīṇātām   & krīṇatām
\end{tabular}
\end{center}
\begin{center}
\begin{tabular}{c*{3}{p{0.23\hsize}}}
  \multicolumn{4}{c}{\textbf{可能法}} \\
     & \multicolumn{3}{l}{爲他:krīṇīyām 等。 爲自:krīnīya 等。}
\end{tabular}
\end{center}

\numberParagraph
子音に終る語根は,二,單,命,爲他の語尾を āna とす
る aś (食ふ),aśāna.

\numberParagraph
この級の或る語根はその母音が弱められる。
\begin{enumerate}[label=(\alph*)]
\item ū に終る語根は母音を短縮する。dhū (振ふ)~ dhunā\-%
ti, pū (淨む)~ punāti.
\item grah (執る)は弱められて gṛh となる。
\item jñā (知る)は鼻音を失ひ jānāti に作る。bandh (縛
る),manth (攪拌す)も同樣である。
\end{enumerate}

\ex{第十}
\begin{longtable}{c*{2}{p{0.45\hsize}}}
 1. & vidyā vinayaṃ dadāti. & 知識は禮譲を賦與する。 \\
 2. & satyaṃ brūhi. & 眞實を語れ。\\
 3. & na vedmi kiṃ mayābhihitam. & 我は我によりて何が云はれしかを知らない。\\
 4. & daridrasya dānaṃ dehi. & 貧人に施物を與へよ。\\
 5. & rajako gardabhaṃ bandhane\-na niyunakti. & 洗濯人は綱を以て驢馬を繫ぐ。\\
 6. & tṛṣṇāṃ chinddhi. & 愛欲を斷て。\\
 7. & śūro raṇe mṛtaḥ svargaṃ prāpṇoti. & 戰いに於て死する勇士は天界に至る。\\
 8. & kathaṃ vetti bhavān me duḥkham. bhadra, vivarṇatā te
 vivṛṇoti śoka-vegam. & 御身は如何にして我が苦痛を知らんや。善きもの
 よ,汝の顔面蒼白は悲痛の劇甚を露はす。\\
 9. & bhadram astu te, Śivaḥ pātu tvām. & 汝の上に幸福あれ。シヷが汝を護れかし。\\
10. & yāty adho vrajaty uccair naraḥ svair eva karmabhiḥ. & 人は自の業によりてのみ下
 に行き上に行く。\\
11. & guṇavaj-jana-saṃsargād yāti svalpo 'pi gauravam. & 德ある人々と交ることによ
 り人は劣等なりとも尊敬せらる。\\
12. & āsīd rājā Nalo nāma Vīrasena\-suto balī. & ヴィーラセーナの子なる力
 强きナラと名けらるゝ王ありき。\\
13. & nīco vadati nakurute, vadati na sādhuḥ karoty eva. & 小人は語りて作さず,善人
 は語らずして作す。\\
14. & niṣiddhas tvaṃ mayānekaśo naśṛṇoṣi me vākyam. & 汝は屢々我によりて警吿さ
 れたるに我が語を聽かず。\\
15. & kiṃ rodiṣi gadgada-vācā. & 汝はとぎれとぎれの語を以て何を泣くや。\\
16. & rājā pīḍitānām anāthānāṃ ca kuryād aśru-pramārjanam. & 王は苦しめられたるもの,
 保護なきものの淚を拭ふべきである。\\
17. & asty uttarasyāṃ diśi devātmā Himālayo nāma nagādhirājaḥ. & 北方に於てヒマーラヤと名
 けられたる神々しき山王ありき。\\
18. & ratna-mālā kutra labdhā yā dīptā sūryam api tiraskaroti.& 何處で太陽をも凌駕するそ
 の輝いた眞珠鬘は得られたか。\\
19. & udeti savitā tamaś cāstameti. & 太陽は昇り而して暗黑は消えたり。\\
20.& padma-pattra-sthitaṃ vāri dhatte muktā-phala-śriyam. & 蓮葉にある水は眞珠の美しさを呈す。\\
21. & yaḥ prasannena manasā bhā\-ṣate karoti vā taṃ sukham an\-veti cchāyeva. & 淨き心もて語り若くは行ふ
 彼に幸福は影の如く隨ひ行く。\\
22.& yaḥ pradoṣeṇa manasā bhā\-ṣate karoti vā taṃ duḥkham anveti cakraṃ yathā vahanam. & 惡しき心もて語り若くは行
 ふ彼に苦は隨ひ行く,恰も車輪が車に隨ふ如し。\\
23. & anāgataṃ yaḥ kurute sa śo\-bhate; sa śocate yo na karoty anāgatam. & 未來を爲す彼は輝く。未來
をなさぬ彼は悲しむ。\\
    & \multicolumn{2}{l}{(śocate と śobhate とは相似たる音なるを注意すべし)。}
\end{longtable}


%%% Local Variables:
%%% mode: latex
%%% TeX-master: "IntroductionToSanskrit"
%%% End:

\section{受動調並に派生動詞}
\subsection{受動調動詞}
\numberParagraph
現在動詞(\ref{np:60}條以下)の受動は語根に ya を附加して作
る。人稱語尾は爲自を用ふ。語根 kṛ (作る)~ krīya.

\begin{center}
\begin{tabular}{c*{3}{p{0.23\hsize}}}
  \multicolumn{4}{c}{\textbf{現實法}} \\
     & 單      & 兩        & 複 \\
  1. & kriye   & kriyāvahe & kriyāmahe \\
  2. & kriyase & kriyethe  & kriyadhve \\
  3. & kriyate & kriyete   & kriyante \\
  \multicolumn{4}{c}{\textbf{第一過去}} \\
  1. & akriye     & akriyāvahi & akriyāmahi \\
  2. & akriyathās & akriyethām & akriyadhvam \\
  3. & akriyata   & akriyetām  & akriyanta
\end{tabular}
\end{center}
\begin{center}
\begin{tabular}{c*{3}{p{0.23\hsize}}}
  \multicolumn{4}{c}{\textbf{命令法}} \\
     & 單       & 兩         & 複 \\
  1. & kriyai   & kriyāvahai & kriyāmahai \\
  2. & kriyasva & kriyethām  & kriyadhvam \\
  3. & kriyatām & kriyetām   & kriyantām \\
  \multicolumn{4}{c}{\textbf{可能法}} \\
  1. & kriyeya   & kriyevahi   & kriyemahi \\
  2. & kriyethās & kriyeyāthām & kriyedhvam \\
  3. & kriyeta   & kriyeyātām  & kriyeran
\end{tabular}
\end{center}

\numberParagraph \label{np:159}
受動は弱語基から作る。尙ほ次の變化が行はれる。
\begin{enumerate}[label=(\alph*), ref=\alph*]
\item 子音に先立つ語根中の鼻音は通例消滅する。bhañj (破
る)~ bhajyate, daṃś (嚙む)~ daśyate, bandh (縛る)~
badhyate. されど nind (責む)は nindyate.
\item \label{item:159b} 語根の始にある va 又は ya は序の如く u 又は i と
なる。vac (語る)~ ucyate, yaj (祀る)~ ijyate, vyadh
(貫く)~ vidhyate.
\item grah (執る)~ gṛhyate, prach (問ふ)~ pṛcchyate.
\end{enumerate}

\numberParagraph
語根の母音は受動詞 ya の前に次の變化をなす。
\begin{enumerate}[label=(\alph*)]
\item i 若くは u は延長せられる。ci (集む)~ cīyate, stu (讚
む)~ stūyate.
\item ā は通常 ī となる。dā (與ふ)~ dīyate, hā (捨つ)~
hīyate, mā (量る)~ mīyate, dhā (置く)~ dhīyate. 但
し jñā (知る)は jñāyate.
\item 子音に隨ふ ṛ は ri 又は ar となる。kṛ (作る)~
kriyate, smṛ (念ず)~ smaryate.
\item ṝ は īr となる。唇音の後に來るものは ūr となる。kṝ
(散布す)~ kīryate, pṝ (滿す)~ pūryate.
\item gai (歌ふ)~ gīyate, dhyai (沈思する)~ dhyāyate, hve
(喚ぶ)~ hūyate.
\end{enumerate}

\numberParagraph
受動調は以上の如き變化によりて能動調と區別せら
れる。然し第四類の或る動詞の現在爲自の形とは只揚音によりての
み區別せらる。nahyate (彼は縛す),nahyáte (彼は縛らる)。

\subsection{催起動詞}
\numberParagraph
催起動詞の語基は aya 級の動詞(\ref{np:73}條)のやうに作られ
變化する。語根 vid (知る)~ vedayati (彼は吿ぐ),bhū (ある)~
bhāvayati (彼は現はす),pat (落つ)~ pātayati (彼は落とす)。

\numberParagraph
ā にて終る語根は大抵 aya の前に p を挿入す。dā (與
ふ)~ dāpaya, sthā (立つ)~ sthāpaya, jñā (知る)には jñāpaya,
jñapaya の兩形がある。

\numberParagraph
語根の中間にある a は屢々これを延長せず,そのまゝに
することがある。gam (行く)~ gamaya, jan (生る)~ janaya.

\numberParagraph
若干の不規則なる語根:ṛ (行く)~ arpayati, i (行く)+
adhi ~ adhyāpayati, sidh (成就す)~ sādhayati, ruh (生長す)~
ropayati.

\numberParagraph
催起動詞の受動調を作るには語基から aya を除去して而
る後受動の ya を加へる。kṛ (作る)催起 kāraya $-$ aya = kār +
ya + te = kāryate.

\subsection{重複動詞}
\numberParagraph
動作の重複隨て强勢を現はす動詞の一種で,これは語根
の重複(\ref{np:146}條)によりて語基を作るのだが重複の綴は母音が强
められ又は延長せられる。

a は ā となり i, ī は e となり u, ū は o となる。語尾変化
に關しては第二種變化に準ず。語根 vid (知る)~ vevid, 三單現
爲他 vevetti

\numberParagraph
通例重複動詞の語基には ya を附ける。而して爲自の變
化をなす。bhid (破る)~ bebhidyate; dhū (搖る)~ dodhūyate.

\subsection{求欲動詞}
\numberParagraph
求欲動詞は重複せる語根に直接又は i を介して sa を附
加して語基を作る。かくて作られたる語基は第六類に準じて變化
する。尙 i を介するものはその語根の母音が通例重韻となる。重
複の母音は i であるが若し語根に u があれば u を用ふ。pac
(煮る)~ pipakṣa (煮んと欲す),kṣip (投ぐ)~ cikṣipsa (投げん
と欲す),tud (打つ)~ tututsa (打たんと欲す),vid (知る)~
vividiṣa 又は vivitsa (知らんと欲す),duh (乳を搾る)\endnote{底本では「乳」は「{\HanazonoA 乳󠄃}」。}~
dudhukṣa (乳を搾らんと欲す)。

\numberParagraph
聲の終の i, u は延長せられる。ṛ と ṝ は īr となり唇音
の次後には ūr となる。ji (勝つ)~ jigīṣa, śru (聞く)~ śuśrūṣa,
kṛ (作す)~ cikīrṣa, mṛ (死す)~ mumūrṣa.

\numberParagraph
或る語根は重複語根が收縮される。āp (得る)~ īpsati
(彼は得んと欲す),dā (與ふ)~ ditsati (彼は與へんと欲す),
labh (得る)~ lipsati (彼は得んと欲す)。

\ex{第十一}
\begin{longtable}{c*{2}{p{0.45\hsize}}}
 1. & gauḥ saṃdhyāyāṃ duhyate. & 牝牛は黎明に搾られる。\\
 2. & svadeśe pūjyate rājā, vidvān sarvatra pūjyate. & 王は自らの國土に於て尊敬
 せらる。賢者は一切の處に尊敬せらる。 \\
 3. & kṣudra-śṛgālo 'yaṃ, tad va\-dhyatām. & これは小さき\ruby{豺}{やまいぬ}である。そは殺さるべし。\\
 4. & pater me vārttā na śrūyate. & 吾にまで主の消息は聞かれず。\\
 5. & Bharata-kṣetre Campā nāma mahā-purī vidyate. & ブラハタの國にチャムパーと名くる大なる市あり。\\
 6. & rājñaḥ samasta-deśaḥ śatru\-bhir vyajītata. & 王の全土は敵によりて征服せられた。\\
 7. & Gaṅgāyā darśanāt sarva\-pāpaiḥ pramucyate naraḥ. & 恒河を一見すれば人は一切の罪障より解脱す。\\
 8. & dīyatāṃ me kiṃcid bhojanam. & 幾許かの食は我に與へらるべし。\\
 9. & mitra, kṣamyatāṃ mayā te 'parādhaḥ kṛtaḥ. & 友よ我によりて爲された罪過は汝にまで忍ばるべし。\\
10. & ādhi-vyādhi-śatair janasyāro\-gyam unmūlyate. & 百の悲哀と病患によりて人の健康は根絕される。\\
11. & āpatsu na saṃpatsu mahatāṃ saktir abhivyajate (añj + abhi + vi). & 幸福に於てゞなく不幸に於
て大人物の能力は示される。\\
12. & bho, madīyam etad gṛhaṃ tac chīghraṃ niṣkramyatām. & おゝこれ我が家なり速かに
それから出らるべきなり(受動)。\\
13. & bindūnāṃ nipātanena krama\-śaḥ pūryate ghaṭaḥ. & 水滴の落下によりて次第に甕は滿たさる。\\
14. & pūrve janmani kṛtaṃ karma daivaṃ kathyate. & 前生に於て作られし業は運命と云はれる。\\
15. & darśaya me sthānaṃ lubdha\-kāgamyam. & 我に獵師の來らざる場處を示せ。\\
16. & vanig bhṝtyaṃ kaṭaṃ kāra\-yati. & 商人は僕をして蓆を敷かしめる。\\
17. & śiṣtaṃ putraṃ ca tāḍayen na tu lālayet. & 弟子と子とを人は叱責すべきだ。決して甘やかすべきでない。\\
18. & purohitaḥ pratyūṣe devatā\-yatane saṃmārjanopalepana\-maṇḍanādikaṃ samājñāpayati. & 家庭僧は朝神壇に淨潔塗附
(牛糞の),装飾等を命ぜり。\\
19. & alabdhaṃ lipseta rājā labd\-naṃ rakṣet. & 王は未だ得ざるものを得んと欲し已に得たるものを守るであらう。\\
20. & namanti phalino vṛkṣā na\-manti guṇino janāḥ, śuṣka\-kāṣṭhaṃ ca mūrkhaś ca bhidyate na ca namyate. & 果實ある樹木は屈し德ある
人も屈す。枯木と愚者とは折るゝも屈せず。
\end{longtable}


%%% Local Variables:
%%% mode: latex
%%% TeX-master: "IntroductionToSanskrit"
%%% End:

\section{分詞}
\subsection{現在並に未來分詞}
\numberParagraph
現在分詞爲他は現在の語基に at (强基 ant, 弱基 at)を
附加して作る。强基は三人稱,複數,現在,爲自の形から i を除き
去りたるものを以てする。隨て第三類動詞並に第二類動詞の或る
もの(\ref{np:138}條)は現在分詞の强基に鼻音を有しない。然し未來分詞
の强基は常に ant である。

語根 bhū (あり)三複現爲他 bhavanti, 現分 bhavat, 强基
bhavant. 語根 hu (供ふ)~三,複,現,爲他 juhvati, 現分
juhvat.

\numberParagraph
未來分詞爲他は未來語基(\ref{np:186}條)と同樣の方法で作ら
れる。語根 bhū (有り)三,複,未來,爲他 bhaviṣyanti ~未來
分詞 bhaviṣyat. これらの分詞の語尾變化並に女性語基の構成に
關しては \ref{np:89}, \ref{np:90}條を見よ。

\numberParagraph \label{np:174}
現在分詞爲自は現在語基に māna を附加して作る。第二
種變化の動詞は māna の代りに āna を加ふ。これらはやはり
三,複,現,爲自形から ante 又は ate を除去したるものに加へ
られる。

語根 div (博戯す)三,複,現,爲自 dīvyante ~現在分詞 dīvya\-%
māna. su (生ず)三,複,現,爲自 sunvate ~現分詞 sunvāna.

\numberParagraph
未來分詞爲自は\endnote{底本では「未來分詞爲自は」ではなく「末來分詞爲自は」。}同樣に未來語基に māna を附加して作
る。語根 dā (與ふ)三複未來爲自 dāsyante ~未來分詞爲自
dāsyamāna.

\numberParagraph \label{np:176}
現在並に未來の受動形に māna を加へて現在並に未來受
動分詞が作られる。語根 tud (打つ),受動 tudya, 現在受動分
詞 tudyamāna.

\subsection{過去能動分詞}
\numberParagraph \label{np:177}
第二過去爲自の分詞は第二過去(\ref{np:192}條)の弱基に vas
(强基には vāṃs, 中基に vat, 弱基に us)を附加して作る。こ
の vas は語基に直接又は i を介して加へられる。語根 gam (行
く)第二過去の弱基 jagm. これに直接又は i を介して vas を
附加す。jagmivas 或は jaganvas. han (殺す),dṛś (見る)も
gam の如く兩形を有す。

\begin{center}
\begin{tabular}{clclclcl}
  語根 & nī (導く)  & 第二過去 & 弱基 & ninī, & 分詞 & ninīvas \\
  〃   & as (投ぐ)  & 〃       & 〃   & ās    & 〃   & āsivas \\
  〃   & pac (煮る) & 〃       & 〃   & pec   & 〃   & pecivas \\
  〃   & yaj (祀る) & 〃       & 〃   & īj    & 〃   & ījivas
\end{tabular}
\end{center}

語尾變化は(\ref{np:94}條)を見よ。

\subsection{過去受動分詞}
\numberParagraph
過去受動分詞は語根に ta 或は na を附加して作る。語
根に强弱語基を分つ場合は最も弱き形を取る。na は語根に直
接,ta は直接又は i を介して附加せられる。他動詞に加へられ
た時は過去受動の義を表はすも自動詞に加へられた時は單に不定
の過去の義となる。lag (附着す)~ lagna, kṛ (作る)~ kṛta, vid
(知る)~ vidita.

\numberParagraph \label{np:179}
次の條々は過去受動分詞の構成に關し重要なるものであ
る。
\begin{enumerate}[label=(\alph*)]
\item na は多くは ṝ 並に\endnote{底本では「並に」ではなく「並にに」。} d に終る語根に附加せられる。tṝ
(超える)~ tīrṇa, pṝ (滿たす)~ pūrṇa, jṝ (老いる)~
jīrṇa, sad (坐す)~ sanna, bhid (破る)~ bhinna. 尙ほ
hā (捨つ)~ hīna, śvi (膨れる)~ śūna, lū (絕つ)~ lūna,
bhañj (破る)~ bhagna.
\item 或る語根は na, ta の兩方を取る。trai (護る)~ trāta
又は trāṇa, tvar (急ぐ)~ tūrṇa 又は tvarita.
\item aya 級の語基並びに催起動詞の語基はその aya を去り
i を介して ta を附加す。cur (盗む)~ corita, budh (覺
む)の催起 bodhayati ~ bodhita.
\item 子音に終る語根には i を介することなくして直に ta を
附加す。隨つて連聲の法則が適用されねばならぬ。語根
tyaj (捨つ)~ tyakta, labh (得る)~ labdha, iṣ (欲する)
~ iṣṭa, dah (燒く)~ dagha, lih (䑛める)~ līḍha, guh (匿
す)~ gūḍha, muh (失神する)~ mūḍha 又は mugdha, sah
(堪ふ)~ soḍha 等。
\item ā に終る語根は或るものは ī に或るものは i に變ず。
pā (飲む)~ pīta, sthā (立つ)~ sthita, dhā (置く)~ hita.
\item 語根中の鼻音は通常消失する。鼻音に終る語根では通
例母音を延長し若くは鼻音が消失する。bandh (縛る)~
baddha, daṃś (咬む)~ daṣṭa, kram (歩む)~ krānta, śam
(靜める)~ śānta, gam (行く)~ gata, han (殺す)~ hata,
man (考ふ)~ mata.

又語根の母音の延長と同時に鼻音を消失する。khan (掘
る)~ khāta, jan (生る)~ jāta.
\item 受動詞構成の場合 u, i に弱められた va, ya を有する
語根(\ref{np:159}條 \ref{item:159b})は此にも同樣の變化をなす。vac (言ふ)
~ ukta, vah (運ぶ)~ ūḍha, yaj (供ふ)~ iṣṭa, vyadh (貫
く)~ viddha.
\item 其の他注意すべきもの:語根 prach (問ふ)~ pṛṣṭa,
grah (取る)~ gṛhīta, dā (與ふ)~ datta.
\end{enumerate}

\numberParagraph
過去受動分詞に vat (强基 vant)を附加して能動過去分
詞が作られる。kṛ (爲す)~ kṛtavat (vān) かれは爲せり。これ
は cakṛvān (\ref{np:177}條)又は cakāra (\ref{np:194}條)と同意義であり。語
尾變化は \ref{np:94}條に出づ。

\numberParagraph
過去分詞爲自は第二過去(\ref{np:192}條)の弱語基に āna を
附加して作られる。語根 bhid (破る)~ bibhidāna, nī (導く)~
ninyāna, stu (讚む)~ tuṣṭuvāna, dā (與ふ)~ dadāna, yaj (供
ふ)~ ijāna (\ref{np:174}--\ref{np:176}條參照)。

\ex{第十二}
\begin{longtable}{c*{2}{p{0.45\hsize}}}
 1. & kiṃ na lajjasa evaṃ bru\-vāṇaḥ. & 如何に汝はかくの如く語りつゝ羞ぢざるか。\\
 2. & siṃhasya vane (\ref{np:231}條) bhra\-mato ravir astaṃgataḥ & 獅子が林を彷徨しつゝある
間に太陽は沒せり。\\
 3. & ajānan dāhārtiṃ patati śala\-bho dīpa-dahanam. & 燃ゆる苦痛を知らずして蛾
は燈火に落ちたり。\\
 4. & apriyāny api kurvāno yaḥ priyaḥ priya eva saḥ. & 非愛のことを爲すも愛する
所の彼は愛人にこそあれ。\\
 5. & muniḥ krodhyamāno 'pi pri\-yaṃ brūyāt. & 聖者は怒らされても愛語を
語るべし。\\
 6. & brāhmaṇasya bhāryā pratidi\-naṃ kuṭumbena saha kalahaṃ kurvāṇā na kṣaṇam api vyaś\-rāmyat.
 & 婆羅門の妻は毎日家族と喧嘩をなしつゝ一瞬と雖も中止せざりき。\\
 7. & narau vivadamānau dhar\-mādhikāriṇaṃ gatavantau prā\-vaktāṃ parasparaṃ dūṣyantau.
 & 二人の男は爭ひつゝ裁判官の許へ來れり。彼等二人は互に耻ぢしめつゝ宣べ立てた。\\
 8. & cauraś cititavān; aho, keno\-pāyenaiṣāṃ dhanaṃ labhe. & 盗人は考えた。あゝ如何な
る方法で彼等の財貨を我は得べき。\\
 9. & bījāṅkuraḥ sūkṣmo 'pi pari\-puṣṭo 'bhirakṣitaś ca kāle pha\-lāni dadāti. & 芽は小くとも育成せられて
守られてあらば時に於て果を生ず。\\
10. & haṃsāv astaṃgate ravau sva\-nīḍa0saṃśrayam akurutām. & 二羽の白鳥は太陽の沒せし
時自分の巢へ避難をなせり。\\
11. & mṛte patyau strī pradāhayed ātmānam. & 夫死したる時婦人は自身を焚燒すべきである。\\
12. & sa daridro yasya tṛṣṇā viśālā manasi parituṣṭe ko 'rthavān ko daridraḥ.
& 貧乏人とはその慾が大きいものゝことである。心に滿足せる時に誰が富人で誰が貧乏人であらうか。\\
13. & parāṅmukhe 'pi daive kṛtyaṃ kuryān medhāvī. & 運命我に背くとも賢者は義務を爲すべきである。\\
14. & brāhmaṇo māghamāse saum\-yānile pravāti meghācchādite gagane mandam mandaṃ var\-%
ṣati parjanye paśu-prārthanā\-rthaṃ grāmāntaraṃ gataḥ. & マーグハ月に於て北風が吹
き,空は雲に覆はれ徐々に雨神が雨を降らす時一人の婆羅門は犠牲の家畜を乞ふために他の村へ行けり。\\
15. & naṣṭaṃ mṛtam atikrāntaṃ nānuśocanti paṇḍitāḥ paṇḍitānāṃ ca mūrkhāṇāṃ viśeṣo 'yaṃ yataḥ smṛtaḥ.
& 失はれたるもの,死せるもの,過ぎしものを賢者は悲しまず賢者と愚者との差はこれなりと云はるるが故に。
\end{longtable}

\subsection{義務分詞}
\numberParagraph
この分詞は未來受動の意義を有し,梵文には屢々用ひ
られるが,三種の構成法によつて作られる,卽ち tavya, anīya 又
は ya を附加するものである。

語根 kṛ (作る)~ kartavya, kartaṇīya, kārya 作らるべき。
\begin{enumerate}[label=(\alph*)]
\item tavya は語根に直接又は i を介して附加せられる。語
根の母音は重韻となる。dā (與ふ)~ dātavya, ji (勝つ)
~ jetavya, bhū (有る)~ bhavitavya, muc (解く)~
moktavya, cur (盗む)~ coritavya.
\item anīya の前には語根の母音は通例同樣に重韻となる。
nī (導く)~ nayanīya, śru (聞く)~śravaṇīya, bhid (破
る)~ bhedanīya, sṛj (投ぐ)~ sarjānīya.
\item ya を附加する時に語根の終の ā は e に變じ,其他の
母音は或る時は變化せず,或る時は重韻又は複重韻となる。
その時重韻の o は直ちに ya の前にあるとき av 又は āv
となる。dā (與ふ)~ deya, ji (勝つ)~ jeya, nī (導く)~
neya, bhū (有る)~ bhāvya 又は bhavya, budh (覺む)~
bodhya, vac (語る)~ vācya, labh (得る)~ labhya.
\end{enumerate}

%%% Local Variables:
%%% mode: latex
%%% TeX-master: "IntroductionToSanskrit"
%%% End:

\section{連續體}
\numberParagraph
連續體を作るには二種の方法がある。語根に或は tvā を
加へ或は ya を加ふ。

\begin{enumerate}[label=(\alph*)]
\item tvā は語根に直接又は i を介して添加せらる。i の添
加に關しては \ref{np:179}條の法則が適用せられる。tyaj (捨つ)
~ tyaktvā, labh (得る)~ labdhvā, kṛ (爲す)~ kṛtvā,
grah (取る)~ gṛhītvā, vac (語る)~ uktvā, han (殺す)
~ hatvā, gam (行く)~ gatvā, man (考ふ)~ matvā,
kram (歩む)~ krāntvā, śam (靜める)~ śāntvā, yaj (供
ふ)~ iṣṭvā, dṛś (見る) dṛṣṭvā.

否定前加語 a は tvā の連續體に附加せられ得る。
akṛtvā (爲さずして)。aya 級並に催起動詞は母音 i を介
して附加せられた後接字の前に ay なる綴が殘される。
cur (盗む)~ corayitvā.
\item ya は前加語を有する語根に附加せられる。語根は一般
に受動後接字(\ref{np:159}條)の附加の場合に於ける變化をなす。

e, ai, o に終るものは ā に變じ,ā に終るものは變化し
ない。kṣip + pra (投げ棄つ)~ prakṣipya, prach + ā (吿
別する)~ āpṛcchya, pṝ + ā (充たす)~ āpūrya, dā + pra
(渡す)~ pradāya.

短い母音に終る語根は ya の代りに tya を附加する。
kṛ + adhi (處理す)~ adhikṛtya, i + adhi (學ぶ)~ adhītya.
同樣に tya は am 並に an に終る語根でもその鼻音を除
去してこれを附加するを得る。それは過去受動分詞を作る
時鼻音の消失するに準ず。gam + ā (來る)~ āgatya 又は
āgamya, jan + pra (發生す)~ prajanya 又は prajāya.
khan + ni (埋む)~ nikhanya 又は nikhāya. 然し man +
ava (輕蔑す)は只 avamatya, han + pra (殺す)も只
prahatya のみ。
\end{enumerate}

\numberParagraph
連續體の他の一種がある。それはあまり多く用ひられな
いが,語根に am を附加して作る。語根若し母音に終る時は複
重韻,若し中間に母音ある時はそれを重韻となす。中間の a は
延長せられ,ā に終るものは ā の前に y を添加する。kṛ (作る)
kāram. vid (知る) vedam, dā (與ふ) dāyam.

%%% Local Variables:
%%% mode: latex
%%% TeX-master: "IntroductionToSanskrit"
%%% End:

\section{不定法}
\numberParagraph
不定法は語根に直接又は i を介して後接字 tum を附加
して作る。その母音は重韻となる。dā (與ふ)~ dātum, ji (勝
つ)~ jetum, nī (導く)~ netum, bhū (有り)~ bhavitum, kṛ (作
る)~ kartum, jīv (生く)~ jīvitum, grah (取る)~ grahītum,
sah (堪ふ)~ soḍhum.

aya 級並に催起動詞は ay なる綴を添へ常に i を介して tum
を附加する。cur (盗む)~ corayitum, budh (覺める)~ bodhayi\-%
tum.

分詞等其他の意義並に用法に關しては \ref{np:232} 條以下參照。

\ex{第十三}
\begin{longtable}{c*{2}{p{0.45\hsize}}}
 1. & na hi bhavati yan na bhāvyaṃ, bhavati bhāvyaṃ vināpi yat\-nena;
kara-tala-gatam api na\-śyati, yasya hi bhavitavyatā nāsti.
& 有るべからざる所のものは有ることなし。有るべきものは努力を加へずして
もあり,存在性なき所のものについてはたとひ掌理にあるも消失す。\\
 2. & mūṣikā gṛha-jātāpi hantavyā\-pakāriṇī. & 鼠は家に生ずるも害あり殺さるべし。\\
 3. & vayasya, na bhetavyam. & 友よ,恐れざれ。\\
 4. & kim evaṃvidhe vyatikare kār\-yam āvābyām. & かくの如き不幸の時我々二
 人は何をなすべきや。\\
 5. & puruṣenodyamo na tyājyaḥ. & 人は努力を廢すべきでない。\\
 6. & yasya gṛhe mātā nāsti bhāryā ca, tenāraṇyaṃ gantavyam. & その家に母と妻となき人は
 森林へ行くべきだ。\\
 7. & avaśyaṃ nidhanaṃ sarvair gantavyam iha mānavaiḥ. & 必ずや一切の人は死に行くべし。\\
 8. & durbalo 'pi ripur nāvajñeyo kathaṃcana. & 敵は弱くとも決して侮るべきでない。\\
 9. & upadeśo na dātavyo yādṛśe tādṛśe janāya. & 出會がしらの人に忠吿は與へらるべきでない。\\
10. & yo vidyām aiśvaryaṃ vāsādya vicarty asamunnaddhaḥ, sa paṇḍita ucyate.
& 知識若くは權力に達して謙虛に虛する所の彼は賢者と云はれる。\\
11. & bhadre, yāvad ahaṃ bhoja\-naṃ gṛhītvā pratyāgacchāmi, tāvat tvayātra sthātavyam.
& 愛するものよ,我が食を齎して歸り來るまで汝はそこに立つべし。\\
12. & mām uddiśya pattraṃ pre\-ṣaya. & 我に宛てゝ書翰を送れ。\\
13. & brāhmaṇenoktam: ko guṇo vidyāyā yena deśāntaraṃ gatvā bhūpatīn paritoṣyārthopārjanā
na kriyate. & 婆羅門は云へり。他國に行きて王者を滿足せしめ富の獲得がなされざるなら
ば何の知識の功績ぞや。\\
14. & śaśako mandaṃ mandaṃ gat\-vā praṇamya siṃhasyāgre sthitaḥ. & 兎は徐々に行きて禮をなし
獅子の前に立てり。\\
15. & kauliko rāja-kanyāṃ dṛṣṭvā kāma-śarair hanyamānaḥ saha\-sā bhūtale nyapatat.
& 織匠は王の娘を見て愛神の箭に打たれつつ突然地面に仆れたり。\\
16. & atha mitraṃ tad-avastham avalokya rathakāras tad-duḥ\-khaduḥkhita āpta-puruṣais taṃ
samutkṣipya sva-gṛham ānā\-yayat. & 時に車匠は友のその狀態を
見てその苦しみを苦しみとし得られた人々によりて彼はかき上げて自分の家につれ來れり。\\
17. & śiṣyā upādhyāyam āpṛcchyā\-nujñāṃ labdhvā pustakāni nīt\-vā pracalitāḥ.
& 弟子等は師を吿別し許可を得て書を取りて開きたり。\\
18. & pathiko grīṣmoṣmaṇā saṃta\-ptaḥ kaṃcin mārga-sthaṃ vṛk\-ṣam āsādya tatraiva prasuptaḥ.
& 旅人は夏の暑さに苦しめられとある路傍の樹に近づきて眠れり。\\
19. & Nalo bhaktāṃ Damayantīṃ tyaktvā śoka-paryākulo 'bhavat. & ナラは貞實なるダマヤン
ティーを捨てゝ悲に亂されたり。\\
20. & anusṛtya satāṃ vartma yat svalpam api tad bahu & 善人の路を追ひ行けば極め
て小なるもそは價値大なり。\\
21. & śarīrāsāmarthyān na padam api vṛddhaḥ siṃhaś calitum aśaknot. & 身體\ruby{羸{弱}}{るい|じゃく}のために老いたる
獅子は一歩をすら行く能はざりき。\\
22. & stenābhyām uktaṃ na sarvam etad dhanaṃ gṛhaṃ prati netuṃ yujyate tad dhāgam atra vana\-%
gahane kvāpi bhūmau nikṣi\-pāya. & 二人の盗賊は云へり。この一切の財物を家に運ぶは
宜しからず。我々はその一部分をかの深林の中のある地點に置くべし。\\
23. & lubdhako mṛgayāṃ kartuṃ pratiṣṭhati. & 獵師は狩をなすために出發せり。\\
24. & deśāntara-stho dayitā-vipra\-yogaṃ soḍhuṃ na śaknomi. & 他國に在る我れは愛人の離別を堪へ能はず。\\
25. & śṛgālaḥ palāyitum icchaṃs tatra sthāna eva siṃhena khaṇ\-ḍaśaḥ kṛto mṛtaś ca.
& 豺は逃れ去らんと欲せしもその場に於て獅子に寸斷せられて死せり。\\
26. & brāhmaṇo bhāryayā saha bhoktum ārabhate. & 婆羅門は妻と共に食し始めた。\\
27. & upakartuṃ priyaṃ vaktuṃ snehaṃ kartuṃ sajjanānāṃ svabhāvaḥ. & 善を行ひ,善を語り,愛を
示すは善人の性質なり。\\
28. & svabhāvo nopadeśena śakyate kartum anyathā, sutaptam api pānīyaṃ punargacchati śīta\-tām.
& 自性は忠吿によりてこれを變ずる能はず。水は熱せられてもつひに冷かになるものなり。
\end{longtable}
%%% Local Variables:
%%% mode: latex
%%% TeX-master: "IntroductionToSanskrit"
%%% End:

\section{未來組織}
未來時の構成法に二種あり。(1)單未來。(2)複說未來。

\subsection{單未來}
\numberParagraph \label{np:186}
單未來は語根に sya を附加して作る。その母音は重韻化
する。sya は直接或は母音 i を介して添加せらる。人稱語尾は現
在の如くである。

\numberParagraph

\begin{tabular}{cl}
  語根 & dā (與ふ)~ dāsya \\
  〃   & kṛ (作る)~ kariṣya
\end{tabular}

\begin{center}
\begin{tabular}{c*{3}{p{0.23\hsize}}}
  \multicolumn{4}{c}{爲他} \\
     & 單      & 兩        & 複 \\
  1. & dāsyāmi & dāsyāvas  & dāsyāmas \\
  2. & dāsyasi & dāsyathas & dāsyatha \\
  3. & dāsyati & dāsyatas  & dāsyanti \\
  \multicolumn{4}{c}{爲自} \\
  1. & dāsye   & dāsyāvahe & dāsyāmahe \\
  2. & dāsyase & dāsyethe  & dāsyadhve \\
  3. & dāsyate & dāsyete   & dāsyante
\end{tabular}
\end{center}
\begin{center}
\begin{tabular}{c*{3}{p{0.23\hsize}}}
  \multicolumn{4}{c}{爲他} \\
     & 單        & 兩          & 複 \\
  1. & kariṣyāmi & kariṣyāvas  & kariṣyāmas \\
  2. & kariṣyasi & kariṣyathas & kariṣyatha \\
  3. & kariṣyati & kariṣyatas  & kariṣyanti \\
  \multicolumn{4}{c}{爲自} \\
  1. & kariṣye   & kariṣyāvahe & kariṣyāmahe \\
  2. & kariṣyase & kariṣyethe  & kariṣyadhve \\
  3. & kariṣyate & kariṣyete   & kariṣyante
\end{tabular}
\end{center}

\numberParagraph
未來時の記號なる sya を附加するに當り子音に終る語根
に會して聲音上の變化あることに注意すべきである。

\begin{tabular}{*{2}{p{0.4\hsize}}}
  śak (能ふ) śakṣyati    & pac (煮る) pakṣyati \\
  prach (問ふ) prakṣyati & tyaj (捨つ) tyakṣyati \\
  labh (得る) labpyati   & vas (住す) vatsyati
\end{tabular}

尙次の如きものもある。

\begin{tabular}{*{2}{p{0.4\hsize}}}
  nī (導く) neṣyati    & budh (覺る) bodhiṣyati \\
  gam (行く) gamiṣyati & grah (取る) grahīṣyati \\
  dṛś (見る) drakṣyati & ji (勝つ) jeṣyati \\
  gai (歌ふ) gāsyati   &
\end{tabular}

\subsection{條件法}
\numberParagraph
未來語基から條件法が作られる。卽ち過去符と第一過去
の語尾を附加する。dā (與ふ)未來語基 dāsya, 條件法 adāsyat (彼
は與へたであらう)。同樣に bhū (あり)~ bhaviṣya ~ abhaviṣyat
(彼は有りしならむ),kṛ (爲す)~ kariṣya ~ akariṣyat (彼は爲せ
しならむ)。

\subsection{複說未來}
\numberParagraph
複說未來は tṛ に終る作者名詞(\ref{np:55}條)と as (有り)と云
ふ助動詞とで構成せられる。三人稱は總て三數に隨つて用ふる(dātā, dā\-%
tārau, dātaras の如く)。其の他の人稱では as の現,爲他,爲
自の形が添へられる。dā (與ふ)作者名詞 dātṛ 主單 dātā + asmi
= dātāsmi, bhū (有り)~ bhavitṛ ~ bhavitā + asi = bhavitāsi 等。

\begin{center}
\begin{tabular}{c*{3}{p{0.23\hsize}}}
  \multicolumn{4}{c}{爲他} \\
     & 單         & 兩           & 複 \\
  1. & bodhitāsmi & bodhitāsvas  & bodhitāsmas \\
  2. & bodhitāsi  & bodhitāsthas & bodhitāstha \\
  3. & bodhitā    & bodhitārau   & bodhitāras \\
  \multicolumn{4}{c}{爲自} \\
  1. & bodhitāhe & bodhitāsvahe & bodhitāsmahe \\
  2. & bodhitāse & bodhitāsāthe & bodhitādhve \\
  3. & bodhitā   & bodhitārau   & bodhitāras
\end{tabular}
\end{center}

其他の例:nī (導く)~ netāsmi, dṛś (見る)~ drṣṭāsmi, jīv
(生く)~ jīvitāsmi.

\numberParagraph
單未來は今日中に生ずべき事實又は汎く將來に於て早晩
發生すべき事實を敍ぶるに用ふ。 通例 śvas (明
日)等の語と共に用ひられる。

\ex{第十四}
\begin{longtable}{c*{2}{p{0.45\hsize}}}
 1. & aham adya gamiṣyāmi. & 私は今日行くだらう。\\
 2. & tau gantārau. & 彼等二人は明日行くであらう。\\
 3. & yena tvaṃ śvo yoddhāsi taṃ dhruvaṃ tvaṃ jetāsi. & 汝は明日戰うであらう所の
人を必ず征服するであらう。\\
 4. & adya varṣiṣyati śvo 'pi vraṣṭā. & 今日雨降らむ。明日も亦雨降るべし。\\
 5. & sūdo 'nnaṃ pakṣyati. & 庖人は食を煮む。\\
 6. & gṛhāt prasthitvā vane vats\-yāmi. & 家より出でゝ我は林中に住まむ。\\
 7. & Gangā-tīraṃ gatvā nadīṃ drakṣyāmi. & 恒河の岸に行きて我は河を見るべし。\\
 8. & asau śīghraṃ nīrogo bhavi\-ṣyati. & 彼は速かに健康となるべし。\\
 9. & astaṃgate sūrye śaśī śiśira\-kara udeṣyati. & 日沒したる時冷かなる光あ
る月は昇るべし。\\
10. & yadi tvaṃ mahā\-pañke pati\-ṣyati tad asaṃśataṃ mariṣyasi.
& 汝若し大なる沼に陷らば必ず死すべし。\\
11. & yadi na pāpaṃ sa naro 'tyakṣyat tadā duḥkhaṃ tasyā\-bhaviṣyat.
& 彼もし罪を捨てざりせばその時彼に不幸がありしならむ。\\
12. & yadi vayaṃ bhūmau patitam agniṃ niravāpayiṣyāma tad agnis tasya gṛham adhakṣyat.
& 若し我等地上に落ちたる火を消さざりせばその火は彼の家を燒きしならむ。\\
13. & yat tvaṃ prakṣyasi tad ahaṃ yathā-śakti prativakṣyāmi. & 汝の問はむことを我は力の
限り答ふべし。\\
14. & kiṃ kariṣyanti vaktāraḥ śrotā yatra na vidyate, nagna\-kṣapaṇake deśe rajakāḥ kiṃ
kariṣyanti. & 聞くものなき所には語るものは何をかなさむ。裸形者の國土にては洗濯人は何をかなさむ。
\end{longtable}


%%% Local Variables:
%%% mode: latex
%%% TeX-master: "IntroductionToSanskrit"
%%% End:

\section{第二過去組織}
第二過去の構成法に二種,重複第二過去と複說第二過去とであ
ある。

\numberParagraph \label{np:192}
重複第二過去は語根の重複によつて作られる。子音に終
る語根に就ては \ref{np:146}條が適用せられる。然し若干の注意すべきも
のがある。ṛ は a によりて重複せられる。i ではない。卽ち kṛ
(作る)~重複第二過去語基 cakṛ.

母音に終る語根に關しては次の重複が注意さるべきである。
\begin{enumerate}[label=(\alph*)]
\item 單一の子音の前に a で始まる語根は a を ā とする。
ad (食ふ)~ ād. 位置によつて長き a 並に ṛ を以て始ま
る語根は ān なる重複をなす。例 ṛc (尊敬する)~ ānarca.
ā にて始まるものは變化しない。āp (得る)~ āp.
\item i, u にて始まる語根は次の如く ī, ū となり,强語基
の場合(\ref{np:193}條)には重韻に强められたる語根の母音 e, o
の前に iy, uv となる。iṣ (欲す)弱基 īṣ, 强基 iyeṣ.
\end{enumerate}

\numberParagraph \label{np:193}
第二過去に於て强語基は單數爲他の一,二,三人稱に用ひ
られ,其の他は弱語基を用ふ。强基は次の方法で强められ弱基か
ら區別せられる。
\begin{enumerate}[label=(\alph*)]
\item ā 以外の母音に終れる,並に中間に a を有する語根は一
單爲他に重韻又は複重韻,二人稱に重韻,三人稱に複重韻の
形を取る。nī (導く)一. 單爲他 ninaya 又は nināya, 二.
ninetha, 三. nināya; kṛ (作る),一. cakara 又は
cakāra, 二. cakartha, 三. cakāra.
\item 其他の短母音は一,二,三人稱を通じて重韻となる。
budh (覺ゆ)~一. bubodha, 二. bubodhitha, 三. bubodha.
\end{enumerate}

\numberParagraph \label{np:194}
第二過去の語尾は或は直接に或は i を介して語根に附加
せられる。それは次の如くである。

\begin{tabular}{ll}
  爲他: & a, tha, a; va, athus, atus; ma, a, us. \\
  爲自: & e, se, e; vahe, āthe, āte; mahe, dhve, re.
\end{tabular}

三單複,爲自の re は常に i を伴ふ。其他の子音で始まる語
尾は tha を除きて大抵は i を挿入する。tha は時には i を挿入
し時には挿入しない。其他のものは任意に i を取り得る。kṛ
(作る),śru (聞く),vṛ (選ぶ),stu (讚む)は慣用的に連結母音
を取らない。

\numberParagraph
語根 kṛ (作る)~强基 cakar, 弱基 cakṛ.

\begin{center}
\begin{tabular}{c*{3}{p{0.23\hsize}}}
  \multicolumn{4}{c}{爲他} \\
     & 單              & 兩        & 複 \\
  1. & cakāra (cakara) & cakṛva    & cakṛma \\
  2. & cakartha        & cakrathus & cakra \\
  3. & cakāra          & cakratus  & cakrus \\
  \multicolumn{4}{c}{爲自} \\
  1. & cakre  & cakṛvahe & cakṛmahe \\
  2. & cakṛṣe & cakrāthe & cakṛdhve \\
  3. & cakre  & cakrāte  & cakrire
\end{tabular}
\end{center}

\numberParagraph
子音を以て終始する語根にして a を中間に有し語根重複
に當り代音を要しないものは强中語基に於て重複するも,弱語基
では通例只 a を e に變ずるのみで重複をなさない。二. 單,爲
他が i を挿入する場合も重複しない。

語根 pac (煮る)~强基 papac, 弱基 pec.

\begin{center}
\begin{tabular}{c*{3}{p{0.23\hsize}}}
  \multicolumn{4}{c}{爲他} \\
     & 單                 & 兩       & 複 \\
  1. & papāca (papaca)    & peciva   & pecima \\
  2. & pecitha (papaktha) & pecathus & peca \\
  3. & papāca             & pecatus  & pecus \\
  \multicolumn{4}{c}{爲自} \\
  1. & pece   & pecivahe & pecimahe \\
  2. & peciṣe & pecāthe  & pecidhve \\
  3. & pece   & pecāte   & pecire
\end{tabular}
\end{center}

\numberParagraph
ā, ai, au に終る語根は一及び三の單爲他に au なる語尾
を用ふ。弱語基にては語根の母音は母音で始まる語尾の前に消
失し,子音で始まる語尾の前にはその母音を弱めて i とする。二
單爲他は ā 又は i に作る。

語根 dā (與ふ)~强基 dadā, 弱基 dad.

\begin{center}
\begin{tabular}{c*{3}{p{0.23\hsize}}}
  \multicolumn{4}{c}{爲他} \\
     & 單                & 兩       & 複 \\
  1. & dadau             & dadiva   & dadima \\
  2. & dadātha (daditha) & dadathus & dada \\
  3. & dadau             & dadatus  & dadus \\
  \multicolumn{4}{c}{爲自} \\
  1. & dade   & dadivahe & dadimahe \\
  2. & dadiṣe & dadāthe  & dadidhve \\
  3. & dade   & dadāte   & dadire
\end{tabular}
\end{center}

\numberParagraph
va を以て始むる語根は u を以て重複音とする。而して
弱語基では語根の va と重複韻を收縮して ū とする。

語根 vac (語る)~强基 uvac, 弱基 ūc.

\begin{center}
\begin{tabular}{c*{3}{p{0.23\hsize}}}
  \multicolumn{4}{c}{爲他} \\
     & 單                 & 兩       & 複 \\
  1. & uvāca (uvaca)      & ūciva    & ūcima \\
  2. & uvacitha (uvaktha) & ūcatheus & ūca \\
  3. & uvāca              & ūcatus   & ūcus \\
  \multicolumn{4}{c}{爲自} \\
  1. & ūce   & ūcivahe & ūcimahe \\
  2. & ūciṣe & ūcāthe  & ūcidhve \\
  3. & ūce   & ūcāte   & ūcire
\end{tabular}
\end{center}

同樣に vad (語る)~强基 uvad, 弱基 ūd, vas (住す)~ uvas,
ūs.

yaj (供ふ)に就ても類推すべきだ。卽ち u の代りに i, ū の
代りに ī となる。iyaj, īj.

\numberParagraph
二つの子音の中間に a を有する或る語根は弱語基に於い
て a を消失する。
gam (行く)~强基 jagam, 弱基 jagm, 一兩,爲他 jagmiva.
同樣に khan (掘る)~ cakhniva, jan (生る)~ jajñiva, han
(殺す)~ jaghniva.

\numberParagraph
稍不規則なるもの。
\begin{enumerate}[label=(\alph*)]
\item bhū (有る)~ babhūva, babhūvitha, babhūva;
babhūviva 乃至 babhūvus.
\item ji (勝つ)~ jigāya, ci (集む)~ cicāya 又は cikāya.
\item 語根 ah (云ふ)\endnote{「云」の字は底本では印刷が乱れている。}は只次の形を存するのみである。單二.
āttha, 三. āha, 兩二. āhathus, 三. āhatus, 複三. āhus.
\end{enumerate}

\numberParagraph
子音を以て終始し中間に i, u, ṛ を有するものは强基に
於て重韻となるのみ。故に只二樣語基を有するのみである。

\begin{tabular}{ll}
  tud (打つ) & tutoda, tutudus. \\
  dṛś (見る) & dadarśa, dadṛśus.
\end{tabular}

vid (知る)は重複しない。只强基に i を e とするのみ。veda,
vettha, veda; vidima, vida, vidus.

\numberParagraph
複說第二過去は語根又は語基に ām を加へ,kṛ, bhū 又
は as の第二過去を加ふ。kṛ は動詞が爲自のみに變化する時は爲
自の形を取る。これは第十類動詞,隨てこれと同樣の構成を有す
る催起動詞又は a, ā 以外の母音に始まりそれが本來或は位置上
長き語根又は ās (坐す)に用ひられ,或は vid (知る),hu (供
ふ)等若干の動詞に用ひることを得。cur (盗む)~ corayāmāsa
tuṣ (滿足す)催起~ toṣaya (滿足せしむ)~ toṣayāmāsa, katha\-%
ya (話す)~ kathayāṃ babhūva, īkṣ (見る)爲自~ īkṣāṃ cakre,
hu (供ふ)~ juhavāṃ cakāra.

\ex{第十五}
\begin{longtable}{c*{2}{p{0.45\hsize}}}
 1. & saras-tīra upaviśya sarasi\-jaṃ dadṛśuḥ. & 池の岸に坐して彼等は蓮を見た。\\
 2. & rāja-puruṣo laguḍena cauraṃ tutoda. & 王の家來は杖で盗人を打つた。\\
 3. & guruḥ śiṣyān nininda. & 師は弟子は叱責した。\\
 4. & kapayo vṛkṣād bhūmau petuḥ. & 猿等は樹より地へ落ちた。\\
 5. & catvāro bālakā rathena naga\-rāya jagmire. & 四人の子供が車で街へ行つた。\\
 6. & mūṣikā bhūmiṃ cakhnire. & 鼠等は血を掘つた。\\
 7. & aham Ayodhyāyāṃ jagāma. & 我は阿輸遮へ行けり。\\
 8. & nṛpo mṛgayāṃ gatvā bahūn mṛgāṃś caikaṃ vyāghraṃ ca jaghāna.
 & 王は獵に行き多くの鹿と一疋の虎とを殺せり。\\
 9. & aśvena gacchantaṃ nṛpaṃ tasya parivārā anujagmuḥ. & 馬にて行く王に彼の家來達
は從つた。\\
10. & rājñaḥ puruṣāḥ sva-rājānaṃ na viduḥ. & 王の家來等は自分の王を知
らざりき。\\
11. & mūṣiko bhūmiṃ khanitvā sva\-vivaraṃ cakāra. & 鼠は地を掘りて自らの穴を作りき。\\
12. & vṛkṣasya mūle dvau bālakau cikrīḍatuḥ. & 樹の下に於いて二人の子は遊べり。\\
13. & tāv etaṃ bālakaṃ tutudatur iti śrutvāhaṃ tau nininda. & 彼等二人が此の小兒を打て
りと聞きて我は彼等二人を叱責せり。\\
14. & kiṃ tvaṃ mahā-śabdaṃ ca\-kartha. & 何故に汝は大なる聲をなせるか。\\
15. & mahā-siṃhaṃ dṛṣṭvā tad\-bhayād vayaṃ mahā-śabdaṃ cakṛma. & かの少女は美はしき歌を歌へり。\\
16. & kavir kīrtiṃ iyeṣa. & 詩人は名譽を欲せり。\\
17. & sa vaṇik-putraḥ pratyahaṃ saktuṃ brāhmaṇebhyo dadau. & 彼の商人の子は毎日麥粉を
婆羅門に與へたり。\\
18. & sa bālakaḥ kūpāj jalaṃ ānināya. & かの子供は井戸から水を持ち來つた。\\
19. & ghṛtena narā devam ījuḥ. & 人々は酥にて神を祀れり。\\
20. & ete janā nṛpam ūcuḥ. & これ等の人々は王に云へり。\\
21. & rājña ājñāṃ śrutvā rāja\-puruṣaḥ pratyuvāca: yāthā deva ājñāpayatītī.
& 王の命令を聞きて王の家來は答へて云へり。王の命のまゝにと。\\
22. & aśvāt patito rājā mamāra. & 馬より落ちし王は死せり。\\
23. & rājño gṛhe dvau brāhmaṇāv ūṣatuḥ. & 王の家に於いて二人の婆羅門は住めりき。\\
24. & nṛpasya śataṃ putrā ba\-bhūvuḥ. & 王は百人の子等ありき。\\
25. & taṃ sa Bhīmaḥ prajākāmas toṣayāmāsa dharmavit. & 子を欲せる法に達せるブヒ
ーマは彼を滿足せしめた \\
26. & taṃ ca dṛḍhaṃ pariṣvajyā\-śrubhiḥ snapayāmāsa. & 而して彼をかたく抱きて淚
もて沐浴せしめた。\\
27. & svīyaṃ rūpaṃ taṃ darśa\-yāmāsa. & 自分の美貌を彼に示した。
\end{longtable}

%%% Local Variables:
%%% mode: latex
%%% TeX-master: "IntroductionToSanskrit"
%%% End:

\section{第三過去組織}
\numberParagraph
第三過去は過去符 a を有し第一過去と同樣の語尾を附加
して作る。その構成法に三種七類が分れる。三種とは (I) 單第
三過去,これに (1) 語根語基と (2) a 語基の二類がある。(II)
重複第三過去,(III) 硬吹氣音第三過去,これに (1) sa 語基,(2)
s 語基,(3) iṣ 語基,(4) siṣ 語基の四類が語ある\endnote{「四類が語ある」はママ。}。

\subsection{單第三過去}
\subsubsection{語根語基}
\numberParagraph \label{np:204}
語根語基の第三過去は語根に直に語尾を加へる。これは
爲他の ā に終る語根又は bhū にのみ用ひられる。その餘に用
ひらることは極めて稀である。ā に終る語根の三複は us なる語
尾を取る。

語根 dā (與ふ)。

\begin{center}
\begin{tabular}{c*{3}{p{0.23\hsize}}}
  \multicolumn{4}{c}{爲他} \\
     & 單   & 兩     & 複 \\
  1. & adām & adāva  & adāma \\
  2. & adās & adātam & adāta \\
  3. & adāt & adātām & adus \\
\end{tabular}
\end{center}

語根 bhū (ある)。

\begin{center}
\begin{tabular}{c*{3}{p{0.23\hsize}}}
  \multicolumn{4}{c}{爲自} \\
     & 單      & 兩      & 複 \\
  1. & abhūvam & abhūva  & abhūma \\
  2. & abhūs   & abhūtam & abhūta \\
  3. & abhūt   & abhūtām & abhūvan
\end{tabular}
\end{center}

爲自の dhā (置く)~二單 abhithās, sthā (住す)の asthithās,
kṛ (作る)の akṛthās, mṛ (死す)の amṛthās. 三單 abhita,
asthitha, akṛta, amṛta.

\subsubsection{a 語基}
\numberParagraph
a 語基の第三過去は a 級の第一過去(\ref{np:64} 條)の如く構
成せられ變化する。但し語根の母音は平音であつて變化しない。

sic (灌ぐ)~第一過去 asiñcat, 第三過去爲他 asicat, 爲自
asicata. lip (塗る)~第一過去 alimpat, 第三過去爲他 alipat,
爲自 alipata.

不規則なるものに dṛś (見る)三單爲自 adarśat, hve (呼ぶ)
ahvat の如きがある。

\subsection{重複第三過去}
\numberParagraph
重複第三過去は語根を重複して作られる。而して子音は
現在の場合(\ref{np:146} 條)と同樣に重複せられる。śri (行く)三單爲
他 aśiśriyat.

\begin{tabular}{ll}
不規則なるもの: & vac (言ふ)~ avocat. \\
                 & pat (落つ)~ apaptat.
\end{tabular}

この形が最も多く用ひられるのは第十類動詞並催起動詞であ
る。語基の aya は除かれる。重複音と語根の母音とは必ずしも
一樣でない。若し中間に a を有するか又は ā, ṛ, ṝ に終る場合
多くは i 又は ī を以て重複の母音とする。

\numberParagraph
第三過去の重複語基は重複音に重きを置き本來の語根は
比較的輕視せられる。かくして重複音の母音は常に長く,本來の
語根の母音は平音又は短母音である。

語根 cur (盗む)~ acūcuram, nī (導く)催起~ anīnayam,
yuj 催起(軛す)~ ayūyujam. dā 催起(與へしむ)~ adīdapat,
sthā 催起(立たしむ)~ atiṣṭhipat.

\subsection{硬複氣音第三過去}
\subsubsection[sa 語基]{sa 語基 a + 語根 + sa}
\numberParagraph
sa 語基に屬する動詞は a 又は ā 以外の母音を有して
ś, ṣ, h に終れるものを以てする。語尾變化は á 級の第一過去
に同じ(\ref{np:64} 條)。但し一單爲自は e でなく i に終り,二三兩は
athām, ātām である。

diś (指す)~一單爲他 adikṣam, 三複 adikṣan, 一單爲自
adikṣi.

\subsubsection[s 語基]{s 語基 a + 語根 + s}
\numberParagraph
以下の三種は語基に强弱を分つ。この點よりして現在組
織の第二種變化に相當する。强基は爲他に用ひられ,弱基中基は
爲自に用ひられる。

\numberParagraph
s 語基は爲他には複重韻,爲自には i, ī, u, ū に終る語根は
ṣam, aśroṣi, kṛ (作す)~ akārṣam, akṛṣi, dṛś (見る)~三,單,
爲他 adrākṣīt (\ref{np:9} 條),bhaj (分つ)~一,單は abhākṣam,
abhakṣi. t 又は th にて始まる語尾の次前,鼻音にあらざる子音
の次後の s は省き去られる。tud (打つ)~二,複,爲自 atautta.
kṣip (投ぐ)~ akṣaipta. 然し man (思ふ)~三,單,爲自
amaṃsta. kṛ (作る)~二,複,爲自 akārṣṭa. dhvam の前に s
は消失する。若し ā 以外の母音に先立たれた時 s は消失して
dhvam は ḍhvam となる。卽ち akṛḍhvam. nī (導く)~
aneḍhvam.

\numberParagraph
nī (導く)强基 nais, 弱基 nes.

\begin{center}
\begin{tabular}{c*{3}{p{0.23\hsize}}}
  \multicolumn{4}{c}{爲他} \\
     & 單      & 兩       & 複 \\
  1. & anaiṣam & anaiṣva  & anaiṣma \\
  2. & anaiṣīs & anaiṣṭam & anaiṣṭa \\
  3. & anaiṣīt & anaiṣṭām & anaiṣus \\
  \multicolumn{4}{c}{爲自} \\
  1. & aneṣi    & aneṣvahi & aneṣmahi \\
  2. & aneṣṭhās & aneṣṭhās & aneḍhvam \\
  3. & aneṣṭa   & aneṣṭa   & aneṣata
\end{tabular}
\end{center}

\subsubsection[iṣ 語基]{iṣ 語基 a + 語基 + iṣ}
\numberParagraph
母音に終る語根の時は爲他に複重韻,爲自に重韻となる。
lū (切斷す)~alāviṣam, alaviṣi. 又語根に a を有して單子音を
以て終る時は爲他を隨意に複重韻となすを得る。vad は然し恒に
複重韻に作る。grah (取る)は語基を grahī に作る。a 以外の
母音を有する語根は爲他,爲自共に重韻に作る。paṭh (學ぶ)~
三,單,爲他 apāṭhīt 或は apaṭhīt. vad は只 avādīt. grah ~
agrāhīt. 爲自~ agrahīṣṭa, budh (覺る)~ abodhīt, abodhiṣṭa.

\subsubsection[siṣ 語基]{siṣ 語基 a + 語根 + siṣ}
\numberParagraph
siṣ 語基は只爲他の形のみが用ひられる。語根は多くは
ā に終り而も \ref{np:204} 條に屬せざるもの,並に e, o, ai 等に終るも
のである。然しこれらは siṣ と會して皆 ā となる。yā (行く)~
ayāsiṣam, glai (疲困す)~ aglāsiṣam. 他に nam (曲る),yam
(止める),ram (樂む)は例外としてこれに屬す。nam ~ anaṃ\-%
siṣam, anaṃsīs, anaṃsīt; anaṃsiṣus.

\subsection{願望法}
\numberParagraph
可能法の第三過去は常に祝福祈願の意を示す。語尾は爲
他 yāsam, yās; yāsva, yāstam, yāstām; yāsma, yāsta,
yāsus. 爲自 īya, īṣṭhās, īṣṭa; īvahi, īyāsthām, īyāstām;
īmahi, īdhvam, īran. bhū (有り)~ bhūyās (汝をしてあらしめ
む),bhūyāt (彼をしてあらしめむ),dā (施す)~ deyāt (彼をし
て施さしめむ),pā (守る)~ pāyāt (彼をして守らしめむ),sthā
(住す)~ stheyās (汝をして住せしめむ),kṛ (作る)~ kriyāt (彼
をして作らしめむ),diś (示す)~ diśyāt (彼をして示さしめむ)。
又次の如き特殊の形がある。bhūyāstam (彼等兩人をしてあらし
めむ),brū (言ふ)~ brūyāsta (汝等をして云はしめむ),pāyāsus
(彼等をして守らしめむ),bhūyāsus (彼等をしてあらしめむ),
vi + dhā ~ vidhāsīṣṭa (彼をして爲さしめむ)。

\subsection{受動第三過去}
\numberParagraph
受動第三過去は三,單,爲自の外は總て能動の語基に爲自
の語尾を附加せるものに同じ。三,單,爲自は語根に i を附加して
作る。この際語根中の母音は重韻となり,語根の終にある母音は
複重韻となる。

\begin{center}
\begin{tabular}{lll}
       & nī (導く)~ anai + i = anāyi  & (彼は導かれたりき)。\\
       & lū (斷つ)~ alau + i = alāvi  & (彼は斷たれたりき)。\\
       & diś (示す)~ adeś + i = adeśi & (彼は示されたりき)。\\
  然し & pad (落つ)~ apād + i = apādi & (彼は落されたりき)。\\
       & dā (與ふ)~ adāy + i = adāyi  & (彼は與へられき)。
\end{tabular}
\end{center}

\ex{第十六}
\begin{longtable}{c*{2}{p{0.45\hsize}}}
 1. & sa pārthivaḥ kadācin mṛga\-yām agamat. & 彼王はある時狩獵に行けり。\\
 2. & ajani te vai putraḥ. & 汝に一人の子は生まれたり。\\
 3. & tad ahaṃ tubhyam eva dadā\-mi ya eva satyam avādīḥ. & 眞理をのみ語れる汝にまで
 我は其れを與ふ。\\
 4. & ud u śriya uṣaso rocamānā asthuḥ. & 曉の光は輝き始めたり。\\
 5. & Śrīnagarān niragāt paṇḍitaḥ. & スリナガルより一人の賢者が出立せり。\\
 6. & astam ayāsīd ravis timireṇā\-vṛtaṃ nabhaḥ. & 日は沈み行けり,天空は暗もて覆われたり\endnote{「暗もて」はママ。}。\\
 7. & bhoḥ prohita bhavad-anujñām anusṛtya baṭave 'haṃ sāvitrīm upādikṣam.
 & おゝ家庭僧よ汝の命に從ひ我は子にまでサーヴィトリー頌を敎へたり。\\
 8. & veṇu-dhamanyāgnim adhmā\-siṣam, tad asmin pradīpte vahnāv āhutīḥ prāsya.
 & 我は竹の管もて火を吹けり,さればこの燃ゆる火の中に供物を投ぜよ。\\
 9. & nitya-karmānuṣṭhānāyāsnāsīs tac chūdrādīn mā spṛkṣaḥ. & 常時の勤行をなさんがため
 に汝は沐浴せり。されば組陀羅等に觸るる勿れ。\\
10. & kiṃ yūyam avocata, punar api kathayata, nāham avahito 'bhūvam.
& 汝等は何を云へりしか。復び語るべし。我は注意せざりき。\\
11. & idam āmra-phalaṃ vṛkṣād apaptat, yadi rocate gṛhītvā svādasva. & この\ruby[g]{檬果}{マンゴー}は樹より落ちた
り。若しそれが宜しからば執りて味へ。\\
12. & prātar ārabhya pañca-sapta\-tiṃ vṛkṣān asicāma. & 我れ今朝より始めて七十五
本の樹を灌水せりき。\\
13. & krīḍārtham upavanam aga\-matāṃ daṃpatī. & 遊戯のために夫妻は林に行けり。\\
14. & iyaṃ bālikā duḥkhavārttāṃ śrutvāmuhat, āśvāsayainām udakena ca siñca.
& この少女は悲しき出來事を聞きて失神せり。彼女を回復せしめ水を灌げ。\\
15. & ye māṃ rūpeṇa cādrākṣur, ye māṃ ghoṣeṇa cānuvaguḥ, mithyā-praharaṇa-prasṛtā na
māṃ drakṣyanti te janāḥ. & 色を以て我を見,聲を以て我を求むる所の彼等の人々は邪行を行ずるものに
して我を見ざるなり。\\
16. & tataḥ sa gardabhaṃ laguḍena tāḍayāmāsa; tenāsau pañcatvam agamat. & そこで彼は杖を以て驢馬を
打つた。かくてそれは死んだ。\\
17. & jyog vā iyam Urvaśī manu\-ṣyeṣv avatsīt. & このウルヷシーはきつど人間の中に長らく住んでゐた。\\
18. & yad idānīm dvau vivadamā\-nau eyātām aham adarśam aham aśrauṣam iti ya eva brū\-yād aham adarśam iti tasmā
eva śraddadhyāma. & 今若し二人が(一方は)我は見た(他のものは)我は聞いたと互に云ひ爭ひつゝ
來るとせんに我々は我は見たと云ふ彼に信賴を置くべきだ。
\end{longtable}

%%% Local Variables:
%%% mode: latex
%%% TeX-master: "IntroductionToSanskrit"
%%% End:


\newpage
\theendnotes


%%% Local Variables:
%%% mode: latex
%%% TeX-master: "IntroductionToSanskrit"
%%% End:

\chapter{文章法}
\section{總說}
\numberParagraph
梵文學はその大部分が詩で出來てゐる關係上,文章は原
始的で未開拓であるとも云へる。その特徴は同格や動詞狀名詞が
幅を利かせて接續句などに取つて替つてゐることである。それに
間接語法は全然缺如してゐる。その他の特徴としては定動詞形が
極めて稀である。吠陀ではそうでもない。定動詞の代りに過去分
詞とか動詞狀名詞が用ひられる。又受動の文章が好んで用ひられ
る。特に注意すべきは獨立於格の使はれることである。

\subsection{言語の順序}
\numberParagraph \label{np:217}
言語の排列は梵語では第一に主格とその屬性(屬格は主
格に先立つ),次に目的並にその從屬物(それは目的に先つ),而
して最後に動詞である。

副詞卽ち賓辭の擴大されたものは通常最初の方に置かれ,かく
て强調せられない接續的の小辭が最初の語に續く。例

\indent{Janakas tu satvaraṃ svīyaṃ nagaraṃ jagāma.}

\indent{「然しジャナカは急いで彼自身の都城に行けり」。}

若し呼格があれば一般に最初に來る。主語の代りに如何なる他
の語でも强調せらるゝものは最初に置かれる。例

\indent{rātrau tvayāmaṭha-madhye na praveṣṭavyam.}

\indent{「夜に於いて汝は寺院に入つてはならぬ」。}

\begin{enumerate}[label=(\alph*), ref=\alph*]
\item \label{item:217a} 特に强調の必要なくば主語は代名詞で書かぬことになつ
てゐる。定動詞の中に含まれてゐる一般の主語,卽ち人はとか世
人はとか云ふものも動詞だけで示される。例 brūyāt 「人は云ふ
べし」。 āhuḥ 「世人は云ふ」。
\item asti なる助動詞は必要なければ一般に略される。その場
合には賓辭が名詞に先立つのが通常だ。śītalā rātriḥ 「夜は冷か
である」。若し賓辭が强調される必要あらば bhavati が用ひられ
る。例 yo vidyayā tapasā janmanā vā vṛddhaḥ sa pūjyo
dhavati dvijānām 「智慧と行力と門地を以て勝れたる彼は再生
族中に尊敬せられてある」。
\item 屬性が名詞に先立ち,合成語に於いて限定の語が最初に來
るやうに,關係的接續句は主文章に先立つ。而してそれは規則正
しく相關詞で始められる。 yasya dhanaṃ tasya balam 「富める
所の彼には力あり」。同樣に yadā --- tadā, yāvat --- tāvat 等。
\end{enumerate}

\subsection{數}
\numberParagraph \label{np:218}
\begin{enumerate}[label=(\arabic*)]
\item 單數の集合を意味する語が合成語の終に用ひられる。
strī-jana (男)「婦女」。\ruby{恁}{こ}うした集合名詞は時としてそのまゝ複
數に使はれる。lokaḥ 又は lokāḥ 「世界は」,「人々は」。
\item 兩數は嚴格に規則正しく使用せられ,二個のものを複數に
することは決して無い。故に兩數は必ず一對をなすものである。
例へば身體の部分 hastau pādau ca 「手と足と」。男性の兩數が時
として同群の雌雄を意味する。jagataḥ pitarau 「世界の父母」。
\item \label{item:2183}
\begin{enumerate}[label=(\alph*), ref=\alph*]
\item \label{item:2183a} 時として複數が尊敬を表はすこともある。tvam 及び
bhavān と云ふよりも yūyam 及び bhavantaḥ の方が一層恭し
い云ひ方である。例 śrutaṃ bhavadbhiḥ 「陛下は御聽き遊ばせ
しか」(これは單數の意である)。この意味で複數 pādāḥ が兩數の
代りに使はれる。例 eṣa devapādān adhikṣipati 「彼は陛下の御
足を汚し奉つた」。固有名詞が時として同樣に用ひられる。例 iti
śrī-Śaṅkarācāryāḥ 「かく聖者シャンカラ師は云へり」。
\item 一人稱複數が時として說者自身のことを意味することが
ある。vayam api kiṃcit pṛcchāmaḥ 「我れも亦或ることを尋ぬ
べし」。kiṃ kurmaḥ sāṃpratam 「何を我々(=汝と我)は今爲
すべきか」。
\item 國名は複數である。事實は所屬國民の名なのである。例
Vidarbheṣu 「ヴィダルブハに於いて」。
\item 或る名詞は常に複數である。āpaḥ (女)「水」;prāṇāḥ 「生
命」,varṣāḥ 「雨」(雨季のこと),dārāḥ (男)「妻」。
\end{enumerate}
\end{enumerate}

\subsection{用例の一致}
\numberParagraph \label{np:219}
格例,人稱,性,數の一致は語尾曲法を有する言語に於
て一般に同じであるが次の諸點が注意さるべきものである。
\begin{enumerate}[label=(\arabic*)]
\item \label{item:2191} 主格が iti と伴つて呼稱,考慮,認知等の動詞に支配さ
れた賓辭的な業格の位置を取る。例 brāhmaṇa iti māṃ viddhi
「我を婆羅門であると知れ」(brāhmaṇaṃ māṃ viddhi とする
代りに)。
\item 兩數又は複數の動詞が二個又はそれ以上の主體に關係す
る時は,二人稱,三人稱よりも寧ろ一人稱,三人稱よりも寧ろ二人
稱を用ふ。例 tvam ahaṃ ca gacchāvaḥ 「汝と我れとは行く」
\item
\begin{enumerate}[label=(\alph*)]
\item 男性女性名詞に一致する兩數若くは複數の形容詞は
男性に,然し若し中性が男性女性に伴ふ時は中性にする(時とし
て單數とする)。例 mṛgayākṣās tathā pānaṃ garhitāni mahī\-%
bhujām 「狩獵と賭事と飲酒とは王にとつて避難さるべきもので
ある」。pakṣa\-vikalaś ca pakṣī śuṣkaś ca taruḥ saraś ca
jala-hīnaṃ sarpaś coddhṛta-daṃṣṭras tulyaṃ loke daridraś ca
「羽を切られた鳥,枯れた木,干上がつた池,齒を拔かれた蛇,そ
れから貧乏人は世の中に同等のものである」。(tulyam が中性單
數であることに注意)。
\item 時としては屬性や賓辭が文法的の性でなく自然の性をも
つ。tvāṃ cintayanto nirāhārāḥ kṛtāḥ prajāḥ 「汝のことを考
へつゝ(男性)臣下達は(女性)食事をする氣もなくなつた」。
\item 指示代名詞は賓辭と性に於いて一致する。例 asau para\-%
mo mantraḥ 「これは(男)最上の意見(男)だ」。

主格に一致せねばならぬ筈の定動詞の代りに用ひられた分詞が
性に於いて近くにある名詞的賓辭に引き着けられる。例 tvaṃ
me mitraṃ jātam 「汝(男)は我が友(中)となれり(中)」。
\end{enumerate}
\item 單數集合名詞は必然的に單數動詞に從はれる。二個の單
數の主語は兩數の賓辭を要し,三個以上のものは複數の賓辭を要
する。然し時には賓辭が最も近い主語と數に於て一致し,その他
のものに對しては腹で積つて補ふことになる。Kāntimatī rā\-%
jyam idaṃ mama cajīvitam api tvad-adhīnam 「カーンティ
マティーとこの王國と我が生命に至るまでも汝の意(單)に任す」。
\item 同樣に一つの複數主語に一致すべき筈の動詞がその
最も近い名詞賓辭に數の上で引き着けられることもあり得る。
sapta-prakṛtayo hy etāḥ samastaṃ rājyam ucyate 「これら
七の構成部分は全王國なりと云はれる(單)」。
\end{enumerate}

\subsection{代名詞}
\numberParagraph
\begin{enumerate}[label=(\arabic*)]
\item 人稱代名詞。
\begin{enumerate}[label=(\alph*)]
\item 動詞の語尾變化から推測し得る
性質上梵語に人稱代名詞を用ふることは若干節約せられる傾向に
ある(\ref{np:217} \ref{item:217a})。
\item 一人稱,二人稱代名詞の附帶詞(語勢なきもの)は文章の
始に立たない。又呼格の後とか ca vā eva ha の如き小辭の前に
も立つを許されない。例 mama mitram 我が友(me mitram と
は云はぬ),devāsmān pāhi 「神よ,我々を守れ」(asmān で
naḥ とは云はぬ)。tasya mama vā gṛham 「彼若くは我が家\endnote{底本では「我が家」ではなく「我がが家」。}」。
\item bhavān 「貴殿」(女性 bhavatī)は tvam 「汝」の敬稱形
であつて(同じ文章中でさへ交替に用ふ)三人稱の動詞を伴ふ。
例 kim āha bhavān 「貴下は何を仰せられますか」。複數 bha\-%
vantaḥ (女性 bhavatyḥ)同樣に解釋される。それは屢々單數の
意味を持つ(\ref{np:218} \ref{item:2183} \ref{item:2183a})。bhavān の二種の合成語が戯曲に見える。
atra-bhavān は呼びかけられた人でも第三者としての人でも或る
そこに居る人に係る。「此に在す貴殿」と云ふことである。tatra\-%
bhavān 「あの貴殿」は舞臺にゐない或る人に係り第三者のこと
に限る。共に三人稱單數の動詞を伴ふ。
\end{enumerate}
\item \textbf{指示代名詞。}
\begin{enumerate}[label=(\alph*)]
\item eṣa 及び ayam 「此れ」は近き又そこに
ゐるものに係る。前者の意味一層强し。共に一,三,單に於て主
語に一致して使はれ「此に」の意味がある。例 eṣa tapasvī
tiṣṭhati 「此に行者がゐる」,ayam asmi 「此に我あり」,ayam
āgatas tava putraḥ 「此に汝の子は來れり」,ayaṃ janaḥ 「此
の人」なる形は屢々「我」と同意味に用ひられる。
\item sa 「彼」及び asau 「それ」は居ない遠方のものに係る。sa
はこの中でも一層限定された指示代名詞で,例へば前に叙べた
ものに關係がある。次の如き特殊な用例がある。それは譯すれば
「あの有名な」とか「例の話の」とのとか云ふ意味が加はる。例
sā ramyā nagarī 「あの有名な美しい都城」。それは「前に云つ
た」と云ふことにもなる。例 so 'ham 「前に云つたやうなさうし
た私は」。名詞に伴はれない場合は三人稱代名詞として役立つ。
ayam も asau も同樣に使はれる。最後に tad が繰り返される時
には「種々の」「多樣の」「總ての種類の」と云ふ意味となる。例
tāni tāni śāstrāṇy adhyaita 「彼は種々の論書を讀めり」。
\end{enumerate}
\item 所有代名詞。īya なる後接字を附加して madīya (私の),
tvadīya (汝の)の如きものが作られる。然しこの用例は人稱代
名詞の屬格が用ひられるから比較的少い。bhavat から bhava\-%
dīya, bhavatka と云ふ二人稱の敬稱を表はす所有代名詞が作ら
れてゐる。
\end{enumerate}

\section{格の用法}
\subsection{主格}
\numberParagraph
アーリアの言語の中では梵語は文章中の主語として主格
をあまり使はない傾向にある。その代りに具格が受動の動詞と共
に用ひられる。例 kenāpi sasya-rakṣakeṇaikānte sthitam 「あ
る穀物守護者によりて一方に於いて立たれた」。この意味はある
穀物守護者が一方に立つてゐたと云ふことに歸する。

\numberParagraph
主格は「である」「と成る」「と見える」と云ふ意味の動
詞と共に又は呼稱,思惟,送致,造作の意味の動詞の受動と共
に說明的に用ひられる。例 tena muninā kukkuro vyāghraḥ
kṛtaḥ 「狗は聖者によりて虎となされたり」。iti によりて從はれ
たる主格は或る場合には業格の代用をなす。\ref{np:219} 條\ref{item:2191}。

\subsection{業格}
\numberParagraph
通常他動詞の目的となるのであるが,其の他に業格には
次のやうな用法がある。
\begin{enumerate}[label=(\arabic*)]
\item 動作の到着點。例 sa Vidarbhān agamat. 「彼はヴィダ
ブㇵへ行けり」。
\
\begin{enumerate}[label=(\alph*)]
\item gam, yā の如き「行く」と云ふ意味の動詞は通常抽象名
詞と結び付き「……となる」若くは自動詞で表はされるやうな意
味となる。例 sa kīrtiṃ yāti. 「彼は名譽に行く」(有名となれ
り),pañcatvaṃ gacchati. 「彼は死へ行く」(彼は死せり)。
\end{enumerate}
\item 時の繼續,場所の擴がりを意味する。例 māsam adhīte
「彼は一箇月間學べり」。yojanaṃ gacchati 「彼は一由旬行け
り」。
\end{enumerate}

\numberParagraph
\begin{enumerate}[label=(\arabic*)]
\item 二重業格は呼稱,思惟,認知,造作,選定の意義を
有する動詞に支配せられる。例 jānāmi tvāṃ prakṛti-puru\-%
ṣam 「我は汝を主なる人物なりと知れり」。
\item 其他語る (brū, vac, ah), 問ふ (prach), 乞ふ(yāc,
prārthaya), 敎ふ (anuśās), 罰する (daṇḍaya), 勝つ (ji),
搾る (duh) の意味を有する動詞。例 antarikṣago vācaṃ vyā\-%
jahāra Nalam 「鳥はナラへ語を發せり」,sākṣyaṃ pṛcched
ṛtaṃ dvijān 「彼は眞の證據を婆羅門達に尋ねたり」,baliṃ yācate
vasudhām 「彼大地に對して供物を求む」。yad anuśāsti mām 「彼
が私に命ずる所のそのもの」,tān sahasraṃ daṇḍayet 「彼は一
千(パナ)を彼等に課すべし」,jitvā rājyaṃ Nalam 「ナラより
王國を勝ち取りて」,ratnāni duduhur dharitrīm 「彼らは寶玉を
大地から搾れり」。
\begin{enumerate}[label=(\alph*)]
\item kathaya (吿ぐ),vedaya (知らしむ)及び ā-diś (命ず)
は話しかけられた人に對して業格の形を取らず。爲格(又は屬格)
となる。
\end{enumerate}
\item 持來る,運ぶ,導く,送るの意味を有する動詞。例 grā\-%
mam ajāṃ nayati. 「彼は山羊を村に持ち行く」,Śakuntalāṃ
patikulaṃ visṛjya 「シャクンタラーを夫の家に送つて」。
\item 催起動詞。例 Rāmaṃ vedam adhyāpayati 「彼はラー
マをして吠陀を學ばしむ」。若しも行爲者に强調の必要ある時は
それを具格となすことを得。tāṃ śvabhiḥ khādayet 「彼女を狗
に食はしむべし」。
\end{enumerate}

\subsection{具格}
\numberParagraph
具格は「……によりて」又は「……と共に」を以て表さ
れるがその根本的槪念は動作が行はれる行爲者,作具(方法)又
は並立者を示す。例 tenoktam 「彼によりて云はれたり」(彼は云
へり),sa khaḍgena vyāpāditaḥ 「彼は劍を以て殺されたり」。
yasya mitreṇa saṃlāpas tato nāstīha puṇyavān 「友と語る
より幸福なるもの世に無し」。

\numberParagraph \label{np:226}
次に示す所は具格の意味の變形である。
\begin{enumerate}[label=(\arabic*)]
\item \label{item:2261} 理由。例 bhavato 'nugraheṇa 「汝の恩惠の故に」,
tenāparādhena tvāṃ daṇḍayāmi 「我れその過失の故に汝を罰
す」,vyāghrabuddhyā 「虎の思想によりて」(彼がそれは虎なり
と考へたが故に),sukha-bhrāntyā 「幸福の妄想の故に」。
\item 一致。例 prakṛtyā 「性質上」,jātyā. 「生れながらに」,
sa mama matena vartate. 「彼は我が意見によりて(一致して)
動作す」。
\item 値段。例 rūpaka-śatena vikrīyamāṇaṃ pustakam 「一
百ルーピーで賣られる書物」,ātmānaṃ satataṃ rakṣed dārair
api dhanair api 「人は常に妻や財產を投じても自己を守らねば
ならぬ」。
\item 時間の繼續。例 dvādaśair varṣair vyākaraṇaṃ śrū\-%
yate 「文典は十二年間に學ばれる」。
\item 動作がなされる方法,媒介物,又は身體の一部分。例
katamena mārgeṇa pranaṣṭāḥ kākāḥ 「鴉は何の方法に於て消
え失せしや」,vājinā carati 「彼は馬によりて行けり」,sa
śvānaṃ skandhenovāha 「彼は狗を肩で運べり」。
\item 優性,劣性又は缺如を意味する語と共に「……に關して」。
例 etābhyāṃ śauryeṇa hīnaḥ 「彼等二人より(從格)勇氣劣れ
り」,pūrvān mahābhāga tayātiśeṣe 「お,幸福なるものよ,汝
はそれを以て祖先を凌駕す」,akṣṇā kāṇaḥ 「一目眇たり」。
\item 必要,用途 arthaḥ prayojanam (疑問又は否定に用ひら
れる)或は kim (何)(kṛ と共に若くは単獨に)等の語と伴ひて
「……について」「……を以て」と譯すべきもの。例 deva-pādā\-%
nāṃ sevakair na prayojanam 「陛下の足に對し奴僕を以て要
なし」,kiṃ tayā kriyate dhenvā 「その牝牛を以て何が爲され
るか」,kiṃ na etena 「我々はこれを以て何かせむ」,kṛtam,
alam 等の用例もこれと同樣である。kṛtam abhyutthānena
「起立する勿れ」。
\item \label{item:2268} 並列又は倶存の意味は副詞 saha, sākam, sārdham, sa\-%
mam を補つて表はされる。時には分離,對立が示される。例
putreṇa saha pitā gataḥ 「父は子供と共に行けり」。mitreṇa
saha citta-viśleṣaḥ 「友との中たがひ」,sa tena vidadhe samaṃ
yuddham. 「彼は彼と戰に從事せり」。この意義が適用されて更に
\begin{enumerate}[label=(\alph*)]
\item 同伴の狀態を表はすもの。例 tau dampatī mahatā
snehena vasataḥ 「夫妻は深き愛を以て住せり」,mahatā
sukhena. 「大なる幸福を以て」。
\item \label{item:2268b} 伴隨,結合,賦與,具有,對立の意味の動詞の受動と伴
ふもの。例 tvayā sahitaḥ 「彼に伴はれて」,dhanena saṃ\-%
panno vihīno vā 「富を具有せる又は缺如せる」,prāṇair viyu\-%
ktaḥ 「生命なき」。
\item 等同類似の意味の形容詞 sama, samāna, sadṛśa, tulya
と伴ふもの。例 Śakreṇa samaḥ 「インドラに等しき」,anena
sadṛśa 「彼の如き」,ayaṃ na me pāda-rajasāpi tulyaḥ 「彼
は我が足の塵にも及ばず」。屬格もこれらの形容詞と共に用ひら
れる。
\end{enumerate}
\end{enumerate}

\subsection{爲格}
\numberParagraph
爲格は間接目的(通常は人)か若くは動作の目的を表は
す。
\begin{enumerate}[label=(\Alph*)]
\item 関節目的の爲格は
\begin{enumerate}[label=(\arabic*)]
\item 直接目的を有して若くはなしに次の他動詞と共に用ひら
れる。
\begin{enumerate}[label=(\alph*)]
\item 與へる (dā, arpaya), 吿げる (cakṣ, śaṃs, kathaya,
khyāpaya, nivedaya) 約束する (prati 又は ā-śru, prati-jñā),
示す (darśaya)。例 viprāya gāṃ dadāti 「彼は牝牛を婆羅門
に與ふ」,kathayāmi te bhūtārtham 「我は汝に眞實を吿ぐ」。
\item 送る,投ぐ。例 Bhojena dūto Raghave visṛṣṭaḥ 「使は
ボージャによりてラグフに送られた」。 śūlāṃś cikṣipū Rāmāya
「彼らは鎗をラーマに投げたり」。
\end{enumerate}
\item 自動詞にして喜ぶ (ruc), 願ふ (lubh, spṛh), 怒る (asūya,
kup, krudh), 害する (druh)。例 rocate mahyam 「我にまで\ruby[S]{樂}{ねが}
はし」(そは我を喜ばす),na rājyāya spṛhaye 「我は王國を願は
ず」,kiṃkarāya kupyati 「彼は奴僕を怒る」(krudh 及び druh
が前加語を有すれば業格を支配す)。
\item 挨拶の語と共に。例 Gaṇeśāya namaḥ 「ガネーシャに
歸敬す」,kuśalaṃ te 「汝にまで健康あれ」,Rāmāya svasti 「ラ
ーマに幸福あれ」,svāgataṃ devyai 「王后にまで歡迎」。
\end{enumerate}
目的の爲格は動作のなされた目的を表はし屢々不定法と
同意義を有す。例 muktaye Hariṃ bhajati 「解脱のためにハ
リを崇拜す」。phalebhyo yāti 「彼は果實のために行く」(果實を
得んために)。asmatputrāṇāṃ nīti-śāstropadeśāya bhavantaḥ
pramāṇam 「貴下は倫理書の敎訓に關して吾が子等の證權であ
る」,yuddhāya prasthitaḥ 「彼は戰爭に出立つた」,punar
darśanāya 「再會を期して,さよなら」。
\begin{enumerate}[label=(\arabic*)]
\item 「適當である」,「……に歸す,(或は)に資す」(kḷp, saṃ-pad,
pra-bhū)。例 bhaktir jñānāya kalpate 「信仰は知識に資す」。
\begin{enumerate}[label=(\alph*)]
\item as と bhū が同樣の方法で用ひられる。然し屢々省略せ
られる。例 laghūnām api saṃśrayo rakṣāyai bhavati 「弱き
ものゝ附着は保護に資す」,ārta-trāṇāya vaḥ śastram 「汝の
武器は苦しめられたるものゝ救ひに役立つ」。
\end{enumerate}
\item 能ふ,始む,努む,決心す,命ず,任ず。例 iyaṃ kathā
kṣatriyasyākarṣaṇāyāśakat 「この物語は勇士の心を引くに堪
ふ」,prāvartata śapathāya 「彼は誓を始めた」,tad-anveṣaṇāya
yatiṣye 「我はそれを求めんと努力すべし」,tena jīvotsargāya
vyavasitam 「彼は彼の命を捨てんと決心せり」,duhitaram
atithi-satkārāyādiśya 「娘に賓客の接待を委かせて」,Rāvaṇo\-%
cchittaye devair niyojitaḥ 「ラーヷナの討滅にと神々によりて任
命せられた」。
\begin{enumerate}[label=(\alph*)]
\item 副詞 alam (十分に)は「……と匹敵す」,「……と相競ふ」
の意に用ひられる。例 daityebhyo Harir alam 「ハリは惡魔
に敵對す」。
\end{enumerate}
\end{enumerate}
\end{enumerate}

\subsection{從格}
\numberParagraph
從格は元來何かゞ進出する所の出發點若は源泉地を表は
すものである。それは「何處から」と云ふ問に答へる。隨て一
般に「から」と譯される。例 aham asmad vanād gantum
icchāmi 「私はこの林から出立せんと欲する」,pāpān nāśa ud\-%
bhavati. 「罪過より破滅は生ず」,niścayān na cacāla saḥ 「彼は決
心から動かざりき」,sva-janebhyaḥ suta-vināśaṃ śuśrāva 「彼は
親族から子の死を聞けり」,tāṃ bandhanād vimucya 「彼女を束
縛から解放して」,virama karmaṇo 'smāt 「この行爲を止めよ」,
pāhi māṃ narakāt 「我を地獄から護れ」(地獄に墮ちぬやうに)。
\begin{enumerate}[label=(\alph*)]
\item \ruby{危懼}{き|く}の原因は恐怖の動詞 (bhī, ud-vij) と共に從格にす
る。例 lubdhakād bibheṣi 「汝は獵師を恐る」,saṃmānād
brāhmaṇo nityam udvijeta 「婆羅門は常に尊敬を受くることを
恐るべきだ」。
\item 別離を表はす動詞は自然に從格を取る。例 bhavabdhyo
viyojitaḥ 「汝から別れた」,sā pati-lokāc ca hīyate 「而して彼女
は夫の處から離されてゐる」(この種の語は具格にもなる)(\ref{np:226}條
\ref{item:2268} \ref{item:2268b})。これと連絡して vañcaya (欺く)の用例(結局は正しい狀
態から離れることである)。vañcayituṃ brāhmaṇaṃ chāgalāt
「婆羅門を山羊で欺かんと」。
\item 從格は起點 (terminus a quo) を表はすものだから遠
方を意味する總ての語及び方位の名稱に伴ふものである。例
dūraṃ grāmāt 「遙かに村から」,grāmāt pūrvo giriḥ 「山は村
の東方にあり」。
\item 同樣に從格は又何かゞ起つた後の時間を表はす。例
bahor dṛṣṭaṃ kālāt 「長い時の後それは見られたり」,saptāhāt
「七日の後」。
\item ārabhya (……より始めて),prabhṛti (以來)等の語に
伴ふ。例 tataḥ prabhṛti 「それより以來」。又 ā なる前置詞に
伴ふ。例 ā nagarāt 「市城に至るまで」,ā mūlāt 「根本より」。
\end{enumerate}

この根本的な意義から次のやうな意義を表はす。
\begin{enumerate}[label=(\arabic*)]
\item 原因,理由,動機。例 laulyād māṃsaṃ bhakṣayati
「彼は貪慾の故に肉を食ふ」。この用法では特に tva を附加した
抽象名詞で註釋などがなされるのを普通とする。例 parvato
'gnimān dhūmatvāt 「山は燃えてあり,烟あるが故に」(具格もこ
の意味に用ひられる(\ref{np:226} \ref{item:2261})。
\item 比較。
\begin{enumerate}[label=(\alph*)]
\item 比較を表はす例 Govindād Rāmo vidvat\-%
tarah 「ラーマはゴーヴィンダよりも學識あり」。karmaṇo jñānam
atiricyate 「知識は行爲よりも優れたり」。この意味で原級の言ひ
表はし方もある。例 bhāryā sarva-lokād api vallabhā bhavati
「妻は全世界と比しても親愛なり」。
\item 「他の」,「異れる」と云ふ意味の語 (itara, apara, bhinna,
pṛthak) と共に。例 Kṛṣṇād anyo Govindaḥ 「ゴーヴィンダは
クリシュナに異る」。
\item 比較を表はす從格に連關して「二倍」とか「三倍」とか
の語と共に用ひられることがある。例 mūlyāt pañca-guṇo
daṇḍaḥ 「價よりご倍の罰金」。
\end{enumerate}
\end{enumerate}

\subsection{屬格}
\numberParagraph \label{np:229}
屬格の本來の意味は疑似的形容詞であり,他の名詞を限
定して「……に屬する」とか「……に關する」とかの意義を有す
る。伴ふ名詞に對してその用法から所有的,主觀的,客觀的,表
分的の區別がある。例 rājñaḥ puruṣaḥ 「王の臣下」(所有)。
rākṣasa-kalatra-pracchādanaṃ bhavataḥ 「貴下の羅刹女隱匿」
(汝が隱した)(主觀)。śaṅkatā tasyāḥ 「彼女の疑によりて」(彼
女ならむとの疑)(客観)(彼女は目的)。dhuryo dhanavatām
「富める人々の中での首班」(部分を表はす)。
\begin{enumerate}[label=(\arabic*)]
\item 屬格は多くの動詞と共に用ひられる。
\begin{enumerate}[label=(\alph*)]
\item 所有的意義では pra-bhū (……を支配する),as, bhū (有
る),vidyate (存在す)。例 ātmanaḥ prabhaviṣyāmi 「我は自
己を支配すべし」。mama pustakaṃ vidyate 「我が書はあり」。
\item 客觀的意義に於いて(業格と同樣に) day (惠む),smṛ
(念ず),anu-kṛ (擬す)と共に用ひられる。例 ete tava dayantām
「これらの人々をして汝の上に惠あらしめよ」,smarati te pra\-%
sādānām 「彼は汝の惠を念ず」,Bhīmasyānukariṣyāmi 「我はブ
ヒーマに倣ふ」。
\item 客觀的意義に於いて(於格と同樣に)「……に益する又は害
をなす」(upa-kṛ, pra-sad, apa-kṛ, apa-rādh), 「……に信賴す
る」(vi-śvas) 「……を忍ぶ」(kṣam) と共に用ひられる。例
mitrāṇām upakurvāṇaḥ 「友を利しつゝ」,kiṃ mayā tasyā
apakṛtam 「如何に我によりて彼女に害をなせしか」,kṣamasva
me 「我を許せ」。
\item \label{item:229d} 「……に就て語る」,「……を期待す」なる意義の動詞に伴
ふ。例 mamādoṣasyāpy evaṃ vadati 「彼は過失なき我に就て
すらかく語る」,sarvam asya mūrkhasya saṃbhāvyate 「一切
はその愚者に對して期待せらる」。
\item 屢々(爲格の間接目的の代りに)贈與,報知,約束,呈示,
送致,禮敬,歡悅,憤怒を意味する動詞と共に用ひられる。mayā
tasyābhayaṃ pradattam 「我れ彼に無畏を與へたり」,kiṃ tava
rocata eṣaḥ 「彼は如何に汝を喜ばすか」。mamānatikruddho
muniḥ 「聖者は甚しく我に不機嫌でない」。
\item 時として(業格の代りに)「滿足す」「飽く」を意味する動
詞と共に用ひられる。例 nāgnis tṛpyati kāṣṭhānām 「火は材
木にて滿足せしめられない」。
\end{enumerate}
\item 屬格は屢々形容詞と共に用ひられる。
\begin{enumerate}[label=(\alph*)]
\item 他動詞の如き役目をなす。例 jarā vināśinī rūpasya
「老齡は美貌を害す」。
\item 又「……に依存する」「……屬する,着く」「……に親愛な
る」の意義あるもの。例 tavāyattaḥ sa pratīkāraḥ 「療法は
汝に依存す」,yat tvayāsya saktaṃ kiṃcid gṛhītam asti tat
samarpaya 「彼のものと汝が執着せる如何なるものもそれを捨
てよ」,ko nāma rājñāṃ priyḥ 「王に親愛なるは抑も誰ぞ」。
\item 「……を熟知せる」「……に通達せる」「……に慣れたる」の
意義あるもの(於格と同樣に)。abhijñaḥ khalv asi loka-vyava\-%
hārāṇām 「汝は實に世の事柄に通じてゐる」,saṃgrāmāṇām
akovidaḥ 「戰に慣れざる」,ucito janaḥ kleśānām 「苦痛に慣れ
たる人々」。
\item 「……の如き」「……等しき」の意義あるもの(具格と同樣に)。
例 Rāmaḥ Kṛṣṇasya tulyaḥ 「ラーマはクリシュナに等し」。
\end{enumerate}
\item 屬格は受動分詞と共に行爲者を表はす。
\begin{enumerate}[label=(\alph*)]
\item 思惟,認知,禮拜を意味する語根から作られた現在の意
義を有する過去分詞。例 rājñāṃ mataḥ 「王に\ruby{嘉}{よみ}せられる」,
vidito bhavān āśrama-sadām iha-sthaḥ 「貴下は此に住せりと
仙者達に知られてゐる」。
\item 未來分詞(具格も同樣に)。例 mama (mayā) sevyo
Hariḥ 「ハリは私に奉仕せらるべし」。
\end{enumerate}
\item 屬格は tas と云ふ方向の副詞と共に用ひられる。例
grāmasya dakṣiṇataḥ 「村の南方に當りて」。時として亦 ena
を語尾とする福祉と伴ふ(業格と同樣に)。例 uttareṇāsya 「この
北方に」。
\item 時の屬格は次のやうな方法で用ひられる。
\begin{enumerate}[label=(\alph*)]
\item 倍加數又は同樣に他の數にして,ある特定の期間に幾度何
かゞ繰り返されしかを表はす。例 śrāddhaṃ trir (\ref{np:118}條)
abdasya nirvapet 「葬祭を年に三度行ふべきだ」,saṃvatsara\-%
syaikam api caret kṛcchraṃ dvijottamaḥ 「婆羅門は年に一度
でも苦行をなすべきだ」。
\item 時間を表はす語は「後」と云ふ意味で,屬格になされる(從
格の場合のやうに)。例 kati-payāhasya 「若干の日の後」,
cirasya kālasya 「長い時の後」。(cirasya だけでもこの意味にな
る)。
\item 名詞と時の表現をもつ屬格の分詞とで「……より以來」と
云ふ意義を有つ。例 adya daśmo māsas tātasyoparatasya
「今日は父の死せしより第十箇月目である」,daśa kalpā anutta\-%
rāṃ samyaksaṃbodhim abhisaṃbuddhasya 「無上なる正等覺を
\ruby[S]{證}{さと}りしより以來十劫を經た」。この構造は獨立屬格に類するもの
である。獨立屬格のことは後段に詳かにする。
\end{enumerate}
\item 二つの屬格は二者の選擇又は差異を表はす。例 vyasana\-%
sya ca mṛtyoś ca vyasanaṃ kaṣṭam ucyate 「惡德と死とは惡德
の方劇甚なりと云はれる」。etāvān evāyuṣmataḥ Śatakratoś ca
viśeṣaḥ 「それだけが御身とインドラとの差異である」。
\end{enumerate}

\subsection{於格}
\numberParagraph
於格は動作の起る場處又は運動の動詞に伴ひ動作の向け
られる場處を示す。前者は大體「……の中に」と譯され,後者は
「……に於いて」と譯される。
\begin{enumerate}[label=(\arabic*)]
\item 次に於格の「何處に」の意味の通常の用例を擧げる。pak\-%
ṣiṇas tasmin vṛkṣe nivasanti 「鳥はその木に棲む」,Vidarbheṣu
「ヴィダルブハに於いて」,ātmānaṃ tava dvāri vyāpādayiṣyāmi
「汝の戸口に於いて我は自殺すべし」,Kāśyām 「カーシーに於い
て」,phalaṃ dṛṣṭaṃ drumeṣu 「果實は木に於いて見られたり」,
āsedur Gaṅgāyām 「彼らはカンガー河に野營せり」,na deveṣu na
yakṣeṣu tādṛg rūpavatī kvacid mānuṣeṣv api cānyeṣu dṛṣṭa\-%
pūrvā 「神々の中にもヤクシャの中にも他の人々の中にもかくの
如き美貌は曾て見られざりき」,mama pārśve 「我の側に」。
\begin{enumerate}[label=(\alph*)]
\item 若し於格が「……の中に」と云ふ意味ある時は表分的屬
格に等しくなる。例 sarveṣu putreṣu Rāmo mama priyata\-%
maḥ 「すべての子等の中にラーマは我が最愛のものである」。
\item 共に住し滯在する所の人は於格の形を取る。例 gurau
vasati 「彼は彼の敎師と共に住む」。
\item 於格は tiṣṭhati (住す)及び vartate (進む)なる動詞と
共に「……に從ふ」「……に應ず」の意を表はす。例 na m
śāsane tiṣṭhasi 「彼は我が命令に從はない」,mātur mate var\-%
tasva 「母の願に應ぜよ」。
\item 於格は原因の結果を表はす。例 daivam eva nṛṇāṃ
vṛddhau kṣaye kāraṇam 「運命こそは人々の繁榮と滅亡の原因
である」。
\item 於格は「摑む」(graph),「結びつく」(bandh)「固着する,
(lag, śliṣ, sañj) 「\ruby{凭}{もた}れる」「依賴又は信用する」(譬喩的に)なる
動詞と共に接觸を表はす。例 keśeṣu gṛhītvā 「髪をつかみて」
pāṇau saṃgṛhya 「手を取りて」,vṛkṣe pāśaṃ babandha 「彼
は縄を木に結べり」,vyasaneṣv asaktaḥ śuraḥ. 「勇士は惡德
に耽らない」,vṛkṣa-mūleṣu saṃśritāḥ 「樹の根に椅れり」,
viśvasiti śatruṣu 「彼は敵に信賴せり」,āśaṃsante surā asyā\-%
dhijye dhanuṣi vijayam 「神々は勝利をその張られた弓に持ち
設けてゐる」。
\item 於格は「……を熟知す」「……に通曉す,練達す」と云ふ
形容詞に伴ふ(屬格と同樣)。例 Rāmo 'kṣadyūte nipuṇaḥ 「ラ
ーマは博戯に巧みである」,nāṭye dakṣā vayam 「我々は劇に練
達す」。
\item 於格は譬喩的に何かある性質狀態が存する人若くは物を
表はす。例 sarvaṃ saṃbhāvayāmy asmin 「我は一切を彼に
期待す」(\ref{np:229} \ref{item:229d}),dṛṣṭa-doṣā mṛgayā svāmini 「獵は王子にとつ
て過惡が見られる」,ārtānām upadeśe na doṣaḥ 「困しめる
ものに對し忠吿するは過失でない」。同樣に言語の意味が說明
されるに當り於格は「……の意義で」と云ふことを表はす。例
kalāpo barhe 「kalāpa と云ふは孔雀の尾と云ふ意味で使はれ
てゐる」。
\item ある動作が起る事情が於格で表はされる。例 āpadi 「不
幸の場合に」,bhāgyeṣu 「幸福に於て」,chidreṣv anarthā
bahulī-bhavanti 「孔隙に於て不幸は增大す」。この一例では於格
が理由を表はす。「弱點があるところで」の意味であり「……に面
して」の意味である,若しも賓辭の分詞が伴ふと獨立於格となる
ものである。
\item 時間の於格,動作の起る時を表はす。これは前の意義の唯
一特殊な適用である。例 varṣāsu 「雨季に於て」,niṣāyām 「夜
に於て」,dine dine 「毎日」。
\item 於格はある事の起つた距離を表はす。例 ito vasati......
adhyardha-yojane maharṣiḥ 「大仙は此處から一由旬半のとこ
ろに住む」。
\end{enumerate}
\item 「何れの點に」といふ疑問に答へる於格は常に落つ,投ぐ,
置くと云ふ動詞並びに業格と同樣に,行く,入る,登る,打つ,齎
らす,送ると云ふ動詞と共に用ひらる。例 bhūmau papāta 「地
上に落ちた」,arau bāṇāt kṣipati 「彼は敵に對して箭を發射
す」,tatraiva bhikaṣā-pātre nidhāya 「その同じ乞鉢に置きて」,
hastam urasi kṛtvā 「胸に手を置きて」,matsyo nadyāṃ
praviveśa 「魚は河に入れり」,samīpa-vartini nagare prasthi\-%
taḥ 「彼は隣りの街へ出立せり」。

この於格の第二義的の用法は次の如くである。
\begin{enumerate}[label=(\alph*)]
\item 動作の向けられ又は關係する所の人若くは物を表はす。
「……の方へ」「……について」「……に關して」の意を有す。例
prāṇiṣu dayāṃ kurvanti sādhvaḥ 「善人は生類に慈愍を行
ふ」,bhava dakṣiṇā parijane 「汝の從者に禮儀正しかれ」,kṣetre
vivadante 「彼等は田地を爭ふ」。
\item 爲格(並に屬格)と同樣に,贈與,報知,約束,購買,賣渡
の意義ある動詞と共に間接目的を表はす。例 sahasrākṣe prati\-%
jñāya 「千眼(のインドラ)に約束して」,śarīraṃ vikrīya dhana\-%
vati 「富人に身體を賣りて」,vitarati guruḥ prājñe vidyām
「師は知識を智慧ある(弟子)に分つ」。
\item 爲格と同樣に,努力,決心,願望,任命,選擇,參加,許
可,可能,適應を意味する語と共に動作の目的を表はす。例
sarvasva-haraṇe yuktaḥ śatruḥ 「敵は一切の奪取を準備せり」,
karmaṇi nyayuṅkta 「彼は事業を任命せり」,patitve varayāmāsa
tam 「彼女は彼を夫と選べり」,asamartho 'yam udara-pūraṇe
'smākam 「彼は我々の口腹を滿すに適せず」。trailokyasyāpi
prabhtvaṃ tasmin yujyate 「三界の主も彼に適當なり」。單
に賓辭的に於格を用ひたるのみで適當を表はす。例 naya-tyāga\-%
śaurya-saṃpanne puruṣe rājyam 處世の知識,捨施,勇氣を
具有せる人に王位は(適當なり)」。
\item 願望,專心,尊敬,友情,信賴,同情,侮蔑,等閑視を意
味する名詞は屢々(屬格と同じく)これらの感情の向けられた
事物の於格と共に用ひられる。例 na kalu Śakuntalāyāṃ
mamābhilāṣaḥ 「吾が愛は決してシャクンタラーに對してあるの
でない」,na me tvayi viśvāsaḥ. 「私は彼に信賴せぬ」。na
laghuṣv api kartavyeṣv anādaraḥ kāryaḥ 「義務の怠慢はたと
ひ小なりともなさるべからず」。
\item 於格は,同樣に好む,專心なる,熱心なる,及びその反對
を意味する形容詞又は過去分詞と共に用ひられる。例 nāryaḥ
kevalaṃ svasukhe ratāḥ 「婦人は只自身の幸福を樂しむ」。
\end{enumerate}
\end{enumerate}

\subsection{獨立於格及び屬格}
\numberParagraph \label{np:231}
二個の動作の主體が一文中に在る場合に獨立於格又は屬
格が用ひられる。前者が寧ろ普通に見る所である。例 gacchatsu
dineṣu nṛpo pṛcchati 「日を經て王は尋ねたり」,goṣu dugdhāsu
sa gataḥ 「牡牛の搾られた時彼は出立せり」,karṇaṃ dadāti
mayi bhāṣamāṇe 「我が語る時彼女は耳傾く」。
\begin{enumerate}[label=(\alph*)]
\item 獨立於格の賓辭は實際には常に分詞である。例外は sat
(ある)なる分詞が屢々省略されることである。例 kathaṃ
karma-kriyā-vighnaḥ satāṃ rakṣatari tvayi. 「君が守護者な
る時,如何に善人の義務遂行に障礙がありませう」。
\item 分詞 sat (ある)(又はその同意語 vartamāna 及び sthita)
が屢々無意味に他の獨立分詞に附加せられることがある。例
sūryodaye 'ndhatāṃ prāpteṣūlūkeṣu 「日昇りて梟は盲目
となる時」。
\item 主語は過去分詞が非人稱的に用ひられる時勿論常に省略
せられる。又分詞が evam, tathā, ittham, iti の如き不變化語
に伴ふ時も省略せられる。例 tenābhyupagate 「彼によりて承
認せられし時」,evaṃ gate. 「よくあることだが」(直譯。かく
行きし時」,tathā kṛte sati (tathānuṣṭhite) 「これがなされて
ある所で」。
\end{enumerate}
\begin{enumerate}[label=(\arabic*), start=2]
\item 獨立屬格は獨立於格より遙かに用例少く而もその適用に
一層制限がある。それは同時の動作に限られ,主語は人であらね
ばならず,賓辭は形に於いてか意義に於いてか現在分詞である。
その意味は「……の間に」「……であるのに」「……の時に」と譯すべ
きである。例 paśyato me paribhraman 「我が見つゝあるにも拘
らず彼は彷徨しつゝ」,evaṃ vadatas tasya sa lubdhako nibhṛ\-%
taḥ sthitaḥ 「彼がかくの如く語る間に獵師は匿れゐたりき」,iti
cintayatas tasya tatra toyārtham āyayuḥ striyaḥ 「彼がかく
考えつゝありし間に婦人たちはそこへ水を汲まんと來れり」。
\end{enumerate}

\subsection{分詞}
\numberParagraph \label{np:232}
分詞は梵語に於て形容詞の性質を有し,その限定する名
詞と性數格に於て一致する。分詞は屢々動詞の機能を有す。例
śṛgālaḥ kopāviṣṭas tam uvāca 「豺は怒に滿たされ彼に云へり」。
niṣiddhas tvaṃ mayānekaśo na sṛṇoṣi 「汝は幾度となく我に
よりて諫止せられしも汝は聞かざりき」,ajalpato jānatas te
śiro yāsyati khaṇḍaśaḥ 「知りつゝ若しも吿げざるならば汝の頭
は分々に裂くべし」,tāḍayiṣyan Bhīmaṃ punar abhyadravat
「打たんとしつゝ復び彼はブヒーマに走りかゝれし」。
\begin{enumerate}[label=(\arabic*)]
\item 現在分詞。この分詞(並に現在の意味を有する過去)は
asti 又は bhavati (彼はあり),āste (坐す),tiṣṭhati (立つ),
vartate (進行す)と共に續く動作を表はすに用ひられる。例
etad eva vanaṃ yasminn abhūma ciram eva purā vasantaḥ
「これは實に我々が已前長らく住しつゝありし同じ林である」,
bhakṣayann āste 「彼は食しつゝ坐せり」,sā yatnena rakṣya\-%
māṇā tiṣṭhati 「彼女は注意深く護られつゝあり」,paripūrṇo
'yaṃ ghaṭaḥ saktubhir vartate 「この甕は麥粉で滿ちてある」。
\item 過去分詞。ta に終る受動分詞,及び vat に終るその能
動形(時として vas に終る第二過去の能動分詞さへも)屢々定
動詞として用ひられる(繫辭は省略)。例 tenedam uktam 「こ
れは彼によりて云はれたり」,sa idam uktavān 「彼はこれを云
へり」。
\begin{enumerate}[label=(\alph*)]
\item 自動詞の受動は非人稱的に用ひられる。さもなくば能動
の意味を有す。例 mayātra ciraṃ sthitam 「長らくそこに私
は立てり」,sa Gaṅgāṃ gataḥ 「彼はガンガーへ行けり」,sa
pathi mṛtaḥ 「彼は路上に死せり」。
\item 或る ta に終る過去分詞は受動と他能動の兩義を有す。
例 prāpta 「得られたる」「到達したる」,praviṣṭa 「……に入ら
れたる」「入りたる」,pīta 「飲まれたる」「飲みたる」,vismṛta
「忘れられたる」「忘れたる」,vibhakta 「分けられたる」「分けた
る」,prasūta 「生まれたる」「生みたる」,ārūḍha 「乘られたる」
「駕せる」。
\item na に終る受動分詞は決して他能動の意味になることが無
い。
\end{enumerate}
\item 義務分詞。これは必須,義務,適當,蓋然を表はし。未
來受動の意を有す。例 mayāvaśyaṃ deśāntaraṃ gantavyam
「我は必然に外國に行かねばならぬ」,hantavyo 'smi na te rājan
「王よ,汝は我を殺してはならぬ」,tatas tenāpi śabdaḥ kar\-%
tavyaḥ 「かくて彼も亦聲を立てるならむ」。
\begin{enumerate}[label=(\alph*)]
\item 時としては未來受動分詞は純粹未來の意味を有す。例
yuvayoḥ pakṣa-balena mayāpi sukhena gantavyam 「汝の翼
の力にても我も亦易々と行くべし」。
\item bhavitavyam 及び bhāvyam は必須又は高度の蓋然を表
はす。賓辭たる形容詞又は名詞は具格に於て主語に一致す。例
tayā saṃnihitayā bhavitavyam 「彼女は必ずや近くにゐなけれ
ばならぬ」,tasya prāṇino balena sumahatā bhavitavyam 「その
獸の力は甚大なるべし」。
\end{enumerate}
\item 連續體。一の動作が始まる前に他の動作が終ることを表
はす(稀には同時のこともある)。主要動作の文典的主體は主格
と一致し,受動の構成では具格と一致する。然し時としては他の
格に一致することもある。例 taṃ praṇamya sa gataḥ 「彼に
禮して彼は行けり」,atha tenātmānaṃ tasyopari prakṣipya
prāṇāḥ parityaktāḥ 「かくて彼は自身を彼の上に投じて命終れ
り」(「投じて」は tena と一致す)。tasya dṛṣṭvaiva vavṛdhe
kāmas tāṃ cāruhāsinīm 「彼の愛は美しき微笑もつ彼女を見て
增し行けり」(「見て」は tasya に一致す)。
\begin{enumerate}[label=(\alph*)]
\item 連續體は前置詞の如く用ひられる。その用法大體次の如
くである。

具格と共に

\indent{uddiśya 「……の方へ」「……に就て」「……に對して」}

\indent{ādāya, gṛhītvā 「とりて」,nītvā 「……を以て」}

\indent{adhiṣṭhāya, avalambya, āśritya, āsthāya 「……によりて」
  「……のために」}

\indent{muktvā, parityajya, varjayitvā, sthāpayitv;a 「……を除
  きて」}

\indent{adhikṛtya 「……に關して」「……に就て」}

從格と共に

\indent{ārabhya 「……より始めて」「……以來」。}
\item 連續體の原形は名詞の具格である。これが kim (何)又
は alam (十分)なる語と共に用ひられて次のやうに元來の意義を
表はす。

\indent{kiṃ tava gopāyitvā 「隱匿して汝に何の(益)ありや」。}

\indent{alaṃ te vanaṃ gatvā 「林に行くを止めよ」。}

又は受動形で表はされた汎意の主語と共に。

\indent{paśūn hatvā yadi svarge gamyate 「若し人獸を殺して天
  界を往くならば」。}
\end{enumerate}
\end{enumerate}

\subsection{不定法}
\numberParagraph
動作の目的を表はすもので一般に目的の爲格に伴ふ。そ
の目的たる事物を業格となし屬格としないので通常の名詞の爲格
と異る。例 taṃ jetuṃ yatate. 「彼は彼に勝たんと努む」と同意であ
る。tasya 屬格)。動詞の直接目的であるから元來の業格の形を
保存する。例 snātuṃ labhate 「彼は沐浴する」。

而して文章の主語とはならぬ。動詞的名詞は通常主語としてそ
の場處を補ふ。例 varaṃ dānaṃ na tu pratigrahaḥ 「與ふる
は取るに勝る」。

不定法は或は名詞と共に用ひられて「時」「機會」を意味し,或
は形容詞と共に用ひられて「適當」「可能」の意となり,或は動詞
と共に「可能」「願望」「開始」の意となる。例 nāyaṃ kālo
vilambitum 「今は遷延すべき時ならず」,avasaro 'yam ātmā\-%
naṃ prakāśayitum 「今は自身を現はす機會なり」,likhitam
api lalāṭe projjhituṃ kaḥ samarthaḥ 「誰か額上に書かれたる
ものを免れ得べき」,ahaṃ tvāṃ praṣṭum āgataḥ 「我れ汝に
尋ねんがため來れり」,kathayituṃ śaknoti 「彼は吿げ得」,
iyeṣa sā kartum 「彼女は爲さんと欲せり」。

\begin{enumerate}[label=(\alph*)]
\item 語根 arh 「値す」の二人稱,三人稱,單數,現實法は鄭
重なる命令の意を含む。例 bhavān māṃ śrotum arhati 「貴
下は何卒私に聽きたまへかし」。
\item 末尾の m を除きたる不定法の形がその動詞の表はす所
をなさんと欲するの意に用ひられる。その合成語は有財釋の形を
取る。例 draṣṭu-kāmaḥ 「見んと欲す」,kiṃ vaktu-manā bha\-%
vān 「何を貴下は語らんとしたまうや」。
\item 梵語の不定法には受動形がない。受動の意味を表はすに
は不定法を支配する動詞が受動に作られる。例 kartuṃ na
yujyate 「そはなされるに適當でない」。例 mayā nītiṃ grāha\-%
yituṃ śakyante 「我によりて儀禮が敎へられ能ふ」,tena
maṇḍapaḥ kārayitum ārabdhaḥ 「彼によりて小屋は作らせられ
始めたり」。
\item 未來,受動分詞 śakya は主語と一致し,又は中性單數
の形となされ得。例 na śakyās te (doṣāḥ) samādhātum 「こ
れらの(過失)は償ひ難し」,sā na śakyam upekṣituṃ kupitā
「怒れる彼女は等閑視すべきでない」,yukta, nyāyya 「適當な
る」も同樣に使用せらる。例 seyaṃ nyāyyā mayā mocayituṃ
bhavattaḥ 「彼女は私によつて貴下から赦免されても宜しい」。
\end{enumerate}

\section{時と法}
\subsection{現在時}
\numberParagraph
現在時の使用法にやや特有なものは
\begin{enumerate}[label=(\arabic*)]
\item 說話の上に歴史的現在が屢々用ひられる,それは持續する
意味を有す。例 Damanakaḥ pṛcchati katham etat 「ダマナ
カは尋ねた。これは何ぞや」,Hiraṇyako bhojanaṃ kṛtvā bile
svapiti 「ヒラヌヤカは食物を作りて穴に於いて眠れり」。
\begin{enumerate}[label=(\alph*)]
\item 時として現在形に purā 「曾て」「以前に」なる副詞が
伴ふ。例 kaśmiṃś cid vṛkṣe purā 'haṃ vāsāmi 「我れ曾て
とある樹下に住したり」。

又屢々 sma なる語が伴ふ。この語は古代梵語では屢々 purā
に伴ひしため遂に單獨に用ひられてもその意味を有するに至つた
ものである。例 kaśmiṃś cid adhiṣṭhāne Somiliko nāma kauliko
asati sma 「とある場處にソーミリカなる織師が住みたりき」。
\item 現在は直接に連絡する過去を意味して使はれる。例
ayam āgacchāmi 「此に我來れり」(今しも恰度來たところである。
現在完了の意)。
\end{enumerate}
\item 現在は又近接せる未來の意を有す。時として purā 「すぐ
に」,yāvat 「丁度」なる語が附加せられる。例 tarhi muktvā
dhanur gacchāmi 「かくて弓を棄てて我は去らむ」,tad yāvac
Chatrughnaṃ preṣayāmi 「故に我は今シャトルグフナを遣はさ
む」。
\begin{enumerate}[label=(\alph*)]
\item 疑問詞と伴ひて將來の動作に關する疑を含む。例 kiṃ
karomi 「我れ何をなすべきや」。
\item 又時には直に或行動をなせと云ふやうな勸吿の意を有す。
例 tarhi gṛham eva praviśāmaḥ 「今や我々は家に入るべ
きだ」。
\end{enumerate}
\end{enumerate}

\subsection{過去時}
\numberParagraph
過去と云はゞ第一,第二,第三は固より,その他 -ta や
-tavat を附した過去分詞,歴史的現在も區別なく遠き時の經過
を表し,等しく一度起つた,又は反覆された,又は連續的な事實
を示す。
\begin{enumerate}[label=(\alph*)]
\item 第二過去は遠き時の經過せし事實の記述に限られ,これ
は記述者の經驗內のことでない。隨てこの過去の一人稱,二人稱
は極めて稀である。
\item 第一過去は歴史的過去を記述する上に記述者自身が目賭
せし過去の事實を述べる。
\item 第三過去は(-ta や -tavat を附加する分詞も含めて)現
在完了の特殊な意味がある。故に對話の上に適用せられる。例
abhūt saṃpādita-svādu-phalo me manorathaḥ 「我が願望は美
はしき結果を齎した」,tubhyaṃ mayā rājyam adāyi 「私は汝
に王位を與ふ」,taṃ dṛṣṭavān asmi 「我れ彼を見たり」。
\item 過去符なき第三過去(稀に第一過去)は否定辭 mā と共
に命令の意を有す。
\item 完全過去の形は梵語に存しない。この意味は前後の關係
より推知せられ,他の過去形連續體を以て表現せられる。
\end{enumerate}

\subsection{未來時}
\numberParagraph
單未來は一般に未來の動作に關する記述に用ひられ,複
說未來は遠き未來に限られ幾分用ひられることも少い。故に雙方
共に同一の動作を記述するに用ひられ屢々互用せられる。
\begin{enumerate}[label=(\alph*)]
\item 未來は時として命令法を伴つて命令の意味を有す。例
bhadre yāsyasi mama tāvad arthitvaṃ śrūyatām 「善良なる
ものよ,汝は行くべし。されどまづ我が願が聽かるべし」。
\end{enumerate}

\subsection{命令法}
\numberParagraph
通常の\ruby{禁遏}{きん|あつ}又は勸吿の意味の外に特殊の用法は次の如き
ものである。
\begin{enumerate}[label=(\alph*)]
\item 第一人稱命令は古代の接續法の痕跡を存するもので「…
…であらう」又は「……をして……せしめよ」と譯すべきだ。例
dīvyāvety (dīvyāva iti) abravīd bhrātā 「兄弟は云へり。い
ざ我々をして博戯せしめよ」,ahaṃ karavāṇi 「我はなすであら
う」。
\item 三人稱,單數,受動,命令法は通常二人稱,能動のそれ
よりも丁寧なる用法である。例 deva śrūyatām 「君よ,願は
くは聽きたまへ」。
\item 願望,祝福を表はすに可能法や願望法を用ふる代りに命令
法が用ひられる。例 ciraṃ jīva 「長壽なれ」,śivās te
panthānaḥ santu 「汝の道程は幸福なれ」。
\item 疑問詞と共に可能性又は疑惑を表はす。例 viṣaṃ bha\-%
vatu mā vāstu phaṭāṭopo bhayaṃkaraḥ 「毒あるも又は無き
も蛇頭の膨大は恐ろし」,kim adhunā karavāma 「我々は今何を爲す
べきか」。
\item 否定辭 mā を伴ふ命令法は比較的稀である。通常は過去
符なき第一過去を以て作られる。又は具格に伴ふ alam 或は
kṛtam によりて表現せられる。
\end{enumerate}

\subsection{可能法}
\numberParagraph
その一般的性質は能力の存在を現はすにあるがその他に
現在梵語に消滅せる接續法に係る意味が種々の點から影の如くに
伴隨する。
\begin{enumerate}[label=(\arabic*)]
\item 主文章に於てはそれは次の如き意味を表はす。
\begin{enumerate}[label=(\alph*)]
\item 願望(屢々 api なる語を加へて)。例 api paśyeyam iha
Rāghavam 「此に我れラーマ王を見るを得たらましかば」。
\item 可能性又は疑惑。例 kadācid go-śabdena budhyate 「い
つか恐らく彼は牛の聲にて目覺めたらむ」,paśyeyuḥ kṣitipata\-%
yaś cāra-dṛṣṭyā 「王達は諜者の目もて見るを得るのだ」。ekaṃ
hanyān na vā hanyād iṣur mukto dhanuṣmatā 「射手の放ち
し箭は一人を傷けしか又は傷けざりしならむ」。
\item 蓋然性。この場合屢々未來に等しい。例 iyaṃ kanyā
nātra tiṣṭhet 「この少女は其處に止まらざるべし」。
\item 勸吿は又は訓誡。例 tvam evaṃ kuryāḥ 「汝はかく爲
すとは\ruby[S]{底}{なに}事ぞや」,āpad arthaṃ dhanaṃ rakṣet 「人は災厄の
ために富を守護すべきである」。
\end{enumerate}
\item 可能法は次の如き接續的の句に用ひられる。
\begin{enumerate}[label=(\alph*)]
\item 一般の關係句。例 kālātikramaṇaṃ vṛtter yo na
kurvīta bhūpatiḥ 「俸給の支拂の時を忽かにせざる所の王は」。
\item 目的の句(「……のために」の意にて)。例 ādiśa me
deśaṃ yatra vaseyam 「私の住すべき所の場處を私に示せ」。
\item 理由を示す句(「……するやうに」の意にて)。例 sa
bhāro bhartavyo yo naraṃ nāvasādayet 「人を壓倒せぬ程の
荷物が持たれるべきである」。
\item 假定句の前件に於て。例 yadi na syān nara-patir
viplaveta naur iva prajā 「若し王なくば人民は船の如く沈沒す
べし」。
\end{enumerate}
\end{enumerate}

\subsection{願望法}
\numberParagraph
これは第三過去の可能法であつて極めて稀に用ひらるる
形であり,祝福の發表に限られ,第一人稱にて說者の願望を述ぶ。
例 vīra-prsavā bhūyāḥ 「御身が勇士を生むやうに」,kṛtārtha
bhūyāsam 「我れ目的を達するやうに」。

命令法も亦この意味に用ひられる。極めて稀ではあるが願望法
が命令法又は通常の可能法と何等區別し得ないこともある。例
idaṃ vaco brūyāsta 「汝等はこの語を宣べよ」,mamaiṣa kāmo
bhūtānāṃ yad bhūyāsur vibhūtayaḥ 「人々が幸福を享受せん
ことは我が欲する所である」。

\subsection{條件法}
\numberParagraph
條件法はその形から云はゞ未來の現實的過去とも云ふべ
く過去の條件を表はすに用ひられ,事實そのことは存在せぬこと
が含まれる。而もそれは過去完了に相當する。又この形は假定句
の前件後件の雙方に用ひられる。例 suvṛṣṭiś ced abhaviṣyad
durbhikṣaṃ nābhaviṣyat 「若しも多量の雨ありせば饑饉はあら
ざりしならむ」。

若し前件に可能法が用ひられたならば後件は假定の意味とな
る。例 yadi na praṇayed rājā daṇḍaṃ śūle matsyān ivā\-%
pakṣyan durbalān balavattarāḥ 「若しも王が處罰を加へない
ならば强者は弱者を串に於いての魚のやうに焙るであらう」。

可能法は假定の現在を云ひ表はすのが持前だが,然しそれが亦
條件法でなくて過去の意味にもなるのである。

\newpage
\theendnotes


%%% Local Variables:
%%% mode: latex
%%% TeX-master: "IntroductionToSanskrit"
%%% End:


\part{文抄}
\texttitle{ナラ王物語}
\addcontentsline{toc}{chapter}{\protect\numberline{}ナラ王物語}%
昔ヴィーラセーナの王子なる力强きナラなる王ありき。

勝れたる德を具し美貌にして馬術に巧みに,

人たる君主の頭首なりき。神の主領なる如く,

太陽の如く,威光を以て一切の上に秀で,

梵行を修し,吠陀に委しく,勇者にしてニシャドハの王なり。

勝負事を好み,眞實を語り,大人物にして軍團の將なり。

勝れし婦女たしに愛せられ,寛厚にして節操正し。

守護者にして弓取るものの最勝者,マヌ自信を親たりに見るが
如くなり。

又ヴィダルブハにブヒーマとて力すぐれたる王あり,

一切の德を具し,勇者なるが,子なくして久しく子を欲せり。

彼の子を欲していみじき努力を爲し,心統一をなせる,

彼へ梵仙ダナマ\endnote{「ダナマ」はママ。}なるものが來れり。ブハーラタよ。

子を欲りせる,法に通ぜる,彼のブヒーマは

夫人と共に,王よ,尊敬もて威光ある彼を喜ばしめぬ。

喜べるダマナは,夫人と共なる彼に賚賜を與へたり,

寶の如き王女と大なる名譽あるすぐれたる三人の王子,

卽ちダマヤンティーと威光あるダマ,ダーンタ,ダマナとなり。

一切の德を具し勇ありて勇氣すぐれたり。

ダマヤンティーは姿めでたく容色すぐれたり。

又幸運を以て世に名を得たり。

\newpage{}
\texttitle{Nalopākhyānam}

āsid rājā Nalo nāma Virasena-suto bali \da

upapanno guṇair iṣṭai rūpavān aśva-kovidaḥ \dd

atiṣṭhan manujendrāṇāṃ mūrdhni devapatir iva \da

upary upari sarveṣām āditya iva tejasā \dd

brahmaṇyo veda-vic chūro Niṣadheṣu mahīpatiḥ \da

akṣa-priyaḥ satya-vādī mahān akṣauhiṇī-patiḥ \dd

īpsito vara-nārīṇām udāraḥ saṃyatendriyaḥ \da

rakṣitā dhanvināṃ śreṣṭhaḥ sākṣād iva Manuḥ svayam \dd

tathaivāsīd Vidarbheṣu Bhīmo bhīma-parākramaḥ \da

śūraḥ sarva-guṇair yuktaḥ prajā-kāmaḥ sa cāprajaḥ \dd

sa prajārthe paraṃ yatnam akarot susamāhitaḥ \da

tam abhyagacchad brahma-rṣir Damano nāma Bhārata \dd

taṃ sa Bhīmaḥ prajā-kāmas toṣayāmāsa dharma-vit \da

mahiṣyā saha rājendra satkāreṇa suvarcasam \dd

tasmai prasanno Damanaḥ sabhāryāya varaṃ dadau \da

kanyā-ratnaṃ kumārāṃś ca trīn udārān mahāyaśāḥ \dd

Damayantīṃ Damaṃ Dāntaṃ Damanaṃ Damanaṃ ca suvarcasam \da

unsapannān guṇaiḥ sarvair bhīmān bhīmaparākramān \dd

Damayantī tu rūpeṇa tejasā yaśasā śriyā \da

saubhāgyena ca lokeṣu yaśaḥ prāpa sumadhyamā \dd

\newpage

さても妙齡に達せし時,装飾つけたる婢女と

侍女との幾百が奉仕せり。シャチーの如くに。

そこにブヒーマの王女は一切の装飾に飾られて輝けり。

侍女の中に避難さるべき點なき彼女は電光の如く,

極めて容色美はしく脩長なる目あり。吉祥天の如く,

天界にもヤクシャ界にも何處にもかくの如き美容もてるは無し。

人界にもその他にも嘗て見られず,又聞かれず。

神々にとりても美はしき心を迷はす若き少女なり。

ナラは又世に於いて人中の虎のやうに强く地上に比びなし。

美貌を以て,親たり形を現ぜしカンダルパのごとくなりき。

彼女の側にては彼等(女達)は好奇心もてナラ王を讚めたり。

ナラ王の側には再三再四ダマヤンティーを讚めたり。

常に樣子を聞ける彼等二人の間にはまだ見ぬ戀心がありき。

クンティーの子よ,互の間にかの戀は增し行けり。

ナラは心もて戀を保ち得ずして

後宮の側なる林に祕かに行きて坐せり。

その時彼は黄金色に輝ける白鳥の群れを見たり。

林に飛び翔る彼等の一羽の鳥を捕へたり。

その時,鳥はナラに人語を語れり,

我をな殺しそ王よ。汝に我はよき事をなすべし。

ニシャドハの王よ,我はダマヤンティーの傍に汝の上を語るべし,

かくて汝以外の男子を彼女は決して考へざるべし。

かくて語りし時王は白鳥を放ちやりぬ,

彼等白鳥は飛び上りてヴィダルブハの方へ行けり。

\newpage

atha tāṃ vayasi prāpte dāsīnāṃ samalaṃkṛtam \da

śatam śatam sakhīnāṃ ca paryupāsac Chacīm iva \dd

tatra sma rājate bhaimī sarvābharaṇa-bhūṣitā \da

sakhī-madhye 'navadyāṅgī vidyut saudāminī yathā \dd

atīva rūpa-saṃpannā Śrīr ivāyata-locanā \da

na deveṣu na yakṣeṣu tādṛg-pūrvāthavā śrutā \da

citta-pramāthinī bālā devānām api sundarī \dd

Nalaś ca nara-śārdūlo lokeṣv apratimo bhuvi \da

Kandarpa iva rūpeṇa mūrtimān abhavat svayam \dd

tasyāḥ samīpe tu Nalaṃ praśaśaṃsuḥ kutūhalāt \da

Naīṣadhasya samīpe tu Damayantīṃ punaḥ punaḥ \dd

tayor adṛṣta-kāmo 'bhūt sṛṇvato satataṃ guṇān \da

anyonyaṃ prati Kauteya sa vyavardhata hṛcchayaḥ \dd

aśaknuvan Nalaḥ kāmaṃ tadā dhārayituṃ hṛdā \da

antaḥpura-samīpa-sthe vana āste raho gataḥ \dd

sa dadarśa tato haṃsān jātarūpa-pariṣkṛtān \da

vane vicaratāṃ teṣām ekaṃ jagrāha pakṣiṇam \dd

tato 'ntarīkṣago vācaṃ vyajahāra Nalaṃ tadā \da

hantavyo 'smi na te rājan kariṣyāmi tava priyam \dd

Damayantī-sakāśe tvāṃ kathayiṣyāmi Naiṣadha \da

yathā tvad-anyaṃ puruṣaṃ na sā maṃsyati karhicit \dd

evam uktas tato haṃsam utsasarja mahīpatiḥ \da

te tu haṃsāḥ samutpatya Vidarbhān agamaṃs tataḥ \dd

\newpage

ヴィダルブハの都城に行きてその時ダマヤンティーの側に

彼等鳥等は飛び下りぬ。かくて彼女は彼等の群を見たり。

彼女はかれら世にもいみじき色を見て侍女たちと共に

心喜び鳥等を捉へんものと急ぎつゝ進み行けり。

時に白鳥等は若き婦女の群の中を遍ねく飛び回れり,

時に少女達は各々彼等白鳥を追ひ回せり,

ダマヤンティーも亦側なるとある白鳥を追ひまわせしが

その白鳥は人語をなしその時ダマヤンティーに云へらく

ダマヤンティーよニシャドハの王なるナラなる君あり,

美貌はアシュヴィンの如く人間に彼に比ぶべきは無し。

若し汝彼の妻たらば,美はしきものよ,

この生れと美貌とはよき果を結ばむ,美しき少女よ。

我らは神々ガンダルヷ人間龍羅刹の世界を

見たるが,未だ曾て我等はかくの如き美しきものを見ず。

汝は又婦女の寶なり,而してナラは人間の最もすぐれしもの。

殊勝なるものが殊勝なるものに合會せんはいみじき事なりかし。

かくの如く白鳥に語りかけられしダマヤンティーは,王よ,

そこに白鳥に言へり,汝はかくナラに語れ。

爾かせむと白鳥はヴィダルブハの少女に云ひて,王よ,

再びニシャドハに來りて王にすべてを吿げたり。
\wosfnt{%
  ナラ王物語は印度二大叙事詩の一なるマハーバーラタの中に見える美し
  い詩篇である。全篇二十六章凡そ九百八十頌のシュローカ調で出來てゐる。
  ナラ王とダマヤンティー姫との間の戀愛から結婚,不幸厄難,而もそれを越
  えて進む勇氣と天祐神助,全卷を貫く誠實貞潔,或る見方ではマハーバー
  ラタ全篇の縮圖とも見られ,印度の理想的人格が手際よく描かれてゐる。
  此に擧げしは最初の部分若干であるが,その文體も平明で華麗,實に典文
  梵語の模範たるに値する。}

\newpage

Vidarbha-nagarīṃ\endnote{底本では ``Vidarbha-'' は ``\rotatebox{90}{V}idarbha-''。} gatvā Damayantyās tadāntike \da

nipetus te garutmantaḥ sā dadarśa ca tān gaṇān \dd

sā tān adbhuta-rūpān vai dṛṣṭvā sakhī-gaṇāvṛtā \da

hṛṣṭā grahītuṃ khagamāṃs tvaramāṇopacakrame \dd

atha haṃsā visasṛpuḥ sarvataḥ pramadāvane \da

ekaikaśas tadā kanyās tān haṃsān samupādravan \dd

Damayantī tu yaṃ haṃsaṃ samupādhāvad antike \da

sa mānuṣīṃ giraṃ kṛtvā Damayantīṃ athābravīt \dd

Damayanti Nalo nāma Niṣadheṣu mahīpatiḥ \da

Aśvinoḥ sadṛśo rūpe na samās tasya mānuṣāḥ \dd

tasya vai yadi bhāryā tvaṃ bhavethā varavarṇini \da

saphalaṃ te bhavej janma rūpaṃ cedaṃ sumadhyame \dd

vayaṃ hi deva-gandharva-mānuṣoraga-rākṣasān \da

dṛṣṭavanto na cāsmābhir dṛṣṭa-pūrvas tathā-vidhaḥ \dd

viśiṣṭāyā tu haṃsena Damayantī viśāṃpate \da

abravīt tatra taṃ haṃsaṃ tvam apy evaṃ Nale vada \dd

tathety uktvāṇḍajaḥ kanyāṃ Vidarbhasya viśāṃpate \da

punar āgatya Niṣadhān Nale sarvaṃ nyavedayat \dd

\rightline{(Mahābhārata III. 6)}


\newpage
\theendnotes

%%% Local Variables:
%%% mode: latex
%%% TeX-master: "IntroductionToSanskrit"
%%% End:

\newpage
\texttitle{婆羅門の{空}想}
\addcontentsline{toc}{chapter}{\protect\numberline{}婆羅門の空想}%

デーヴィーコータ城にデーヷシャルマンと名くる婆羅門ありき。
彼れ祭祀の時麥の煎粉を容れたる一皿を得たり。かくてこれを携
へて彼は陶器を滿たせる陶工の小屋の一隅に於て眠りつゝ考へ
たり。もし我れこの皿の煎粉を賣りて十カパルダカを得ばその時
こゝにてこれを以て壺皿を買ひ,種々の方法にて增殖せる金錢に
て又\ruby{檳榔子}{びん|ろう|じ}衣服等を買ひ,賣りて十萬を以て數ふる富を生じて四
度の結婚をなすべし。かくて彼等四人の妻の中にて若く美貌ある
ものに他に勝る寵愛をなすべし。その間に嫉妬を生じたるその婦
女達が爭をなさばその時は予は怒りて彼等妻女をかくの如く杖に
て打つべし。かく云ひて彼は杖を投げたり。その時かの麥の煎粉
の皿は碎かれ多くの陶器は壞されたり。かくて陶器の壞れたる音
によりて來れる陶工は見て彼の婆羅門は叱責せられ小屋の外に出
されたり。\wosfnt{%
  これは印度訓話ヒトーバデーシャ中の物語である。ヒトーバデーシャは印
  度訓話の最も著名なるもので,殊にこの婆羅門の物語の如きは世界各地へ
  姿を變じて傳はつて行つた。エソポスやラフオンテーンに乳搾り娘の{空}想
  の物語として有名である。印度は物語の無盡の藏とも云ふべきで,パン
  チャヤタントラと云ひカトハーサリットサーガラと云ひ,さては佛敎のジャ
  ータカ本生說話に至るまで擧げ來ればその資料の豊富なること實に驚嘆そ
  のものである。ヒトーバデーシャは四篇から成る短い作品だが,その物語は
  精選せられ,引用の詩に見るべきもの多く,散文は實に簡明にして要を
  得,何れの點よりしても模範的作品と謂うべきである。}

\newpage

\texttitle{Tyaktāśaḥ}

asti Devīkoṭa-nagare Devaśarmā nāma brāhmaṇaḥ \da{} tena
viṣuvat-samaye saktu-bhṛta-śarāva ekaḥ prāptaḥ \da{} tatas tam
ādāyāsau bhāṇḍa-pūrṇa-kumbhakāra-maṇḍapikaika-deśe sup\-%
taḥ sann acintayat \da{} yady aham imaṃ saktu-śarāvaṃ vikrīya
daśa kapardakān prāpnomi tadā tair iha samaye ghaṭa\-%
śarāvān upakrīya vikrīyānekadhā vṛddhais tair dhanaiḥ
punaḥ pūga-vastrādikam upakrīya vikrīya lakṣasaṃkhyātāni
dhanāny utpādya vivāha-catuṣṭayaṃ karomi \da{} tatas tāsu
patnīṣu yā rūpa-yauvanavatī tasyām adhikānurāgaṃ karomi \da{}
anantaraṃ saṃjāterṣyās tat-sapatnyo yadā dvaṃdvaṃ kur\-%
vanti tadā kopākulo 'haṃ tāḥ sapatnīr itthaṃ laguḍena
tāḍayiṣyāmi \da{} ity abhidhāya tena laguḍaḥ kṣiptaḥ \da{} atha sa
saktu-śarāvaś ca cūrṇito bhāṇḍāni ca bahūni bhagnāni \da{} tato
bhagna-bhāṇḍa-śabdenāgata-kumbhakāreṇa dṛṣṭvā sa brāh\-%
maṇas tiraskṛto maṇḍapikā-garbhād bahiḥ-kṛtaḥ \dd{}

\rightline{(Hitopadeśa VI)}


%%% Local Variables:
%%% mode: latex
%%% TeX-master: "IntroductionToSanskrit"
%%% End:

\newpage

\texttitle{阿難誘惑せらる}
\addcontentsline{toc}{chapter}{\protect\numberline{}阿難誘惑せらる}%
是の如く我れによりて聞かれたり。ある時世尊はシュラーヷス
ティー逝多林なる給孤獨の園に住したまへり。時に\ruby{具壽}{ぐ|じゅ}阿難陀は
午前に衣を整へ鉢と上衣を取りシュラーヷスティー大城に行乞のた
めに進み行けり。時に具壽阿難陀は行乞をなし,飯食の後,とある
井の方に進み行けり,その時にその井に於いてプラクリティと名
くる賤族の少女は水を汲みてゐたりき。時に具壽阿難陀は賤族の
少女にこれを云へり。「我に水を得させよ,姉妹よ,我れは飲まん
とす」。かく云はれたる賤族の少女プラクリティは具壽なる阿難
陀にこれを云へり。「我れは賤族の少女なり,大德阿難陀よ」。
「姉妹よ,我は種族生家を問へるにあらず。たゞ若し許さるべくば
水を與へよ我は飲まんとす」。時に賤族の少女プラクリティは具壽
阿難陀に水を與へたり。時に具壽なる阿難陀は水を飲みて進み行
けり。時に賤族の少女プラクリティは具壽阿難陀の身に於いて顔
に於いて一切善く審らかに相を執りて如理に作意して愛戀の心を
生ぜり。聖者阿難陀は我が夫たるべしと。而して我母は大咒師な
り,彼は聖者阿難陀を引き出すを得べし。時に賤族の少女プラク
リティは水瓶を一方に投げて自からの母にこれを云へり。「阿母よ

\newpage
\texttitle{Ānanda ākṛṣṭaḥ}
evaṃ mayā śrutam \da{} ekasmin samayo Bhagavāṃś Chrā\-%
vastyāṃ viharati sma Jetavane 'nāthapiṇḍadasyārame \da{}
athāyuṣmān Ānandaḥ pūrvāhne nivāsya pātra-cīvaram ādāya
Śrāvastīṃ mahānagarīṃ piṇḍāya prāvikṣat \da{} atha āyuṣmān
Ānandaḥ Śrāvastīṃ piṇḍāya caritvā kṛta-bhakta-kṛtyo
yenānyatamam udapānaṃ tenopasaṃkrāntaḥ \da{} tena khalu
samayena tasminn udapāne Prakṛtir nāma mātaṅgadāriko\-%
dakam uddharate sma \da{} athāyuṣmān Ānandaḥ Prakṛtiṃ
mātaṅgadārikām etad avocat \da{} dehi me bhagini pānīyaṃ
pāsyāmi \da{} evam ukte Prakṛtir mātaṅgadārikā āyuṣmantam
Ānandam idam avocat \da{} mātaṅgadārikāham asmi bhadan\-%
tānanda \da{} nāhaṃ he bhagini kulaṃ vā jātiṃ vā pṛcchāmi
api tu sacet te parityaktaṃ pānīyaṃ dehi pāsyāmi \da{} atha
Prakṛtir mātaṅgadārikā ayuṣmaṭa Ānandāya pānīyam adāt \da{}
athāyuṣmān Ānandaḥ pānīyaṃ pītvā prakrāntaḥ \da{} atha
Prakṛtir mātaṅgadārikā āyuṣmato Ānandasya śarīre mukhe
sarve sādhu ca suṣṭhu ca nimittam udgṛhītvā yoniśo mana\-%
sikāreṇāviṣṭā saṃrāga-cittam utpādayati sma \da{} āryo me
Ānandaḥ svāmī syād iti \da{} mātā ca me mahā-vidyā-dharī sā
śakṣyaty āryam Ānandaṃ ānayitum \da{} atha Prakṛtir mātaṅ\-%
gadārikā pānīya-ghaṭam ekānte nikṣipya svāṃ jananīm idam

\newpage
\noindent
今や知らるべし。阿難陀と名くる沙門瞿答摩\endnote{「瞿答摩」はおそらく Gautama の音訳。}の聲聞の弟子あり。
我は彼を夫となさんとす。阿母よ,彼を引き出し得べきや」。彼女
は娘に云へり。「娘よ死せるか離慾ならざる限り我は阿難陀を引
出し得。されどコーサラの王波斯匿は沙門瞿答摩に極めて尊敬奉
事せり,若し彼にして知らばこの旃陀羅族の不幸は生ずべし。而
して沙門瞿答摩は離慾なりと聞く。離慾なる彼は一切の咒を超越
せり」。かくの如く云はれし時,賤族の女プラクリティは母にこれ
を云へり。「阿母よ,若し沙門瞿答摩は離慾にて,彼の前より沙門
阿難陀を得ざるならば我は生命を捨つべし。若し得ば我は生くべ
し」。「娘よ,汝は生命を棄つる勿れ。我れ汝に沙門阿難陀を引き
出すべし」。時に賤族の女プラクリティの母は家の中に牛糞の塗料
を作りて壇を塗り。吉祥草を散らし,火を燃し,百八瓣のアルカ
の花を取りて咒を唱へつゝ一々のアルカの花を咒しつつ火中に投
じたり。そこにこの咒あり。
\begin{addmargin}[2em]{0em}
\indent
アマレー,ヴィマレー,クンクメー,スマネー,それによ
りて汝は覺者なり,希望もて輝かさる。神は雨らし輝く。驚
きより響く。大王の增加せしめんために\ 諸神に諸人に諸乾
闥婆は火執の故に明火執の故に。阿難陀の來現と交會と進出
と執受のために\ 我は供養す。莎婆訶。
\end{addmargin}

\newpage
avocat \da{} yat khalv evam amba jānīyā Ānando nāma śramaṇa-
Gautamasya śrāvaka upasthāyakas tam ahaṃ svāminam
icchāmi śakṣyasi tam amba ānayitum \da{} sā tām avocat \da{}
śaktāhaṃ putri Ānandam ānayituṃ sthāpayitvā yo mṛtaḥ
syād yo vā vīta-rāgaḥ \da{} api ca rājā Prasenajit Kauśalaḥ
śramaṇa-Gautamam atīva sevate bhajate paryupāsate \da{} yadi
jānīyāt so 'yaṃ caṇḍāla-kulasyānarthāya pratipadyeta \da{}
śramaṇaś ca Gautamo vīta-rāgaḥ śrūyate vīta-rāgas sa punaḥ
sarva-mantrān abhibhavati \da{} evam uktā Prakṛtir mātaṅga\-%
dārikā mātaram idam avocat \da{} saced amba śramaṇo Gautamo
vīta-rāgas tasyāntikāc chramaṇam Ānandaṃ na pratilapsye
jīvitaṃ parityajeyaṃ sacet pratilapsye jīvāmi \da{} mā tvaṃ
putri jīvitaṃ parityakṣyasi ānayāmi te śramaṇam Ānandam \da{}
atha Prakṛter mātaṇgadārikāyā mātā madhye gṛhāṅgaṇasya
gomayenālepanaṃ kṛtvā vedīm ālipya darbhān saṃstīryāgniṃ
prajvālyāṣṭaśatam arka-puṣpāṇāṃ gṛhītvā mantrān āvar\-%
tayamānā ekaikam arka-puṣpaṃ parijapyāgnau pratikṣipati
sma \da{} tatreyaṃ vidyā bhavati \da{}
\begin{addmargin}[2em]{0em}
\indent
amale vimale kuṅkume sumane yena buddho 'si vidyuta
icchayā devo varṣati vidyotati garjati vismayān mahā\-%
rājasya samabhivardhayituṃ devebhyo manuṣyebhyo gan\-%
dharvebhyaḥ śikhigrahād eva viśikhigrahād eva Ānanda-
syāgamanāya saṃgamanāya kramaṇāya grahaṇāya grahaṇāya ju\-%
homi svāhā \dd
\end{addmargin}

\newpage
時に具壽阿難陀の心は引き寄せられたり。彼は精舎より出でゝ
旃陀羅の屋舎の方に往けり。旃陀羅の女は具壽阿難陀の來れる
を遠く望み見て娘プラクリティにこれを云へり。「娘よ,ここに彼
れ沙門阿難陀は來れり,臥床を設けよ」。時に賤族の少女は歡喜滿
足欣悅の意もて具壽阿難陀の臥床を設けたり。時に具壽阿難陀は
旃陀羅女の屋舎の方に往けり,往きて壇に近づきて立てり。一方
に立ち又彼具壽阿難陀は啼泣せり,淚を流しつゝかくの如く云へり。
「我は厄難に臨めり,而して世尊は我を憶念したまはざるなり」。
時に世尊は具壽阿難陀を憶念したまへり。憶念して正覺者の咒に
よりて旃陀羅の咒を破壞したまへり。咒に云く
\begin{addmargin}[2em]{0em}
「一切有情にまで堅固不死不導吉祥\\
\ 湖水の如く清澄無過失寂靜一切處無畏\\
\ 災禍は鎮靜し怖畏は移動する所に\\
\ 天よ一切成就の修行者は彼を敬禮す。\\
\ この眞實語を以て比丘阿難陀に吉祥あれ」
\end{addmargin}
時に具壽阿難陀は旃陀羅の咒破られて旃陀羅の屋舎より出でて
自らの精舎の方に往き始めたり。賤族の女プラクリティは具壽阿
難陀の歸り去るを見て自らの母にこれを云へり。「こゝに母よ,か
の沙門阿難陀は歸り去る」。彼へ母は云へり。「娘よ,定めて沙門

\newpage
athāyuṣmata Ānandasya cittam ākṣiptam \da{} sa vihārān
niṣkramya yena caṇḍāla-gṛhaṃ tenopasaṃkrāmati \da{} adrāk\-%
ṣīc caṇḍālī āyuṣmantam Ānandaṃ dūrād evāgacchantaṃ
dṛṣṭvā ca punaḥ Prakṛtiṃ duhitaram idam avocat \da{} ayam
asau putri śramaṇānanda āgacchati śayanaṃ prajñapaya \da{}
atha Prakṛtir mātaṅgadārikā hṛṣta-tuṣṭa-pramudita-manā
āyuṣmata Ānandasya śayyām prajñapayati \da{} atha āyuṣmān
Ānando yena caṇḍālī-gṛhaṃ tenopasaṃkrāntaḥ \da{} upasaṃ\-%
krāmya vedīm upaniśrityāsthāt \da{} ekānta-sthitaḥ sa punar
āyuṣmān Ānandaḥ praroditi aśrūṇi pravartayamāna evam
āha \da{} vyasana-prāpto 'ham asmi na ca me Bhagavān samanvā\-%
harati \da{} atha Bhagavān āyuṣmantam Ānandaṃ samanvā\-%
harati sma samanvāhṛtya saṃbuddha-mantraiś caṇḍāla\-%
mantrān pratihanti sma \da{} tatreyaṃ vidyā
\begin{addmargin}[2em]{0em}
sthitir acyutir anītiḥ svasti sarvaprāṇibhyaḥ \da{}\\
saraḥ prasannaṃ nirdoṣaṃ praśāntaṃ sarvato 'bhayam \da{}\\
ītayo yatra śāmyanti bhayāni calitāni ca \dd{}\\
taṃ vai deva namasyanti sarva-siddhāś ca yoginaḥ \da{}\\
etena satya-vākyena svasty Ānandāya bhikṣave \dd{}
\end{addmargin}
athāyuṣmān Ānandaḥ pratihata-caṇdāla-mantraś caṇḍāla\-%
gṛhān niṣkramya yena svako vihāras tenopasaṃkramitum
ārabdhaḥ \da{} adrākṣīt Prakṛtir mātaṅgadārikā Ānandam āyuṣ\-%
mantaṃ pratigacchantam dṛṣṭvā ca punaḥ svāṃ jananīm
idam avoat \da{} ayam asau mātaḥ śramaṇānandaḥ pratiga-

\newpage
\noindent
瞿答摩の憶念(原文訂正)せしならむ,その故に我が咒は破られて
あるならむ」。プラクリティ亦云へり。「母よ,如何に沙門瞿答摩の
咒は我等のそれより力强きや」。彼女へ母は云へり。「沙門瞿答摩
の咒は一層强力なり。それらの咒は娘よ,一切世界を超越す,そ
れらの咒を瞿答摩は欲するならば破る,されど世界は沙門瞿答摩
の咒を破る能はじ。かくの如く沙門瞿答摩の咒は一層强大なり。\wosfnt{%
  この一節はディヴャーヷダーナ(殊勝說話)の第三十三章から取つた。殊
  勝說話は全篇三十八章の說話から成る佛敎經典で,その内容も樣式も種々
  雜多で連絡統一を缺き,成立もさう古いとは云へぬが,然しその中に極め
  て古い要素を含んでゐることが注意される。文體は極めて簡潔明快なる純
  梵語であるが,まゝ從來の文學中に用例なき方言を發見する。三十八の說
  話中その三十までは支那譯有部律其の他の經中に散在する。第二十六章か
  ら第二十九章に至る阿育王物語は興味多いものであり,阿育王經や阿育王
  傳(大正藏第五十卷)等に相當する。

  此に出した阿難誘惑の說話は,虎耳說話の一部分をなすもので支那譯經
  典の摩登伽經,舎頭諫太子二十八宿經(大正藏第二十一卷)等に一致し,
  釋尊が賤族の女を得度せしめて敎團の一員とせしことに對し,市民の不滿
  激昂,王への愁訴,王の訪問,釋尊の種姓に何等の差別なしとの說示がこの
  一章の梗槪である。此には只最初の部分である阿難が賤族の女から戀せ
  られ,咒術によりて破戒の危機に臨みし一段を擧げた。阿難が釋尊に親し
  く仕へながら修道に於いては割合に進み得ず。未だ離慾の境地に到達し得
  なかつたと傳へられることは興味多いことである。この阿難を拉し來りて
  旃陀羅の少女を配し,凄愴なる咒法執行の一場面を描いた所は經典として
  やや特殊の手法を見る。尙ほ印度の井戸には釣瓶の如き汲水具が無い。水
  を汲むものは自身これを携へて行くのである。これ阿難の自ら飲み得ず,
  女に水を乞ふ所以。又種族を異にすれば下種のものから飲食を受けられな
  いのである。況んや最下の旃陀羅族に於いては猶更のことである。これら
  を注意して讀むべきである。

  尙文中に見える咒についてはその一部分に純粹梵語の文法では解き得ら
  れない語がある。アマレー,ヴィマレー,クンクメー,スマネーの四語の如き
  は若し純粹梵語として解釋せば於格であるかの如く見える。然しこれは決
  して於格ではない。俗語の中にはかうした用例が屢々あるのだが,これは呼
  格である。般若心經の中に出て來るガテー,ガテー等の咒も亦さうである。
  その意義も茫然として捕捉し難いのが常である。此の咒の如きはまだ比較
  的意義の判然してゐる部分が多い方と云はねばならぬ。この方面の研究に
  ついては學会に未だ完成した業績が擧つてゐない。アタルヷ吠陀あたりか
  らみつしり研究してかからねばならぬ相當重大な問題が課せられてゐる。
  }


\newpage
cchati \da{} tāṃ mātāha \da{} niyataṃ putri śramaṇena Gautamena
samanvāhṛto (gato) bhaviṣyati tena mama mantrāḥ pratihatā
bhaviṣyanti \da{} Prakṛtir āha \da{} kiṃ punar amba dalavattarāḥ
śramaṇasya Gautamasya mantrā nāsmākam \da{} tāṃ mātāha \da{}
balavattarāḥ śramaṇasya Gautamasya mantrā nāsmākaṃ
ye putri mantrāḥ sarva-lokasya prabhavanti tān mantrān
śramaṇo Gautama ākaṅksamāṇāḥ pratihanti na punar lokaḥ
prabhavati śramaṇasya Gautamasya mantrān pratihantum
evaṃ balavattarāḥ śramaṇasya Gautamasya mantrāḥ \dd{}

\rightline{(Divyāvadāna XXXIII)}

\newpage
\theendnotes

%%% Local Variables:
%%% mode: latex
%%% TeX-master: "IntroductionToSanskrit"
%%% End:

\newpage

\texttitle{般若心經}
\addcontentsline{toc}{chapter}{\protect\numberline{}般若心經}%

觀自在菩薩は甚深なる般若波羅蜜多に於て行を行じつつ觀じ
き。五蘊あり。而してそれらを彼は自性空なりと觀じき。此に
舎利弗よ,色は空なり,空は色なり。色より空は異らず,空よ
り色は異らず。色なる所のものは空なり,空なる所のものは色な
り。是の如く實に受想行識も然り。此に舎利弗よ,一切諸法
は空相あり,不生不滅なり,不垢不淨,不滅不滿なり。故に舎利
弗よ空の中に色なく受なく想なく行なく識なし。眼耳鼻舌身意無
し。色聲香味觸法なし。限界乃至意識界無し。明なく無明無し。
明の盡なく,無明の盡なし。乃至老死なく老死の盡なし。苦集滅
道無く,智なく,得無し,無所得の故に。菩薩は般若波羅蜜多によ
りて住しつゝ心罣礙無し。心罣礙無きが故に怖畏無く,顚倒を離
れ,涅槃を究竟す。三世安立の諸仏は般若波羅蜜多によりて無上
正等覺を證得せり。故に知るべし,般若波羅蜜多は大明咒,無上

\newpage
\texttitle{Hṛdaya-sūtram}
Avalokiteśvara-bodhisattvo gambhīrāyāṃ prajñā-pārami\-%
tāyāṃ cartāṃ caramāṇo vyavalokayati sma \da{} pañca-skan\-%
dhāḥ \da{} tāṃś ca svadhāva-śūnyān paśyati sma iha Śāriputra
rūpaṃ ṣūnyatā śūnyataiva rūpaṃ rūpān na pṛthak śūnyatā
śūnyatāyā na pṛthag rūpaṃ yad rūpaṃ sā śūnyatā yā śūn\-%
yatā tad rūpam \da{} evam eva vedanā-saṃjñā-saṃskāra-vijñā\-%
nāni \da{} iha Śāriputra sarva-dharmāḥ śūnyatā-lakṣaṇā anu\-%
tpannā aniruddhā amalā na vimalā nonā na paripūrṇāḥ \da{}
tasmāc Chāriputra śūnyatāyāṃ na rūpaṃ na vedanā na
saṃjñā na saṃskārā na vijñānāni na cakṣuḥ-śrotra-ghrāṇa\-%
jihvā-kāya-manāṃsi \da{} na rūpa-śabdha-gandha-rasa-spraṣ\-%
ṭavya-dharmāḥ \da{} na cakṣur-dhātur yāvan na mano\-%
vijñāna-dhātuḥ na vidyā nāvidyā na vidyā-kṣayo nāvidyā\-%
kṣayo yāvan na jarā-maraṇaṃ na jarā-maraṇa-kṣayo na
duḥkha-samudaya-nirodha-mārgā na jñānaṃ na prāptir
aprāptitvena \da{} bodhisattvasya prajñā-pāramitām āśritya
viharato 'cittāvaraṇaḥ cittāvaraṇa-nāstitvād atrasto vipar\-%
yāsātikrānto niṣṭha-nirvāṇaḥ \da{} try-adhva-vyavasthitāḥ sarva\-%
buddhāḥ prajñā-pāramitām āsrityānuttarāṃ samyak-saṃ\-%
bodhim adhisaṃbuddhāḥ \da{} tasmāj jñātavyaṃ prajñā-pāramitā
mahā-mantro mahā-vidyā-mantro 'nuttara-mantro 'samasama

\newpage
咒,無等等咒なり。一切の苦を鎮め,不虚なるが故に眞實なり,
般若波羅蜜多の中に說かれたる咒あり。卽ち,行けり,行けり。
彼岸に行けり。皆彼岸に行けり。覺よ。スヷーハー。

\rightline{(以上般若波羅蜜多心經畢る)}\wosfnt{%
  般若心經は恐らく佛敎經典の中で最も短いものであらう。而もその内容
  は尨大浩瀚なる大般若がこの纔かに一紙にも足らぬ經典中によく撮要さ
  れてゐる。故に現在諸宗に於いて朝昏にこれを讀誦してゐるのも少くな
  い。般若の行は五蘊皆空に在ることを示し,色卽是空,空卽是色,受想
  行識,一切萬有に於いて空性を宣言し,一切の菩薩もこれによりて涅槃を究
  竟し,三世の諸佛もこれによりて正覺を成ずと云ひ,最後に楬諦等の陀羅
  尼までが附加せられてゐる。

  この梵本は有名な法隆寺傳來の貝多羅葉に書かれたものを底本とする。
  淨嚴,慈雲等の日本梵學者が丁寧に綿密に傳寫したものである。明治十三
  年の交,諸種の資料と共にこれも英國牛津のマックスミューレル敎授の許へ
  送られ,南條笠原二氏が銳意研鑽し,明治十七年古代貝葉(The Ancient
  Palm-leaves)と題して牛津紀要(Anecdota Oxoniensia)の中に於
  いて刊行せられてゐる。

  漢譯には羅什,玄弉\endnote{「玄弉」はママ。}の手になれるもの共に大正藏第八卷中に收載せられ
  てある。尙ほ梵本に廣略二本がある。普通に略本の方が行はれてゐて心經
  と云はば略本の方を意味するやうである。廣本の方では起首に如是我聞と
  あり,結尾に歡喜信受奉行とある通常經典の形式を具へてゐる。廣本の漢
  譯も前記大正藏第八卷中に種々收載せられてゐる。略本は恐らく廣本から
  略抄されてそれが行はれるやうになつたものと思はれる。尙ほこの心經の
  原形とも云ふべきものは大品般若の中の一節例へば習應品などの章句であ
  らうとのことである。

  梵本の整理については一部分の讀み方がまだ學徒の間に懸案となつてゐ
  て未解決のままである。牛津本のマックスミューレル敎授の讀み方には確か
  に訂正せねばならぬ部分が若干あつて,それを榊博士が解說梵語學の上で
  注意せられたのは大に多とするに足る。然し榊博士の本文でも現今では訂
  正せられねばならぬ部分が若干ある。殊にその一は「心無罣礙」の一節で
  ある。

  此に擧げた本文は實は私一流の試案であつて果してこれが最後のものか
  どうかそれは保證もできぬ。然し法隆寺貝葉の文面を基礎とし玄奘等の漢
  譯に準據するとかうした讀み方も一應許されるかと思ふ。これは曾てマユ
  ーラ誌上に發表したこともあり,荻原博士は聖語研究に私のこの読み方を
  批評しつゝ自分の說を出してゐられる。然し荻原博士の說は燉煌出土と云
  はれる別のテキストの上に根據を置くもので,それはそれで話が違ふ。私
  は法隆寺貝葉を根柢としてその上に修正を施したらばどうなるかと云ふと
  ころをねらつたまでである。尙ほ「故に知るべし」の語も中性にしてあ
  る。これも玄奘譯の意義から推して前後の關係を辿ると男性ではいけない
  やうだ。法隆寺貝葉についてはマックスミューレル敎授がこの本文を發表し
  て學界をあつと云はせた時にウェーバー敎授がそんなものは贋造品である。
  一體寺院の傳說は取り上げる價値のないものだと云つたので,マックスミュ
  ーレル敎授が躍起となり,論難往復のあつたいきさつが南條先生へ宛てた
  マクスミューレル教授\endnote{「マクスミューレル教授」はママ。}の書簡に見えてゐる。當年のことを回想すると面白
  い。今後の學徒はどうか存分に梵文の解讀力を發揮して適正な讀み方を考
  へ出して頂きたいと切願する。短い經典とは云へ決して等閑に附すべきで
  はない。}


\newpage
mantraḥ sarva-duḥkha-praśamanaḥ satyam amithyatvāt pra\-%
jñā-pāramitāyām ukto mantraḥ \da{} tadyathā gate gate pāragate
pāra-saṃgate bodhi svāhā \dd{} iti prajñā-pāramitā-hṛdayaṃ
samāptam \dd{}


\newpage
\theendnotes


%%% Local Variables:
%%% mode: latex
%%% TeX-master: "IntroductionToSanskrit"
%%% End:

\newpage

\texttitle{阿彌陀經}
\addcontentsline{toc}{chapter}{\protect\numberline{}阿彌陀經}%

舎利弗よ,如何にそれを汝は思ふや,何の理由ありてか彼の如
來は無量壽と名けらるゝや。舎利弗よ,彼の如來並に彼等人々
の壽量は量られず。この理由によりてかの如來は無量壽と名けら
る。舎利弗よ,而して彼の如來は無上なる正等覺を證得せし以來
十劫を經たり。

舎利弗よ,如何にそれを汝は思ふや,何の理由ありてか,彼の
如來は無量光と名けらるるや。舎利弗よ,彼の如來の光明は一切
佛國に於て無礙なり,この理由によりて彼の如來は無量光と名け
る。又舎利弗よ,かの如來の聲聞衆は無量なり,その清淨なる應
供の量を擧ぐることは容易にあらず。舎利弗よ,かくの如きの佛
國功德莊嚴を以てその佛國は飾られたり。

又舎利弗よ,無量壽如來の佛國に於て生るる有情は清淨なる菩
薩にして不退轉なり。一生所繫なり。舎利弗よ,その菩薩の量を
擧ぐることは無量無數なりとの數に行くを除きては容易にあら
ず。又舎利弗よ,その佛國に於て有情は發願すべし。その故は如
何。そこに實に是の如き正士と共に俱會はあるべし。舎利弗よ,
少善根のみによりてはその無量壽如來の佛國に於いて有情は生ぜ

\newpage

\texttitle{Sukhāvatīvyūhaḥ}
tat kiṃ manyase Śāriputra kena kāraṇena sa tathāgato
'mitāyur nāmocyate \da{} tasya khalu punaḥ Śāriputra taihāga\-%
tasya teṣāṃ ca manuṣyāṇām aparimitam āyuḥ-pramāṇam \da{}
tena kāraṇena sa tathāgato 'mitāyur nāmocyate \da{} tasya ca
Śāriputra tathāgatasya daśa kalpā anuttarāṃ samyaksaṃ\-%
bodhim abhisaṃbuddhasya \dd

tat kiṃ manyase Śāriputra kena kāraṇena sa tathāgato
'mitābho nāmocyate \da{} tasya khalu punaḥ Śāriputra tathāga\-%
tasyābhāpratihatā sarva-buddha-kṣetreṣu \da{} tena kāraṇena sa
tathāgato 'mitābho nāmocyate \da{} tasya ca Śāriputra tathāga\-%
tasyāprameyaḥ śrāvaka-saṃgho yeṣāṃ na sukaraṃ pramāṇam
ākhyātuṃ śuddhānām arhatām \da{} evaṃ-rūpaiḥ Śāriputra
buddha-kṣetra-guṇa-vyūhaiḥ samalaṃkṛtaṃ tad-buddha-
kṣetram \dd

punar aparaṃ Śāriputra ye 'mitāyuṣas tathāgatasya buddha-
kṣetre sattvā upapannāḥ śuddhā bodhisattvā avinivartanīyā
eka-jāti-pratibaddhās teṣāṃ Śāriputra bodhisattvānāṃ na
sukaraṃ pramāṇam ākhyātum anyatrāprameyāsaṃkhyeyā
iti saṃkhyāṃ gacchanti \da{} tatra khalu punaḥ Śāriputra
buddha-kṣetre sattvaiḥ praṇidhānaṃ kartavyam \da tat kasmād
dhetoḥ \da{} tatra hi nāma tathā-rūpaiḥ sat-puruṣaiḥ saha sama-

\newpage
\noindent
ず。舎利弗よ,如何なる善男子又は善女子にてもかの世尊無量壽
如來の名號を聞くべし,聞きて作意すべし。若くは一夜,若くは二
夜,若くは三夜,若くは四夜,若くは五夜,若くは六夜,若くは七夜
の間,不散亂の意にて作意すべし。彼の善男子又は善女子は死す
べし。彼の死する時,彼の無量壽如來は聲聞衆に圍繞せられ,菩薩
の衆に圍繞せられて前に至るべし。彼は不顚倒の心にて死すべ
し。彼は死してかの無量壽如來の佛國卽ち樂有世界に生るべし。
故に舎利弗よ,この義利を見つゝ我は是の如く說く。恭敬して善
男子又は善女子はその佛國に於いて發願すべし。\wosfnt{%
  日本の全土,津々浦々の涯までも,この阿彌陀經ほどよく行き渡つて讀
  まれてゐるお經はあるまい。實に日本佛敎僧侶の過半數はこれを佛前に諷
  誦する。否\ruby{啻}{ただ}に僧侶のみでなく,信徒にしてよくこれを讀むものも尠くは
  ない。蓋しこれわが日本國に於いての最も有緣の經典の一と謂はねばなら
  ぬ。

  有緣の經典であることに就いて,此に特記せねばならぬことは,この梵
  文が日本に於いて完全に保存せられてゐて,未だ世界何れの處からも發見
  せられないといふ事實である。勿論これは印度なり西域なりのある地方
  で,ある時代に筆寫せられ支那を經て日本へ傳來したのであらうが,今日
  では印度にも支那にも何れの處にも發見されずに,それが奇妙にも日本に
  だけ存在したといふことは實に不可思議の因緣あるを思はざるを得ない。
  恐らくこれは平安初期の留學僧の將來にかかるものであらうが慈雲の梵
  篋三本の記錄によれば梵文寫本が三部あつたと云ふ。旣に常明の梵漢阿彌
  陀經はこれより十年も前に刊行されてゐる。これらの資料を基礎として明
  治十三年マックスミューレル敎授南條笠原二氏の共同研究成り明治十六年の
  第一公刊となつたのである。恁うした因緣で日本が梵文を世界の舞臺へ送
  り出したことは日本國民として銘記せねばならぬ痛快事の一である。

  此に掲出したのは經の中心をなす彌陀の名義から執持名號,俱會一處を
  說く一段である。支那譯には羅什玄弉のものあれど,羅什譯最も行はれ
  る。比較研究すべきである。}

\newpage
vadhānaṃ bhavati nāvara-mātrakeṇa Śāriputra kuśala-mūle\-%
nāmitāyuṣas tathāgatasya buddhakṣetre sattvā upapadyante \da{}
yaḥ kaścic Chāriputra kulaputro vā kula-duhitā vā tasya bhaga\-%
vato 'mitāyuṣas tathāgatasya nāmadheyaṃ śroṣyati śrutvā
ca manasikariṣyati eka-rātraṃ vā dvi-rātraṃ vā tri-rātraṃ vā
catūrātraṃ vā pañca-rātraṃ vā ṣaḍ-rātrām vā sapta-rātraṃ
vāvikṣipta-citto manasikarṣyati yadā sa kulaputra vā kula\-%
duhitā vā kālaṃ kariṣyati tasya kālaṃ kurvataḥ so 'mitāyus
tathāgataḥ śrāvaka-saṃgha-parivṛto dobhisattva-gaṇa-puras\-%
kṛtaḥ purataḥ sthāsyati so 'viparyasta-cittaḥ kālaṃ kariṣyati
ca \da{} sa kālaṃ kṛtvā tasyaivāmitāyuṣas tathāgatasya buddha\-%
kṣetra Sukhāvatyāṃ lokadhātāv upapatsyate \da{} tasmāt tarhi
Śāriputra idam artha-vaśaṃ saṃpaśyamāna evaṃ vadāmi \da{}
satkṛtya kula-putreṇa vā kula-duhitrā vā tatra buddha-kṣetre
citta-praṇidhānaṃ kartavyam \dd


%%% Local Variables:
%%% mode: latex
%%% TeX-master: "IntroductionToSanskrit"
%%% End:


\end{document}

%%% Local Variables:
%%% mode: latex
%%% TeX-master: t
%%% TeX-engine: luatex
%%% End:
