\newpage

\texttitle{般若心經}
\addcontentsline{toc}{chapter}{\protect\numberline{}般若心經}%

觀自在菩薩は甚深なる般若波羅蜜多に於て行を行じつつ觀じ
き。五蘊あり。而してそれらを彼は自性空なりと觀じき。此に
舎利弗よ,色は空なり,空は色なり。色より空は異らず,空よ
り色は異らず。色なる所のものは空なり,空なる所のものは色な
り。是の如く實に受想行識も然り。此に舎利弗よ,一切諸法
は空相あり,不生不滅なり,不垢不淨,不滅不滿なり。故に舎利
弗よ空の中に色なく受なく想なく行なく識なし。眼耳鼻舌身意無
し。色聲香味觸法なし。限界乃至意識界無し。明なく無明無し。
明の盡なく,無明の盡なし。乃至老死なく老死の盡なし。苦集滅
道無く,智なく,得無し,無所得の故に。菩薩は般若波羅蜜多によ
りて住しつゝ心罣礙無し。心罣礙無きが故に怖畏無く,顚倒を離
れ,涅槃を究竟す。三世安立の諸仏は般若波羅蜜多によりて無上
正等覺を證得せり。故に知るべし,般若波羅蜜多は大明咒,無上

\newpage
\texttitle{Hṛdaya-sūtram}
Avalokiteśvara-bodhisattvo gambhīrāyāṃ prajñā-pārami\-%
tāyāṃ cartāṃ caramāṇo vyavalokayati sma \da{} pañca-skan\-%
dhāḥ \da{} tāṃś ca svadhāva-śūnyān paśyati sma iha Śāriputra
rūpaṃ ṣūnyatā śūnyataiva rūpaṃ rūpān na pṛthak śūnyatā
śūnyatāyā na pṛthag rūpaṃ yad rūpaṃ sā śūnyatā yā śūn\-%
yatā tad rūpam \da{} evam eva vedanā-saṃjñā-saṃskāra-vijñā\-%
nāni \da{} iha Śāriputra sarva-dharmāḥ śūnyatā-lakṣaṇā anu\-%
tpannā aniruddhā amalā na vimalā nonā na paripūrṇāḥ \da{}
tasmāc Chāriputra śūnyatāyāṃ na rūpaṃ na vedanā na
saṃjñā na saṃskārā na vijñānāni na cakṣuḥ-śrotra-ghrāṇa\-%
jihvā-kāya-manāṃsi \da{} na rūpa-śabdha-gandha-rasa-spraṣ\-%
ṭavya-dharmāḥ \da{} na cakṣur-dhātur yāvan na mano\-%
vijñāna-dhātuḥ na vidyā nāvidyā na vidyā-kṣayo nāvidyā\-%
kṣayo yāvan na jarā-maraṇaṃ na jarā-maraṇa-kṣayo na
duḥkha-samudaya-nirodha-mārgā na jñānaṃ na prāptir
aprāptitvena \da{} bodhisattvasya prajñā-pāramitām āśritya
viharato 'cittāvaraṇaḥ cittāvaraṇa-nāstitvād atrasto vipar\-%
yāsātikrānto niṣṭha-nirvāṇaḥ \da{} try-adhva-vyavasthitāḥ sarva\-%
buddhāḥ prajñā-pāramitām āsrityānuttarāṃ samyak-saṃ\-%
bodhim adhisaṃbuddhāḥ \da{} tasmāj jñātavyaṃ prajñā-pāramitā
mahā-mantro mahā-vidyā-mantro 'nuttara-mantro 'samasama

\newpage
咒,無等等咒なり。一切の苦を鎮め,不虚なるが故に眞實なり,
般若波羅蜜多の中に說かれたる咒あり。卽ち,行けり,行けり。
彼岸に行けり。皆彼岸に行けり。覺よ。スヷーハー。

\rightline{(以上般若波羅蜜多心經畢る)}\wosfnt{%
  般若心經は恐らく佛敎經典の中で最も短いものであらう。而もその内容
  は尨大浩瀚なる大般若がこの纔かに一紙にも足らぬ經典中によく撮要さ
  れてゐる。故に現在諸宗に於いて朝昏にこれを讀誦してゐるのも少くな
  い。般若の行は五蘊皆空に在ることを示し,色卽是空,空卽是色,受想
  行識,一切萬有に於いて空性を宣言し,一切の菩薩もこれによりて涅槃を究
  竟し,三世の諸佛もこれによりて正覺を成ずと云ひ,最後に楬諦等の陀羅
  尼までが附加せられてゐる。

  この梵本は有名な法隆寺傳來の貝多羅葉に書かれたものを底本とする。
  淨嚴,慈雲等の日本梵學者が丁寧に綿密に傳寫したものである。明治十三
  年の交,諸種の資料と共にこれも英國牛津のマックスミューレル敎授の許へ
  送られ,南條笠原二氏が銳意研鑽し,明治十七年古代貝葉(The Ancient
  Palm-leaves)と題して牛津紀要(Anecdota Oxoniensia)の中に於
  いて刊行せられてゐる。

  漢譯には羅什,玄弉\endnote{「玄弉」はママ。}の手になれるもの共に大正藏第八卷中に收載せられ
  てある。尙ほ梵本に廣略二本がある。普通に略本の方が行はれてゐて心經
  と云はば略本の方を意味するやうである。廣本の方では起首に如是我聞と
  あり,結尾に歡喜信受奉行とある通常經典の形式を具へてゐる。廣本の漢
  譯も前記大正藏第八卷中に種々收載せられてゐる。略本は恐らく廣本から
  略抄されてそれが行はれるやうになつたものと思はれる。尙ほこの心經の
  原形とも云ふべきものは大品般若の中の一節例へば習應品などの章句であ
  らうとのことである。

  梵本の整理については一部分の讀み方がまだ學徒の間に懸案となつてゐ
  て未解決のままである。牛津本のマックスミューレル敎授の讀み方には確か
  に訂正せねばならぬ部分が若干あつて,それを榊博士が解說梵語學の上で
  注意せられたのは大に多とするに足る。然し榊博士の本文でも現今では訂
  正せられねばならぬ部分が若干ある。殊にその一は「心無罣礙」の一節で
  ある。

  此に擧げた本文は實は私一流の試案であつて果してこれが最後のものか
  どうかそれは保證もできぬ。然し法隆寺貝葉の文面を基礎とし玄奘等の漢
  譯に準據するとかうした讀み方も一應許されるかと思ふ。これは曾てマユ
  ーラ誌上に發表したこともあり,荻原博士は聖語研究に私のこの読み方を
  批評しつゝ自分の說を出してゐられる。然し荻原博士の說は燉煌出土と云
  はれる別のテキストの上に根據を置くもので,それはそれで話が違ふ。私
  は法隆寺貝葉を根柢としてその上に修正を施したらばどうなるかと云ふと
  ころをねらつたまでである。尙ほ「故に知るべし」の語も中性にしてあ
  る。これも玄奘譯の意義から推して前後の關係を辿ると男性ではいけない
  やうだ。法隆寺貝葉についてはマックスミューレル敎授がこの本文を發表し
  て學界をあつと云はせた時にウェーバー敎授がそんなものは贋造品である。
  一體寺院の傳說は取り上げる價値のないものだと云つたので,マックスミュ
  ーレル敎授が躍起となり,論難往復のあつたいきさつが南條先生へ宛てた
  マクスミューレル教授\endnote{「マクスミューレル教授」はママ。}の書簡に見えてゐる。當年のことを回想すると面白
  い。今後の學徒はどうか存分に梵文の解讀力を發揮して適正な讀み方を考
  へ出して頂きたいと切願する。短い經典とは云へ決して等閑に附すべきで
  はない。}


\newpage
mantraḥ sarva-duḥkha-praśamanaḥ satyam amithyatvāt pra\-%
jñā-pāramitāyām ukto mantraḥ \da{} tadyathā gate gate pāragate
pāra-saṃgate bodhi svāhā \dd{} iti prajñā-pāramitā-hṛdayaṃ
samāptam \dd{}


\newpage
\theendnotes


%%% Local Variables:
%%% mode: latex
%%% TeX-master: "IntroductionToSanskrit"
%%% End:
