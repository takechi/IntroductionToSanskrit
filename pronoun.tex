\section{代名詞}
\subsection{人稱代名詞}
\numberParagraph
一人稱代名詞(吾)語基は 單 mad, 複 asmad.

\begin{center}
\begin{tabular}{c*{3}{p{0.2\hsize}}}
     & 單                    & 兩                        & 複 \\
  主 & \rdelim\}{2}{*}[aham] & \rdelim\}{3}{*}[āvām]     & \rdelim\}{2}{*}[vayam] \\
  呼 &                       &                           & \\
  業 & mām, mā               &                           & asmān, nas \\
  具 & mayā                  & \rdelim\}{3}{*}[āvābhyām] & asmābhis \\
  爲 & mahyam, me            &                           & asmabhyam, nas \\
  從 & mat                   &                           & asmat \\
  屬 & mama, me              & \rdelim\}{2}{*}[āvayos]   & asmākam, nas \\
  於 & mayi                  &                           & asmāsu
\end{tabular}
\end{center}

兩數の業爲屬に nau を用ふることがある。

\numberParagraph
二人稱代名詞(汝)語基は 單 tvad, 複 yuṣmad.

\begin{center}
\begin{tabular}{c*{3}{p{0.2\hsize}}}
     & 單                    & 兩                         & 複 \\
  主 & \rdelim\}{2}{*}[tvam] & \rdelim\}{3}{*}[yuvām]     & \rdelim\}{2}{*}[yūyam] \\
  呼 &                       &                            & \\
  業 & tvām, tvā             &                            & yuṣmān, vas \\
  具 & tvayā                 & \rdelim\}{3}{*}[yuvābhyām] & yuṣmābhis \\
  爲 & tubhyam, te           &                            & yuṣmabhyam, vas \\
  從 & tvat                  &                            & yuṣmat \\
  屬 & tava, te              & \rdelim\}{2}{*}[yuvayos]   & yuṣmākam, vas \\
  於 & tvayi                 &                            & yuṣmāsu
\end{tabular}
\end{center}

兩數の業爲屬に vām を用ふることがある。

\subsection{指示代名詞}
\numberParagraph
語基 tad. これは同時に三人稱代名詞(彼,夫)でもあり得
る,\endnote{底本では女性複{数}具格(tābhis)が{空}欄。}

\begin{center}
\begin{tabular}{c*{9}{p{0.085\hsize}}}
     & \multicolumn{3}{c}{單}                                                                                & \multicolumn{3}{c}{兩}                                           & \multicolumn{3}{c}{複} \\
     & \multicolumn{3}{c}{\downbracefill}                                                                    & \multicolumn{3}{c}{\downbracefill}                               & \multicolumn{3}{c}{\downbracefill} \\
     & 男                                & 中                   & 女                      & 男                   & 中                  & 女                  & 男                  & 中                     & 女 \\
  主 & \rdelim\}{2}{*}[sas\footnotemark] & \rdelim\}{3}{*}[tat] & \rdelim\}{2}{*}[sā]     & \rdelim\}{3}{*}[tau] & \multirow{3}{*}{te} & \multirow{3}{*}{te} & \rdelim\}{2}{*}[te] & \rdelim\}{3}{*}[tāni]  & \multirow{3}{*}{tās} \\
  呼 &                                   &                      &                         &                      &                     &                     &                     &                        & \\
  業 & tam                               &                      & tām                     &                      &                     &                     & tān                 &                        & \\
     & \multicolumn{2}{c}{\upbracefill}                         &                         & \multicolumn{3}{c}{\upbracefill}                                 & \multicolumn{2}{c}{\upbracefill}             & \\
  具 & \multicolumn{2}{c}{tena}                                 & tayā                    & \multicolumn{3}{c}{\rdelim\}{3}{*}[tābhyām]}                     & \multicolumn{2}{c}{tais}                     & tābhis \\
  爲 & \multicolumn{2}{c}{tasmai}                               & tasyai                  &                      &                     &                     & \multicolumn{2}{c}{\rdelim\}{2}{*}[tebhyas]} & \multirow{2}{*}{tābhyas} \\
  從 & \multicolumn{2}{c}{tasmāt}                               & \rdelim\}{2}{*}[tasyās] &                      &                     &                     &                   &                          & \\
  屬 & \multicolumn{2}{c}{tasya}                                &                         & \multicolumn{3}{c}{\rdelim\}{2}{*}[tayos]}                       & \multicolumn{2}{c}{teṣām}                    & tāsām \\
  於 & \multicolumn{2}{c}{tasmin}                               & tasyām                  &                      &                     &                     & \multicolumn{2}{c}{teṣu}                     & tāsu
\end{tabular}
\footnotetext{{\ref{np:24}條註參照。}}
\end{center}

\numberParagraph
語基 etad (此れ)の變化は tad に準ず。

主,單 eṣas, eṣā, etad.

\numberParagraph
語基 enad (彼)は只業の三數と具の單,屬於の兩がある
のみである。卽ち:

\begin{center}
\begin{tabular}{c*{3}{p{0.15\hsize}}}
     & \multicolumn{3}{c}{單} \\
     & 男   & 中                        & 女 \\
  業 & enam & enat                      & enām \\
     & \multicolumn{2}{c}{\upbracefill} & \\
  具 & \multicolumn{2}{c}{enena}        & enayā
\end{tabular}
\end{center}
\begin{center}
\begin{tabular}{c*{3}{p{0.15\hsize}}}
     & \multicolumn{3}{c}{兩} \\
  業 & enau & ene                      & ene \\
     & \multicolumn{3}{c}{\upbracefill} \\
  屬 & \multicolumn{3}{c}{\rdelim\}{2}{*}[enayos]} \\
  於 & \\
\end{tabular}
\end{center}
\begin{center}
\begin{tabular}{c*{3}{p{0.15\hsize}}}
     & \multicolumn{3}{c}{複} \\
  業 & enān & enāni & enās
\end{tabular}
\end{center}

\numberParagraph
語基 idam (これ),adas (其れ)。

\begin{center}
\begin{tabular}{c*{9}{p{0.1\hsize}}}
     & \multicolumn{6}{c}{單數} \\
     & 男                    & 中                    & 女                     & 男                    & 中                    & 女 \\
  主 & \rdelim\}{2}{*}[ayam] & \rdelim\}{3}{*}[idam] & \rdelim\}{2}{*}[iyam]  & \multirow{2}{*}{asau} & \rdelim\}{3}{*}[adas] & \rdelim\}{2}{*}[asau] \\
  呼 &                       &                       &                        &                       &                       & \\
  業 & imam                  &                       & imām                   & amum                  &                       & amūm \\
     & \multicolumn{2}{c}{\upbracefill}              &                        & \multicolumn{2}{c}{\upbracefill}              & \\
  具 & \multicolumn{2}{c}{anena}                     & anayā                  & \multicolumn{2}{c}{amunā}                     & amuyā \\
  爲 & \multicolumn{2}{c}{asmai}                     & asyai                  & \multicolumn{2}{c}{amuṣmai}                   & amuṣyai \\
  從 & \multicolumn{2}{c}{asmāt}                     & \rdelim\}{2}{*}[asyās] & \multicolumn{2}{c}{amuṣmāt}                   & \rdelim\}{2}{*}[amuṣyās] \\
  屬 & \multicolumn{2}{c}{asya}                      &                        & \multicolumn{2}{c}{amuṣya}                    & \\
  於 & \multicolumn{2}{c}{asmin}                     & asyām                  & \multicolumn{2}{c}{amuṣmin}                   & amuṣyām
\end{tabular}
\end{center}
\begin{center}
\begin{tabular}{c*{4}{p{0.1\hsize}}}
     & \multicolumn{4}{c}{兩數} \\
     & 男                    & 中                   & 女                    & 男中女 \\
  主 & \rdelim\}{3}{*}[imau] & \multirow{3}{*}{ime} & \multirow{3}{*}{ime}  & \multirow{3}{*}{amū} \\
  呼 &                       &                      &                       & \\
  業 &                       &                      &                       & \\
     & \multicolumn{3}{c}{\upbracefill}                                     & \\
  具 & \multicolumn{3}{c}{\rdelim\}{3}{*}[ābhyām]}                          & \multirow{3}{*}{amūbhyām} \\
  爲 &                                                                      & \\
  從 &                                                                      & \\
  屬 & \multicolumn{3}{c}{\rdelim\}{2}{*}[ābhyām]}                          & \multirow{2}{*}{amuyos} \\
  於 &                                                                      &
\end{tabular}
\end{center}
\begin{center}
\begin{tabular}{c*{9}{p{0.1\hsize}}}
     & \multicolumn{6}{c}{複數} \\
     & 男                   & 中                     & 女                      & 男                    & 中                     & 女 \\
  主 & \rdelim\}{2}{*}[ime] & \rdelim\}{3}{*}[imāni] & \multirow{3}{*}{imās}   & \rdelim\}{2}{*}[amī]  & \rdelim\}{3}{*}[amūni] & \multirow{3}{*}{amūs} \\
  呼 &                      &                        &                         &                       &                        & \\
  業 & imān                 &                        &                         & amūn                  &                        & \\
     & \multicolumn{2}{c}{\upbracefill}              &                         & \multicolumn{2}{c}{\upbracefill}               & \\
  具 & \multicolumn{2}{c}{ebhis}                     & ābhis                   & \multicolumn{2}{c}{amībhis}                    & amūbhis \\
  爲 & \multicolumn{2}{c}{\rdelim\}{2}{*}[ebhyas]}   & \multirow{2}{*}{ābhyas} & \multicolumn{2}{c}{\rdelim\}{2}{*}[amībhyas]}  & \multirow{2}{*}{amūbhyas} \\
  從 &                                               &                         &                                                & \\
  屬 & \multicolumn{2}{c}{eṣām}                      & āsām                    & \multicolumn{2}{c}{amīṣām}                     & amūṣām \\
  於 & \multicolumn{2}{c}{eṣu}                       & āsu                     & \multicolumn{2}{c}{amīṣu}                      & amūṣu
\end{tabular}
\end{center}

\subsection{關係代名詞}
\numberParagraph
語基 yad (所のそれは),變化は tad に準ず。

\begin{center}
\begin{tabular}{c*{9}{p{0.085\hsize}}}
     & \multicolumn{3}{c}{單}                                                               & \multicolumn{3}{c}{兩}                                           & \multicolumn{3}{c}{複} \\
     & \multicolumn{3}{c}{\downbracefill}                                                   & \multicolumn{3}{c}{\downbracefill}                               & \multicolumn{3}{c}{\downbracefill} \\
     & 男                               & 中                      & 女                      & 男                   & 中                  & 女                  & 男                  & 中                     & 女 \\
  主 & \rdelim\}{2}{*}[yas]             & \rdelim\}{3}{*}[yat]    & \rdelim\}{2}{*}[yā]     & \rdelim\}{3}{*}[yau] & \multirow{3}{*}{ye} & \multirow{3}{*}{ye} & \rdelim\}{2}{*}[ye] & \rdelim\}{3}{*}[yāni]  & \multirow{3}{*}{yās} \\
  呼 &                                  &                         &                         &                      &                     &                     &                     &                        & \\
  業 & yam                              &                         & yām                     &                      &                     &                     & yān                 &                        & \\
     & \multicolumn{2}{c}{\upbracefill}                           &                         & \multicolumn{3}{c}{\upbracefill}                                 & \multicolumn{2}{c}{\upbracefill}             & \\
  具 & \multicolumn{2}{l}{yena}                                   & yayā                    & \multicolumn{3}{l}{\rdelim\}{3}{*}[yābhyām]}                     & \multicolumn{2}{l}{yais}                     & yābhis \\
  爲 & \multicolumn{2}{l}{yasmai}                                 & yasyai                  &                      &                     &                     & \multicolumn{2}{l}{\rdelim\}{2}{*}[yebhyas] \hfill} & \multirow{2}{*}{yābhyas} \\
  從 & \multicolumn{2}{l}{yasmāt}                                 & \rdelim\}{2}{*}[yasyās] &                      &                     &                     &                   &                          & \\
  屬 & \multicolumn{2}{l}{yasya}                                  &                         & \multicolumn{3}{l}{\rdelim\}{2}{*}[yayos]}                       & \multicolumn{2}{l}{yeṣām}                    & yāsām \\
  於 & \multicolumn{2}{l}{yasmin}                                 & yasyām                  &                      &                     &                     & \multicolumn{2}{l}{yeṣu}                     & yāsu
\end{tabular}
\end{center}

\subsection{疑問代名詞}
\numberParagraph
語基 kim (誰,何)亦 tad に準ず。

\begin{center}
\begin{tabular}{c*{9}{p{0.085\hsize}}}
     & \multicolumn{3}{c}{單}                                                               & \multicolumn{3}{c}{兩}                                           & \multicolumn{3}{c}{複} \\
     & \multicolumn{3}{c}{\downbracefill}                                                   & \multicolumn{3}{c}{\downbracefill}                               & \multicolumn{3}{c}{\downbracefill} \\
     & 男                               & 中                      & 女                      & 男                   & 中                  & 女                  & 男                  & 中                     & 女 \\
  主 & \rdelim\}{2}{*}[kas]             & \rdelim\}{3}{*}[kim]    & \rdelim\}{2}{*}[kā]     & \rdelim\}{3}{*}[kau] & \multirow{3}{*}{ke} & \multirow{3}{*}{ke} & \rdelim\}{2}{*}[ke] & \rdelim\}{3}{*}[kāni]  & \multirow{3}{*}{kās} \\
  呼 &                                  &                         &                         &                      &                     &                     &                     &                        & \\
  業 & kam                              &                         & kām                     &                      &                     &                     & kān                 &                        & \\
     & \multicolumn{2}{c}{\upbracefill}                           &                         & \multicolumn{3}{c}{\upbracefill}                                 & \multicolumn{2}{c}{\upbracefill}             & \\
  具 & \multicolumn{2}{l}{kena}                                   & kayā                    & \multicolumn{3}{l}{\rdelim\}{3}{*}[kābhyām]}                     & \multicolumn{2}{l}{kais}                     & kābhis \\
  爲 & \multicolumn{2}{l}{kasmai}                                 & kasyai                  &                      &                     &                     & \multicolumn{2}{l}{\rdelim\}{2}{*}[kebhyas] \hfill} & \multirow{2}{*}{kābhyas} \\
  從 & \multicolumn{2}{l}{kasmāt}                                 & \rdelim\}{2}{*}[kasyās] &                      &                     &                     &                   &                          & \\
  屬 & \multicolumn{2}{l}{kasya}                                  &                         & \multicolumn{3}{l}{\rdelim\}{2}{*}[kayos]}                       & \multicolumn{2}{l}{keṣām}                    & kāsām \\
  於 & \multicolumn{2}{l}{kasmin}                                 & kasyām                  &                      &                     &                     & \multicolumn{2}{l}{keṣu}                     & kāsu
\end{tabular}
\end{center}

\numberParagraph
疑問代名詞に cit, api, cana を附加すれば不定の意味を
有す。kaścit (誰にても),kimapi (何にても),na kaścit (決し
て何人も……せぬ)。

\subsection{代名詞的形容詞}
\numberParagraph
或る形容詞は代名詞的の變化をなす。
\begin{enumerate}[label=(\alph*)]
\item anyatara (他の)(主 anyas, anyā, anyat),katara (二
つの中誰,孰れ),katama (多くの中誰,孰れ),anyatara
(二つの中隨一),yatara (二の中の孰れ),yatama (多くの
中の孰れ),itara (他の)。
\item \label{item:108b}同樣に sarva (總ての),viśva (總ての),eka (一の),
ekatara (二の中の隨一の)は tad に準ず。只單中性主呼
業が m の語尾を附加せらるる差異に注意せねばならぬ。
\item adhara (劣れる,西方の),antara (外の),apara (他
の),uttara (上の),avara (後の,西方の),dakṣiṇa (右方
の,南の),para (後の),pūrva (前の),sva (自の)は \ref{item:108b}
に準じて變化す。只男中單,從,於と主複に於ては名詞變化
をなすを得。
\end{enumerate}

\subsection{代名詞的名詞}
\numberParagraph
名詞の或るものは屢々意義上代名詞の如く見做される。
\begin{enumerate}[label=(\alph*)]
\item ātman (我)は再歸代名詞として用ひられる。ātmānaṃ
naraḥ parīkṣeta (人は自らを省るべきだ)。
\item bhavat (君)なる語基は二人稱の意義に用ひられた代名
詞である。kva gacchati (又は gacchasi)bhavān (君は
何處に行く)。
\end{enumerate}

%%% Local Variables:
%%% mode: latex
%%% TeX-master: "IntroductionToSanskrit"
%%% End:
