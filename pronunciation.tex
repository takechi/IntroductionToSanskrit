\chapter{聲法}
\label{cha:pronunciation}

\section{發音}
\numberParagraph
以上の字母の發音に就て,大體近似せる音を假名を以て
示したが,二三の注意すべき點を下に擧げる。ṛ, ṝ は特殊の母
音である。リ リーと發音すれば大差なし。ḷ は li の音に近い。e,
o は夫々エー,オーと長く,ai, au もアーイ,アーウである。長音
符號はなきも必ず長く發音する。ṅ は ng の如く,kha, gha
等の h は聞えるやうに,クハ,グハであつて,カ,ガでない。ñ は
ny の如く,ṭ, ḍ, ṇ 等は舌端を卷き上顎齒根へ觸れて發音する。
ś は舌の根と上顎の間を氣息が擦過して發音せられる。具體的に
は舌の中央を高める心地にてシャを發音すれば可い。ṣ は sh の
如くシャである。h は喉音であつて全く口腔の聲音期關に關係せ
ぬ。ṃ も ḥ も必ず母音に隨ふものであつて ṃ は母音となるべ
き氣息が鼻腔に入つて外に出づるより生ずる音である。ḥ は母音
の發生に伴つて起る一種の摩擦音である。故に前に來る母音の影
響を受けて aḥ, iḥ, uḥ は夫々 アハ,イヒ,ウフ等となる。

\numberParagraph
子母音全部を發聲機關に隨つて喉,上顎,舌,齒,唇の五種に
分類し,又半母音,吹氣音の命名あること前述の如くである。更に
子音中含氣あり,無氣あり,硬あり,軟あり。含氣とは h 音を含む
ものであり,無氣とは含まざるものを云ふ。硬とは無響,軟とは
有響と見るべく,發音に當り聲帶の振動を伴ふか伴はざるかによ
りて次の如く有響無響を分つ。卽ち k, kh, c, ch, ṭ, ṭh, t, th, p,
ph, ś, ṣ, s の十三は無響であり其の他は子母音を通じて有響と
知るべきである。

\numberParagraph
單母音は a 音を添へて强められ,その質を變化して\textbf{複母音}
となる。複母音には二重の段階がある。第一重を重韻,グナ(guṇa)
と云ひ第二重を複重韻,ヴィリッド\textsubscript{ヒ}(vṛddhi)と云ふ。卽ち i, ī の
重韻は e であつて u, ū の重韻は o である。又 ai, au は夫々
その\textbf{複重韻}である a, ā の重韻は無く ā はその複重韻である。尚
ほ ṛ, ṝ の重韻は ar で\textbf{複重韻}は ār であり ḷ の重韻は al であ
つて複重韻は存在しない。又 ṛ の重韻は時に ra となり複重韻
は rā となることもある。

\begin{center}
\fbox{
\begin{tabular}{*{6}{c}}
  平音   & a \ ā & i \ ī        & u \ ū        & ṛ \ ṝ        & ḷ \\
         &       & \upbracefill & \upbracefill & \upbracefill & \\
  重韻   &       & e            & o            & ar(ra)       & al \\
  複重韻 & ā     & ai           & au           & ār(rā)       & \\
\end{tabular}}
\end{center}

造語の場合に於ける次の例を見てこれらの變化を知るべきである。

\begin{tabular}{lll}
  div (輝く) & deva (天) & daiva (運命) \\
  lul (轉ず) & lola (貪慾の) & \\
  bhṛ (\ruby{擔}{にな}ふ) & bhara (保持する) & bhāra (荷物) \\
  kṛ (作す) & kartṛ (作者) & kārya (義務) \\
  kḷp (適當にする) & \multicolumn{2}{l}{kalpate (彼は適當である)} \\
\end{tabular}

\section{連聲法(Saṃdhi)}
\numberParagraph
梵語には語と語の間に特殊の處置が施される。卽ち語末と
語首の音が相會してそこに性質を變ずるか,又は消失するか,又
は何か他の音が挿入せられるか等の變化がある。これを連聲法
(サンドヒ)と云ふ。これには母音と母音とに相會する場合,子音と
子音と相會する場合を分つて説くことにする。

\subsection{語と語の連聲法}
\subsubsection{母音相互間の連聲法}
\numberParagraph\label{np:11}
等しき單母音はその長韻となる。

\eg{
  adya api = adyāpi (今日も亦)\\
  yadi icchasi = yadīcchash (若しも汝が欲するならば)\\
  sādhu uktam = sādhūktam (よくも云はれた)。}

\numberParagraph
a, ā は等しからざる單母音と共に重韻となり e, o, ai, au
と共に複重韻となる。
\eg{
  ca iti = ceti (而して以上)\\
  tena uktam = tenoktam (彼は云へり)\\
  adhunā ṛṣir yajate = adhunarṣir yajate (今聖仙は
  祭祀す)\endnote{底本は「祀」ではなく「{\HanazonoA 祀󠄄}」。} \\
  kva eti = kvaiti (彼は何處に行く)\\
  atra oṣadhī tiṣṭhati = atrauṣadhī tiṣṭhati (此に藥
  草がある)。}

\numberParagraph \label{np:13}
i, ī, u, ū, ṛ, ṝ は等しからざる母音の次前に在る時
各自相當の半母音 y, v, r となる。
\eg{
  yadi evam = yady evam (若し此の如くならば)\\
  astu etat = astv etat (これをしてあらしめよ)。}

\numberParagraph \label{np:14}
a 以外の母音の次前に來る e, o は轉じて a となる(稀に
は ay, av)。
\eg{
  vane iha = vana iha (此に林に於て)\\
  prabho ehi = prabha ehi (王よ來れ)}

e 又は o の次後の a は省略せらる。而して省字 \,' を用ふ。\\
\hfill \ref{np:6} 條(c)
\eg{
  te atra = te 'tra (彼等はそこに)\\
  so api = so 'pi (彼も亦)。}

\numberParagraph
ai は母音の次前に來る時は ā となり,au は通常は āv となる。
\eg{
  bhāryāyai akathayat = bhāryāyā akathayat (彼は妻
  に語れり)\\
  putrau āgacchataḥ = putrāv āgacchataḥ (二人の子
  等は來れり)。}

\nt{格例法及活用法の兩數の語尾 ī, ū, e 及び amī (代名詞)の ī は
次後に母音來るも變化せず。}

\subsubsection{子音相互間の連聲法}
\numberParagraph
二個以上の子音が聲の終に來る時はその第一の聲のみを
存し,餘は總て除き去る。
\eg{
  prāṅks = prāṅ (東方の)\\
  bhavants = bhavan (bhū の現在分詞)〔主・男・單〕\\
  ahant = ahan (彼は打ちき)}
但し r が語根に屬する(又は語根に代用せらるる)k, ṭ, t, p
に先立つ場合は殘存せしむ。
\eg{
  ūrk (力)(k は j の代用)\\
  amārṭ (彼は拭へり)(語根 mṛj, ṭ は j の代用)。}

\numberParagraph \label{np:17}
連聲の法則は語の終に立つを許さるる次の九個の音の孰
れかに還元したる後適用せらるべきである。

\begin{center}
\begin{tabular}{cccccc}
  k & ṭ & t & p & & \\
  ṅ & ṇ & n & m & 並に & ḥ
\end{tabular}
\end{center}
三十四個の子音はこれらの九個の孰れかに還元せらる。卽ち語末
の音は硬無氣であらねばならぬ。隨て語末の音は必ずこれに變へ
られる。上顎音(ś を含む)及び h は k 又は ṭ に(ñ は ṅ
に),ṣ は ṭ に,s 及び r は ḥ に置き替へられる。y, l, v は決
して語末に來ない。

\subsubsection{同化の法則}
\numberParagraph \label{np:18}
硬音は軟音の次前には軟音となり,軟子音は硬子音の次前
には硬となる。
\eg{
  vanāt āgacchati = vanād āgacchati (彼は林より來
  れり)\\
  āpad kāle = āpat kāle (不幸の時に)。}

\numberParagraph
硬軟共に鼻音の前には鼻音となる。
\eg{
  vāc-mātreṇa = vāṅ-mātreṇa (言葉のみを以て)\\
  saṭ-māsān = saṇ-māsān (六ケ月の間)\\
  yāvat na = yāvan na (乃至......無し)。}

\numberParagraph
語の始の h は次前の子音の軟含氣音と變ず。
\eg{
  samyak-huta = samyag-ghuta (正しく供へられたる)\\
  tak kasmāt hetoḥ = tat kasmād dhetoḥ (其の故は
  如何)。}

\numberParagraph
齒音は次下の上顎音下音又は l と同化し,次に來る ś と
共に cch なとる\endnote{「なとる」はママ。}。
\eg{
  yāvat jīvam = yāvaj jīvam (生命の限り)\\
  tad jalam = taj jalam (その水は)\\
  vidyut-latā = vidyul-latā (電蘿)\\
  tad śrutvā = tac chrutvā (それを聞きて)。}

\subsubsection{s と r に關する法則}
\numberParagraph \label{np:22}
硬音の前の s.\\
硬齒音の前には變化せず。\\
次下の硬顎音舌音と同化して ś,時としては ṣ となる。\\
その他の硬音の前(並に文章の終末には)s は ḥ となる。
\eg{
  tatas tena bhaṇitam (かくて彼は云へり)\\
  tatas ca = tataś ca (而もかくて)\\
  tatas kṣipati = tataḥ kṣipati (彼は投ぐ)。}

\numberParagraph \label{np:23}
a, ā 以外の母音の次後の s は軟音の前に r となる。r の前に
は消失して先立つ短母音は延長せらる。
\eg{
  agnis idhyate = agnir idhyate (火は點ぜられたり)\\
  tayos ekahḥ = tayor ekaḥ (二人のうちの一人)\\
  kavis racayati kāvyam = kavī racayati kāvyam (詩
  人は一の詩を作れり)。}

\nt{
  bhos 間投詞は軟音の前にその s を去る。
  \eg{
    bhos mitra = bho mitra (おゝ友よ)。}}

\numberParagraph \label{np:24}
語の終の as は軟子音又は a の次前にある時は o となる。
\eg{
  devas jayatu = devo jayatu (王をして勝利あらしめよ)\\
  putras api = putro 'pi (子も亦)}
その餘の母音の次前に來る時は s 消失す。
\eg{
  siṃhas āha = siṃha āha (獅子は云へり)。}

\nt{
  代名詞の sas eṣas は總て sa eṣa となる。只 a の前には so
  eṣo となり。文章の結尾に saḥ eṣaḥ となるのみ。}

\numberParagraph
語末の r は語末の s の如く取り扱はれる。只軟音の前に變
化しない。
\eg{
  punaḥ punar upakuryāt (人は再三善をなすべし)}
r の前には消失して先立つ短母音は延長せられる(\ref{np:23}條)。

\subsubsection{鼻音の變化}
\numberParagraph
語末の n は軟の上顎音下音及び ś の前に在る時これと
同類の鼻音となる。
\eg{
  vṛttāntān jānāmi = vṛttāntāñ jānāmi (我は出來事
  を知る)}
又 s は變じて ch となることもある。
\eg{
  tān śaśāpa = tāñ śaśāpa 又は tāñ chaśāpa (彼は彼
  等を\ruby{詛}{のろ}へり)}
l の次前の n は ṃl 又は \anunasikam{}l となる。
\eg{
  asmin loke = asmiṃl loke (この世界に於いて)。}

\numberParagraph
語末の n と次下の硬の上顎音下音及び齒音との間にはこ
れらの音に相當する硬吹氣音を挿入し且つ n は隨韻となる。
\eg{
  kasmin cid vane = kasmiṃś cid vane (とある林に於
  いて)\\
  agaman tataḥ = agamaṃs tataḥ (かくて彼は往けり)。}

\numberParagraph \label{np:28}
語末の m は總ての子音の前に隨韻となる。母音の前には
變化しない。
\eg{
  uktam ca = uktaṃ ca (而してそれは云はれたり)\\
  kim karomi = kiṃ karomi (吾れは何を爲すべきか)\\
  sam-gacchati = saṃgacchati (彼は共に來れり)。}

\numberParagraph
m 以外の鼻音が語末にありて短母音に先立たれた時は重
複せられる。
\eg{
  tasmin adrau = tasminn adrau (その山の上に)\\
  smaran api = smarann api (彼は念じつゝも)。}
\nt{
  語の始の ch は短母音又は前置詞 ā 又は否定詞 mā の次後に來
  る時は cch となる。}

\subsection{一語中の連聲法}
格例法,活用法又は造語法に關し,後接字を加ふる場合に聲音
の變化を起こすことがある.大體上述の原則を適用すべきも若干の
注意すべきものがある。

\numberParagraph
一綴の名詞又は時としては動詞の語根及びその語幹 i,
ī, u, ū は母音にて始まる語尾の前には iy, uv と變化す。此等
が子音の次前に在るとき殊に然り。
\eg{
  bhū + i = bhuvi (地上に)\\
  su + e = suve (我は生む)\\
  śaknu + anti = śaknuvanti (彼らは能ふ)。}

\numberParagraph \label{np:31}
複母音 e, ai, o, au は次後の母音並に y の前に ay, āy,
av, āv となる。
\eg{
  nī = ne + ana = nay + ana (眼)\\
  bhū + a + ti = bho = bhav + a + ti (彼は有り)}

\numberParagraph
語根に屬せる r 又は v の次前なる i または u は其語根の
次後に子音來る時は槪ね延長せられる。
\eg{
  div + yati = dīvyati (彼は賭事を遊ぶ)\\
  pur + bhiḥ = pūrbhiḥ (都邑によりて)。}

\numberParagraph
語根又は語基の末尾にある子音は母音,半母音又は鼻音に
て始まる後接字の次前にある時は變化しない。其の他の後接字の
次前にある時は \ref{np:11} 以下の諸條を適用する。
\eg{
  marut + am = marutam (風を)\\
  vāc + ya = vācya (言はるべき)
  \item[されど] marut + bhyas = marudbhyaḥ}

\numberParagraph
t 又は th は後接字の聲の始にして軟含氣音の次後に在る
時は軟音となり且つその吹氣音 h を己に移す。
\eg{
  labh + ta = labdha (得られたる)\\
  rundh + thaḥ = runddhaḥ (汝等兩人は拒む)。}

\numberParagraph \label{np:35}
g, d, b で始まり gh, dh, bh, h で終る語根が \ref{np:17} 條によ
りてその含氣を失つた場合に始の音が含氣となる。これを代償の
法則と云ふ。例:duh (乳搾る)~dhuk, budh (賢なる)~bhut.

\numberParagraph
齒音は舌音の次後に來る時は通常は舌音となる。
\eg{
  iṣ + ta = iṣṭa (欲したる)\\
  dviṣ + dhi = dviḍḍhi (憎め)\\
  īḍ + te = īṭṭe (彼は\ruby{讚}{ほ}む)
  \item[されど] saṭsu (六に於て)。}

\numberParagraph
ś は t の次前に在る時は ṣ となる。
\eg{
  dṛś + ta = dṛṣṭa (見られたる)。}
この外の子音の次前にあるとき ś と ṣ は \ref{np:17} 條に準ず。

\numberParagraph
n は母音又は n, m, y, v にて從はれ、ṛ, ṝ, r, ṣ に先立たれ
た時には ṇ に變ず。但し母音喉音唇音 ṃ, y, v, h の中間に入り
來るを妨げない。
\eg{
  kar + ana = kāraṇa (因)\\
  brahman + ya = brāhmaṇya (信仰ある)\\
  pūṣan の屬單 pūṣṇāḥ \\
  grah + nāti = gṛhṇāti (彼は取る)\\
  Rāmāyaṇa (ラーマ物語)}

\nt{
  前接字 nis, parā, pari 又は pra にて先だたれた前接字 ni 並
  に大槪の語根の n は舌音化す。}

\numberParagraph \label{np:39}
s は k, r, l 又は a, ā 以外の母音によりて直に先立たれた
時 ṣ に變ず。但し s が語尾に在るか又は r が直に從ふ場合を
除く。隨韻止聲の中間に入り來るを妨げない。
\eg{
  dhanus + aḥ = dhnuṣaḥ (弓)屬單 \\
  dhanus = dhanūṃṣi (複主)\\
  vac + syati = (vak + syati =) vakṣyati (彼は語るべし)\\
  sarpis = sarpiṣā (淨酪を持って)具單
\item[されど] sarpiḥ (語の終)\\
  manasā (a に先立たる)\\
  tamisram (r 從ふ)。}

\newpage
\theendnotes

%%% Local Variables:
%%% mode: latex
%%% TeX-master: "IntroductionToSanskrit"
%%% End:
