\chapter*{序言}
\label{cha:preface}
かれこれ三十年以上も梵語を敎へてゐる經驗から梵語の文典に
ついて種々意見もあるが,大體に於て文典學習の方法に二通りの
型がある。卽ち言語の組織の理解と言語そのものの實習である。
これは必ずしも梵語に限つたことではないが梵語のやうに複雜な
體系を有つてゐる言語では尚ほ更のことであらうと思ふ。言語の
組織に就ては名詞變化なり動詞變化なり順序を追うて秩序整然と
排列されたものが必要である。然るに實習となると必ずしも順序
は追はなくても可い。あちらからもこちらからも随時に必要なも
のを取り出さねばならぬ。言ひ換へれば名詞變化が濟まねば動詞
變化に移られぬといふやうなものでない。名詞を少しやつたと思
ふうちに動詞が必要となつて來るのが實狀である。

ところが從來多くの文典は組織の排列に重きを置くか實習に重
きを置くかの孰れかを狙つてゐるので,組織に重きを置けば實際
の運用が鈍るし,實習に重きを置けば組織の纒めが疎かになり易
い。この兩端を折衷することができぬものかと私は常に考へ來
つた所であつた。この文典が存在理由を要求し得るとせばこの點
に多少の考慮を拂つたといふことである。勿論非常に限られた頁
數で十分のことは望み得られないが,然しこれによつて初學入門
の人が相當複雜なこの語の要領を把握してくれるには十分役立つ
だらうといふ自信だけはもつてゐる。

本書は次の二書が資料となつてゐる。
\begin{itemize}
\item Richard Fick: Praktische Grammatik der Sanskrit-
Sprache für den Selbstunterricht. Wien und Leipzig.
\item Macdonell: A Sanskrit Grammar for Students. London. 1927
\end{itemize}

又参考書として次のものがある。
\begin{itemize}
\item Whiteney. W. D.: Sanskrit Grammar,including both the
classical language, and the older dialects of Veda and Brahmana. Leipzig, 1896.
\item Monier Williams, M.: Practical Grammar of the San\-skrit language. Oxford, 1887.
\item Bhandarkar, R.G.: First Book of Sanskrit, Bombay, 1926. Second Book of Sanskrit, Bombay, 1928.
\item Stenzler, A.F.---Pischel, R.---Geldner, K.F.: Elementar\-buch der Sanskrit-Sprache, Giessen, 1923.
\item Bühler G.: Leitfaden für den Elementarkursus des Sanskrit, Wien, 1883.
\end{itemize}

尚ほ邦語で書かれたものでは次の如きものがある。
\begin{itemize}
\item 榊亮三郎氏 解説梵語學 明治四十年\\
\hfil 京都 眞言宗高等中學校。
\item 荻原雲來氏 實習梵語學 昭和七年(十二版)\\
\hfil 東京 丙午社。
\item 阿滿得壽氏 梵語文法綱要 大正十四年\\
\hfil 京都 伏見 壽竹堂。
\end{itemize}

%%% Local Variables:
%%% mode: latex
%%% TeX-master: "IntroductionToSanskrit"
%%% End:
