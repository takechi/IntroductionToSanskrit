\newpage

\texttitle{阿彌陀經}
\addcontentsline{toc}{chapter}{\protect\numberline{}阿彌陀經}%

舎利弗よ,如何にそれを汝は思ふや,何の理由ありてか彼の如
來は無量壽と名けらるゝや。舎利弗よ,彼の如來並に彼等人々
の壽量は量られず。この理由によりてかの如來は無量壽と名けら
る。舎利弗よ,而して彼の如來は無上なる正等覺を證得せし以來
十劫を經たり。

舎利弗よ,如何にそれを汝は思ふや,何の理由ありてか,彼の
如來は無量光と名けらるるや。舎利弗よ,彼の如來の光明は一切
佛國に於て無礙なり,この理由によりて彼の如來は無量光と名け
る。又舎利弗よ,かの如來の聲聞衆は無量なり,その清淨なる應
供の量を擧ぐることは容易にあらず。舎利弗よ,かくの如きの佛
國功德莊嚴を以てその佛國は飾られたり。

又舎利弗よ,無量壽如來の佛國に於て生るる有情は清淨なる菩
薩にして不退轉なり。一生所繫なり。舎利弗よ,その菩薩の量を
擧ぐることは無量無數なりとの數に行くを除きては容易にあら
ず。又舎利弗よ,その佛國に於て有情は發願すべし。その故は如
何。そこに實に是の如き正士と共に俱會はあるべし。舎利弗よ,
少善根のみによりてはその無量壽如來の佛國に於いて有情は生ぜ

\newpage

\texttitle{Sukhāvatīvyūhaḥ}
tat kiṃ manyase Śāriputra kena kāraṇena sa tathāgato
'mitāyur nāmocyate \da{} tasya khalu punaḥ Śāriputra taihāga\-%
tasya teṣāṃ ca manuṣyāṇām aparimitam āyuḥ-pramāṇam \da{}
tena kāraṇena sa tathāgato 'mitāyur nāmocyate \da{} tasya ca
Śāriputra tathāgatasya daśa kalpā anuttarāṃ samyaksaṃ\-%
bodhim abhisaṃbuddhasya \dd

tat kiṃ manyase Śāriputra kena kāraṇena sa tathāgato
'mitābho nāmocyate \da{} tasya khalu punaḥ Śāriputra tathāga\-%
tasyābhāpratihatā sarva-buddha-kṣetreṣu \da{} tena kāraṇena sa
tathāgato 'mitābho nāmocyate \da{} tasya ca Śāriputra tathāga\-%
tasyāprameyaḥ śrāvaka-saṃgho yeṣāṃ na sukaraṃ pramāṇam
ākhyātuṃ śuddhānām arhatām \da{} evaṃ-rūpaiḥ Śāriputra
buddha-kṣetra-guṇa-vyūhaiḥ samalaṃkṛtaṃ tad-buddha-
kṣetram \dd

punar aparaṃ Śāriputra ye 'mitāyuṣas tathāgatasya buddha-
kṣetre sattvā upapannāḥ śuddhā bodhisattvā avinivartanīyā
eka-jāti-pratibaddhās teṣāṃ Śāriputra bodhisattvānāṃ na
sukaraṃ pramāṇam ākhyātum anyatrāprameyāsaṃkhyeyā
iti saṃkhyāṃ gacchanti \da{} tatra khalu punaḥ Śāriputra
buddha-kṣetre sattvaiḥ praṇidhānaṃ kartavyam \da tat kasmād
dhetoḥ \da{} tatra hi nāma tathā-rūpaiḥ sat-puruṣaiḥ saha sama-

\newpage
\noindent
ず。舎利弗よ,如何なる善男子又は善女子にてもかの世尊無量壽
如來の名號を聞くべし,聞きて作意すべし。若くは一夜,若くは二
夜,若くは三夜,若くは四夜,若くは五夜,若くは六夜,若くは七夜
の間,不散亂の意にて作意すべし。彼の善男子又は善女子は死す
べし。彼の死する時,彼の無量壽如來は聲聞衆に圍繞せられ,菩薩
の衆に圍繞せられて前に至るべし。彼は不顚倒の心にて死すべ
し。彼は死してかの無量壽如來の佛國卽ち樂有世界に生るべし。
故に舎利弗よ,この義利を見つゝ我は是の如く說く。恭敬して善
男子又は善女子はその佛國に於いて發願すべし。\wosfnt{%
  日本の全土,津々浦々の涯までも,この阿彌陀經ほどよく行き渡つて讀
  まれてゐるお經はあるまい。實に日本佛敎僧侶の過半數はこれを佛前に諷
  誦する。否\ruby{啻}{ただ}に僧侶のみでなく,信徒にしてよくこれを讀むものも尠くは
  ない。蓋しこれわが日本國に於いての最も有緣の經典の一と謂はねばなら
  ぬ。

  有緣の經典であることに就いて,此に特記せねばならぬことは,この梵
  文が日本に於いて完全に保存せられてゐて,未だ世界何れの處からも發見
  せられないといふ事實である。勿論これは印度なり西域なりのある地方
  で,ある時代に筆寫せられ支那を經て日本へ傳來したのであらうが,今日
  では印度にも支那にも何れの處にも發見されずに,それが奇妙にも日本に
  だけ存在したといふことは實に不可思議の因緣あるを思はざるを得ない。
  恐らくこれは平安初期の留學僧の將來にかかるものであらうが慈雲の梵
  篋三本の記錄によれば梵文寫本が三部あつたと云ふ。旣に常明の梵漢阿彌
  陀經はこれより十年も前に刊行されてゐる。これらの資料を基礎として明
  治十三年マックスミューレル敎授南條笠原二氏の共同研究成り明治十六年の
  第一公刊となつたのである。恁うした因緣で日本が梵文を世界の舞臺へ送
  り出したことは日本國民として銘記せねばならぬ痛快事の一である。

  此に掲出したのは經の中心をなす彌陀の名義から執持名號,俱會一處を
  說く一段である。支那譯には羅什玄弉のものあれど,羅什譯最も行はれ
  る。比較研究すべきである。}

\newpage
vadhānaṃ bhavati nāvara-mātrakeṇa Śāriputra kuśala-mūle\-%
nāmitāyuṣas tathāgatasya buddhakṣetre sattvā upapadyante \da{}
yaḥ kaścic Chāriputra kulaputro vā kula-duhitā vā tasya bhaga\-%
vato 'mitāyuṣas tathāgatasya nāmadheyaṃ śroṣyati śrutvā
ca manasikariṣyati eka-rātraṃ vā dvi-rātraṃ vā tri-rātraṃ vā
catūrātraṃ vā pañca-rātraṃ vā ṣaḍ-rātrām vā sapta-rātraṃ
vāvikṣipta-citto manasikarṣyati yadā sa kulaputra vā kula\-%
duhitā vā kālaṃ kariṣyati tasya kālaṃ kurvataḥ so 'mitāyus
tathāgataḥ śrāvaka-saṃgha-parivṛto dobhisattva-gaṇa-puras\-%
kṛtaḥ purataḥ sthāsyati so 'viparyasta-cittaḥ kālaṃ kariṣyati
ca \da{} sa kālaṃ kṛtvā tasyaivāmitāyuṣas tathāgatasya buddha\-%
kṣetra Sukhāvatyāṃ lokadhātāv upapatsyate \da{} tasmāt tarhi
Śāriputra idam artha-vaśaṃ saṃpaśyamāna evaṃ vadāmi \da{}
satkṛtya kula-putreṇa vā kula-duhitrā vā tatra buddha-kṣetre
citta-praṇidhānaṃ kartavyam \dd


%%% Local Variables:
%%% mode: latex
%%% TeX-master: "IntroductionToSanskrit"
%%% End:
