\section{活用法(續き)}
\subsection{第二種變化}
\numberParagraph \label{np:136}
第二種變化の現在語基は强弱の二語基を有し,强語基は
次の如く用ひられる。
\begin{enumerate}[label=(\alph*)]
\item 現實法爲他單數の第一二三人稱
\item 命令法爲他の第一人稱全部並に第三人稱單數。
\end{enumerate}

\numberParagraph
第二種變化の現在組織の語尾は第一種變化のそれと稍異
る。その特徴次の如し。
\begin{enumerate}[label=(\alph*)]
\item \textbf{可能法の語尾。}

爲他 yām, yās, yāt; yāva, yātam, yātām; yāma,
yāta, yus.

爲自 īya, īthās, īta; īvahī, īyāthām, īyātām; īmahi,
īdhvam, īran.
\item \textbf{命令法の二人稱單數} 爲他に子音の後に dhi, 母音の後
に hi を加ふ。但し若干の例外がある。
\item 命令法第三人稱複數,爲他が重複級の動詞並に語根級の
あるものには an の代りに us である。又語根級の ā に
終る語根並に dvis は us 又は an の孰れにてもよい。
\item 三人稱複數爲自は ante, anta, antām の代りに ate,
ata, atām となる。重複級並に或る語根級の動詞は亦現實
法爲他 anti の代りに ati, 命令法,爲他 antu の代りに
atu の形を取る。
\end{enumerate}

\subsubsection{第二類 語根級}
\numberParagraph \label{np:138}
語根級の構造。語根のまゝに直接語尾を附加する。强形
に於て語根の母音は重韻化す。第一過去 三,複の us の前に語根
の終りの ā は消失する。例:語根 yā ~ ayus (彼等は行けり)。

\numberParagraph
語根 dviṣ (憎む)强基 dveṣ, 弱基 dviṣ.

\begin{center}
\begin{tabular}{c*{3}{p{0.23\hsize}}}
  \multicolumn{4}{c}{\textbf{現實法}} \\
  \multicolumn{4}{c}{爲他} \\
     & 單                                & 兩       & 複 \\
  1. & dveṣmi                            & dviṣvas  & dviṣmas \\
  2. & dvekṣi (\ref{np:17}, \ref{np:39}) & dviṣṭhas & dviṣṭha \\
  3. & dveṣṭi (\ref{np:36})              & dviṣṭas  & dviṣanti \\
  \multicolumn{4}{c}{爲自} \\
  1. & dviṣe  & dviṣvahe & dyiṣmahe \\
  2. & dvikṣe & dviṣāthe & dviḍḍhve (\ref{np:17}, \ref{np:36}) \\
  3. & dviṣṭe & dviṣāte  & dviṣate
\end{tabular}
\end{center}
\begin{center}
\begin{tabular}{c*{3}{p{0.23\hsize}}}
  \multicolumn{4}{c}{\textbf{第一過去}} \\
  \multicolumn{4}{c}{爲他} \\
     & 單                  & 兩       & 複 \\
  1. & adveṣam             & adviṣva  & adviṣma \\
  2. & adveṭ (\ref{np:17}) & adviṣṭam & adviṣṭa \\
  3. & adveṭ (\ref{np:17}) & adviṣṭām & adviṣan 又は adviṣus \\
  \multicolumn{4}{c}{爲自} \\
  1. & adviṣi    & adviṣvahi  & adviṣmahi \\
  2. & adviṣṭhās & adviṣāthām & adviḍḍhvam \\
  3. & adviṣṭa   & adviṣātām  & adviṣata
\end{tabular}
\end{center}
\begin{center}
\begin{tabular}{c*{3}{p{0.23\hsize}}}
  \multicolumn{4}{c}{\textbf{可能法}} \\
  \multicolumn{4}{c}{爲他} \\
     & 單      & 兩        & 複 \\
  1. & dviṣyām & dviṣyāva  & dviṣyāma \\
  2. & dviṣyās & dviṣyātam & dviṣyāta \\
  3. & dviṣyāt & dviṣyātām & dviṣyus \\
  \multicolumn{4}{c}{爲自} \\
  1. & dviṣīya   & dviṣīvahi   & dviṣīmahi \\
  2. & dviṣīthās & dviṣīyāthām & dviṣīdhvam \\
  3. & dviṣīta   & dviṣīyātām  & dviṣīran
\end{tabular}
\end{center}
\begin{center}
\begin{tabular}{c*{3}{p{0.23\hsize}}}
  \multicolumn{4}{c}{\textbf{命令法}} \\
  \multicolumn{4}{c}{爲他} \\
     & 單      & 兩      & 複 \\
  1. & dveṣāni & dveṣāva & dveṣāma \\
  2. & dviḍḍḥi & dviṣṭam & dviṣṭa \\
  3. & dveṣṭu  & dviṣṭām & dviṣantu \\
  \multicolumn{4}{c}{爲自} \\
  1. & dveṣai  & dveṣāvahai & dveṣāmahai \\
  2. & dvikṣva & dviṣāthām  & dviḍḍhvam \\
  3. & dviṣṭām & dviṣātām   & dviṣatām
\end{tabular}
\end{center}

\begin{center}\textbf{≪若干の不規則なる語根≫}\end{center}

\numberParagraph
語根 rud (泣く),svap (眠る),an (呼吸す),śvas (溜
息する),jakṣ (食ふ)に加へられる語尾が子音又は y 以外の半母
音で始まる場合,語基の尾を i に作り,二並に三の單,第一過,爲
他の時はそれを a 又は ī に作る。rodimi, rodiṣi, roditi; rudivas;
rudanti; 第一過 arodam, arodīs 又は arodas, arodīt 又は
arodat; arudiva; arudan; 可能法 rudyām 等。命令法 rodāni,
rudihi, roditu 等。

\numberParagraph
語根 as (あり)はその弱語基に於て a を失ふ。第一過去
は次の如くである。
\begin{center}
\begin{tabular}{c*{6}{p{0.15\hsize}}}
     & \multicolumn{3}{c}{\textbf{現實法}} & \multicolumn{3}{c}{\textbf{第一過去}} \\
     & 單    & 兩    & 複    & 單   & 兩    & 複 \\
  1. & asini & svas  & smas  & āsam & āsva  & āsma \\
  2. & asi   & sthas & stha  & āsīs & āstam & āsta \\
  3. & asti  & stas  & santi & asīt & āstām & āsan \\
     & \multicolumn{3}{c}{\textbf{命令法}} & \multicolumn{3}{c}{\textbf{可能法}} \\
  1. & asāni & asāva & asāma & syām & syāva  & syāma \\
  2. & edhi  & stam  & sta   & syās & syātam & syāta \\
  3. & astu  & stām  & santu & syāt & syātām & syus \\
\end{tabular}
\end{center}

\numberParagraph
語基 brū (言ふ)は强形の時子音に始まる語尾の前に
bravī に作る。現實法,現,爲他 bravīmi, bravīṣi, bravīti;
brūvas; bruvanti; 第一過去 abravam, avravīs, abravīt;
abrūva; abruvan; 可能法 brūyām 等。命令法 bravāni, brūhi,
bravītu 等。爲自は bruve, brūṣe, brūte; 三複 bruvate.

\numberParagraph
u にて終る語根は强形の時子音にて始まる語尾の前には
重韻にあらずして複重韻となる。stu (讚む)の現實,現,爲他は
staumi, stauṣi, stauti; 命令 stavāni, stuhi, stautu; 第一過
astavam, astaus, astaut; 三複 astuvan. 尙ほ stu は强語形
を stavī に作ることもあるからその時は三單現爲他は stavīti で
ある。

\numberParagraph
語根 han (殺す)の弱語形は m, y, v を除き子音に始ま
る語尾の前にその n を消失し,母音に始まる語尾の前には a を
除去して h は gh となる。現 hanmi, haṃsi, hanti; hanvas,
hathas, hatas; hanmas, hatha, ghnanti; 第一過去 ahanam,
ahan, ahan; ahanva, ahatām; ahanma, ahata,
aghnan. 二單命令爲他は jahi.

\subsubsection{第三類 重複級}
\numberParagraph \label{np:145}
第三類の現在組織は語根を重複して作る。强形に於て語
根の母音,第一過去の三複爲他の語尾 us の前にも終りの母音は
重韻化し,ā に終る語根はこの us の前にその ā を消失する。

\numberParagraph \label{np:146} \textbf{重複に關する原則。}
\begin{enumerate}[label=(\alph*)]
\item 重複とは語根の一部分通例は第一の子音とその次の母
音とが語根の前に置かれることである。tud (打つ)~
tutud, budh (知る)~ bubudh.
\item 含氣音はそれに相當する無氣音にて重複する。dhā (置
く)~ dadhā, bhuj (受用す)bubhuj.
\item 喉音は同種類の顎音にて,h は j にて重複せられる。
kṛ (作る)~cakṛ, gam (行く)~ jagam, hu (供ふ)~juhu,
khan (掘る)~ cakhan.
\item 連續せる子音が語根の始にある時はその第一のものを
重複す。śru (聞く)~śuśru, kram (歩む)~ cakram. 連續
せる子音の始が硬含氣音なる時は後の音又はそれに代るべ
き音で重複を行う。stu (讚む)~ tuṣtu, spṛś (觸る)~ pas\-%
parś, sthā (立つ)~ tiṣṭha.
\item 語根の母音は重複音中に現はるべきであるが若し長母
音ならば短母音が重複に用らる。dhā (置く)~ dadhā. 又
ṛ には i を用ふ。bhṛ (擔ふ)~ bibhṛ.
\end{enumerate}

\numberParagraph
語根 hu (供ふ),强形 juho, 弱形 juhu.
\begin{center}
\begin{tabular}{c*{3}{p{0.23\hsize}}}
  \multicolumn{4}{c}{\textbf{現實法}} \\
  \multicolumn{4}{c}{爲他} \\
     & 單     & 兩       & 複 \\
  1. & juhomi & juhuvas  & juhumas \\
  2. & juhoṣi & juhuthas & juhutha \\
  3. & juhoti & juhutas  & juhvati \\
  \multicolumn{4}{c}{爲自} \\
  1. & juhve  & juhuvahe & juhumahe \\
  2. & juhuṣe & juhvāthe & juhudhve \\
  3. & juhute & juhvāte  & juhvate
\end{tabular}
\end{center}
\begin{center}
\begin{tabular}{c*{3}{p{0.23\hsize}}}
  \multicolumn{4}{c}{\textbf{第一過去}} \\
  \multicolumn{4}{c}{爲他} \\
     & 單       & 兩       & 複 \\
  1. & ajuhavam & ajuhuva  & ajuhuma \\
  2. & ajuhos   & ajuhutam & ajuhuta \\
  3. & ajuhot   & ajuhutām & ajuhavus \\
  \multicolumn{4}{c}{爲自} \\
  1. & ajuhvi    & ajuhuvahi  & ajuhumahi \\
  2. & ajuhuthās & ajuhvāthām & ajuhudhvam \\
  3. & ajuhuta   & ajuhvātām  & ajuhvata
\end{tabular}
\end{center}
\begin{center}
\begin{tabular}{c*{3}{p{0.23\hsize}}}
  \multicolumn{4}{c}{\textbf{命令法}} \\
  \multicolumn{4}{c}{爲他} \\
     & 單       & 兩       & 複 \\
  1. & juhavāni & juhavāva & juhavāma \\
  2. & juhudhi  & juhutam  & juhuta \\
  3. & juhotu   & juhutām  & juhvatu \\
  \multicolumn{4}{c}{爲自} \\
  1. & juhavai & juhavāvahai & juhavāmahai \\
  2. & juhuṣva & juhvāthām   & juhudhvam \\
  3. & juhutām & juhvātām    & juhvatām
\end{tabular}
\end{center}
\begin{center}
\begin{tabular}{c*{3}{p{0.23\hsize}}}
  \multicolumn{4}{c}{\textbf{可能法}} \\
     & \multicolumn{3}{l}{爲他:juhuyām 等。 爲自:juhvīya 等。}
\end{tabular}
\end{center}

\begin{center}\textbf{≪重複級の若干の不規則なる語根≫}\end{center}

\numberParagraph
語根 dā (與ふ)並に dhā (置く)は序の如く弱語形を
dad 及び dadh に作る。而してこの dadh は t, th に始まる語
尾と共に dhatt, dhatth となる(\ref{np:35}條)。二單,命,爲他は dehi,
dhehi である。

\begin{center}
\begin{tabular}{c*{3}{p{0.23\hsize}}}
  \multicolumn{4}{c}{\textbf{dā の命令法,爲他}} \\
  1. & dadāni & dadāva & dadāma \\
  2. & dehi   & dattam & datta \\
  3. & dadātu & dattām & dadatu \\
  \multicolumn{4}{c}{\textbf{dhā の現實法,爲他}} \\
     & 單      & 兩       & 複 \\
  1. & dadhāmi & dadhvas  & dadhmas \\
  2. & dadhāsi & dhatthas & dhattha \\
  3. & dadhāti & dhattas  & dadhati
\end{tabular}
\end{center}

\numberParagraph
ā に終る或る語根は重複に母音 ī を以てす。弱語基は子
音語尾の前に ī となり母音語尾の前に消失す。mā (量る)三,單,
現,爲自 mimīte, 三,複 mimate. 又語根 hā 爲他(去る)の弱
語基は子音語尾の前に jahi 若くは jahī となる。母音語尾の前
若くは可能法の場合には jah となる。兩,現 jahivas 又は jahī\-%
vas, 二,單,命令 jahāhi, jahīhī 又は jahihī.

\subsubsection{第五類 nu 級}
\numberParagraph
第五類は語根に no を加へて强語基,nu を加へ弱語基を
作る。su (搾る)の强基 suno, 弱基 sunu. 母音に終る語根は v
又は m にて始まる語尾の前にその u を省くことがある。而して
二,單,命,他は語尾を附けない。sunuvas 又は sunvas; sunu\-%
mahe 又は sunmahe; 命 sunu. 子音に終る語根は u を省き得
ないし,又三複の語尾 anti の前にその u は uv に變る。而して
二,單,命,他は hi を附ける。āp (得る)は āpnuvas, āpnumas;
āpnuvanti; 命 āpnuhi.

\begin{center}
\begin{tabular}{cp{0.2\hsize}p{0.25\hsize}p{0.25\hsize}}
  \multicolumn{4}{c}{\textbf{現實法}} \\
  \multicolumn{4}{c}{爲他} \\
     & 單     & 兩               & 複 \\
  1. & sunomi & sunuvas (sunvas) & sunumas (sunmas) \\
  2. & suuoṣi & sunuthas         & suntha \\
  3. & sunoti & sunutas          & sunvanti \\
  \multicolumn{4}{c}{爲自} \\
  1. & sunve  & sunuvahe (sunvahe) & sunumahe (sunmahe) \\
  2. & sunuṣe & sunvāthe           & sunudhve \\
  3. & sunute & sunvāte            & sunvate
\end{tabular}
\end{center}
\begin{center}
\begin{tabular}{cp{0.18\hsize}p{0.26\hsize}p{0.26\hsize}}
  \multicolumn{4}{c}{\textbf{第一過去}} \\
  \multicolumn{4}{c}{爲他} \\
     & 單       & 兩                & 複 \\
  1. & asunavam & asunuva (asunva)  & asunuma (asunma) \\
  2. & asunos   & asunutam          & asunuta \\
  3. & asunot   & asunutām          & asunvan \\
  \multicolumn{4}{c}{爲自} \\
  1. & asunvi    & asunuvahi (asunvahi) & asunumahi (asunmahi) \\
  2. & asunuthās & asunvāthām           & asunudhvam \\
  3. & asunuta   & asunvātām            & asunvata
\end{tabular}
\end{center}
\begin{center}
\begin{tabular}{c*{3}{p{0.23\hsize}}}
  \multicolumn{4}{c}{\textbf{命令法}} \\
  \multicolumn{4}{c}{爲他} \\
     & 單       & 兩       & 複 \\
  1. & sunavāni & sunavāva & sunavāma \\
  2. & sunu     & sunutam  & sunuta \\
  3. & sunotu   & sunutām  & sunvantu \\
  \multicolumn{4}{c}{爲自} \\
  1. & sunavai & sunavāvahai & sunavāmahai \\
  2. & sunuṣva & sunvāthām   & sunudhvam \\
  3. & sunutām & sunvātām    & sunvatām
\end{tabular}
\end{center}
\begin{center}
\begin{tabular}{c*{3}{p{0.23\hsize}}}
  \multicolumn{4}{c}{\textbf{可能法}} \\
     & \multicolumn{3}{l}{爲他:sunuyām 等。 爲自:sunvīya 等。}
\end{tabular}
\end{center}

\numberParagraph
śru (聞く)の現在語基は śṛṇu, 隨つて śṛṇo. 現,爲他
śṛṇomi, śṛṇosi, śṛṇoti; śṛṇuvas 又は śṛṇvas, śṛṇuthas,
śṛṇutas; śṛṇumas 又は śṛṇmas, śṛṇutha, śṛṇvanti.

\subsubsection{第七類 鼻音級}
\numberParagraph
第七類鼻音級。强語基は na を挿入,弱語基はその終り
の子音に同類の鼻音を挿入して作る。bhid (破る),强語基
bhinad, 弱語基 bhind.

\begin{center}
\begin{tabular}{c*{3}{p{0.23\hsize}}}
  \multicolumn{4}{c}{\textbf{現實法}} \\
  \multicolumn{4}{c}{爲他} \\
     & 單       & 兩        & 複 \\
  1. & bhinadmi & bhindvas  & bhindmas \\
  2. & bhinatsi & bhintthas & bhinttha \\
  3. & bhinatti & bhinttas  & bhindanti \\
  \multicolumn{4}{c}{爲自} \\
  1. & bhinde  & bhindvahe & bhindmahe \\
  2. & bhintse & bhindāthe & bhinddhve \\
  3. & bhintte & bhindāte  & bhindate
\end{tabular}
\end{center}
\begin{center}
\begin{tabular}{c*{3}{p{0.23\hsize}}}
  \multicolumn{4}{c}{\textbf{第一過去}} \\
  \multicolumn{4}{c}{爲他} \\
     & 單        & 兩        & 複 \\
  1. & abhinadam & abhindva  & abhindma \\
  2. & abhinat   & abhinttam & abhintta \\
  3. & abhinat   & abhinttām & abhindan \\
  \multicolumn{4}{c}{爲自} \\
  1. & abhindi    & abhindvahi  & abhindmahi \\
  2. & abhintthās & abhindāthām & abhinddhvam \\
  3. & abhintta   & abhindātām  & abhindata
\end{tabular}
\end{center}
\begin{center}
\begin{tabular}{c*{3}{p{0.23\hsize}}}
  \multicolumn{4}{c}{\textbf{命令法}} \\
  \multicolumn{4}{c}{爲他} \\
     & 單        & 兩        & 複 \\
  1. & bhinadāni & bhinadāva & bhinadāma \\
  2. & bhinddhi  & bhinttam  & bhintta \\
  3. & bhinattu  & bhinttām  & bhindantu \\
  \multicolumn{4}{c}{爲自} \\
  1. & bhinadai & bhinadāvahai & bhinadāmahai \\
  2. & bhintsva & bhindāthām   & bhinddhvam \\
  3. & bhinttām & bhindātām    & bhindatām
\end{tabular}
\end{center}
\begin{center}
\begin{tabular}{c*{3}{p{0.23\hsize}}}
  \multicolumn{4}{c}{\textbf{可能法}} \\
     & \multicolumn{3}{l}{爲他:bhindyām 等。 爲自:bhindīya 等。}
\end{tabular}
\end{center}

\numberParagraph
是の如く yuj (結合す)~ yunajmi, yunakṣi, yunakti;
yuñjumas, yuṅktha, yuñjanti. piṣ (碎く)~ pinaṣmi, pinakṣi,
pinaṣṭi; piṃṣmas, piṃṣtha, piṃṣanti. rundh (止める)~
runadhmi, runatsi, runaddhi; rundhmas, runddha, rundhanti.

\subsubsection{第八類 u 級}
\numberParagraph
第八類 u 級。强語基は語根に o を加へ,弱語基は u を
加ふ。その u は m 又は v にて始まる語尾の前には省くことが
ある。その部類に屬する語根は kṛ (作る)の例外は別として皆 n
に終るから,結局は第五類の su と同樣となる。例せば tan (擴
ぐ)~强基 tano, 弱基 tanu. 尙ほその數も極めて少い。但し kṛ
は例外とは云へ,用ひらるゝこと極めて多いから次に表示する。
kṛ の强基は重韻の形を取り kar + o = karo であり,弱基は
kur となり u を加へて kuru に作る。この u はm, v を以て始
まる語尾の前には省かれねばならぬ。

\begin{center}
\begin{tabular}{c*{3}{p{0.23\hsize}}}
  \multicolumn{4}{c}{\textbf{現實法}} \\
  \multicolumn{4}{c}{爲他} \\
     & 單     & 兩       & 複 \\
  1. & karomi & kurvas   & kurmas \\
  2. & karoṣi & kuruthas & kurutha \\
  3. & karoti & kurutas  & kurvanti \\
  \multicolumn{4}{c}{爲自} \\
  1. & kurve  & kurvahe  & kurmahe \\
  2. & kuruṣe & kurvāthe & kurudhve \\
  3. & kurute & kurvāte  & kurvate
\end{tabular}
\end{center}
\begin{center}
\begin{tabular}{c*{3}{p{0.23\hsize}}}
  \multicolumn{4}{c}{\textbf{第一過去}} \\
  \multicolumn{4}{c}{爲他} \\
     & 單       & 兩       & 複 \\
  1. & akaravam & akurva   & akurma \\
  2. & akaros   & akurutam & akuruta \\
  3. & akarot   & akurutām & akurvan \\
  \multicolumn{4}{c}{爲自} \\
  1. & akurvi    & akurvahi   & akurmahi \\
  2. & akuruthās & akurvāthām & akurudhvam \\
  3. & akuruta   & akurvātām  & akurvata
\end{tabular}
\end{center}
\begin{center}
\begin{tabular}{c*{3}{p{0.23\hsize}}}
  \multicolumn{4}{c}{\textbf{命令法}} \\
  \multicolumn{4}{c}{爲他} \\
     & 單       & 兩       & 複 \\
  1. & karavāni & karavāva & karavāma \\
  2. & kuru     & kurutam  & kuruta \\
  3. & karotu   & kurutām  & kurvantu \\
  \multicolumn{4}{c}{爲自} \\
  1. & karavai & karavāvahai & karavāmahai \\
  2. & kuruṣva & kurvāthām   & kurudhvam \\
  3. & kurutām & kurvātām    & kurvatām
\end{tabular}
\end{center}
\begin{center}
\begin{tabular}{c*{3}{p{0.23\hsize}}}
  \multicolumn{4}{c}{\textbf{可能法}} \\
     & \multicolumn{3}{l}{爲他:kuryām 等。 爲自:kurvīya 等。}
\end{tabular}
\end{center}

\subsubsection{第九類 nā 級}
\numberParagraph
第九類 nā 級。强基は nā を加へ,弱基は子音にて始ま
る語尾の前には nī, 母音にて始まる語尾の前には n を加へて作
る。krī (買ふ)の强基 krīṇā, 弱基 krīṇī 又は krīn.

\begin{center}
\begin{tabular}{c*{3}{p{0.23\hsize}}}
  \multicolumn{4}{c}{\textbf{現實法}} \\
  \multicolumn{4}{c}{爲他} \\
     & 單      & 兩        & 複 \\
  1. & krīṇāmi & krīṇīvas  & krīṇīmas \\
  2. & krīṇāsi & krīṇīthas & krīṇītha \\
  3. & krīṇāti & krīṇītas  & krīṇanti \\
  \multicolumn{4}{c}{爲自} \\
  1. & krīne   & krīṇīvahe & krīṇīmahe \\
  2. & krīṇīṣe & krīṇāthe  & krīṇīdhve \\
  3. & krīṇīte & krīṇāte   & krīṇate
\end{tabular}
\end{center}
\begin{center}
\begin{tabular}{c*{3}{p{0.23\hsize}}}
  \multicolumn{4}{c}{\textbf{第一過去}\endnote{底本では爲自二人称複数が ``akrīṇīdhvam'' ではなく ``akriṇīdhvam''.}} \\
  \multicolumn{4}{c}{爲他} \\
     & 單      & 兩        & 複 \\
  1. & akrīṇām & akīṇīva   & akrīṇīma \\
  2. & akrīnās & akrīṇītam & akrīṇīta \\
  3. & akrīṇāt & akrīṇītām & akrīṇan \\
  \multicolumn{4}{c}{爲自} \\
  1. & akrīṇi     & akrīṇīvahi & akrīṇīmahi \\
  2. & akrīṇīthās & akrīṇāthām & akrīṇīdhvam \\
  3. & akrīṇīta   & akrīṇātām  & akrīṇata
\end{tabular}
\end{center}
\begin{center}
\begin{tabular}{c*{3}{p{0.23\hsize}}}
  \multicolumn{4}{c}{\textbf{命令法}} \\
  \multicolumn{4}{c}{爲他} \\
     & 單       & 兩       & 複 \\
  1. & krīṇāni  & krīṇāva  & krīṇāma \\
  2. & krīṇīhi  & krīṇītam & krīṇīta \\
  3. & krīṇātu  & krīṇītām & krīṇantu \\
  \multicolumn{4}{c}{爲自} \\
  1. & krīṇai   & krīṇāvahai & krīṇāmahai \\
  2. & krīṇīṣva & krīṇāthām  & krīṇīdhvam \\
  3. & krīṇītām & krīṇātām   & krīṇatām
\end{tabular}
\end{center}
\begin{center}
\begin{tabular}{c*{3}{p{0.23\hsize}}}
  \multicolumn{4}{c}{\textbf{可能法}} \\
     & \multicolumn{3}{l}{爲他:krīṇīyām 等。 爲自:krīnīya 等。}
\end{tabular}
\end{center}

\numberParagraph
子音に終る語根は,二,單,命,爲他の語尾を āna とす
る aś (食ふ),aśāna.

\numberParagraph
この級の或る語根はその母音が弱められる。
\begin{enumerate}[label=(\alph*)]
\item ū に終る語根は母音を短縮する。dhū (振ふ)~ dhunā\-%
ti, pū (淨む)~ punāti.
\item grah (執る)は弱められて gṛh となる。
\item jñā (知る)は鼻音を失ひ jānāti に作る。bandh (縛
る),manth (攪拌す)も同樣である。
\end{enumerate}

\ex{第十}
\begin{longtable}{c*{2}{p{0.45\hsize}}}
 1. & vidyā vinayaṃ dadāti. & 知識は禮譲を賦與する。 \\
 2. & satyaṃ brūhi. & 眞實を語れ。\\
 3. & na vedmi kiṃ mayābhihitam. & 我は我によりて何が云はれしかを知らない。\\
 4. & daridrasya dānaṃ dehi. & 貧人に施物を與へよ。\\
 5. & rajako gardabhaṃ bandhane\-na niyunakti. & 洗濯人は綱を以て驢馬を繫ぐ。\\
 6. & tṛṣṇāṃ chinddhi. & 愛欲を斷て。\\
 7. & śūro raṇe mṛtaḥ svargaṃ prāpṇoti. & 戰いに於て死する勇士は天界に至る。\\
 8. & kathaṃ vetti bhavān me duḥkham. bhadra, vivarṇatā te
 vivṛṇoti śoka-vegam. & 御身は如何にして我が苦痛を知らんや。善きもの
 よ,汝の顔面蒼白は悲痛の劇甚を露はす。\\
 9. & bhadram astu te, Śivaḥ pātu tvām. & 汝の上に幸福あれ。シヷが汝を護れかし。\\
10. & yāty adho vrajaty uccair naraḥ svair eva karmabhiḥ. & 人は自の業によりてのみ下
 に行き上に行く。\\
11. & guṇavaj-jana-saṃsargād yāti svalpo 'pi gauravam. & 德ある人々と交ることによ
 り人は劣等なりとも尊敬せらる。\\
12. & āsīd rājā Nalo nāma Vīrasena\-suto balī. & ヴィーラセーナの子なる力
 强きナラと名けらるゝ王ありき。\\
13. & nīco vadati nakurute, vadati na sādhuḥ karoty eva. & 小人は語りて作さず,善人
 は語らずして作す。\\
14. & niṣiddhas tvaṃ mayānekaśo naśṛṇoṣi me vākyam. & 汝は屢々我によりて警吿さ
 れたるに我が語を聽かず。\\
15. & kiṃ rodiṣi gadgada-vācā. & 汝はとぎれとぎれの語を以て何を泣くや。\\
16. & rājā pīḍitānām anāthānāṃ ca kuryād aśru-pramārjanam. & 王は苦しめられたるもの,
 保護なきものの淚を拭ふべきである。\\
17. & asty uttarasyāṃ diśi devātmā Himālayo nāma nagādhirājaḥ. & 北方に於てヒマーラヤと名
 けられたる神々しき山王ありき。\\
18. & ratna-mālā kutra labdhā yā dīptā sūryam api tiraskaroti.& 何處で太陽をも凌駕するそ
 の輝いた眞珠鬘は得られたか。\\
19. & udeti savitā tamaś cāstameti. & 太陽は昇り而して暗黑は消えたり。\\
20.& padma-pattra-sthitaṃ vāri dhatte muktā-phala-śriyam. & 蓮葉にある水は眞珠の美しさを呈す。\\
21. & yaḥ prasannena manasā bhā\-ṣate karoti vā taṃ sukham an\-veti cchāyeva. & 淨き心もて語り若くは行ふ
 彼に幸福は影の如く隨ひ行く。\\
22.& yaḥ pradoṣeṇa manasā bhā\-ṣate karoti vā taṃ duḥkham anveti cakraṃ yathā vahanam. & 惡しき心もて語り若くは行
 ふ彼に苦は隨ひ行く,恰も車輪が車に隨ふ如し。\\
23. & anāgataṃ yaḥ kurute sa śo\-bhate; sa śocate yo na karoty anāgatam. & 未來を爲す彼は輝く。未來
をなさぬ彼は悲しむ。\\
    & \multicolumn{2}{l}{(śocate と śobhate とは相似たる音なるを注意すべし)。}
\end{longtable}


%%% Local Variables:
%%% mode: latex
%%% TeX-master: "IntroductionToSanskrit"
%%% End:
