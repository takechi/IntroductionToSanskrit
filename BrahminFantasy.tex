\newpage
\texttitle{婆羅門の{空}想}
\addcontentsline{toc}{chapter}{\protect\numberline{}婆羅門の空想}%

デーヴィーコータ城にデーヷシャルマンと名くる婆羅門ありき。
彼れ祭祀の時麥の煎粉を容れたる一皿を得たり。かくてこれを携
へて彼は陶器を滿たせる陶工の小屋の一隅に於て眠りつゝ考へ
たり。もし我れこの皿の煎粉を賣りて十カパルダカを得ばその時
こゝにてこれを以て壺皿を買ひ,種々の方法にて增殖せる金錢に
て又\ruby{檳榔子}{びん|ろう|じ}衣服等を買ひ,賣りて十萬を以て數ふる富を生じて四
度の結婚をなすべし。かくて彼等四人の妻の中にて若く美貌ある
ものに他に勝る寵愛をなすべし。その間に嫉妬を生じたるその婦
女達が爭をなさばその時は予は怒りて彼等妻女をかくの如く杖に
て打つべし。かく云ひて彼は杖を投げたり。その時かの麥の煎粉
の皿は碎かれ多くの陶器は壞されたり。かくて陶器の壞れたる音
によりて來れる陶工は見て彼の婆羅門は叱責せられ小屋の外に出
されたり。\wosfnt{%
  これは印度訓話ヒトーバデーシャ中の物語である。ヒトーバデーシャは印
  度訓話の最も著名なるもので,殊にこの婆羅門の物語の如きは世界各地へ
  姿を變じて傳はつて行つた。エソポスやラフオンテーンに乳搾り娘の{空}想
  の物語として有名である。印度は物語の無盡の藏とも云ふべきで,パン
  チャヤタントラと云ひカトハーサリットサーガラと云ひ,さては佛敎のジャ
  ータカ本生說話に至るまで擧げ來ればその資料の豊富なること實に驚嘆そ
  のものである。ヒトーバデーシャは四篇から成る短い作品だが,その物語は
  精選せられ,引用の詩に見るべきもの多く,散文は實に簡明にして要を
  得,何れの點よりしても模範的作品と謂うべきである。}

\newpage

\texttitle{Tyaktāśaḥ}

asti Devīkoṭa-nagare Devaśarmā nāma brāhmaṇaḥ \da{} tena
viṣuvat-samaye saktu-bhṛta-śarāva ekaḥ prāptaḥ \da{} tatas tam
ādāyāsau bhāṇḍa-pūrṇa-kumbhakāra-maṇḍapikaika-deśe sup\-%
taḥ sann acintayat \da{} yady aham imaṃ saktu-śarāvaṃ vikrīya
daśa kapardakān prāpnomi tadā tair iha samaye ghaṭa\-%
śarāvān upakrīya vikrīyānekadhā vṛddhais tair dhanaiḥ
punaḥ pūga-vastrādikam upakrīya vikrīya lakṣasaṃkhyātāni
dhanāny utpādya vivāha-catuṣṭayaṃ karomi \da{} tatas tāsu
patnīṣu yā rūpa-yauvanavatī tasyām adhikānurāgaṃ karomi \da{}
anantaraṃ saṃjāterṣyās tat-sapatnyo yadā dvaṃdvaṃ kur\-%
vanti tadā kopākulo 'haṃ tāḥ sapatnīr itthaṃ laguḍena
tāḍayiṣyāmi \da{} ity abhidhāya tena laguḍaḥ kṣiptaḥ \da{} atha sa
saktu-śarāvaś ca cūrṇito bhāṇḍāni ca bahūni bhagnāni \da{} tato
bhagna-bhāṇḍa-śabdenāgata-kumbhakāreṇa dṛṣṭvā sa brāh\-%
maṇas tiraskṛto maṇḍapikā-garbhād bahiḥ-kṛtaḥ \dd{}

\rightline{(Hitopadeśa VI)}


%%% Local Variables:
%%% mode: latex
%%% TeX-master: "IntroductionToSanskrit"
%%% End:
