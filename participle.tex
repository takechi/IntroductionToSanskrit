\section{分詞}
\subsection{現在並に未來分詞}
\numberParagraph
現在分詞爲他は現在の語基に at (强基 ant, 弱基 at)を
附加して作る。强基は三人稱,複數,現在,爲自の形から i を除き
去りたるものを以てする。隨て第三類動詞並に第二類動詞の或る
もの(\ref{np:138}條)は現在分詞の强基に鼻音を有しない。然し未來分詞
の强基は常に ant である。

語根 bhū (あり)三複現爲他 bhavanti, 現分 bhavat, 强基
bhavant. 語根 hu (供ふ)~三,複,現,爲他 juhvati, 現分
juhvat.

\numberParagraph
未來分詞爲他は未來語基(\ref{np:186}條)と同樣の方法で作ら
れる。語根 bhū (有り)三,複,未來,爲他 bhaviṣyanti ~未來
分詞 bhaviṣyat. これらの分詞の語尾變化並に女性語基の構成に
關しては \ref{np:89}, \ref{np:90}條を見よ。

\numberParagraph \label{np:174}
現在分詞爲自は現在語基に māna を附加して作る。第二
種變化の動詞は māna の代りに āna を加ふ。これらはやはり
三,複,現,爲自形から ante 又は ate を除去したるものに加へ
られる。

語根 div (博戯す)三,複,現,爲自 dīvyante ~現在分詞 dīvya\-%
māna. su (生ず)三,複,現,爲自 sunvate ~現分詞 sunvāna.

\numberParagraph
未來分詞爲自は\endnote{底本では「未來分詞爲自は」ではなく「末來分詞爲自は」。}同樣に未來語基に māna を附加して作
る。語根 dā (與ふ)三複未來爲自 dāsyante ~未來分詞爲自
dāsyamāna.

\numberParagraph \label{np:176}
現在並に未來の受動形に māna を加へて現在並に未來受
動分詞が作られる。語根 tud (打つ),受動 tudya, 現在受動分
詞 tudyamāna.

\subsection{過去能動分詞}
\numberParagraph \label{np:177}
第二過去爲自の分詞は第二過去(\ref{np:192}條)の弱基に vas
(强基には vāṃs, 中基に vat, 弱基に us)を附加して作る。こ
の vas は語基に直接又は i を介して加へられる。語根 gam (行
く)第二過去の弱基 jagm. これに直接又は i を介して vas を
附加す。jagmivas 或は jaganvas. han (殺す),dṛś (見る)も
gam の如く兩形を有す。

\begin{center}
\begin{tabular}{clclclcl}
  語根 & nī (導く)  & 第二過去 & 弱基 & ninī, & 分詞 & ninīvas \\
  〃   & as (投ぐ)  & 〃       & 〃   & ās    & 〃   & āsivas \\
  〃   & pac (煮る) & 〃       & 〃   & pec   & 〃   & pecivas \\
  〃   & yaj (祀る) & 〃       & 〃   & īj    & 〃   & ījivas
\end{tabular}
\end{center}

語尾變化は(\ref{np:94}條)を見よ。

\subsection{過去受動分詞}
\numberParagraph
過去受動分詞は語根に ta 或は na を附加して作る。語
根に强弱語基を分つ場合は最も弱き形を取る。na は語根に直
接,ta は直接又は i を介して附加せられる。他動詞に加へられ
た時は過去受動の義を表はすも自動詞に加へられた時は單に不定
の過去の義となる。lag (附着す)~ lagna, kṛ (作る)~ kṛta, vid
(知る)~ vidita.

\numberParagraph \label{np:179}
次の條々は過去受動分詞の構成に關し重要なるものであ
る。
\begin{enumerate}[label=(\alph*)]
\item na は多くは ṝ 並に\endnote{底本では「並に」ではなく「並にに」。} d に終る語根に附加せられる。tṝ
(超える)~ tīrṇa, pṝ (滿たす)~ pūrṇa, jṝ (老いる)~
jīrṇa, sad (坐す)~ sanna, bhid (破る)~ bhinna. 尙ほ
hā (捨つ)~ hīna, śvi (膨れる)~ śūna, lū (絕つ)~ lūna,
bhañj (破る)~ bhagna.
\item 或る語根は na, ta の兩方を取る。trai (護る)~ trāta
又は trāṇa, tvar (急ぐ)~ tūrṇa 又は tvarita.
\item aya 級の語基並びに催起動詞の語基はその aya を去り
i を介して ta を附加す。cur (盗む)~ corita, budh (覺
む)の催起 bodhayati ~ bodhita.
\item 子音に終る語根には i を介することなくして直に ta を
附加す。隨つて連聲の法則が適用されねばならぬ。語根
tyaj (捨つ)~ tyakta, labh (得る)~ labdha, iṣ (欲する)
~ iṣṭa, dah (燒く)~ dagha, lih (䑛める)~ līḍha, guh (匿
す)~ gūḍha, muh (失神する)~ mūḍha 又は mugdha, sah
(堪ふ)~ soḍha 等。
\item ā に終る語根は或るものは ī に或るものは i に變ず。
pā (飲む)~ pīta, sthā (立つ)~ sthita, dhā (置く)~ hita.
\item 語根中の鼻音は通常消失する。鼻音に終る語根では通
例母音を延長し若くは鼻音が消失する。bandh (縛る)~
baddha, daṃś (咬む)~ daṣṭa, kram (歩む)~ krānta, śam
(靜める)~ śānta, gam (行く)~ gata, han (殺す)~ hata,
man (考ふ)~ mata.

又語根の母音の延長と同時に鼻音を消失する。khan (掘
る)~ khāta, jan (生る)~ jāta.
\item 受動詞構成の場合 u, i に弱められた va, ya を有する
語根(\ref{np:159}條 \ref{item:159b})は此にも同樣の變化をなす。vac (言ふ)
~ ukta, vah (運ぶ)~ ūḍha, yaj (供ふ)~ iṣṭa, vyadh (貫
く)~ viddha.
\item 其の他注意すべきもの:語根 prach (問ふ)~ pṛṣṭa,
grah (取る)~ gṛhīta, dā (與ふ)~ datta.
\end{enumerate}

\numberParagraph
過去受動分詞に vat (强基 vant)を附加して能動過去分
詞が作られる。kṛ (爲す)~ kṛtavat (vān) かれは爲せり。これ
は cakṛvān (\ref{np:177}條)又は cakāra (\ref{np:194}條)と同意義であり。語
尾變化は \ref{np:94}條に出づ。

\numberParagraph
過去分詞爲自は第二過去(\ref{np:192}條)の弱語基に āna を
附加して作られる。語根 bhid (破る)~ bibhidāna, nī (導く)~
ninyāna, stu (讚む)~ tuṣṭuvāna, dā (與ふ)~ dadāna, yaj (供
ふ)~ ijāna (\ref{np:174}--\ref{np:176}條參照)。

\ex{第十二}
\begin{longtable}{c*{2}{p{0.45\hsize}}}
 1. & kiṃ na lajjasa evaṃ bru\-vāṇaḥ. & 如何に汝はかくの如く語りつゝ羞ぢざるか。\\
 2. & siṃhasya vane (\ref{np:231}條) bhra\-mato ravir astaṃgataḥ & 獅子が林を彷徨しつゝある
間に太陽は沒せり。\\
 3. & ajānan dāhārtiṃ patati śala\-bho dīpa-dahanam. & 燃ゆる苦痛を知らずして蛾
は燈火に落ちたり。\\
 4. & apriyāny api kurvāno yaḥ priyaḥ priya eva saḥ. & 非愛のことを爲すも愛する
所の彼は愛人にこそあれ。\\
 5. & muniḥ krodhyamāno 'pi pri\-yaṃ brūyāt. & 聖者は怒らされても愛語を
語るべし。\\
 6. & brāhmaṇasya bhāryā pratidi\-naṃ kuṭumbena saha kalahaṃ kurvāṇā na kṣaṇam api vyaś\-rāmyat.
 & 婆羅門の妻は毎日家族と喧嘩をなしつゝ一瞬と雖も中止せざりき。\\
 7. & narau vivadamānau dhar\-mādhikāriṇaṃ gatavantau prā\-vaktāṃ parasparaṃ dūṣyantau.
 & 二人の男は爭ひつゝ裁判官の許へ來れり。彼等二人は互に耻ぢしめつゝ宣べ立てた。\\
 8. & cauraś cititavān; aho, keno\-pāyenaiṣāṃ dhanaṃ labhe. & 盗人は考えた。あゝ如何な
る方法で彼等の財貨を我は得べき。\\
 9. & bījāṅkuraḥ sūkṣmo 'pi pari\-puṣṭo 'bhirakṣitaś ca kāle pha\-lāni dadāti. & 芽は小くとも育成せられて
守られてあらば時に於て果を生ず。\\
10. & haṃsāv astaṃgate ravau sva\-nīḍa0saṃśrayam akurutām. & 二羽の白鳥は太陽の沒せし
時自分の巢へ避難をなせり。\\
11. & mṛte patyau strī pradāhayed ātmānam. & 夫死したる時婦人は自身を焚燒すべきである。\\
12. & sa daridro yasya tṛṣṇā viśālā manasi parituṣṭe ko 'rthavān ko daridraḥ.
& 貧乏人とはその慾が大きいものゝことである。心に滿足せる時に誰が富人で誰が貧乏人であらうか。\\
13. & parāṅmukhe 'pi daive kṛtyaṃ kuryān medhāvī. & 運命我に背くとも賢者は義務を爲すべきである。\\
14. & brāhmaṇo māghamāse saum\-yānile pravāti meghācchādite gagane mandam mandaṃ var\-%
ṣati parjanye paśu-prārthanā\-rthaṃ grāmāntaraṃ gataḥ. & マーグハ月に於て北風が吹
き,空は雲に覆はれ徐々に雨神が雨を降らす時一人の婆羅門は犠牲の家畜を乞ふために他の村へ行けり。\\
15. & naṣṭaṃ mṛtam atikrāntaṃ nānuśocanti paṇḍitāḥ paṇḍitānāṃ ca mūrkhāṇāṃ viśeṣo 'yaṃ yataḥ smṛtaḥ.
& 失はれたるもの,死せるもの,過ぎしものを賢者は悲しまず賢者と愚者との差はこれなりと云はるるが故に。
\end{longtable}

\subsection{義務分詞}
\numberParagraph
この分詞は未來受動の意義を有し,梵文には屢々用ひ
られるが,三種の構成法によつて作られる,卽ち tavya, anīya 又
は ya を附加するものである。

語根 kṛ (作る)~ kartavya, kartaṇīya, kārya 作らるべき。
\begin{enumerate}[label=(\alph*)]
\item tavya は語根に直接又は i を介して附加せられる。語
根の母音は重韻となる。dā (與ふ)~ dātavya, ji (勝つ)
~ jetavya, bhū (有る)~ bhavitavya, muc (解く)~
moktavya, cur (盗む)~ coritavya.
\item anīya の前には語根の母音は通例同樣に重韻となる。
nī (導く)~ nayanīya, śru (聞く)~śravaṇīya, bhid (破
る)~ bhedanīya, sṛj (投ぐ)~ sarjānīya.
\item ya を附加する時に語根の終の ā は e に變じ,其他の
母音は或る時は變化せず,或る時は重韻又は複重韻となる。
その時重韻の o は直ちに ya の前にあるとき av 又は āv
となる。dā (與ふ)~ deya, ji (勝つ)~ jeya, nī (導く)~
neya, bhū (有る)~ bhāvya 又は bhavya, budh (覺む)~
bodhya, vac (語る)~ vācya, labh (得る)~ labhya.
\end{enumerate}

%%% Local Variables:
%%% mode: latex
%%% TeX-master: "IntroductionToSanskrit"
%%% End:
