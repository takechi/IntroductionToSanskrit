\section{受動調並に派生動詞}
\subsection{受動調動詞}
\numberParagraph
現在動詞(\ref{np:60}條以下)の受動は語根に ya を附加して作
る。人稱語尾は爲自を用ふ。語根 kṛ (作る)~ krīya.

\begin{center}
\begin{tabular}{c*{3}{p{0.23\hsize}}}
  \multicolumn{4}{c}{\textbf{現實法}} \\
     & 單      & 兩        & 複 \\
  1. & kriye   & kriyāvahe & kriyāmahe \\
  2. & kriyase & kriyethe  & kriyadhve \\
  3. & kriyate & kriyete   & kriyante \\
  \multicolumn{4}{c}{\textbf{第一過去}} \\
  1. & akriye     & akriyāvahi & akriyāmahi \\
  2. & akriyathās & akriyethām & akriyadhvam \\
  3. & akriyata   & akriyetām  & akriyanta
\end{tabular}
\end{center}
\begin{center}
\begin{tabular}{c*{3}{p{0.23\hsize}}}
  \multicolumn{4}{c}{\textbf{命令法}} \\
     & 單       & 兩         & 複 \\
  1. & kriyai   & kriyāvahai & kriyāmahai \\
  2. & kriyasva & kriyethām  & kriyadhvam \\
  3. & kriyatām & kriyetām   & kriyantām \\
  \multicolumn{4}{c}{\textbf{可能法}} \\
  1. & kriyeya   & kriyevahi   & kriyemahi \\
  2. & kriyethās & kriyeyāthām & kriyedhvam \\
  3. & kriyeta   & kriyeyātām  & kriyeran
\end{tabular}
\end{center}

\numberParagraph
受動は弱語基から作る。尙ほ次の變化が行はれる。
\begin{enumerate}[label=(\alph*)]
\item 子音に先立つ語根中の鼻音は通例消滅する。bhañj (破
る)~ bhajyate, daṃś (嚙む)~ daśyate, bandh (縛る)~
badhyate. されど nind (責む)は nindyate.
\item 語根の始にある va 又は ya は序の如く u 又は i と
なる。vac (語る)~ ucyate, yaj (祀る)~ ijyate, vyadh
(貫く)~ vidhyate.
\item grah (執る)~ gṛhyate, prach (問ふ)~ pṛcchyate.
\end{enumerate}

\numberParagraph
語根の母音は受動詞 ya の前に次の變化をなす。
\begin{enumerate}[label=(\alph*)]
\item i 若くは u は延長せられる。ci (集む)~ cīyate, stu (讚
む)~ stūyate.
\item ā は通常 ī となる。dā (與ふ)~ dīyate, hā (捨つ)~
hīyate, mā (量る)~ mīyate, dhā (置く)~ dhīyate. 但
し jñā (知る)は jñāyate.
\item 子音に隨ふ ṛ は ri 又は ar となる。kṛ (作る)~
kriyate, smṛ (念ず)~ smaryate.
\item ṝ は īr となる。唇音の後に來るものは ūr となる。kṝ
(散布す)~ kīryate, pṝ (滿す)~ pūryate.
\item gai (歌ふ)~ gīyate, dhyai (沈思する)~ dhyāyate, hve
(喚ぶ)~ hūyate.
\end{enumerate}

\numberParagraph
受動調は以上の如き變化によりて能動調と區別せら
れる。然し第四類の或る動詞の現在爲自の形とは只揚音によりての
み區別せらる。nahyate (彼は縛す),nahyáte (彼は縛らる)。

\subsection{催起動詞}
\numberParagraph
催起動詞の語基は aya 級の動詞(\ref{np:73}條)のやうに作られ
變化する。語根 vid (知る)~ vedayati (彼は吿ぐ),bhū (ある)~
bhāvayati (彼は現はす),pat (落つ)~ pātayati (彼は落とす)。

\numberParagraph
ā にて終る語根は大抵 aya の前に p を挿入す。dā (與
ふ)~ dāpaya, sthā (立つ)~ sthāpaya, jñā (知る)には jñāpaya,
jñapaya の兩形がある。

\numberParagraph
語根の中間にある a は屢々これを延長せず,そのまゝに
することがある。gam (行く)~ gamaya, jan (生る)~ janaya.

\numberParagraph
若干の不規則なる語根:ṛ (行く)~ arpayati, i (行く)+
adhi ~ adhyāpayati, sidh (成就す)~ sādhayati, ruh (生長す)~
ropayati.

\numberParagraph
催起動詞の受動調を作るには語基から aya を除去して而
る後受動の ya を加へる。kṛ (作る)催起 kāraya $-$ aya = kār +
ya + te = kāryate.

\subsection{重複動詞}
\numberParagraph
動作の重複隨て强勢を現はす動詞の一種で,これは語根
の重複(\ref{np:146}條)によりて語基を作るのだが重複の綴は母音が强
められ又は延長せられる。

a は ā となり i, ī は e となり u, ū は o となる。語尾変化
に關しては第二種變化に準ず。語根 vid (知る)~ vevid, 三單現
爲他 vevetti

\numberParagraph
通例重複動詞の語基には ya を附ける。而して爲自の變
化をなす。bhid (破る)~ bebhidyate; dhū (搖る)~ dodhūyate.

\subsection{求欲動詞}
\numberParagraph
求欲動詞は重複せる語根に直接又は i を介して sa を附
加して語基を作る。かくて作られたる語基は第六類に準じて變化
する。尙 i を介するものはその語根の母音が通例重韻となる。重
複の母音は i であるが若し語根に u があれば u を用ふ。pac
(煮る)~ pipakṣa (煮んと欲す),kṣip (投ぐ)~ cikṣipsa (投げん
と欲す),tud (打つ)~ tututsa (打たんと欲す),vid (知る)~
vividiṣa 又は vivitsa (知らんと欲す),duh (乳を搾る)\endnote{底本では「乳」は「{\HanazonoA 乳󠄃}」。}~
dudhukṣa (乳を搾らんと欲す)。

\numberParagraph
聲の終の i, u は延長せられる。ṛ と ṝ は īr となり唇音
の次後には ūr となる。ji (勝つ)~ jigīṣa, śru (聞く)~ śuśrūṣa,
kṛ (作す)~ cikīrṣa, mṛ (死す)~ mumūrṣa.

\numberParagraph
或る語根は重複語根が收縮される。āp (得る)~ īpsati
(彼は得んと欲す),dā (與ふ)~ ditsati (彼は與へんと欲す),
labh (得る)~ lipsati (彼は得んと欲す)。

\ex{第十一}
\begin{longtable}{c*{2}{p{0.45\hsize}}}
 1. & gauḥ saṃdhyāyāṃ duhyate. & 牝牛は黎明に搾られる。\\
 2. & svadeśe pūjyate rājā, vidvān sarvatra pūjyate. & 王は自らの國土に於て尊敬
 せらる。賢者は一切の處に尊敬せらる。 \\
 3. & kṣudra-śṛgālo 'yaṃ, tad va\-dhyatām. & これは小さき\ruby{豺}{やまいぬ}である。そは殺さるべし。
\end{longtable}


%%% Local Variables:
%%% mode: latex
%%% TeX-master: "IntroductionToSanskrit"
%%% End:
