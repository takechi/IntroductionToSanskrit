\section{受動調並に派生動詞}
\subsection{受動調動詞}
\numberParagraph
現在動詞(\ref{np:60}條以下)の受動は語根に ya を附加して作
る。人稱語尾は爲自を用ふ。語根 kṛ (作る)~ krīya.

\begin{center}
\begin{tabular}{c*{3}{p{0.23\hsize}}}
  \multicolumn{4}{c}{\textbf{現實法}} \\
     & 單      & 兩        & 複 \\
  1. & kriye   & kriyāvahe & kriyāmahe \\
  2. & kriyase & kriyethe  & kriyadhve \\
  3. & kriyate & kriyete   & kriyante \\
  \multicolumn{4}{c}{\textbf{第一過去}} \\
  1. & akriye     & akriyāvahi & akriyāmahi \\
  2. & akriyathās & akriyethām & akriyadhvam \\
  3. & akriyata   & akriyetām  & akriyanta
\end{tabular}
\end{center}
\begin{center}
\begin{tabular}{c*{3}{p{0.23\hsize}}}
  \multicolumn{4}{c}{\textbf{命令法}} \\
     & 單       & 兩         & 複 \\
  1. & kriyai   & kriyāvahai & kriyāmahai \\
  2. & kriyasva & kriyethām  & kriyadhvam \\
  3. & kriyatām & kriyetām   & kriyantām \\
  \multicolumn{4}{c}{\textbf{可能法}} \\
  1. & kriyeya   & kriyevahi   & kriyemahi \\
  2. & kriyethās & kriyeyāthām & kriyedhvam \\
  3. & kriyeta   & kriyeyātām  & kriyeran
\end{tabular}
\end{center}



%%% Local Variables:
%%% mode: latex
%%% TeX-master: "IntroductionToSanskrit"
%%% End:
