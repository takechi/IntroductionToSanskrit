\section{合成語法}
\numberParagraph
二個以上の語を合成することは梵語には極めて著しい特
徴である。Veda や Brāhmaṇa ではせいぜい二語ぐらゐの合成
語が現はれるに過ぎないが,後期梵語ではこの傾向が次第に加は
つて來る。合成語は語尾變化の繁雜を簡易にすることがその發生
の動機の一である。詩人 Kālidāsa の「雲の使ひ」(I, 30) に
は河の流を形容して vīci-kṣobha-stanita-vihaga-śreṇi-kāñcī-
guṇāyāḥ と云ふ語があるが,これは「波-亂-囀-鳥-列-帶-
絲」と云ふ七語から成る合成語で,その意味は「波の亂れが(恰
うど)囀る鳥の一列(のやうなの)を一筋の帶とした所の」と云
ふことである。語と語の間の關聯を適當に理解するにはこの合成
語を正しく讀むことが重要な問題の一である。

\numberParagraph
印度の文典家は合成語を六種に分つて說く。卽ち六合釋
(ṣaṭ-samāsa 殺三磨娑)である。佛敎經典にも相當合成語が見え
る。而してその讀み方の如何が敎義重大の関係を有する。印度
佛敎々學の華やかなりし那爛陀寺時代の學匠達は經典の釋義の上
に常に六合釋を以て鎬を削つて諍論した形迹がある。支那に經典
が飜譯されても漢文の上に現はれた合成語の解釋が何等かの規範
なしには不便を感ずるので,之を看て取つた玄奘は印度文典の方
規を採用して經典解釋に新機軸を出した。これが六合釋である。
かくしてこれは久しく佛典硏究家にも規矩準繩となつたものであ
る。

\numberParagraph
六合釋とは相違,依主,持業,帶數,隣近,有財である。
然しこれらは性質上三種として說明せられる。卽ち 1. 並列合成
語(相違),2. 決定合成語(この中に依主,持業,帶數,有財を含
む),3. 副詞合成語(隣近)である。

\numberParagraph
合成語の各個の間には連聲法の規定(\ref{np:10}--\ref{np:29}條)が適用
せられる。又最後の語の外,各語はみな語基の形を取る。三語基
あるものは中,二語基あるものは弱を用ふ。rājan の如き n に
終るものはその n を除去して用ふ。

\numberParagraph
mahat は前分として mahā となる。akṣi (眼)は後分と
して akṣa, ahan (日)は aha 又は ahna に作る。sakhi (友)は
sakha, rātri (夜)は rātra, path (路)は patha, manas (意)は
manasa, varcas (輝)は varcasa となる。又時として反對に變化
することもある。gandha (香)が gandhi, go (牝牛)は母音の以
前に gava, 語末には gava 又は gu.

\subsection{並列合成語(相違釋 dvaṃdva)}
\numberParagraph
この合成語は各部分が互に等しき位にあるもので,各語
が連接し(copulative)又はその孰れかを選び取る(alternative)
意義を有す。
\begin{enumerate}[label=(\alph*)]
\item 語が二個より成るか三個以上より成るかで兩數又は複
數の語尾を追加する。candrādityau (月と日),kākākhu-%
mṛga-kūrmāḥ (鴉と鼠と鹿と龜と)。
\item 若し合成語が個々のものを意味しないでそのまゝで一
つの槪念を表はすやうな時には集合名詞として中性單數の
形を取る。pāṇi-pādam (手足)。gavāśvam (牛馬)。
\end{enumerate}

\subsection{決定合成語}
\subsubsection{依主釋(tatpuruṣa)}
\numberParagraph
依主とは後分が前分に依つて限定せらるゝものである。
前分は後分に對して格の關係を有す。例:grāma-gata (村へ行
きたる)の前分は業格,Indra-gupta (インドラに護られたる)の
前分は具格,svarga-patita (天界から墮ちたる)の前分は從格,
rāja-putra (王子)の前分は屬格である。

\numberParagraph
依主の前分が格の形を有することもある。vācaṃ-yama
(聲を制して),Gavāṃ-pati (牛主,憍梵波提),padme śaya (蓮華
の上に橫はれる)。

\numberParagraph
合成語の最後分として動詞の語根が用ひられる。veda-%
vid (吠陀に通ぜる),短母音の語根には t を加ふ。Aśva-jit (馬
勝)。語根の母音 ā は男中性の語として短縮せらる。abhyāsa-%
stha (近くにある)。

\subsubsection{持業釋(Karmadhāraya)}
\numberParagraph
依主の前分が形容詞副詞にして後分を限定する時は持業
釋と云ふ。nīlotpala (靑き蓮花),paramānanda (最上の歡喜),
ati-dīrgha (極めて長き)。前分が名詞であることもある。kusu-%
ma-sukumāra (花の如く軟かなる),puruṣa-siṃha (人獅子),か
くの如き場合は獅子の如き人の意味で前分が主體後分が譬喩であ
る。

\subsubsection{帶數釋(dvigu)}
\numberParagraph
依主の前分が數詞にして其の語形は中性又は ī にて終る
女性名詞である場合を帶數釋と云ふ。tri-rātra (三夜),pañca-%
gava 又は pañca-gavī (五牛)。

\subsection{所有合成語(有財釋 bahuvrīhi)}
\numberParagraph
これは最後分が名詞又は名詞の義に用ひられた形容詞で
あつて,全體が一つの形容詞として用ひられたる決定合成語であ
る。この合成語は「……を持てる」と云ふ意義を現はす。dīrgha-%
bāhu は「長き臂」でなくして「長き臂を持てる」の意。mauna-%
vrata は「沈默の戒を持てる」で「沈默の戒を受けたる」といふ
こと。manda-mati「劣れる智慧をもてる」卽ち「智慧劣りたる」
の意。cintā-para「思惟を最上とせる」卽ち「思惟に專一なる」。

\numberParagraph
有財は形容詞の作用を有するから,形容される名詞によ
りて性を定める。故に ā に終る最後分は男性又は中性の名詞に關
係する時は a となる。vidyā から alpha-vidya (少しく知れる),
jihvā から dvi-jihva (二舌を持てる),dhāryā から sa-bhārya
(妻を伴へる)が作られる。

\numberParagraph
hasta, pāṇi 等「手」と云ふ意味の語は最後分となつて
「……を手に持てる」の意味となる。pātra-hasta (器を手にせる),
padmapāṇi (蓮花を手にせる)等。

\subsection{副詞合成語即隣近釋(avyayībhāva)}
\numberParagraph
前分が不變詞であつて後分が名稱詞であり,而して副詞
的に用ひられた合成語である。この合成語は中,單,業の語尾を
取る。例:yatheccham (欲するまゝに),yāvaj-jīvam (生涯の
間),pratidinam (每日)。

支那の註釋家が六合釋を云々する中にこの隣近釋は全く誤解さ
れてゐるやうである。大抵,例には「長安住」と云ふのが出る。
長安に住してゐないでも近くに住んでゐるからそれを長安住と云
ふのだと云つていゐる。これでは何のことかわからぬ。これは「近
長安」(upanagaram) とでもあつた副詞的の句であつたのだら
う。これならば「都城の近くに」の意味で例として當てはまる。
隣近の意味は慈恩の義林章に「俱時の法義用增勝なるを以て自體
を彼に從へて而も其の名を立つ」とあつて,俱時の法とは文典的
に云ふならば主辭と賓辭である。これは同時に存在する二個の槪
念,俱時の法である。義は意味。用は働らき。增勝とは一層說明
を要する場合といふこと。自體を彼に従へるとは語そのものを彼
の主辭賓辭に從屬せしめてこれを設けたのであると云ふ。主辭賓
辭の意味を更に說明するもので,その場合その傍に隣近せしめる
ものである。卽ち副詞的の用法のことである。慈恩ではよくわか
つてゐたものが,後代の人がわからぬために何のことかわからぬ
儘に傳へられて來たものらしい。

\subsection{動詞合成法}
\numberParagraph
動詞は前接字又は副詞と合成せられその語根の意義は少
しく變化する。加へらるゝ前接字は

\begin{center}
\begin{tabular}{*{3}{p{0.3\hsize}}}
  ati (超えて)  & ava (下に) & parā (彼方に) \\
  adhi (上に)   & ā (まで)   & pari (回りに) \\
  anu (隨つて)  & ud (上に)  & pra (前に) \\
  antar (間に)  & upa (近く) & prati (反對に) \\
  apa (外に)    & ni (下に)  & vi (別に) \\
  abhi (對して) & nis (外に) & sam (共に)
\end{tabular}
\end{center}

副詞の例:

alam (十分に) + kṛ (作る) = alaṃkṛ (飾る)。

astam (下に) + gam (行く) = astaṃgam (沈む)。

āvis (明に) + bhū (ある) = āvirbhū (現はる)。

\numberParagraph
語根 as (有る),bhū (なる),kṛ (作す)を名詞の後に
加へて「……がある」,「……となる」,「……と作す」の意とする。
a 語基は ī に變じ,i, u はこれを延長し,ṛ は rī に變ず。śukla
(白き)は śuklīkṛ (白くす),śuci (淨き)は śucībhū (淨くなる),
mṛdu (軟らかき)は mṛdū-syāt [as の三單可能法] (軟らかくあるべ
し),mātṛ (母)は mātrīkṛ (母となす),又 bhasman (灰)は
bhasmīkṛ (灰と化す)に作る。

\ex{第九}
\begin{longtable}{c*{2}{p{0.45\hsize}}}
 1. & puruṣā api baṇā api guṇa\-cyutāḥ kasya na bhayāya. & グナ(德,弦)を離れた人も
箭も誰に取つてか恐怖を起さざらむ。\\
 2. & gajo gharmārtaś chāyārthī tamāla-vṛkṣaṃ samāśritaḥ. & 熱に苦められ蔭を欲する象
はタマーラ樹に椅れり。\\
 3. & Mumbā-pura-nivāso mamāro\-gyāya na kalpate. & ボンベイ市の滯留は私の健
康に適せず。\\
 4. & brāhmaṇā vidyopārjanārthaṃ Kanyakubjaṃ gatāḥ. & 婆羅門等は知識獲得のため
にカニヤクブジヤへ行けり。\\
 5. & jalada-ninada-muditāḥ śikhino nṛtyanti. & 雲の響を喜んで孔雀は舞ふ。\\
 6. & bālā kāntam indīvara-dala-prabhā-caura-cakṣuḥ kṣipati. & 少女は愛人に靑蓮の瓣
の光を盗む眼ざしを投げたり。\\
 7. & asāraḥ saṃsāro 'yam, giri\-nadī-vegopamaṃ yauvanaṃ, tṛṇāgni-samaṃ, śarad\-%
abhra-cchātā-sadṛśā bhogāḥ, svapna-sadṛśo mitra-putra-kala-tra-bhṛtya-varga-saṃyogaḥ:
evaṃ mayā samyak parijñātam. & この世界は空虛である。靑春は山より落つる\ruby{駃}{はや}き流
の如く,生命は枯草を火に投ずる如く,享樂は秋雲の影に似,友人,子,妻,奴僕の群の相會する
ことは夢の如し。かくわれによりて正しく認識せられたり。\\
 8. & arthaḥ puruṣāṇāṃ ṣaḍbhir upāyair bhavati bhikṣayā nṛpa\-sevayā kṛṣi-karmaṇā vidyopārja\-nayā
vyavahāreṇa vaṇik-karma\-ṇā vā. & 富は人にとりては六種の方法によりて存在す。乞食によりて,王に對する奉
仕によりて農耕の業によりて,知識の獲得によりて金融業又は商業によりて。\\
 9. & brāhmaṇī paruṣatara-vacanaiḥ patiṃ bhartsayamānābhāṣata bho na mayā tava hasta-lagna-yā
kvāpi labdhaṃ sukhaṃ na miṣṭānnasyādanaṃ na hasta-pāda-kaṇṭha-bhūṣaṇam. & 婆羅門の妻は一層\ruby{麤}{あ}らき言
葉で夫を叱りつゝ云へり おゝ汝の手に椅りすがる私は何處にも幸福を得ず。舐める食物のうまさ
もなし。手足頸の裝飾もなし。\\
10. & eko doṣo guṇa saṃnipāte nimajjatīndoḥ kiraṇeṣv ivāṅkaḥ. & 德の積聚に於ける一の過失
は月の光線の中に於ける斑點の如く影を消す。\\
11. & bho, svalpa-kāyo bhavān & おゝ汝は小なる身體をもてることよ。\\
12. & kaulikaḥ kṛta-maraṇa-niścayaś cāpa-pāṇir garuḍārūḍho yuddhā\-ya prasthitaḥ. & 織匠は死の決心をなして弓
を手にしガルダに駕して戰鬪のために出發せり。\\
13. & brāhmaṇena mitrasyātīva-pīvara-tanuḥ paśuḥ pradattaḥ. & 一人の婆羅門によりて友に
極めて身體肥へたる供犧の家畜が與へられたり。\\
14. & śubhe 'hani prahṛṣṭa-manāḥ kumāro mitreṇa saha gurujanā-nujñāto deśāntaraṃ gataḥ. & 美はしき日に心悅べる王子
は友と共に長上の命によりて國外に行けり。\\
15. & kṣapaṇakāḥ śrāvakasya gṛhe gacchanti prāṇa-dharaṇa-mā\-trāṃ cāśana-kriyāṃ kurvanti. & 乞食僧等は俗人の家に行け
り,而して生命の保持だけの食事をなせり。\\
16. & te dhanyā ye vīta-rāgā guru-vacana-ratās tyakta-saṃsāra-sangā veda-jñāne vilīnā vane
yauvanaṃ nayante. & 貪慾を離れ,師の語を樂み,輪廻の執着を捨て吠陀の智に沈潛し,森林に靑春
を過す彼等は幸福なり。\\
17. & varaṃ prāṇa-parityāgo na māna-parikhaṇḍanaṃ prāṇa-tyāgaḥ kṣaṇaṃ caiva māna-bhaṅgo dine dine & 矜持を失はんよりは寧ろ生
命を捨てむこと勝る。生命の捨は瞬間なり。矜持の破壞は日日のことなり。
\end{longtable}

%%% Local Variables:
%%% mode: latex
%%% TeX-master: "IntroductionToSanskrit"
%%% End:
