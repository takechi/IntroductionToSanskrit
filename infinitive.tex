\section{不定法}
\numberParagraph
不定法は語根に直接又は i を介して後接字 tum を附加
して作る。その母音は重韻となる。dā (與ふ)~ dātum, ji (勝
つ)~ jetum, nī (導く)~ netum, bhū (有り)~ bhavitum, kṛ (作
る)~ kartum, jīv (生く)~ jīvitum, grah (取る)~ grahītum,
sah (堪ふ)~ soḍhum.

aya 級並に催起動詞は ay なる綴を添へ常に i を介して tum
を附加する。cur (盗む)~ corayitum, budh (覺める)~ bodhayi\-%
tum.

分詞等其他の意義並に用法に關しては \ref{np:232} 條以下參照。

\ex{第十三}
\begin{longtable}{c*{2}{p{0.45\hsize}}}
 1. & na hi bhavati yan na bhāvyaṃ, bhavati bhāvyaṃ vināpi yat\-nena;
kara-tala-gatam api na\-śyati, yasya hi bhavitavyatā nāsti.
& 有るべからざる所のものは有ることなし。有るべきものは努力を加へずして
もあり,存在性なき所のものについてはたとひ掌理にあるも消失す。\\
 2. & mūṣikā gṛha-jātāpi hantavyā\-pakāriṇī. & 鼠は家に生ずるも害あり殺さるべし。\\
 3. & vayasya, na bhetavyam. & 友よ,恐れざれ。\\
 4. & kim evaṃvidhe vyatikare kār\-yam āvābyām. & かくの如き不幸の時我々二
 人は何をなすべきや。\\
 5. & puruṣenodyamo na tyājyaḥ. & 人は努力を廢すべきでない。\\
 6. & yasya gṛhe mātā nāsti bhāryā ca, tenāraṇyaṃ gantavyam. & その家に母と妻となき人は
 森林へ行くべきだ。\\
 7. & avaśyaṃ nidhanaṃ sarvair gantavyam iha mānavaiḥ. & 必ずや一切の人は死に行くべし。\\
 8. & durbalo 'pi ripur nāvajñeyo kathaṃcana. & 敵は弱くとも決して侮るべきでない。\\
 9. & upadeśo na dātavyo yādṛśe tādṛśe janāya. & 出會がしらの人に忠吿は與へらるべきでない。\\
10. & yo vidyām aiśvaryaṃ vāsādya vicarty asamunnaddhaḥ, sa paṇḍita ucyate.
& 知識若くは權力に達して謙虛に虛する所の彼は賢者と云はれる。\\
11. & bhadre, yāvad ahaṃ bhoja\-naṃ gṛhītvā pratyāgacchāmi, tāvat tvayātra sthātavyam.
& 愛するものよ,我が食を齎して歸り來るまで汝はそこに立つべし。\\
12. & mām uddiśya pattraṃ pre\-ṣaya. & 我に宛てゝ書翰を送れ。\\
13. & brāhmaṇenoktam: ko guṇo vidyāyā yena deśāntaraṃ gatvā bhūpatīn paritoṣyārthopārjanā
na kriyate. & 婆羅門は云へり。他國に行きて王者を滿足せしめ富の獲得がなされざるなら
ば何の知識の功績ぞや。\\
14. & śaśako mandaṃ mandaṃ gat\-vā praṇamya siṃhasyāgre sthitaḥ. & 兎は徐々に行きて禮をなし
獅子の前に立てり。\\
15. & kauliko rāja-kanyāṃ dṛṣṭvā kāma-śarair hanyamānaḥ saha\-sā bhūtale nyapatat.
& 織匠は王の娘を見て愛神の箭に打たれつつ突然地面に仆れたり。\\
16. & atha mitraṃ tad-avastham avalokya rathakāras tad-duḥ\-khaduḥkhita āpta-puruṣais taṃ
samutkṣipya sva-gṛham ānā\-yayat. & 時に車匠は友のその狀態を
見てその苦しみを苦しみとし得られた人々によりて彼はかき上げて自分の家につれ來れり。\\
17. & śiṣyā upādhyāyam āpṛcchyā\-nujñāṃ labdhvā pustakāni nīt\-vā pracalitāḥ.
& 弟子等は師を吿別し許可を得て書を取りて開きたり。\\
18. & pathiko grīṣmoṣmaṇā saṃta\-ptaḥ kaṃcin mārga-sthaṃ vṛk\-ṣam āsādya tatraiva prasuptaḥ.
& 旅人は夏の暑さに苦しめられとある路傍の樹に近づきて眠れり。\\
19. & Nalo bhaktāṃ Damayantīṃ tyaktvā śoka-paryākulo 'bhavat. & ナラは貞實なるダマヤン
ティーを捨てゝ悲に亂されたり。\\
20. & anusṛtya satāṃ vartma yat svalpam api tad bahu & 善人の路を追ひ行けば極め
て小なるもそは價値大なり。\\
21. & śarīrāsāmarthyān na padam api vṛddhaḥ siṃhaś calitum aśaknot. & 身體\ruby{羸{弱}}{るい|じゃく}のために老いたる
獅子は一歩をすら行く能はざりき。\\
22. & stenābhyām uktaṃ na sarvam etad dhanaṃ gṛhaṃ prati netuṃ yujyate tad dhāgam atra vana\-%
gahane kvāpi bhūmau nikṣi\-pāya. & 二人の盗賊は云へり。この一切の財物を家に運ぶは
宜しからず。我々はその一部分をかの深林の中のある地點に置くべし。\\
23. & lubdhako mṛgayāṃ kartuṃ pratiṣṭhati. & 獵師は狩をなすために出發せり。\\
24. & deśāntara-stho dayitā-vipra\-yogaṃ soḍhuṃ na śaknomi. & 他國に在る我れは愛人の離別を堪へ能はず。\\
25. & śṛgālaḥ palāyitum icchaṃs tatra sthāna eva siṃhena khaṇ\-ḍaśaḥ kṛto mṛtaś ca.
& 豺は逃れ去らんと欲せしもその場に於て獅子に寸斷せられて死せり。\\
26. & brāhmaṇo bhāryayā saha bhoktum ārabhate. & 婆羅門は妻と共に食し始めた。\\
27. & upakartuṃ priyaṃ vaktuṃ snehaṃ kartuṃ sajjanānāṃ svabhāvaḥ. & 善を行ひ,善を語り,愛を
示すは善人の性質なり。\\
28. & svabhāvo nopadeśena śakyate kartum anyathā, sutaptam api pānīyaṃ punargacchati śīta\-tām.
& 自性は忠吿によりてこれを變ずる能はず。水は熱せられてもつひに冷かになるものなり。
\end{longtable}
%%% Local Variables:
%%% mode: latex
%%% TeX-master: "IntroductionToSanskrit"
%%% End:
