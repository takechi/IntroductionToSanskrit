\section{連續體}
\numberParagraph
連續體を作るには二種の方法がある。語根に或は tvā を
加へ或は ya を加ふ。

\begin{enumerate}[label=(\alph*)]
\item tvā は語根に直接又は i を介して添加せらる。i の添
加に關しては \ref{np:179}條の法則が適用せられる。tyaj (捨つ)
~ tyaktvā, labh (得る)~ labdhvā, kṛ (爲す)~ kṛtvā,
grah (取る)~ gṛhītvā, vac (語る)~ uktvā, han (殺す)
~ hatvā, gam (行く)~ gatvā, man (考ふ)~ matvā,
kram (歩む)~ krāntvā, śam (靜める)~ śāntvā, yaj (供
ふ)~ iṣṭvā, dṛś (見る) dṛṣṭvā.

否定前加語 a は tvā の連續體に附加せられ得る。
akṛtvā (爲さずして)。aya 級並に催起動詞は母音 i を介
して附加せられた後接字の前に ay なる綴が殘される。
cur (盗む)~ corayitvā.
\item ya は前加語を有する語根に附加せられる。語根は一般
に受動後接字(\ref{np:159}條)の附加の場合に於ける變化をなす。

e, ai, o に終るものは ā に變じ,ā に終るものは變化し
ない。kṣip + pra (投げ棄つ)~ prakṣipya, prach + ā (吿
別する)~ āpṛcchya, pṝ + ā (充たす)~ āpūrya, dā + pra
(渡す)~ pradāya.

短い母音に終る語根は ya の代りに tya を附加する。
kṛ + adhi (處理す)~ adhikṛtya, i + adhi (學ぶ)~ adhītya.
同樣に tya は am 並に an に終る語根でもその鼻音を除
去してこれを附加するを得る。それは過去受動分詞を作る
時鼻音の消失するに準ず。gam + ā (來る)~ āgatya 又は
āgamya, jan + pra (發生す)~ prajanya 又は prajāya.
khan + ni (埋む)~ nikhanya 又は nikhāya. 然し man +
ava (輕蔑す)は只 avamatya, han + pra (殺す)も只
prahatya のみ。
\end{enumerate}

\numberParagraph
連續體の他の一種がある。それはあまり多く用ひられな
いが,語根に am を附加して作る。語根若し母音に終る時は複
重韻,若し中間に母音ある時はそれを重韻となす。中間の a は
延長せられ,ā に終るものは ā の前に y を添加する。kṛ (作る)
kāram. vid (知る) vedam, dā (與ふ) dāyam.

%%% Local Variables:
%%% mode: latex
%%% TeX-master: "IntroductionToSanskrit"
%%% End:
