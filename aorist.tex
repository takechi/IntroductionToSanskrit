\section{第三過去組織}
\numberParagraph
第三過去は過去符 a を有し第一過去と同樣の語尾を附加
して作る。その構成法に三種七類が分れる。三種とは (I) 單第
三過去,これに (1) 語根語基と (2) a 語基の二類がある。(II)
重複第三過去,(III) 硬吹氣音第三過去,これに (1) sa 語基,(2)
s 語基,(3) iṣ 語基,(4) siṣ 語基の四類が語ある\endnote{「四類が語ある」はママ。}。

\subsection{單第三過去}
\subsubsection{語根語基}
\numberParagraph \label{np:204}
語根語基の第三過去は語根に直に語尾を加へる。これは
爲他の ā に終る語根又は bhū にのみ用ひられる。その餘に用
ひらることは極めて稀である。ā に終る語根の三複は us なる語
尾を取る。

語根 dā (與ふ)。

\begin{center}
\begin{tabular}{c*{3}{p{0.23\hsize}}}
  \multicolumn{4}{c}{爲他} \\
     & 單   & 兩     & 複 \\
  1. & adām & adāva  & adāma \\
  2. & adās & adātam & adāta \\
  3. & adāt & adātām & adus \\
\end{tabular}
\end{center}

語根 bhū (ある)。

\begin{center}
\begin{tabular}{c*{3}{p{0.23\hsize}}}
  \multicolumn{4}{c}{爲自} \\
     & 單      & 兩      & 複 \\
  1. & abhūvam & abhūva  & abhūma \\
  2. & abhūs   & abhūtam & abhūta \\
  3. & abhūt   & abhūtām & abhūvan
\end{tabular}
\end{center}

爲自の dhā (置く)~二單 abhithās, sthā (住す)の asthithās,
kṛ (作る)の akṛthās, mṛ (死す)の amṛthās. 三單 abhita,
asthitha, akṛta, amṛta.

\subsubsection{a 語基}
\numberParagraph
a 語基の第三過去は a 級の第一過去(\ref{np:64} 條)の如く構
成せられ變化する。但し語根の母音は平音であつて變化しない。

sic (灌ぐ)~第一過去 asiñcat, 第三過去爲他 asicat, 爲自
asicata. lip (塗る)~第一過去 alimpat, 第三過去爲他 alipat,
爲自 alipata.

不規則なるものに dṛś (見る)三單爲自 adarśat, hve (呼ぶ)
ahvat の如きがある。

\subsection{重複第三過去}
\numberParagraph
重複第三過去は語根を重複して作られる。而して子音は
現在の場合(\ref{np:146} 條)と同樣に重複せられる。śri (行く)三單爲
他 aśiśriyat.

\begin{tabular}{ll}
不規則なるもの: & vac (言ふ)~ avocat. \\
                 & pat (落つ)~ apaptat.
\end{tabular}

この形が最も多く用ひられるのは第十類動詞並催起動詞であ
る。語基の aya は除かれる。重複音と語根の母音とは必ずしも
一樣でない。若し中間に a を有するか又は ā, ṛ, ṝ に終る場合
多くは i 又は ī を以て重複の母音とする。

\numberParagraph
第三過去の重複語基は重複音に重きを置き本來の語根は
比較的輕視せられる。かくして重複音の母音は常に長く,本來の
語根の母音は平音又は短母音である。

語根 cur (盗む)~ acūcuram, nī (導く)催起~ anīnayam,
yuj 催起(軛す)~ ayūyujam. dā 催起(與へしむ)~ adīdapat,
sthā 催起(立たしむ)~ atiṣṭhipat.

\subsection{硬複氣音第三過去}
\subsubsection[sa 語基]{sa 語基 a + 語根 + sa}
\numberParagraph
sa 語基に屬する動詞は a 又は ā 以外の母音を有して
ś, ṣ, h に終れるものを以てする。語尾變化は á 級の第一過去
に同じ(\ref{np:64} 條)。但し一單爲自は e でなく i に終り,二三兩は
athām, ātām である。

diś (指す)~一單爲他 adikṣam, 三複 adikṣan, 一單爲自
adikṣi.

\subsubsection[s 語基]{s 語基 a + 語根 + s}
\numberParagraph
以下の三種は語基に强弱を分つ。この點よりして現在組
織の第二種變化に相當する。强基は爲他に用ひられ,弱基中基は
爲自に用ひられる。

\numberParagraph
s 語基は爲他には複重韻,爲自には i, ī, u, ū に終る語根は
ṣam, aśroṣi, kṛ (作す)~ akārṣam, akṛṣi, dṛś (見る)~三,單,
爲他 adrākṣīt (\ref{np:9} 條),bhaj (分つ)~一,單は abhākṣam,
abhakṣi. t 又は th にて始まる語尾の次前,鼻音にあらざる子音
の次後の s は省き去られる。tud (打つ)~二,複,爲自 atautta.
kṣip (投ぐ)~ akṣaipta. 然し man (思ふ)~三,單,爲自
amaṃsta. kṛ (作る)~二,複,爲自 akārṣṭa. dhvam の前に s
は消失する。若し ā 以外の母音に先立たれた時 s は消失して
dhvam は ḍhvam となる。卽ち akṛḍhvam. nī (導く)~
aneḍhvam.

\numberParagraph
nī (導く)强基 nais, 弱基 nes.

\begin{center}
\begin{tabular}{c*{3}{p{0.23\hsize}}}
  \multicolumn{4}{c}{爲他} \\
     & 單      & 兩       & 複 \\
  1. & anaiṣam & anaiṣva  & anaiṣma \\
  2. & anaiṣīs & anaiṣṭam & anaiṣṭa \\
  3. & anaiṣīt & anaiṣṭām & anaiṣus \\
  \multicolumn{4}{c}{爲自} \\
  1. & aneṣi    & aneṣvahi & aneṣmahi \\
  2. & aneṣṭhās & aneṣṭhās & aneḍhvam \\
  3. & aneṣṭa   & aneṣṭa   & aneṣata
\end{tabular}
\end{center}

\subsubsection[iṣ 語基]{iṣ 語基 a + 語基 + iṣ}
\numberParagraph
母音に終る語根の時は爲他に複重韻,爲自に重韻となる。
lū (切斷す)~alāviṣam, alaviṣi. 又語根に a を有して單子音を
以て終る時は爲他を隨意に複重韻となすを得る。vad は然し恒に
複重韻に作る。grah (取る)は語基を grahī に作る。a 以外の
母音を有する語根は爲他,爲自共に重韻に作る。paṭh (學ぶ)~
三,單,爲他 apāṭhīt 或は apaṭhīt. vad は只 avādīt. grah ~
agrāhīt. 爲自~ agrahīṣṭa, budh (覺る)~ abodhīt, abodhiṣṭa.

\subsubsection[siṣ 語基]{siṣ 語基 a + 語根 + siṣ}
\numberParagraph
siṣ 語基は只爲他の形のみが用ひられる。語根は多くは
ā に終り而も \ref{np:204} 條に屬せざるもの,並に e, o, ai 等に終るも
のである。然しこれらは siṣ と會して皆 ā となる。yā (行く)~
ayāsiṣam, glai (疲困す)~ aglāsiṣam. 他に nam (曲る),yam
(止める),ram (樂む)は例外としてこれに屬す。nam ~ anaṃ\-%
siṣam, anaṃsīs, anaṃsīt; anaṃsiṣus.

\subsection{願望法}
\numberParagraph
可能法の第三過去は常に祝福祈願の意を示す。語尾は爲
他 yāsam, yās; yāsva, yāstam, yāstām; yāsma, yāsta,
yāsus. 爲自 īya, īṣṭhās, īṣṭa; īvahi, īyāsthām, īyāstām;
īmahi, īdhvam, īran. bhū (有り)~ bhūyās (汝をしてあらしめ
む),bhūyāt (彼をしてあらしめむ),dā (施す)~ deyāt (彼をし
て施さしめむ),pā (守る)~ pāyāt (彼をして守らしめむ),sthā
(住す)~ stheyās (汝をして住せしめむ),kṛ (作る)~ kriyāt (彼
をして作らしめむ),diś (示す)~ diśyāt (彼をして示さしめむ)。
又次の如き特殊の形がある。bhūyāstam (彼等兩人をしてあらし
めむ),brū (言ふ)~ brūyāsta (汝等をして云はしめむ),pāyāsus
(彼等をして守らしめむ),bhūyāsus (彼等をしてあらしめむ),
vi + dhā ~ vidhāsīṣṭa (彼をして爲さしめむ)。

\subsection{受動第三過去}
\numberParagraph
受動第三過去は三,單,爲自の外は總て能動の語基に爲自
の語尾を附加せるものに同じ。三,單,爲自は語根に i を附加して
作る。この際語根中の母音は重韻となり,語根の終にある母音は
複重韻となる。

\begin{center}
\begin{tabular}{lll}
       & nī (導く)~ anai + i = anāyi  & (彼は導かれたりき)。\\
       & lū (斷つ)~ alau + i = alāvi  & (彼は斷たれたりき)。\\
       & diś (示す)~ adeś + i = adeśi & (彼は示されたりき)。\\
  然し & pad (落つ)~ apād + i = apādi & (彼は落されたりき)。\\
       & dā (與ふ)~ adāy + i = adāyi  & (彼は與へられき)。
\end{tabular}
\end{center}

\ex{第十六}
\begin{longtable}{c*{2}{p{0.45\hsize}}}
 1. & sa pārthivaḥ kadācin mṛga\-yām agamat. & 彼王はある時狩獵に行けり。\\
 2. & ajani te vai putraḥ. & 汝に一人の子は生まれたり。\\
 3. & tad ahaṃ tubhyam eva dadā\-mi ya eva satyam avādīḥ. & 眞理をのみ語れる汝にまで
 我は其れを與ふ。\\
 4. & ud u śriya uṣaso rocamānā asthuḥ. & 曉の光は輝き始めたり。\\
 5. & Śrīnagarān niragāt paṇḍitaḥ. & スリナガルより一人の賢者が出立せり。\\
 6. & astam ayāsīd ravis timireṇā\-vṛtaṃ nabhaḥ. & 日は沈み行けり,天空は暗もて覆われたり\endnote{「暗もて」はママ。}。\\
 7. & bhoḥ prohita bhavad-anujñām anusṛtya baṭave 'haṃ sāvitrīm upādikṣam.
 & おゝ家庭僧よ汝の命に從ひ我は子にまでサーヴィトリー頌を敎へたり。\\
 8. & veṇu-dhamanyāgnim adhmā\-siṣam, tad asmin pradīpte vahnāv āhutīḥ prāsya.
 & 我は竹の管もて火を吹けり,さればこの燃ゆる火の中に供物を投ぜよ。\\
 9. & nitya-karmānuṣṭhānāyāsnāsīs tac chūdrādīn mā spṛkṣaḥ. & 常時の勤行をなさんがため
 に汝は沐浴せり。されば組陀羅等に觸るる勿れ。\\
10. & kiṃ yūyam avocata, punar api kathayata, nāham avahito 'bhūvam.
& 汝等は何を云へりしか。復び語るべし。我は注意せざりき。\\
11. & idam āmra-phalaṃ vṛkṣād apaptat, yadi rocate gṛhītvā svādasva. & この\ruby[g]{檬果}{マンゴー}は樹より落ちた
り。若しそれが宜しからば執りて味へ。\\
12. & prātar ārabhya pañca-sapta\-tiṃ vṛkṣān asicāma. & 我れ今朝より始めて七十五
本の樹を灌水せりき。\\
13. & krīḍārtham upavanam aga\-matāṃ daṃpatī. & 遊戯のために夫妻は林に行けり。\\
14. & iyaṃ bālikā duḥkhavārttāṃ śrutvāmuhat, āśvāsayainām udakena ca siñca.
& この少女は悲しき出來事を聞きて失神せり。彼女を回復せしめ水を灌げ。\\
15. & ye māṃ rūpeṇa cādrākṣur, ye māṃ ghoṣeṇa cānuvaguḥ, mithyā-praharaṇa-prasṛtā na
māṃ drakṣyanti te janāḥ. & 色を以て我を見,聲を以て我を求むる所の彼等の人々は邪行を行ずるものに
して我を見ざるなり。\\
16. & tataḥ sa gardabhaṃ laguḍena tāḍayāmāsa; tenāsau pañcatvam agamat. & そこで彼は杖を以て驢馬を
打つた。かくてそれは死んだ。\\
17. & jyog vā iyam Urvaśī manu\-ṣyeṣv avatsīt. & このウルヷシーはきつど人間の中に長らく住んでゐた。\\
18. & yad idānīm dvau vivadamā\-nau eyātām aham adarśam aham aśrauṣam iti ya eva brū\-yād aham adarśam iti tasmā
eva śraddadhyāma. & 今若し二人が(一方は)我は見た(他のものは)我は聞いたと互に云ひ爭ひつゝ
來るとせんに我々は我は見たと云ふ彼に信賴を置くべきだ。
\end{longtable}

%%% Local Variables:
%%% mode: latex
%%% TeX-master: "IntroductionToSanskrit"
%%% End:
