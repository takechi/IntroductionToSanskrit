\section{第三過去組織}
\numberParagraph
第三過去は過去符 a を有し第一過去と同樣の語尾を附加
して作る。その構成法に三種七類が分れる。三種とは (I) 單第
三過去,これに (1) 語根語基と (2) a 語基の二類がある。(II)
重複第三過去,(III) 硬吹氣音第三過去,これに (1) sa 語基,(2)
s 語基,(3) iṣ 語基,(4) siṣ 語基の四類が語ある\endnote{「四類が語ある」はママ。}。

\subsection{單第三過去}
\subsubsection{語根語基}
\numberParagraph
語根語基の第三過去は語根に直に語尾を加へる。これは
爲他の ā に終る語根又は bhū にのみ用ひられる。その餘に用
ひらることは極めて稀である。ā に終る語根の三複は us なる語
尾を取る。

語根 dā (與ふ)。

\begin{center}
\begin{tabular}{c*{3}{p{0.23\hsize}}}
  \multicolumn{4}{c}{爲他} \\
     & 單   & 兩     & 複 \\
  1. & adām & adāva  & adāma \\
  2. & adās & adātam & adāta \\
  3. & adāt & adātām & adus \\
\end{tabular}
\end{center}

語根 bhū (ある)。

\begin{center}
\begin{tabular}{c*{3}{p{0.23\hsize}}}
  \multicolumn{4}{c}{爲自} \\
     & 單      & 兩      & 複 \\
  1. & abhūvam & abhūva  & abhūma \\
  2. & abhūs   & abhūtam & abhūta \\
  3. & abhūt   & abhūtām & abhūvan
\end{tabular}
\end{center}

爲自の dhā (置く)~二單 abhithās, sthā (住す)の asthithās,
kṛ (作る)の akṛthās, mṛ (死す)の amṛthās. 三單 abhita,
asthitha, akṛta, amṛta.

\subsubsection{a 語基}
\numberParagraph
a 語基の第三過去は a 級の第一過去(\ref{np:64} 條)の如く構
成せられ變化する。但し語根の母音は平音であつて變化しない。

sic (灌ぐ)~第一過去 asiñcat, 第三過去爲他 asicat, 爲自
asicata. lip (塗る)~第一過去 alimpat, 第三過去爲他 alipat,
爲自 alipata.

不規則なるものに dṛś (見る)三單爲自 adarśat, hve (呼ぶ)
ahvat の如きがある。

\subsection{重複第三過去}
\numberParagraph
重複第三過去は語根を重複して作られる。而して子音は
現在の場合(\ref{np:146} 條)と同樣に重複せられる。śri (行く)三單爲
他 aśiśriyat.

\begin{tabular}{ll}
不規則なるもの: & vac (言ふ)~ avocat. \\
                 & pat (落つ)~ apaptat.
\end{tabular}

この形が最も多く用ひられるのは第十類動詞並催起動詞であ
る。語基の aya は除かれる。重複音と語根の母音とは必ずしも
一樣でない。若し中間に a を有するか又は ā, ṛ, ṝ に終る場合
多くは i 又は ī を以て重複の母音とする。

\numberParagraph
第三過去の重複語基は重複音に重きを置き本來の語根は
比較的輕視せられる。かくして重複音の母音は常に長く,本來の
語根の母音は平音又は短母音である。

語根 cur (盗む)~ acūcuram, nī (導く)催起~ anīnayam,
yuj 催起(軛す)~ ayūyujam. dā 催起(與へしむ)~ adīdapat,
sthā 催起(立たしむ)~ atiṣṭhipat.

\subsection{硬複氣音第三過去}
\subsubsection{sa 語基 a + 語根 + sa}
\numberParagraph
sa 語基に屬する動詞は a 又は ā 以外の母音を有して
ś, ṣ, h に終れるものを以てする。語尾變化は á 級の第一過去
に同じ(\ref{np:64} 條)。但し一單爲自は e でなく i に終り,二三兩は
athām, ātām である。

diś (指す)~一單爲他 adikṣam, 三複 adikṣan, 一單爲自
adikṣi.

\subsubsection{s 語基 a + 語根 + s}
\numberParagraph
以下の三種は語基に强弱を分つ。この點よりして現在組
織の第二種變化に相當する。强基は爲他に用ひられ,弱基中基は
爲自に用ひられる。

\numberParagraph
s 語基は爲他には複重韻,爲自には i, ī, u, ū に終る語根は
ṣam, aśroṣi, kṛ (作す)~ akārṣam, akṛṣi, dṛś (見る)~三,單,
爲他 adrākṣīt (\ref{np:9} 條),bhaj (分つ)~一,單は abhākṣam,
abhakṣi. t 又は th にて始まる語尾の次前,鼻音にあらざる子音
の次後の s は省き去られる。tud (打つ)~二,複,爲自 atautta.
kṣip (投ぐ)~ akṣaipta. 然し man (思ふ)~三,單,爲自
amaṃsta. kṛ (作る)~二,複,爲自 akārṣṭa. dhvam の前に s
は消失する。若し ā 以外の母音に先立たれた時 s は消失して
dhvam は ḍhvam となる。卽ち akṛḍhvam. nī (導く)~
aneḍhvam.

\numberParagraph
nī (導く)强基 nais, 弱基 nes.

\begin{center}
\begin{tabular}{c*{3}{p{0.23\hsize}}}
  \multicolumn{4}{c}{爲他} \\
     & 單      & 兩       & 複 \\
  1. & anaiṣam & anaiṣva  & anaiṣma \\
  2. & anaiṣīs & anaiṣṭam & anaiṣṭa \\
  3. & anaiṣīt & anaiṣṭām & anaiṣus \\
  \multicolumn{4}{c}{爲自} \\
  1. & aneṣi    & aneṣvahi & aneṣmahi \\
  2. & aneṣṭhās & aneṣṭhās & aneḍhvam \\
  3. & aneṣṭa   & aneṣṭa   & aneṣata
\end{tabular}
\end{center}


%%% Local Variables:
%%% mode: latex
%%% TeX-master: "IntroductionToSanskrit"
%%% End:
