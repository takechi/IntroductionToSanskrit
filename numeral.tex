\section{數詞}
\subsection{順數}
\numberParagraph
\begin{longtable}{rlrl}
 1 & eka     & 13 & trayodaśa \\
 2 & dvi     & 14 & caturdaśa \\
 3 & tri     & 15 & pañcadaśa \\
 4 & catur   & 16 & ṣoḍaśa \\
 5 & pañca   & 17 & saptadaśa \\
 6 & ṣaṣ     & 18 & aṣṭādaśa \\
 7 & sapta   & 19 & navadaśa\footnotemark \\
 8 & aṣṭa    & 20 & viṃśati \\
 9 & nava    & 21 & ekaviṃśati \\
10 & daśa    & 22 & dvāviṃśati \\
11 & ekādaśa & 23 & trayoviṃśati \\
12 & dvādaśa & 24 & caturviṃśati \\
\\
25 & pañcaviṃśati              &   60 & ṣaṣṭi \\
26 & śaḍviṃśati                &   70 & saptati \\
27 & saptaviṃśati              &   80 & aśīti \\
28 & aṣṭāviṃśati               &   90 & navati \\
29 & navaviṃśati\footnotemark  &  100 & śata \\
30 & triṃśat                   &  200 & dve śate 又は dviśata\\
40 & catvāriṃśat               &  300 & trīṇi śatāni, 又は triśata \\
50 & pañcāśat                  & 1000 & sahasra \\
\end{longtable}
\footnotetext{又は ūnaviṃśati, ekonaviṃśati.}
\footnotetext{ūnatriṃśat.}

\numberParagraph
2, 3, 8 の數は 20, 30 に結合せられる時は夫々 dvā,
trayas, aṣṭā となり,80 に對しては dvi, tri, aṣṭa である。40
から 70 までと 90 とに對しては兩形が用ひられる。

19, 20 等は ūna (減ぜられし)又は ekona (一を減ぜし)を添へ
て呼ぶことがある。101, 102 等は adhika (越えたる,加へたる)
の語を添へて呼ぶ。ekādhikaṃ śatam 又は ekādhikaśatam.

\numberParagraph
eka (一)は單數並に複數(eke 或るもの)に於て三性に
變化し,代名詞的變化をなす。dvi (二)は規則的に dva の兩數
に變化せられ,tri と catur は次の如く變化する。

\begin{center}
\begin{tabular}{c*{9}{p{0.12\hsize}}}
     & 男                      & 中                     & 女                         & 男                        & 中                       & 女 \\
  主 & \rdelim\}{2}{*}[trayas] & \rdelim\}{3}{*}[trīṇi] & \multirow{3}{*}{tisras}    & \rdelim\}{2}{*}[catvāras] & \rdelim\}{3}{*}[catvāri] & \multirow{3}{*}{catasras} \\
  呼 &                         &                        &                            &                           &                          & \\
  業 & trīn                    &                        &                            & caturas                   &                          & \\
     & \multicolumn{2}{c}{\upbracefill}                 &                            & \multicolumn{2}{c}{\upbracefill}                     & \\
  具 & \multicolumn{2}{c}{tribhis}                      & tisṛbhis                   & \multicolumn{2}{c}{caturbhis}                        & catasṛbhis \\
  爲 & \multicolumn{2}{c}{\rdelim\}{2}{*}[tribhyas]}    & \multirow{2}{*}{tisṛbhyas} & \multicolumn{2}{c}{\rdelim\}{2}{*}[caturbhyas]}      & \multirow{2}{*}{catasṛbhyas} \\
  從 &                                                  &                            &                                                      & \\
  屬 & \multicolumn{2}{c}{trayāṇām}                     & tisṛṇām                    & \multicolumn{2}{c}{caturṇām}                         & catasṛṇām \\
  於 & \multicolumn{2}{c}{triṣu}                        & tisṛṣu                     & \multicolumn{2}{c}{caturṣu}                          & catasṛṣu
\end{tabular}
\end{center}

\numberParagraph
5 から 19 に至る迄は性の區別なく、若干の不規則はあ
るも複數として變化する。只主業は語基のまゝである。

主業呼 pañca, 具 pañcabhis, 爲從 pañcabhyas, 屬 pañcā\-%
nām, 於 pañcasu.

主業呼 ṣaṭ, 具 ṣaḍbhis, 屬 ṣaṇṇām, 於 ṣaṭsu.

主業呼 aṣṭa 又は aṣṭau, 具 aṣṭabhis 又は aṣṭābhis, 爲從
aṣṭabhyas, 又は aṣṭābhyas, 屬 aṣṭānām, 於 aṣṭasu 又は
aṣṭāsu.

\numberParagraph
20 より以上の數詞は名詞である。20 より 99 までは女
性單數,100, 1,000, 10,000 及び 100,000 は中性單數である。
數へられたる事物は或は複數となして數詞と同格にし,或は屬格
複數となして並べ,或は合成語となす。「二十人の婦女に圍まれ
たる王」と云ふ時 rājā viṃśatyā nārībhiḥ parivṛtaḥ 又は
rājā viṃśatyā nārīṇāṃ parivṛtaḥ; ṣaṣṭyāṃ varṣeṣu (六十年
の間),catvāri sahasrāṇi varṣāṇām (四千年),varṣa-śatam
(百年)。

\subsection{序數}
\numberParagraph
\begin{longtable}{rlrl}
 1. & prathama &  7. & saptama \\
 2. & dvitīya  &  8. & aṣṭama \\
 3. & tṛtīya   &  9. & navama \\
 4. & caturtha & 10. & daśama \\
 5. & pañcama  & 11. & ekādaśa \\
 6. & ṣaṣṭha   & 12. & dvādaśa \\
\\
20. & viṃśatitama, viṃśa          &   70. & saptatitama \\
30. & triṃśattama, triṃśa         &   80. & aśītitama \\
40. & catvāriṃśattama, catvāriṃśa &   90. & navātitama \\
50. & pañcāśattama, pañcaśa       &  100. & śatatama \\
60. & ṣaṣṭitama                   & 1000. & sahasratama \\
61. & ekaṣaṣṭitama, ekaṣaṣṭa      &       & \\
\end{longtable}

\numberParagraph
序數は a 語基男中性の變化に準ず。女性は ī を附加す
(caturthī, pañcamī)。但し最初の三は ā 語基(prathamā,
dvitīyā, tṛtīyā)なるを異とする。

\numberParagraph
prathama, dvitīya, tṛtīya は或る場合に代名詞的變化を
なし得。但し prathama は主複,dvitīya, tṛtīya は爲從,於,單
に於いてである。

\subsection{數の副詞}
\numberParagraph
\begin{enumerate}[label=(\alph*)]
\item sakṛt (一度),dvis (二度),tris (三度),catus (四
度),pañcakṛtvas (五度),ṣaṭkṛtvas (六度)等。
\item ekadhā (一重に),dvidhā 又は dvedhā (二重に),
tridhā 又は tredhā (三重に),caturdhā (四重に)等。
\item ekaśas (單獨に),dviśas (一對づゝに),triśas (三つ
宛で)等。
\end{enumerate}

\ex{第八}
\begin{longtable}{c*{2}{p{0.45\hsize}}}
 1. & yasyārthās tasya mitrāṇi, & 富める彼には友あり,\\
    & yasyārthās tasya bāndhavāḥ, & 富める彼には親族あり,\\
    & yasyārthaḥ sa pumā\anunasikamd{}l loke, & 富める彼は世間に於て人なり,\\
    & yasyārthāḥ sa hi paṇḍitaḥ. & 富める彼は實に學者なり。\\
 2. & bho, ko bhavān. & おゝ君は誰なるか。\\
 3. & na bhavati mad dhanyataraḥ. & 吾れよりも幸福なるものはあらず。\\
 4. & siddhāḥ sarve 'smākaṃ ma\-norathāḥ. & 我等の希望の總ては成就した。\\
 5. & bho vañcitā vayam anena. & おゝ我々は彼によりて欺かれた。\\
 6. & mitra, kiyatā mūlynaitat pustakaṃ gṛhītam. rūpakānāṃ śatena. & 友よ\ruby[g]{幾何}{いくばく}にてこの書が得ら
 れしか。百ルーピー。\\
 7. & asminn eva latāgṛhe tvam abhavaḥ. & その蔓草の家に於て汝はありき。\\
 8. & kṛtaṃ kim ebhis tava vipri\-yaṃ, yad aniṣṭam eṣāṃ cinta\-yasi. & 汝が彼等に就て禍と考へる
 所の汝に不快な何ものが是等によりて爲されたるか。\\
 9. & tvaṃ me jīvitaṃ tvaṃ me hṛdayaṃ dvitīyaṃ tvaṃ kaumu\-dī nayanayor amṛtaṃ tvam aṅge. & 汝は吾が生命である,汝は
 吾が第二の心である,汝は兩眼に於て月光なり。身に於て汝は甘露なり。\\
10. & sarvasyātithir guruḥ. & 一切にとつて賓客は尊敬すべきだ。\\
11. & kā sā purī, ko vā deśaḥ. & かの町は何ぞ,又如何なる地方なるか。\\
12. & nīrasāny api rocante naḥ kar\-pāsasya phlāni. & 味は無きも綿の實は我々に願はしくある。\\
13. & caturdaśa sahasrāṇi śatrūṇāṃ raṇe hatāni. & 一萬四千の敵は戰に於いて殺された。\\
14. & ete trayaḥ puruṣaya gariṣṭhā bhavanti: ācāryaḥ, pitā, mā\-tā ca. & これら三人は人の中で最も
重んぜられてある。卽ち師,父及び母である。\\
15. & te putrā ye pitur bhaktāḥ, & 父に從順なるは子であり,\\
    & sa pitā yas tu poṣakaḥ, & 養育するものは父であり,\\
    & tan mitraṃ yatra viśvāsaḥ, & 信賴あるものは友であり,\\
    & sā bhāryā yasya nirvṛtiḥ. & 滿足あるものは妻である。
\end{longtable}

%%% Local Variables:
%%% mode: latex
%%% TeX-master: "IntroductionToSanskrit"
%%% End:
